\documentclass[11pt,fleqn]{book} % Default font size and left-justified equations

%--------------------------------------------------------------------------
% Document geometry % Page margins
%--------------------------------------------------------------------------
\usepackage[margin=1in,headsep=10pt,a4paper]{geometry}

%--------------------------------------------------------------------------
% MinionPro fonts
%--------------------------------------------------------------------------
\IfFileExists{MinionPro.sty}{%
  \usepackage{MnSymbol}
  \usepackage{MinionPro}
  \usepackage{MyriadPro}
}{%
  \usepackage[T1]{fontenc}
  \usepackage{newtxtext}
  %\usepackage{eulervm}
  \usepackage{amssymb}%needed for \mathbb
}

%--------------------------------------------------------------------------
% Color scheme used in this book
%--------------------------------------------------------------------------
\usepackage{xcolor} % Required for specifying colors by name
\definecolor{ocre}{RGB}{243,102,25} % Define the orange color used for highlighting throughout the book
\definecolor{dkgreen}{rgb}{0,0.6,0}
\definecolor{delim}{RGB}{20,105,176}
\definecolor{background}{HTML}{EEEEEE}
\definecolor{Red}{rgb}{0.8666,0.03137,0.02352}
\definecolor{Blue}{rgb}{0.00784,0.67059,0.91764}
\definecolor{Darkgreen}{rgb}{0,0.68235,0}
\definecolor{Green}{rgb}{0,0.8,0}
\definecolor{Royalblue}{rgb}{0,0.2,0.91764}
\definecolor{Brickred}{rgb}{0.644541,0.164065,0.164065}
\definecolor{Brown}{rgb}{0.6,0.4,0.4}
\definecolor{Orange}{rgb}{1,0.647059,0}
\definecolor{Indigo}{rgb}{0.746105,0,0.996109}
\definecolor{Violet}{rgb}{0.308598,0.183597,0.308598}
\definecolor{Lightgrey}{rgb}{0.762951,0.762951,0.762951}
\definecolor{Darkgrey}{rgb}{0.503548,0.503548,0.503548}
\definecolor{Pink}{rgb}{1,0.6,0.6}
\definecolor{DarkBlue}{rgb}{0,0.08,0.45}
\definecolor{OliveDrab}{rgb}{0.41961,0.55686,0.13725}
\definecolor{LightMagenta}{cmyk}{0.1,0.8,0,0.1}
\definecolor{OliveD}{HTML}{6B8E23}
\definecolor{authorNoteColor}{rgb}{.8,0,0}
\newcommand{\Ochre}{\color{ocre}}
\newcommand{\Red}{\color{Brickred}}
\newcommand{\Blue}{\color{Royalblue}}
\newcommand{\Green}{\color{Darkgreen}}
\newcommand{\DarkGreen}{\color{Darkgreen}}
\newcommand{\Violet}{\color{Violet}}
\newcommand{\Indigo}{\color{Indigo}}
\newcommand{\Orange}{\color{Orange}}
\newcommand{\Brown}{\color{Brown}}
\newcommand{\OliveD}{\color{OliveD}}
\newcommand{\Pink}{\color{Pink}}


%--------------------------------------------------------------------------
% Bibliography : biblatex does not work with tufte-book
%--------------------------------------------------------------------------
\usepackage[style=numeric,
            citestyle=numeric-comp,
            sorting=none,
            sortcites=true,
            autopunct=true,
            babel=hyphen,
            hyperref=true,
            abbreviate=false,
            backref=true,
            backend=biber]{biblatex}
\addbibresource{../Bibliography.bib} % BibTeX bibliography file
\defbibheading{bibempty}{}

%--------------------------------------------------------------------------
% Graphics
%--------------------------------------------------------------------------
\usepackage[pdftex]{graphicx}
\DeclareGraphicsExtensions{{.png},{.pdf},{.jpg},{jpeg}}
\graphicspath{ {../figures/}}
\setkeys{Gin}{width=\linewidth,totalheight=\textheight,keepaspectratio}

%--------------------------------------------------------------------------
% Links
%--------------------------------------------------------------------------
\usepackage{hyperref}
\hypersetup{pdftex, colorlinks=true, linkcolor=blue, citecolor=blue, filecolor=blue, urlcolor=blue, pdftitle=, pdfauthor= , pdfsubject=, pdfkeywords=}

%--------------------------------------------------------------------------
% Code listing
%--------------------------------------------------------------------------
\usepackage{listings}

\lstset{ %
  basicstyle=\scriptsize\ttfamily,           % the size of the fonts that are used for the code
  backgroundcolor=\color{white},      % choose the background color. You must add \usepackage{color}
  showspaces=false,               % show spaces adding particular underscores
  showstringspaces=false,         % underline spaces within strings
  showtabs=false,                 % show tabs within strings adding particular underscores
  tabsize=2,                      % sets default tabsize to 2 spaces
  captionpos=b,                   % sets the caption-position to bottom
  breaklines=true,                % sets automatic line breaking
  breakatwhitespace=false,        % sets if automatic breaks should only happen at whitespace
  commentstyle=\color{dkgreen}\upshape,       % comment style
  escapeinside={\%*}{*)},            % if you want to add LaTeX within your code
  morekeywords={*,MPM,ICE,MPMICE}               % if you want to add more keywords to the set
}

\colorlet{punct}{red!60!black}
\colorlet{numb}{magenta!60!black}

\lstdefinelanguage{XML}
{
  morestring=[b]",
  morestring=[s]{>}{<},
  morecomment=[s]{<?}{?>},
  morestring=[s]{"}{"},
  morecomment=[s]{?}{?},
  morecomment=[s]{!--}{--},
  stringstyle=\color{black},
  identifierstyle=\color{DarkBlue},
  keywordstyle=\color{Brickred},
  backgroundcolor=\color{background},
  frame=lines,
  morekeywords={xmlns,version,type}% list your attributes here
}

\lstdefinelanguage{C++}
{
  keywordstyle=\color{Brickred},
  stringstyle=\color{red},
  identifierstyle=\color{DarkBlue},
  backgroundcolor=\color{background},
  frame=lines,
  morecomment=[l][\color{magenta}]{\#}
}

\lstdefinelanguage{Cpp}
{
  language=C++,
  commentstyle=\color{dkgreen}\upshape,   
  stringstyle=\color{red},
  identifierstyle=\color{DarkBlue},
  backgroundcolor=\color{background},
  frame=lines,
  deletekeywords={...},
  escapeinside={\%*}{*)},
  keywordstyle=\color{Brickred},
  morekeywords={ProblemSpecP, ProcessorGroup, SimulationController,%
                AMRSimulationController, RegridderCommon, SolverInterface,%
                SolverFactory, UintahParallelComponent, SimulationInterface,%
                ComponentFactory, LoadBalancerCommon, LoadBalancerFactory,%
                DataArchiver, Output, SchedulerCommon, SchedulerFactory,%
                ProblemSetupException, Exception,%
                Task, Level, Patch, Ghost, Example,%
                GridP, SimulationStateP, LevelP, SchedulerP,%
                PatchSubset, MaterialSubset, DataWarehouse,% 
                SFCXVariable, PerPatch, IntVector, NodeIterator,%
                VarLabel, MPMMaterial, PatchSet, ParticleSubset,%
                MPMFlags, FlowModel, ConstitutiveModel, MyModel, MyFlow, PlasticityState,%
                particleIndex, TangentModulusTensor, Matrix3, Vector,
                ElasticModuli, TabularPlasticity, ModelState_Tabular, ParticleVariable,
                constParticleVariable, delt_vartype, ElasticModuliModelFactory,%
                YieldConditionFactory, ElasticModuliModel, YieldCondition, CMData,
                ParticleLabelVariableMap, ParameterDict, TabularData, IndependentVar,
                DependentVar, IndexKey, json, DoubleVec1D, DoubleVec2D,
                int,char,double,float,unsigned, size_t,%
                string, istringstream, cerr, exit}, 
  morecomment=[l][\color{magenta}]{\#}
}


\lstdefinelanguage{JSON}{
    numbers=left,
    numberstyle=\scriptsize,
    stepnumber=1,
    numbersep=8pt,
    showstringspaces=false,
    breaklines=true,
    frame=lines,
    backgroundcolor=\color{background},
    literate=
     *{0}{{{\color{numb}0}}}{1}
      {1}{{{\color{numb}1}}}{1}
      {2}{{{\color{numb}2}}}{1}
      {3}{{{\color{numb}3}}}{1}
      {4}{{{\color{numb}4}}}{1}
      {5}{{{\color{numb}5}}}{1}
      {6}{{{\color{numb}6}}}{1}
      {7}{{{\color{numb}7}}}{1}
      {8}{{{\color{numb}8}}}{1}
      {9}{{{\color{numb}9}}}{1}
      {:}{{{\color{punct}{:}}}}{1}
      {,}{{{\color{punct}{,}}}}{1}
      {\{}{{{\color{delim}{\{}}}}{1}
      {\}}{{{\color{delim}{\}}}}}{1}
      {[}{{{\color{delim}{[}}}}{1}
      {]}{{{\color{delim}{]}}}}{1},
}


%--------------------------------------------------------------------------
% Colored boxes:  Must come after graphicx and verbatim
% and Tikz
%--------------------------------------------------------------------------
\RequirePackage{tikz}
\usetikzlibrary{shadings,shadows}
\usetikzlibrary{decorations.pathmorphing}
\usetikzlibrary{patterns}
\usetikzlibrary{intersections}
\usetikzlibrary{calc}

\usepackage{tcolorbox}
\tcbuselibrary{most}
\newtcolorbox{NoteBox}[1][]{%
  colback=yellow!50,
  colframe=yellow!20!black,
  before skip=2mm,after skip=3mm,
  boxrule=0.4pt,left=5mm,right=2mm,top=1mm,bottom=1mm,
  sharp corners,
  rounded corners=southeast,arc is angular,arc=3mm,
  underlay={%
    \path[fill=tcbcol@back!80!black] ([yshift=3mm]interior.south east)--++(-0.4,-0.1)--++(0.1,-0.2);
    \path[draw=tcbcol@frame,shorten <=-0.05mm,shorten >=-0.05mm] ([yshift=3mm]interior.south east)--++(-0.4,-0.1)--++(0.1,-0.2);
    \path[fill=yellow!50!black,draw=none] (interior.south west) rectangle node[white]{\Huge\bfseries !} ([xshift=4mm]interior.north west);
    },
  drop fuzzy shadow,%
  #1}

\newtcolorbox{WarningBox}[1][]{%
  colback=pink!50,
  colframe=pink!20!black,
  before skip=2mm,after skip=3mm,
  boxrule=0.4pt,left=5mm,right=2mm,top=1mm,bottom=1mm,
  sharp corners,
  rounded corners=southeast,arc is angular,arc=3mm,
  underlay={%
    \path[fill=tcbcol@back!80!black] ([yshift=3mm]interior.south east)--++(-0.4,-0.1)--++(0.1,-0.2);
    \path[draw=tcbcol@frame,shorten <=-0.05mm,shorten >=-0.05mm] ([yshift=3mm]interior.south east)--++(-0.4,-0.1)--++(0.1,-0.2);
    \path[fill=pink!50!black,draw=none] (interior.south west) rectangle node[white]{\Huge\bfseries !} ([xshift=4mm]interior.north west);
    },
  drop fuzzy shadow,%
  #1}

\tcbuselibrary{skins}
\newtcolorbox[auto counter,number within=chapter]{ExampleBox}[1][]{%
  enhanced,
  colback=white,
  colframe=green!65!black,
  enlarge top by=10mm,
  overlay={%
    \path[fill=blue!65,line width=.4mm] (frame.north west)--++(17mm,0)coordinate(n2)--++(0,8mm)--++(-20mm,0) arc (-90:90:-4mm)--cycle;
    \node at ([shift={(5mm,4mm)}]frame.north west){\color{white}{\textbf{\sffamily EXAMPLE}}};
    \path[fill=green!65!blue] ([xshift=.4mm]n2)--++(0,8mm)--++(7mm,0)--++(0,-8mm)--cycle;
    \node at ([shift={(4mm,4mm)}]n2){\color{white}{\textbf{\sffamily \thetcbcounter}}};
    %\node at ([shift={(18mm,4mm)}]n2){\itshape\textbf{\sffamily Solution}};
  },
  #1}

\usetikzlibrary{calc}
\usetikzlibrary{positioning}
\tcbuselibrary{skins}
\newtcolorbox[auto counter,number within=section]{SummaryBox}[2][]{%
  enhanced,
  colback=white,
  colframe=ocre,
  enlarge top by=10mm,
  overlay={%
    \path[fill=ocre!65,line width=.4mm] (frame.north west)--++(17mm,0)coordinate(n2)--++(0,8mm)--++(-20mm,0) arc (-90:90:-4mm)--cycle;
    \node at ([shift={(5mm,4mm)}]frame.north west){\color{white}{\textbf{\sffamily Summary}}};
    \path[fill=ocre] ([xshift=.4mm]n2)--++(0,8mm)--++(10mm,0)--++(0,-8mm)--cycle;
    \node (A) at ([shift={(5mm,4mm)}]n2){\color{white}{\textbf{\sffamily \thetcbcounter}}};
    \node [right=0.5cm of A] {\itshape\textbf{\sffamily #2}};
  },
  #1}


%--------------------------------------------------------------------------
% Structure of the document
%--------------------------------------------------------------------------
%\input{00LayoutStructure}

%--------------------------------------------------------------------------
% Other packges
%--------------------------------------------------------------------------
%\usepackage{bm}%to get bold math symbols
%\usepackage[pdftex]{graphicx}
%\usepackage{verbatim}
%\usepackage[tt]{titlepic}%package i downloaded to put a pic in the titlepage
%\usepackage{helvet}
%\usepackage[usenames,dvipsnames]{xcolor}%options to use names like redviolet and others
%\usepackage{xspace}
%\usepackage[mathscr]{eucal}%Defines which font to use with \mathscr
%\usepackage{enumerate}
%\usepackage{comment}
%\usepackage{mathrsfs}
%\usepackage{floatrow}%automatically centers graphics without needing a \center command on each and every figure.

% Index
%\usepackage{calc} % For simpler calculation - used for spacing the index letter headings correctly
%\usepackage{makeidx} % Required to make an index

%-------------------------------------------------------
% For appendix
%-------------------------------------------------------
%\usepackage{esint}
\usepackage[toc,page,title,titletoc]{appendix}

%\usepackage{makeidx} % Used to generate the index

%\usepackage{xspace} % Used for printing a trailing space better than using a tilde (~) using the \xspace command

%\usepackage{epsf}
%\usepackage{epsfig}
%\usepackage{boxedminipage}
%\usepackage{stmaryrd}
%\usepackage{cancel}

%-------------------------------------------------------
% For text wrap around figures and captions for subfigures
%-------------------------------------------------------
\usepackage{wrapfig}
\usepackage{subcaption}

%-------------------------------------------------------
% For underlines
%-------------------------------------------------------
\usepackage[normalem]{ulem}

%-------------------------------------------------------
% For algorithms
%-------------------------------------------------------
\usepackage{algorithm, algpseudocode, float}
%\usepackage{algorithmic}

\makeatletter
\newenvironment{breakablealgorithm}
  {% \begin{breakablealgorithm}
   \begin{center}
     \refstepcounter{algorithm}% New algorithm
     \hrule height.8pt depth0pt \kern2pt% \@fs@pre for \@fs@ruled
     \renewcommand{\caption}[2][\relax]{% Make a new \caption
       {\raggedright\textbf{\ALG@name~\thealgorithm} ##2\par}%
       \ifx\relax##1\relax % #1 is \relax
         \addcontentsline{loa}{algorithm}{\protect\numberline{\thealgorithm}##2}%
       \else % #1 is not \relax
         \addcontentsline{loa}{algorithm}{\protect\numberline{\thealgorithm}##1}%
       \fi
       \kern2pt\hrule\kern2pt
     }
  }{% \end{breakablealgorithm}
     \kern2pt\hrule\relax% \@fs@post for \@fs@ruled
   \end{center}
  }
\makeatother
\renewcommand{\algorithmiccomment}[1]{\hfill{\color{ocre}$\triangleright$\textit{#1}}}



%\input{00LatexMacros.tex}

%-------------------------------------------------------
% For index
%-------------------------------------------------------
\makeindex 

%-------------------------------------------------------
% Background picture
%-------------------------------------------------------
\newcommand\BackgroundPic{%
\put(0,0){%
\parbox[b][\paperheight]{\paperwidth}{%
\vfill
\centering
\includegraphics[width=0.5\paperwidth,height=0.5\paperheight,%
keepaspectratio]{periodic_ellipsoids_tighter_fit.png}%
\vfill
}}}

%-------------------------------------------------------
% Versioning
%-------------------------------------------------------
\makeatletter
\def\MyYear#1{%
  \def\yy@##1##2##3##4;{##3##4}%
  \expandafter\yy@#1;
}
\def\MyMonth#1{%
  \def\yy@##1;{0##1}%
  \def\yy@##1##2;{##1##2}%
  \expandafter\yy@#1;
}
\makeatother
\newcommand{\version}{Version \MyYear{\the\year}.\MyMonth{\the\month}}



\input{../00LayoutStructure.tex}
\input{../00LatexMacros.tex}


\begin{document}

  %----------------------------------------------------------------------------------------
  %       TITLE PAGE
  %----------------------------------------------------------------------------------------
  \begingroup
    \thispagestyle{empty}
    \AddToShipoutPicture*{\BackgroundPic} % Image background
    \centering
    \vspace*{1cm}
    \par\normalfont\fontsize{35}{35}\sffamily\selectfont
    \textcolor{black}{GranularSim Installation Guide}\par % Book title
    \vspace*{0.5cm}
    {\Large \textcolor{black}{Version \version}}\par
    {\Large \textcolor{black}{\today}}\par
    \vspace*{13cm}
    {\Large \textcolor{black}{Biswajit Banerjee}}\par % Author name
  \endgroup

    %----------------------------------------------------------------------------------------
  %       COPYRIGHT PAGE
  %----------------------------------------------------------------------------------------

  \newpage
  ~\vfill
  \thispagestyle{empty}

  \noindent Copyright \copyright\ 2013 Callaghan Innovation\\ % Copyright notice
  \noindent Copyright \copyright\ 2015-2016 Parresia Research Limited\\ % Copyright notice

  %\noindent \textsc{Published by Publisher}\\ % Publisher

  %\noindent \textsc{book-website.com}\\ % URL

  %\noindent Licensed under the Creative Commons Attribution-NonCommercial 3.0 Unported License (the ``License''). You may not use this file except in compliance with the License. You may obtain a copy of the License at \url{http://creativecommons.org/licenses/by-nc/3.0}. Unless required by applicable law or agreed to in writing, software distributed under the License is distributed on an \textsc{``AS IS'' BASIS, WITHOUT WARRANTIES OR CONDITIONS OF ANY KIND}, either express or implied. See the License for the specific language governing permissions and limitations under the License.\\ % License information

  \noindent The contents of this manual can and will change significantly over time.  Please make sure that all the information is up to date.
  %\noindent \textit{First printing, March 2013} % Printing/edition date

  %----------------------------------------------------------------------------------------
  %       TABLE OF CONTENTS
  %----------------------------------------------------------------------------------------
  \chapterimage{fire-container-explosion-scirun-cut.png} % Table of contents heading image
  \pagestyle{empty} % No headers
  \tableofcontents % Print the table of contents itself
  \cleardoublepage % Forces the first chapter to start on an odd page so it's on the right
  \pagestyle{fancy} % Print headers again

  \setlength{\parindent}{0pt}
  \setlength{\parskip}{5pt}




\chapter{Overview of GranularSim} \label{sec:overview} 
\GranularSim is a rewritten and updated version of the parallel discrete element code \ParaEllip
developed at the University of Colorado at Boulder, with the intent of having a more general
code that can be improved and appended to more easily.  The primary use case is the simulation
of soils and other granular materials.

The main components of \GranularSim include a discrete element (DEM) code that can handle
ellipsoidal particles, a peridynamics (PD) code that couples to DEM particles, and
a smoothed particle hydrodynamics (SPH) code that also couples to DEM particles.
In addition, the code is able to handle special boundary conditions that are needed
for hierarchical upscaling.

\GranularSim provides outputs in VTK format and results can be visualized either
with \ParaView or with \Visit (\url{https://wci.llnl.gov/simulation/computer-codes/visit/}).

\section{Obtaining the Code}
\GranularSim can be obtained either as a zipped archive (approx. 550 Mb) from
\begin{lstlisting}[language=sh, backgroundcolor=\color{background}]
https://github.com/bbanerjee/ParSim/archive/master.zip
\end{lstlisting}
or cloned from Github using either HTTP
\begin{lstlisting}[language=sh, backgroundcolor=\color{background}]
git clone --recursive https://github.com/bbanerjee/ParSim.git
\end{lstlisting}
or SSH
\begin{lstlisting}[language=sh, backgroundcolor=\color{background}]
git clone --recursive git@github.com:bbanerjee/ParSim.git
\end{lstlisting}
The above commands check out the \Parsim source tree and
installs it into a directory called \Textsfc{ParSim} in the user's home
directory.  The \Parsim code contains several research codes.  The
main \GranularSim code can be found in the \Textsfc{ParSim/GranularSim/src}
directory.  Manuals for \GranularSim are in \Textsfc{ParSim/GranularSim/Manuals}.

You may also need to install \Textsfc{git} on your machine if you 
want to clone the repository from \Textsfc{Github}:
\begin{lstlisting}[backgroundcolor=\color{background}]
sudo apt-get install git-core
\end{lstlisting}

\begin{WarningBox}
The \Textsfc{--recursive} flag is needed in order to make sure that the
submodules \Textsfc{googletest}, \Textsfc{cppcodec}, and \Textsfc{qhull} are also populated
during the download process.
\end{WarningBox}

\section{Developer version}
If you are a developer, we will suggest that you use the following longer process.
\begin{enumerate}
  \item  Create a \Textsfc{GitHub} account
\begin{lstlisting}[backgroundcolor=\color{background}]
   https://github.com/signup/free
\end{lstlisting}

  \item  Install \Textsfc{git} on your machine
\begin{lstlisting}[backgroundcolor=\color{background}]
   sudo apt-get install git-core
\end{lstlisting}

  \item  Setup git configuration
\begin{lstlisting}[backgroundcolor=\color{background}]
   git config --global user.email "your_email@your_address.com"
   git config --global user.name "your_github_username"
\end{lstlisting}

  \item Email your \Textsfc{github} username to the owner of \Parsim.

  \item Create a clone of the ParSim repository
\begin{lstlisting}[backgroundcolor=\color{background}]
   git clone https://github.com/bbanerjee/ParSim
\end{lstlisting}

  \item Check branches/master
\begin{lstlisting}[backgroundcolor=\color{background}]
  git branch -a
\end{lstlisting}

  \item Configure \Textsfc{ssh} for secure access (if needed)
\begin{lstlisting}[backgroundcolor=\color{background}]
  cd ~/.ssh
  mkdir backup
  cp * backup/
  rm id_rsa*
  ssh-keygen -t rsa -C "your_email@your_address.com"
  sudo apt-get install xclip
  cd ~/ParSim/
  xclip -sel clip < ~/.ssh/id_rsa.pub
  ssh -T git@github.com
  ssh-add
  ssh -T git@github.com
\end{lstlisting}

  \item For password-less access with \Textsfc{ssh}, use
\begin{lstlisting}[backgroundcolor=\color{background}]
  git remote set-url origin git@github.com:bbanerjee/ParSim.git
\end{lstlisting}

  \item To change the default message editor, do
\begin{lstlisting}[backgroundcolor=\color{background}]
  git config --global core.editor "vim"
\end{lstlisting}
\end{enumerate}

\begin{WarningBox}
If you are new to \GranularSim, we strongly suggest that you fork the code from GitHub and submit 
your changes via pull requests.
\end{WarningBox}

\subsection{Getting the latest version, adding files etc.}
\begin{enumerate}
  \item Check status
\begin{lstlisting}[backgroundcolor=\color{background}]
  git status
  git diff origin/master
\end{lstlisting}

  \item Update local repository
\begin{lstlisting}[backgroundcolor=\color{background}]
  git pull
\end{lstlisting}

  \item Add file to your local repository
\begin{lstlisting}[backgroundcolor=\color{background}]
  git add README
  git commit -m 'Added README file for git'
\end{lstlisting}

  \item Update main repository with your changes
\begin{lstlisting}[backgroundcolor=\color{background}]
  git push origin master
\end{lstlisting}
\end{enumerate}

\begin{WarningBox}
If you are new to \GranularSim, we strongly suggest that you fork the code from GitHub and submit 
your changes via pull requests.
\end{WarningBox}

\chapter{Prerequisites and library dependencies}
\GranularSim depends on several tools and libraries that are
commonly available or easily installed on various Linux or Unix like
OS distributions.  

\begin{WarningBox}
The following instructions are tailored for installations on Ubuntu/Debian-like systems.
\end{WarningBox}

\section{Using a package manager}
If you are using a package manager to install the prerequisites, follow the instructions
in this section.  Otherwise, go to section~\ref{sec:build_from_src}.

\subsection{Compilers}
You will need a C++ compiler, either \Textsfc{g++} or \Textsfc{clang}:
\begin{lstlisting}[backgroundcolor=\color{background}]
sudo apt-get install g++
\end{lstlisting}
\begin{lstlisting}[backgroundcolor=\color{background}]
sudo apt-get install clang-3.8
\end{lstlisting}

You will also need a C Compiler such as \Textsfc{gcc}:
\begin{lstlisting}[backgroundcolor=\color{background}]
sudo apt-get install gcc
\end{lstlisting}

\subsection{Cmake}
You will probably need to install \Textsfc{Cmake} to control the software compilation process. To do that:
\begin{lstlisting}[backgroundcolor=\color{background}]
     sudo apt-get install cmake
\end{lstlisting}

\subsection{Boost}
You will need \Textsfc{Boost} for parts of the code (and also some of the unit tests) to compile.
\GranularSim uses the boost MPI and serialization libraries.  It's easiest to install everything from Boost.
\begin{lstlisting}[backgroundcolor=\color{background}]
sudo apt-get install libboost-all-dev
\end{lstlisting}

\subsection{MPI libraries}
Install the \Textsfc{OpenMPI} libraries:
\begin{lstlisting}[backgroundcolor=\color{background}]
sudo apt-get install mpi-default-dev
\end{lstlisting}

\subsection{OpenMP libraries}
For threaded simulations you will need the \Textsfc{OpenMP} library.  This library is included with
both \Textsfc{gcc} and \Textsfc{clang}.  All \GranularSim programs are compiled with \Textsfc{OpenMP} enabled.

\subsection{Eigen3 library}
Some parts of \GranularSim use the \Textsfc{Eigen3} library. You will have to install 
this library if you don't have it in your system:
\begin{lstlisting}[backgroundcolor=\color{background}]
sudo apt-get install libeigen3-dev
\end{lstlisting}

\subsection{XML libraries}
If you are reading data in \Vaango format into \GranularSim, yo will need the 
\Textsfc{libxml2} library:
\begin{lstlisting}[backgroundcolor=\color{background}]
sudo apt-get install libxml2
sudo apt-get install libxml2-dev
\end{lstlisting}

\GranularSim reads and writes data in Ellip3D format.  You will need \Textsfc{zenxml} to
read that data.  This header-only library is included with the source code.

\subsection{JSON libraries}
The \Textsfc{json} header-only library is also cloned into the repository when you use
the \Textsfc{--recursive} flag.  You don't need any further installation.

\subsection{Zlib compression library}
You will also need to install the development version of \Textsfc{zlib}.
\begin{lstlisting}[backgroundcolor=\color{background}]
sudo apt-get install zlib1g zlib1g-dev
\end{lstlisting}

\subsection{GoogleTest library}
The \Textsfc{googletest} libraries are cloned into the repository when you use
the \Textsfc{--recursive} flag.  You don't need any further installation.

\subsection{VTK libraries}
In order for output to be written in \VTK format, you will also need the \Textsfc{VTK} libraries. Use
\begin{lstlisting}[backgroundcolor=\color{background}]
sudo apt-get install libvtk5-dev
sudo apt-get install python-vtk tcl-vtk libvtk-java libvtk5-qt4-dev
\end{lstlisting}

\subsection{Qhull library}
\begin{WarningBox}
You will have to compile \Textsfc{Qhull} before you can compile \Textsfc{GranularSim}.
\end{WarningBox}

The \Textsfc{Qhull} source code is a submodule of the repository.  For \Textsfc{gcc} builds,
do the following to compile \Textsfc{Qhull}:
\begin{lstlisting}[backgroundcolor=\color{background}]
cd ParSim/GranularSim/src/qhull/build
cmake ..
make
\end{lstlisting}
If you want to compile with \Textsfc{clang}, use
\begin{lstlisting}[backgroundcolor=\color{background}]
cd ParSim/GranularSim/src/qhull
mkdir build_clang
cd build_clang
CC=/usr/local/bin/clang CXX=/usr/local/bin/clang++ cmake ..
make
\end{lstlisting}
The actual paths to the \Textsfc{clang} compilers may vary depending on your system setup.

\section{Building from source} \label{sec:build_from_src}
Building on desktop is easier if you use a package manager.  However, in some situations you may have to 
build from source.  We will discuss an example of building the code on a Cray machine below.

\begin{WarningBox}
We have been successful in building on a Cray with the \Textsfc{gcc} compiler but failed with
\Textsfc{clang} because the C++11 headers were not available on that machine.  Our experience is 
discussed below but yours may differ.
\end{WarningBox}

\subsection{Check what is already installed}

\begin{WarningBox}
Sometimes the head node and the compile node may have different versions of the \Textsfc{gcc} libraries.  
\end{WarningBox}

To check the modules that are available on the compilation node, log into that node using a command such as
\begin{lstlisting}[backgroundcolor=\color{background}]
  ssh scompile@machine.domain
\end{lstlisting}
and then run
\begin{lstlisting}[backgroundcolor=\color{background}]
  module avail
\end{lstlisting}

In this example we will assume that \Textsfc{gcc 6.1.0}, \Textsfc{cmake}, \Textsfc{git},
and \Textsfc{zlib} are already installed and available as the
default option.  We will load those modules using
\begin{lstlisting}[backgroundcolor=\color{background}]
  module load gcc/6.1.0
  module load cmake/3.6.2
  module load git
  module load zlib
\end{lstlisting}

\subsection{Building VTK}

Typically, you will not find \Textsfc{VTK} installed on the machine and will have to build the
associated libraries.  We have found that \Textsfc{VTK-7.1.1} can be built with \Textsfc{gcc-6.1.0} while
older versions do not build properly.

To download the \Textsfc{VTK} source code use
\begin{lstlisting}[backgroundcolor=\color{background}]
mkdir VTK-7.1.1
cd VTK-7.1.1
wget https://gitlab.kitware.com/vtk/vtk/repository/v7.1.1/archive.tar.gz
tar xvfz archive.tar.gz
\end{lstlisting}

To build the code, created a \Textsfc{build} directory and run
\begin{lstlisting}[backgroundcolor=\color{background}]
mkdir build
cd build
ccmake ../vtk-v7.1.1-b86da7eef93f75c4a7f524b3644523ae6b651bc4/
\end{lstlisting}

We suggest that you adjust the build flags in the \Textsfc{ccmake} graphical interface so that
the build script downloads all the necessary libraries. Also specify that the build be an optimized build 
and that no tests are to be run.  You will also need to specify the install prefix to be a local 
\Textsfc{vtk} directory.

To run the build script use:
\begin{lstlisting}[backgroundcolor=\color{background}]
make -j8
make install
\end{lstlisting}

\subsection{Building Boost}

We have been able to successfully build \Textsfc{boost-1.65.1}.  
\begin{WarningBox}
To build the Boost \Textsfc{mpi} libraries, you may have to create the file
\Textsfc{/home/user/user-config.jam} containing the single line \Textsfc{using mpi ;} 
before you do the bootstrap procedure.
\end{WarningBox}

The process is quite straightfoward:
\begin{lstlisting}[backgroundcolor=\color{background}]
wget https://dl.bintray.com/boostorg/release/1.65.1/source/boost_1_65_1.tar.gz
tar xvfz boost_1_65_1.tar.gz
cd boost_1_65_1/
./bootstrap.sh --prefix=../boost --with-libraries=mpi,serialization --show-libraries
./b2  link=shared install
\end{lstlisting}

One problem that we ran into with modern compilers is that some preprocessor directives are 
not understand by the compiler.  If you run into that problem, locate the file
\begin{lstlisting}[backgroundcolor=\color{background}]
boost/include/boost/archive/detail/iserializer.hpp
\end{lstlisting}

and replace
\begin{lstlisting}[backgroundcolor=\color{background}]
#ifndef BOOST_MSVC
    #define DONT_USE_HAS_NEW_OPERATOR (                    \
           BOOST_WORKAROUND(__IBMCPP__, < 1210)            \
        || defined(__SUNPRO_CC) && (__SUNPRO_CC < 0x590)   \
    )
#else
    #define DONT_USE_HAS_NEW_OPERATOR 0
#endif
\end{lstlisting}
with
\begin{lstlisting}[backgroundcolor=\color{background}]
#ifndef BOOST_MSVC
        #if BOOST_WORKAROUND(__IBMCPP__, < 1210)               \
            || defined(__SUNPRO_CC) && (__SUNPRO_CC < 0x590)
                #define DONT_USE_HAS_NEW_OPERATOR 1
        #else
                #define DONT_USE_HAS_NEW_OPERATOR 0
        #endif
#else
    #define DONT_USE_HAS_NEW_OPERATOR 0
#endif
\end{lstlisting}

\subsection{Building libxml2}
Building \Textsfc{libxml2} is straightforward.  Just follow the sequence of steps below:

\begin{lstlisting}[backgroundcolor=\color{background}]
wget ftp://xmlsoft.org/libxml2/libxml2-2.9.7.tar.gz
tar xvfz libxml2-2.9.7.tar.gz
module load zlib
module unload python
./configure --prefix=<home-directory>/libxml2 --enable-shared --without-python
make -j4
make install
\end{lstlisting}

\subsection{Building eigen3}
The \Textsfc{eigen3} build is also straightforward:

\begin{lstlisting}[backgroundcolor=\color{background}]
wget http://bitbucket.org/eigen/eigen/get/3.3.4.tar.gz
tar xvfz 3.3.4.tar.gz
mkdir eigen-build
cd eigen-build
cmake ../eigen-eigen-5a0156e40feb/ -DCMAKE_INSTALL_PREFIX=../eigen3
make install
\end{lstlisting}

\subsection{Building qhull}
The \Textsfc{qhull} source is downloaded as a submodule of the \Textsfc{ParSim} source code.
Building \Textsfc{qhull} is done in-source:
\begin{lstlisting}[backgroundcolor=\color{background}]
cd ParSim/GranularSim/src/qhull
cd build
cmake .. -DCMAKE_INSTALL_PREFIX=.
make
make install
\end{lstlisting}

\subsection{Building googletests and zenxml}
Googletests are built automatically while building \Textsfc{GranularSim} while \Textsfc{zenxml}
is a header-only file that requires C++14 capabilities in the compiler.  Both are downloaded when the
\Textsfc{ParSim} repository is cloned.

\chapter{Configuring and Building GranularSim}

\section{Basic instructions}
Once you are sure that all the prerequisites have been installed, you can build \GranularSim.

\subsection{ The GranularSim repository}
After you get the code from GitHub, follow these steps:
\begin{itemize}
  \item Go to the \Textsfc{GranularSim} directory:

\begin{lstlisting}[backgroundcolor=\color{background}]
     cd ParSim/GranularSim
\end{lstlisting}

  \item  The source code is in the directory \Textsfc{src}.
\end{itemize}

\subsection{ Out-of-source optimized build}
\begin{itemize}
  \item For the optimized build, create a new directory called \Textsfc{opt} under \Textsfc{GranularSim}:

\begin{lstlisting}[backgroundcolor=\color{background}]
    mkdir opt
\end{lstlisting}

  \item followed by:

\begin{lstlisting}[backgroundcolor=\color{background}]
    cd opt
\end{lstlisting}

  \item To create the make files do:

\begin{lstlisting}[backgroundcolor=\color{background}]
    cmake ../src
\end{lstlisting}
\end{itemize}

\subsection{ Out-of-source debug build}
\begin{itemize}
 \item For the debug build, create a directory \Textsfc{dbg} under \Textsfc{GranularSim}:

\begin{lstlisting}[backgroundcolor=\color{background}]
    mkdir dbg
    cd dbg
\end{lstlisting}

 \item To create the makefiles for the debug build, use

\begin{lstlisting}[backgroundcolor=\color{background}]
    cmake -DCMAKE_BUILD_TYPE=Debug ../src
\end{lstlisting}

\end{itemize}

\subsection{ Clang compiler:}
If you want to used the \Textsfc{clang} compiler instead of \Textsfc{gcc}:

\begin{lstlisting}[backgroundcolor=\color{background}]
    cmake ../src -DUSE_CLANG=1
\end{lstlisting}

\section{Compiling the code:}
Next you need to compile the files from \Textsfc{src}. So go to the \Textsfc{opt} directory and then type : 
\begin{lstlisting}[backgroundcolor=\color{background}]
make -j4
\end{lstlisting}

After this compilation you have all the executable files in your \Textsfc{opt} directory.
The same process can  be used for the debug build.

\section{Building GranularSim with compiled libraries}
When you have built your own libraries the build procedure for \Textsfc{GranularSim} is slightly different. 
For example, for the optimized build, do the following:

\begin{lstlisting}[backgroundcolor=\color{background}]
module load gcc
module load zlib
module load openmpi
git clone --recursive https://github.com/bbanerjee/ParSim.git
cd ParSim/GranularSim/
mkdir opt
cd opt
CC=gcc CXX=g++ \
cmake -DCMAKE_BUILD_TYPE=Release \
      -DBOOST_ROOT=/projects/user_name/boost \
      -DVTK_DIR=/projects/user_name/vtk/lib/cmake/vtk-7.1/ -DWITH_VTK7=1 \
      -DEIGEN3_INCLUDE_DIR=/projects/user_name/eigen3/include/eigen3 \
      ../src
make -j4
\end{lstlisting}

\chapter{Installing VisIt}
Visualization of \GranularSim data is currently possible using
\Visit. The \Visit package from LLNL is general purpose
visualization software that offers all of the usual capabilities for
rendering scientific data.  It is still developed and maintained by
LLNL staff, and its interface to \GranularSim data is supported by
the \VTK team. 

\section{Directory structure}
The following directory structure is suggested for building VisIt:

\begin{lstlisting}[backgroundcolor=\color{background}]
/myPath/VisIt
/myPath/VisIt/thirdparty
/myPath/VisIt/2.10.2
\end{lstlisting}

\section{Using the VisIt install script}
Download the build script from \\
\url{http://portal.nersc.gov/project/visit/releases/2.10.2/visit-install2_10_2}.
Follow the instructions at \url{http://portal.nersc.gov/project/visit/releases/2.10.2/INSTALL_NOTES}.
\GranularSim writes its output in the \VTK file format.  If \Visit does not recognize the
\VTK file format, you may have to follow the detailed instructions below.

\section{Using the VisIt build script}
\begin{enumerate}
  \item Get the \Textsfc{build\_visit} script from \url{http://portal.nersc.gov/project/visit/releases/2.10.2/build_visit2_10_2}.

  \item Set the MPI compiler to be used (must be the same version used to build VTK):
\begin{lstlisting}[backgroundcolor=\color{background}]
export PAR_COMPILER=`which mpic++` 
export PAR_COMPILER_CXX=`which mpicxx` 
export PAR_INCLUDE ="-I/usr/lib/openmpi/include/"
\end{lstlisting}

  \item Use the \Textsfc{build\_visit} script to download, build, and install the thirdparty 
        libraries required by VisIt. 

  \item Go the the \Visit thirdparty directory:
\begin{lstlisting}[backgroundcolor=\color{background}]
cd /mypath/VisIt/thirdparty
\end{lstlisting}

  \item Run \Textsfc{build\_visit}:
\begin{lstlisting}[backgroundcolor=\color{background}]
/myPath/VisIt/2.10.2/build_visit --console --thirdparty-path /myPath/VisIt/thirdparty --no-visit --parallel --fortran --uintah
\end{lstlisting}
       \begin{NoteBox}
       To add additional libraries to enable other database readers using the \Textsfc{build\_visit} script 
       you will have to add the appropriate flags when you run \Textsfc{build\_visit}.
       \end{NoteBox}

  \item The \Textsfc{build\_visit} script will take a while to finish.  After the script finishes it 
        will produce a machine specific \Textsfc{cmake} file called 
        \Textsfc{\textless your\_machine\_name \textgreater.cmake}. This file will be used when 
        compiling visit. 

  \item Move the \Textsfc{\textless your\_machine\_name \textgreater.cmake} file
        to the \Visit source \Textsfc{config-site} directory:
\begin{lstlisting}[backgroundcolor=\color{background}]
mv <your_machine_name>.cmake /myPath/VisIt/2.10.2/src/config-site
\end{lstlisting}

  \item Go to the \Visit source directory:
\begin{lstlisting}[backgroundcolor=\color{background}]
cd /myPath/VisIt/2.10.2/src
\end{lstlisting}

  \item Run \Textsfc{cmake} using the \Textsfc{cmake} that was built by the \Textsfc{build\_visit} script
\begin{lstlisting}[backgroundcolor=\color{background}]
/myPath/VisIt/thirdparty/cmake/<version>/<OS-gcc_version>bin/cmake .
\end{lstlisting}

  \item Finally make \Visit:
\begin{lstlisting}[backgroundcolor=\color{background}]
make -j 8
\end{lstlisting}
        One can install VisIt or run it directly from the bin directory.
\end{enumerate}


\subsection{Remote visualization}
In order to read data on a remote machine, you will need to have a
version of \Visit and the \VTK data format reader plugin installed on that
machine. You will also need to make sure that the \Visit executable is
in your path. You may need to edit your \Textsfc{.\textless shell\textgreater rc} 
file so that the path to \Visit is included in your shell's \Textsfc{PATH} variable.

\begin{NoteBox}
The version of VisIt installed on your local desktop and the
remote machine should be the same.
\end{NoteBox}

\subsection{Host Profile}
Next you will want to set up a Host Profile for your remote
machine. Select \Textsfc{Host Profiles} from the \Textsfc{Options} menu and set up a
new profile as shown in figure~\ref{VisItHostProfile}.
\begin{figure}
  \centering
  \includegraphics[width=.425\textwidth]{VisItHostProfile.png}
  \caption{Setting up Host Profile}
  \label{VisItHostProfile}
\end{figure}

After filling in the remote machine information, 
check the \Textsfc{Tunnel data connections
through SSH} option. 
Click on \Textsfc{Apply} and then do a \Textsfc{Save
Settings} in the \Textsfc{Options} menu.

For the remote visualization option to work, you must have ports 5600
- 5609 open. You can try this by running the following on your local
desktop (on Linux distributions),
\begin{lstlisting}
traceroute -p 560[0-9] <remote machine>
\end{lstlisting}

If the tunneling option does not work, try the \Textsfc{Use local machine name} option.

Now you should be able select \Textsfc{Open} from the \Visit \Textsfc{File} menu. After
selecting your host from the host entry list, you will be prompted for
a password on the remote machine (unless you have set up passwordless
ssh access).

Once the ssh login has completed, you should see the directory
listing. You can then change directories to your data directory and load the
data.

%
%----------------------------------------------------------------------------------------
%       BIBLIOGRAPHY
%----------------------------------------------------------------------------------------

\chapter*{Bibliography}
\addcontentsline{toc}{chapter}{\textcolor{ocre}{Bibliography}}

%\bibliographystyle{plain}
%\bibliography{Bibliography}
\printbibliography[heading=bibempty]




\end{document}
