\documentclass[a4paper]{beamer}
\graphicspath{{FIGS/}}

\title[Fracture Effects]
{Fracture Effects Update}

\author{Biswajit Banerjee}

\date[15 April, 2013]{15 April, 2013}

\subject{Fracture Effects}

\mode<presentation>
{
  \usetheme{CambridgeUS}
  \usefonttheme{professionalfonts}
}

\usepackage[english]{babel}
%\usepackage[latin1]{inputenc}
\usepackage{palatino}
\usepackage[T1]{fontenc}
\usepackage[urw-garamond]{mathdesign}
\usepackage{colortbl}
\usepackage{media9}
\usepackage{listings}
\usepackage{adjustbox}

\RequirePackage{tikz}

\defbeamertemplate*{title page}{customized}[1][]
{
  \vfill
  \hbox{
    \vbox {
      \begin{beamercolorbox}[wd=0.8\paperwidth,sep=8pt,left,#1]{title}
        \vskip5em
        %\hskip-1em
        \usebeamerfont{title}\usebeamercolor[fg]{title}\inserttitle
        \vskip2em
        %\hskip-1em
        \usebeamerfont{subtitle}\usebeamercolor[fg]{subtitle}\insertsubtitle\par
        %\hskip-1em
        \usebeamerfont{author}\insertauthor\par
        %\hskip-1em
        \usebeamerfont{date}\insertdate\par
      \end{beamercolorbox}
    }
  }
  \begin{tikzpicture}[remember picture, overlay]
    \node [xshift=0,yshift=0] at (0.21\paperwidth,0.7\paperheight) {
      \includegraphics[height=0.7cm]{FIGS/CallaghanLogo.png}
    };
  \end{tikzpicture}
}

%\setbeamertemplate{background}{
%  \includegraphics[width=\paperwidth,height=\paperheight]%
%                  {../IRLTitlePage.png}
%}
\setbeamertemplate{background}{
  \includegraphics[width=\paperwidth,height=\paperheight]%
                  {FIGS/CallaghanBackground.png}
}

\setbeamertemplate{frametitle}
{\vskip12pt
  \leavevmode
  \hbox{%
  \begin{beamercolorbox}[wd=\paperwidth,ht=1.8ex,dp=1ex]{frametitle}%
    \raggedright\hspace*{2em}\usebeamerfont{frametitle}\usebeamercolor[fg]{frametitle}\insertframetitle
  \end{beamercolorbox}
  }%
}

\definecolor{callaghanbg}{rgb}{0.6,0.68627,0.22745}
\definecolor{callaghangrey}{rgb}{0.25098,0.25098,0.25490}
\definecolor{callaghangreen}{rgb}{0.1098,0.46667,0.14510}
\definecolor{callaghanlight}{rgb}{0.827,0.922,0.894}
\definecolor{callaghandark}{rgb}{0.553,0.796,0.729}
\definecolor{callaghanviolet}{rgb}{0.54118,0.16863,0.88627}
\setbeamercolor*{frametitle}{fg=callaghanbg}
\setbeamercolor*{title}{fg=callaghanbg}
\setbeamercolor*{subtitle}{fg=callaghangrey}
\setbeamercolor*{author}{fg=callaghangrey}
\setbeamercolor*{institute}{fg=callaghangrey}
\setbeamercolor*{date}{fg=callaghangrey}
\setbeamercolor*{footline}{fg=callaghangrey}
\setbeamercolor*{item}{fg=callaghanbg}

\setbeamerfont{frametitle}{size=\LARGE, series=\bfseries}
\setbeamerfont{title}{size=\huge, series=\bfseries}
\setbeamerfont{author}{size=\Large, series=\bfseries}
\setbeamerfont{date}{size=\Large, series=\normalfont}
\setbeamerfont{footline}{size=\small, series=\normalfont}

%\setbeamertemplate{footline}{}
\setbeamertemplate{headline}{}
\setbeamertemplate{navigation symbols}{} 

\setbeamersize{text margin left=1cm}
\setbeamersize{text margin right=1cm}


\newcolumntype{A}{%
  >{\scriptsize\columncolor{callaghanlight}[0.95\tabcolsep]}c}
\newcolumntype{B}{%
  >{\scriptsize\columncolor{callaghandark}[0.95\tabcolsep]}c}

\definecolor{red}{rgb}{1,0,0}
\definecolor{green}{rgb}{0,1,0}
\definecolor{blue}{rgb}{0,0,1}
\definecolor{Red}{rgb}{0.8666,0.03137,0.02352}
\definecolor{Blue}{rgb}{0.00784,0.67059,0.91764}
\definecolor{Darkgreen}{rgb}{0,0.68235,0}
\definecolor{Green}{rgb}{0,0.8,0}
\definecolor{Bl}{rgb}{0,0.2,0.91764}
\definecolor{Royalblue}{rgb}{0,0.2,0.91764}
\definecolor{Brickred}{rgb}{0.644541,0.164065,0.164065}
\definecolor{Brown}{rgb}{0.6,0.4,0.4}
\definecolor{Orange}{rgb}{1,0.647059,0}
\definecolor{Orangy}{rgb}{1,0.647059,0.7}
\definecolor{Indigo}{rgb}{0.746105,0,0.996109}
\definecolor{Violet}{rgb}{0.308598,0.183597,0.308598}
\definecolor{Lightgrey}{rgb}{0.762951,0.762951,0.762951}
\definecolor{Darkgrey}{rgb}{0.503548,0.503548,0.503548}
\definecolor{Pink}{rgb}{1,0.6,0.6}
\definecolor{MyLightMagenta}{cmyk}{0.1,0.8,0,0.1}
\definecolor{MyDarkBlue}{rgb}{0,0.08,0.45}
\definecolor{Cornflower}{rgb}{0.41961,0.49804,0.90196}
\definecolor{TomThumb}{rgb}{0.16863,0.29020,0.18039}
\definecolor{Tan}{rgb}{0.549020,0.298039,0.090196}
\newcommand{\Red}{\color{Brickred}}
\newcommand{\Blue}{\color{Royalblue}}
\newcommand{\DarkBlue}{\color{MyDarkBlue}}
\newcommand{\Green}{\color{TomThumb}}
\newcommand{\Violet}{\color{callaghanviolet}}
\newcommand{\Orange}{\color{Orangy}}
\newcommand{\LtBlue}{\color{Cornflower}}
\newcommand{\Tan}{\color{Tan}}
\newcommand{\Grey}{\color{callaghangrey}}
\newcommand{\opencircle}{\mbox{\Large$\circ\,$}}  % moved Large outside maths
\newcommand{\opensquare}{\mbox{$\square$}}
\newcommand{\opendiamond}{\mbox{$\lozenge$}}
\newcommand{\opentriangledown}{\mbox{$\triangledown$}}
\newcommand{\openpentagram}{\mbox{$\star$}}
\newcommand{\opentriangleup}{\mbox{$\vartriangle$}}
\newcommand{\openhexagram}{\mbox{$\ast$}}
\newcommand{\opencross}{\mbox{$\times$}}
\newcommand{\opentriangleleft}{\mbox{$\triangleleft$}}
\newcommand{\openplus}{\mbox{$+$}}
\newcommand{\opentriangleright}{\mbox{$\triangleright$}}

\newcommand{\AvE}{\ensuremath{\langle E \rangle}}
\newcommand{\Barc}{\ensuremath{\bar{c}_0}}
\newcommand{\BStilde}{\ensuremath{\widetilde{\boldsymbol{S}}}}
\newcommand{\CalD}{\ensuremath{\mathcal{D}}}
\newcommand{\CalJ}{\ensuremath{\mathcal{J}}}
\newcommand{\Ep}{\ensuremath{\varepsilon_{p}}} 
\newcommand{\TExp}[1]{\ensuremath{\times\text{10}^{#1}}}
\newcommand{\Qndot}{\ensuremath{\dot{q}_n}} 
\newcommand{\That}{\ensuremath{\widehat{T}}}
\newcommand{\Tcore}{\ensuremath{\text{core}}}
\newcommand{\Ttt}{\ensuremath{\text{tt}}}
\newcommand{\Ttb}{\ensuremath{\text{tb}}}
\newcommand{\Teff}{\ensuremath{\text{eff}}}

\newcommand{\Int}{\ensuremath{\int_{\Omega}}}
\newcommand{\Sigi}{\ensuremath{\sigma_i}}
\newcommand{\Siga}{\ensuremath{\sigma_a}}
\newcommand{\Sige}{\ensuremath{\sigma_e}}
\newcommand{\Siges}{\ensuremath{\sigma_{es0}}}
\newcommand{\Se}{\ensuremath{\sigma_{es}}}
\newcommand{\Shp}{\ensuremath{\boldsymbol{\sigma^s}}}
\newcommand{\Ssigma}{\ensuremath{\cfrac{d\sigma^s}{dx}}}
\newcommand{\We}{\ensuremath{\mathbf{w}}}
\newcommand{\Cf}{\ensuremath{\cfrac{d}{dx}}}
\newcommand{\Sum}{\ensuremath{\sum_{i=1}^3}}
\newcommand{\SPum}{\ensuremath{\sum_{p=1}^2}}
\newcommand{\SP}{\ensuremath{\sum_p}}
\newcommand{\SJum}{\ensuremath{\sum_{j=1}^3}}
\newcommand{\Pdelta}{\ensuremath{\Int \SPum M_p \delta \left[x-X_p\right]}}
\newcommand{\Lb}{\ensuremath{\lbrace}}
\newcommand{\Rb}{\ensuremath{\rbrace}}
\newcommand{\BVe}{\ensuremath{\boldsymbol{\Ve}}}
\newcommand{\Bfk}[3]{\ensuremath{\mathbf{{#1}^{#2}_{#3}}}}
\newcommand{\DVe}{\ensuremath{\dot \varepsilon}}
\newcommand{\Ex}{\ensuremath{\mathbf{e}_1}}
\newcommand{\Ey}{\ensuremath{\mathbf{e}_2}}
\newcommand{\Ez}{\ensuremath{\mathbf{e}_3}}
\newcommand{\Exp}{\ensuremath{\mathbf{e}^{'}_1}}
\newcommand{\Eyp}{\ensuremath{\mathbf{e}^{'}_2}}
\newcommand{\Ezp}{\ensuremath{\mathbf{e}^{'}_3}}
\newcommand{\Dotepi}{\ensuremath{\dot{\epsilon}_{0i}}}
\newcommand{\Muo}{\ensuremath{\mu_0}}
\newcommand{\Dt}{\ensuremath{\dfrac{\partial}{\partial t}}}
\newcommand{\SThr}{\ensuremath{\sqrt{3}}}
\newcommand{\GraduT}{\ensuremath{\left(\Gradu\right)^T}}
\newcommand{\PartMuP}{\ensuremath{\partial{\mu}/\partial{p}}}
\newcommand{\APartialA}[4]{\ensuremath{{\cfrac{\partial #1}{\partial #2}}+\cfrac{\partial #3}{\partial #4}}}	 

\newenvironment{compress}{\baselineskip=0.5\baselineskip}{\par}

\newcommand{\Intx}{\ensuremath{\int_{x_a}^{x_b}}}
\newcommand{\IntX}{\ensuremath{\int_{X_a}^{X_b}}}
\newcommand{\Intiso}{\ensuremath{\int_{-1}^{1}}}
\newcommand{\Norm}[1]{\ensuremath{\lVert#1\rVert}}
\newcommand{\NormG}[2]{\ensuremath{\lVert#1\rVert_{#2}}}
\newcommand{\Var}[1]{\ensuremath{\delta #1}}
\newcommand{\DelT}{\ensuremath{\Delta t}}
\newcommand{\Dela}{\ensuremath{\Delta a}}
\newcommand{\Delb}{\ensuremath{\Delta b}}
\newcommand{\Delt}{\ensuremath{\Delta t}}
\newcommand{\CalF}{\ensuremath{\mathcal{F}}}
\newcommand{\CalH}{\ensuremath{\mathcal{H}}}
\newcommand{\CalL}{\ensuremath{\mathcal{L}}}
\newcommand{\CalN}{\ensuremath{\mathcal{N}}}
\newcommand{\CalP}{\ensuremath{\mathcal{P}}}
\newcommand{\CalS}{\ensuremath{\mathcal{S}}}
\newcommand{\BCalS}{\ensuremath{\boldsymbol{\CalS}}}
\newcommand{\CalV}{\ensuremath{\mathcal{V}}}
\newcommand{\CalW}{\ensuremath{\mathcal{W}}}
\newcommand{\CalX}{\ensuremath{\mathcal{X}}}
\newcommand{\Comp}[2]{\ensuremath{#1 \circ #2}}
\newcommand{\Map}[3]{\ensuremath{#1 : #2 \rightarrow #3}}
\newcommand{\MapTo}[3]{\ensuremath{#1 : #2 \mapsto #3}}
\newcommand{\Real}[1]{\ensuremath{\mathbb{R}^{#1}}}
\providecommand{\abs}[1]{\lvert#1\rvert}
\providecommand{\norm}[1]{\lVert#1\rVert}
\newcommand{\Jaumann}[1]{\ensuremath{\overset{\triangle}{#1}}}
\newcommand{\Truesdell}[1]{\ensuremath{\overset{\circ}{#1}}}
\newcommand{\erf}{\text{erf}}
\newcommand{\Epi}{\ensuremath{\epsilon_{pi}}}
%\newcommand{\Epdot}[1]{\ensuremath{\dot{\epsilon}_{p#1}}}
\newcommand{\Epdot}[1]{\ensuremath{\dot{\epsilon}_{#1}}}
\newcommand{\Xidot}{\ensuremath{\dot{\xi}}}
\newcommand{\Sa}{\ensuremath{\sigma_a}}
\newcommand{\Shi}{\ensuremath{\hat\sigma_i}}
\newcommand{\She}{\ensuremath{\hat\sigma_e}}
\newcommand{\Gi}{\ensuremath{g_{0i}}}
\newcommand{\Ge}{\ensuremath{g_{0e}}}
\newcommand{\Ei}{\ensuremath{\dot\epsilon_{0i}}}
\newcommand{\Ee}{\ensuremath{\dot\epsilon_{0e}}}
\newcommand{\En}{\ensuremath{\epsilon_{n}}}
\newcommand{\Pci}{\ensuremath{p_{i}}}
\newcommand{\Pce}{\ensuremath{p_{e}}}
\newcommand{\Qci}{\ensuremath{q_{i}}}
\newcommand{\Qce}{\ensuremath{q_{e}}}
\newcommand{\To}{\ensuremath{\theta_{0}}}
\newcommand{\TIV}{\ensuremath{\theta_{IV}}}
\newcommand{\Ses}{\ensuremath{\hat\sigma_{es0}}}
\newcommand{\Ges}{\ensuremath{g_{0es}}}
\newcommand{\Ees}{\ensuremath{\dot\epsilon_{es0}}}
\newcommand{\Ve}{\ensuremath{\varepsilon}}
\newcommand{\BHat}[1]{\ensuremath{\hat{\boldsymbol{#1}}}}
\newcommand{\BTx}{\ensuremath{\tilde{\boldsymbol{x}}}}
\newcommand{\Beh}{\ensuremath{\hat{\boldsymbol{e}}}}
\newcommand{\BHex}{\ensuremath{\hat{\boldsymbol{e}}_1}}
\newcommand{\BHey}{\ensuremath{\hat{\boldsymbol{e}}_2}}
\newcommand{\BHez}{\ensuremath{\hat{\boldsymbol{e}}_3}}
\newcommand{\BHn}[1]{\ensuremath{\hat{\boldsymbol{n}}_{#1}}}
\newcommand{\BHe}[1]{\ensuremath{\hat{\boldsymbol{e}}_{#1}}}
\newcommand{\BHg}[1]{\ensuremath{\hat{\boldsymbol{g}}_{#1}}}
\newcommand{\BHG}[1]{\ensuremath{\hat{\boldsymbol{G}}_{#1}}}
\newcommand{\Bnhat}{\ensuremath{\hat{\boldsymbol{n}}}}
\newcommand{\Hn}{\ensuremath{\hat{\boldsymbol{n}}}}
\newcommand{\Mba}{\ensuremath{\mathbf{a}}}
\newcommand{\Mbb}{\ensuremath{\mathbf{b}}}
\newcommand{\Mbd}{\ensuremath{\mathbf{d}}}
\newcommand{\Mbf}{\ensuremath{\mathbf{f}}}
\newcommand{\Mbr}{\ensuremath{\mathbf{r}}}
\newcommand{\Mbu}{\ensuremath{\mathbf{u}}}
\newcommand{\Mbv}{\ensuremath{\mathbf{v}}}
\newcommand{\Mbx}{\ensuremath{\mathbf{x}}}
\newcommand{\MbA}{\ensuremath{\mathbf{A}}}
\newcommand{\MbB}{\ensuremath{\mathbf{B}}}
\newcommand{\MbC}{\ensuremath{\mathbf{C}}}
\newcommand{\MbD}{\ensuremath{\mathbf{D}}}
\newcommand{\MbI}{\ensuremath{\mathbf{I}}}
\newcommand{\MbK}{\ensuremath{\mathbf{K}}}
\newcommand{\MbM}{\ensuremath{\mathbf{M}}}
\newcommand{\MbN}{\ensuremath{\mathbf{N}}}
\newcommand{\MbT}{\ensuremath{\mathbf{T}}}
\newcommand{\MbX}{\ensuremath{\mathbf{X}}}
\newcommand{\Mbzero}{\ensuremath{\mathbf{0}}}
\newcommand{\MbSig}{\ensuremath{\boldsymbol{\sigma}}}
\newcommand{\Mb}{\ensuremath{\left[\mathsf{b}\right]}}
\newcommand{\Mu}{\ensuremath{\left[\mathsf{u}\right]}}
\newcommand{\Mv}{\ensuremath{\left[\mathsf{v}\right]}}
\newcommand{\Mw}{\ensuremath{\left[\mathsf{w}\right]}}
\newcommand{\Mx}{\ensuremath{\left[\mathsf{x}\right]}}
\newcommand{\MA}{\ensuremath{\left[\mathsf{A}\right]}}
\newcommand{\MC}{\ensuremath{\left[\mathsf{C}\right]}}
\newcommand{\MD}{\ensuremath{\left[\mathsf{D}\right]}}
\newcommand{\ML}{\ensuremath{\left[\mathsf{L}\right]}}
\newcommand{\MN}{\ensuremath{\left[\mathsf{N}\right]}}
\newcommand{\MP}{\ensuremath{\left[\mathsf{P}\right]}}
\newcommand{\MT}{\ensuremath{\left[\mathsf{T}\right]}}
\newcommand{\SfA}{\ensuremath{\boldsymbol{\mathsf{A}}}}
\newcommand{\SfC}{\boldsymbol{\mathsf{C}}}
\newcommand{\SfD}{\ensuremath{\boldsymbol{\mathsf{D}}}}
\newcommand{\SfI}{\ensuremath{\boldsymbol{\mathsf{I}}}}
\newcommand{\SfL}{\ensuremath{\boldsymbol{\mathsf{L}}}}
\newcommand{\SfS}{\ensuremath{\boldsymbol{\mathsf{S}}}}
\newcommand{\SfT}{\ensuremath{\boldsymbol{\mathsf{T}}}}
\newcommand{\Msig}{\ensuremath{\left[\boldsymbol{\sigma}\right]}}
\newcommand{\Meps}{\ensuremath{\left[\boldsymbol{\varepsilon}\right]}}
\newcommand{\Epsxx}{\ensuremath{\varepsilon_{11}}}
\newcommand{\Epsyy}{\ensuremath{\varepsilon_{22}}}
\newcommand{\Epszz}{\ensuremath{\varepsilon_{33}}}
\newcommand{\Epsyz}{\ensuremath{\varepsilon_{23}}}
\newcommand{\Epszx}{\ensuremath{\varepsilon_{31}}}
\newcommand{\Epsxy}{\ensuremath{\varepsilon_{12}}}
\newcommand{\Sigxx}{\ensuremath{\sigma_{11}}}
\newcommand{\Sigyy}{\ensuremath{\sigma_{22}}}
\newcommand{\Sigzz}{\ensuremath{\sigma_{33}}}
\newcommand{\Sigyz}{\ensuremath{\sigma_{23}}}
\newcommand{\Sigzx}{\ensuremath{\sigma_{31}}}
\newcommand{\Sigxy}{\ensuremath{\sigma_{12}}}
\newcommand{\Eps}[1]{\ensuremath{\varepsilon_{#1}}}
\newcommand{\Sig}[1]{\ensuremath{\sigma_{#1}}}
\newcommand{\X}{\ensuremath{X_1}}
\newcommand{\Y}{\ensuremath{X_2}}
\newcommand{\Z}{\ensuremath{X_3}}
\newcommand{\Balpha}{\ensuremath{\boldsymbol{\alpha}}}
\newcommand{\Bbeta}{\ensuremath{\boldsymbol{\beta}}}
\newcommand{\Bchi}{\ensuremath{\boldsymbol{\chi}}}
\newcommand{\Bpsi}{\ensuremath{\boldsymbol{\psi}}}
\newcommand{\Beps}{\ensuremath{\boldsymbol{\varepsilon}}}
\newcommand{\BVeps}{\ensuremath{\boldsymbol{\varepsilon}}}
\newcommand{\Bbeps}{\ensuremath{\bar{\boldsymbol{\varepsilon}}}}
\newcommand{\Bkappa}{\ensuremath{\boldsymbol{\kappa}}}
\newcommand{\Bnabla}{\ensuremath{\boldsymbol{\nabla}}}
\newcommand{\Bomega}{\ensuremath{\boldsymbol{\omega}}}
\newcommand{\Bsig}{\ensuremath{\boldsymbol{\sigma}}}
\newcommand{\Btau}{\ensuremath{\boldsymbol{\tau}}}
\newcommand{\Bvarphi}{\ensuremath{\boldsymbol{\varphi}}}
\newcommand{\Blambda}{\ensuremath{\boldsymbol{\lambda}}}
\newcommand{\Btheta}{\ensuremath{\boldsymbol{\theta}}}
\newcommand{\Brho}{\ensuremath{\boldsymbol{\rho}}}
\newcommand{\Bmu}{\ensuremath{\boldsymbol{\mu}}}
\newcommand{\Bxi}{\ensuremath{\boldsymbol{\xi}}}
\newcommand{\Beta}{\ensuremath{\boldsymbol{\eta}}}
\newcommand{\Bone}{\ensuremath{\boldsymbol{\mathit{1}}}}
\newcommand{\Bzero}{\ensuremath{\boldsymbol{\mathit{0}}}}
\newcommand{\BoneHat}{\ensuremath{\widehat{\Bone}}}
\newcommand{\BoneV}{\ensuremath{\boldsymbol{1}}}
\newcommand{\BzeroV}{\ensuremath{\boldsymbol{0}}}
\newcommand{\Ba}{\ensuremath{\mathbf{a}}}
\newcommand{\Bb}{\ensuremath{\mathbf{b}}}
\newcommand{\BbT}{\ensuremath{\boldsymbol{b}}}
\newcommand{\Bc}{\ensuremath{\mathbf{c}}}
\newcommand{\Bd}{\ensuremath{\boldsymbol{d}}}
\newcommand{\Be}{\ensuremath{\mathbf{e}}}
\newcommand{\BeT}{\ensuremath{\bolsymbol{e}}}
\newcommand{\Bf}{\ensuremath{\mathbf{f}}}
\newcommand{\BfT}{\ensuremath{\boldsymbol{f}}}
\newcommand{\Bg}{\ensuremath{\boldsymbol{g}}}
\newcommand{\Bj}{\ensuremath{\boldsymbol{j}}}
\newcommand{\Bl}{\ensuremath{\boldsymbol{l}}}
\newcommand{\Bm}{\ensuremath{\boldsymbol{m}}}
\newcommand{\Bnn}{\ensuremath{\boldsymbol{n}}}
\newcommand{\Bn}{\ensuremath{\mathbf{n}}}
\newcommand{\Bo}{\ensuremath{\mathbf{o}}}
\newcommand{\BoT}{\ensuremath{\boldsymbol{o}}}
\newcommand{\Bp}{\ensuremath{\boldsymbol{p}}}
\newcommand{\Bq}{\ensuremath{\mathbf{q}}}
\newcommand{\Br}{\ensuremath{\boldsymbol{r}}}
\newcommand{\Bsv}{\ensuremath{\mathbf{s}}}
\newcommand{\Bs}{\ensuremath{\boldsymbol{s}}}
\newcommand{\Bt}{\ensuremath{\mathbf{t}}}
\newcommand{\Bu}{\ensuremath{\mathbf{u}}}
\newcommand{\Bv}{\ensuremath{\mathbf{v}}}
\newcommand{\Bw}{\ensuremath{\mathbf{w}}}
\newcommand{\BwT}{\ensuremath{\boldsymbol{w}}}
\newcommand{\Bx}{\ensuremath{\mathbf{x}}}
\newcommand{\By}{\ensuremath{\mathbf{y}}}
\newcommand{\BA}{\ensuremath{\boldsymbol{A}}}
\newcommand{\BB}{\ensuremath{\boldsymbol{B}}}
\newcommand{\BC}{\ensuremath{\boldsymbol{C}}}
\newcommand{\BD}{\ensuremath{\boldsymbol{D}}}
\newcommand{\BE}{\ensuremath{\boldsymbol{E}}}
\newcommand{\BEv}{\ensuremath{\mathbf{E}}}
\newcommand{\BF}{\ensuremath{\boldsymbol{F}}}
\newcommand{\BG}{\ensuremath{\boldsymbol{G}}}
\newcommand{\BH}{\ensuremath{\boldsymbol{H}}}
\newcommand{\BI}{\ensuremath{\boldsymbol{I}}}
\newcommand{\BJ}{\ensuremath{\boldsymbol{J}}}
\newcommand{\BK}{\ensuremath{\boldsymbol{K}}}
\newcommand{\BL}{\ensuremath{\boldsymbol{L}}}
\newcommand{\BM}{\ensuremath{\boldsymbol{M}}}
\newcommand{\BN}{\ensuremath{\boldsymbol{N}}}
\newcommand{\BNv}{\ensuremath{\mathbf{N}}}
\newcommand{\BP}{\ensuremath{\boldsymbol{P}}}
\newcommand{\BPv}{\ensuremath{\mathbf{P}}}
\newcommand{\BQ}{\ensuremath{\boldsymbol{Q}}}
\newcommand{\BR}{\ensuremath{\boldsymbol{R}}}
\newcommand{\BS}{\ensuremath{\boldsymbol{S}}}
\newcommand{\BT}{\ensuremath{\boldsymbol{T}}}
\newcommand{\BU}{\ensuremath{\boldsymbol{U}}}
\newcommand{\BV}{\ensuremath{\boldsymbol{V}}}
\newcommand{\BTv}{\ensuremath{\mathbf{T}}}
\newcommand{\BW}{\ensuremath{\boldsymbol{W}}}
\newcommand{\BX}{\ensuremath{\mathbf{X}}}
\newcommand{\BXT}{\ensuremath{\boldsymbol{X}}}
\newcommand{\BY}{\ensuremath{\boldsymbol{Y}}}
\newcommand{\Tor}{\ensuremath{\text{or}}}
\newcommand{\Tr}[1]{\ensuremath{\text{tr}\left(#1\right)}}
\newcommand{\Dev}{\ensuremath{\text{~dev}}}
\newcommand{\Tand}{\ensuremath{\text{and}}}
\newcommand{\Trial}{\ensuremath{\text{trial}}}
\newcommand{\Tint}{\ensuremath{\text{int}}}
\newcommand{\Text}{\ensuremath{\text{ext}}}
\newcommand{\Tkin}{\ensuremath{\text{kin}}}
\newcommand{\Tbody}{\ensuremath{\text{body}}}
\newcommand{\Tmin}{\ensuremath{\text{min}}}
\newcommand{\Tmax}{\ensuremath{\text{max}}}
\newcommand{\Tbod}{\ensuremath{\text{bod}}}
\newcommand{\Tdev}{\ensuremath{\text{dev}}}
\newcommand{\Ttop}{\ensuremath{\text{top}}}
\newcommand{\Tbot}{\ensuremath{\text{bot}}}
\newcommand{\Tcen}{\ensuremath{\text{cen}}}
\newcommand{\Trot}{\ensuremath{\text{rot}}}
\newcommand{\Tcorr}{\ensuremath{\text{corr}}}
\newcommand{\Tvol}{\ensuremath{\text{vol}}}
\newcommand{\Tg}{\ensuremath{\text{g}}}
\newcommand{\Tp}{\ensuremath{\text{p}}}

\newcommand{\Half}{\ensuremath{\frac{1}{2}}}
\newcommand{\Sthr}{\ensuremath{\sqrt{3}}}
\newcommand{\STT}{\ensuremath{\frac{\sqrt{3}}{2}}}
\newcommand{\TT}{\ensuremath{\frac{3}{2}}}
\newcommand{\Third}{\ensuremath{\frac{1}{3}}}
\newcommand{\Sixth}{\ensuremath{\frac{1}{6}}}
\newcommand{\Ninth}{\ensuremath{\frac{1}{9}}}
\newcommand{\Inner}[2]{\ensuremath{\langle#1,~#2\rangle}}
\newcommand{\Bcross}[2]{\ensuremath{#1\boldsymbol{\times}#2}}
\newcommand{\Bdot}[2]{\ensuremath{#1\cdot#2}}
\newcommand{\Dyad}[2]{\ensuremath{#1\boldsymbol{\otimes}#2}}
\newcommand{\Grad}[1]{\ensuremath{\Bnabla #1}}
\newcommand{\Grads}[1]{\ensuremath{\Bnabla^s #1}}
\newcommand{\Lap}[1]{\ensuremath{\nabla^2 #1}}
\newcommand{\Biharm}[1]{\ensuremath{\nabla^4 #1}}
\newcommand{\Div}[1]{\ensuremath{\Bdot{\Bnabla}{#1}}}
\newcommand{\Curl}[1]{\ensuremath{\Bcross{\Bnabla}{#1}}}
\newcommand{\Gradu}{\ensuremath{\Grad{\Bu}}}
\newcommand{\Divu}{\ensuremath{\Div{\Bu}}}
\newcommand{\Curlu}{\ensuremath{\Curl{\Bu}}}
\newcommand{\Gradv}{\ensuremath{\Grad{\Bv}}}
\newcommand{\GradvT}{\ensuremath{(\Gradv)^T}}
\newcommand{\Divv}{\ensuremath{\Div{\Bv}}}
\newcommand{\Curlv}{\ensuremath{\Curl{\Bv}}}
\newcommand{\Dualn}{\ensuremath{\Bdual{\Bn}{\Bn}}}
\newcommand{\Over}[1]{\ensuremath{\frac{1}{#1}}}
\newcommand{\Jump}[1]{\ensuremath{\llbracket#1\rrbracket}}
\newcommand{\Blimitx}[1]{\ensuremath{\left[#1\right]_{x_a}^{x_b}}}
\newcommand{\LimDelt}{\ensuremath{\lim_{\Delt\rightarrow 0}}}
\newcommand{\Diff}[2]{\ensuremath{\frac{d #1}{d #2}}}
\newcommand{\Deriv}[2]{\ensuremath{\frac{d #1}{d #2}}}
\newcommand{\MDeriv}[2]{\ensuremath{\frac{D #1}{D #2}}}
\newcommand{\DDeriv}[2]{\ensuremath{\cfrac{d^2#1}{d#2^2}}}
\newcommand{\Partial}[2]{\ensuremath{\frac{\partial #1}{\partial #2}}}
\newcommand{\PPartial}[2]{\ensuremath{\frac{\partial^2 #1}{\partial #2^2}}}
\newcommand{\PPartialA}[3]{\ensuremath{\frac{\partial^2 #1}{\partial #2\partial#3}}}
\newcommand{\FPartial}[2]{\ensuremath{\frac{\partial^4 #1}{\partial #2^4}}}
\newcommand{\FPartialA}[3]{\ensuremath{\frac{\partial^4 #1}{\partial #2^2
         \partial #3^2}}}
\newcommand{\Bonesnp}{\ensuremath{\Bone - \Dyad{\Bn^n_p}{\Bn^n_p}}}
\newcommand{\Bonesp}{\ensuremath{\Bone - \Dyad{\Bn_p}{\Bn_p}}}
\newcommand{\Bones}{\ensuremath{\Bone - \Dyad{\Bn}{\Bn}}}
\newcommand{\IntOmega}{\ensuremath{\int_{\Omega}}}
\newcommand{\IntDomega}{\ensuremath{\int_{\Domega}}}
\newcommand{\IntOmegac}{\ensuremath{\int_{\Omega_c}}}
\newcommand{\IntOmegapc}{\ensuremath{\int_{\Omega_p\cap\Omega_c}}}
\newcommand{\IntOmegap}{\ensuremath{\int_{\Omega_p\cap\Omega}}}
\newcommand{\IntOmegaq}{\ensuremath{\int_{\Omega_q\cap\Omega}}}
\newcommand{\IntOmegaA}{\ensuremath{\int_{\Omega_0}}}
\newcommand{\IntGammat}{\ensuremath{\int_{\Gamma_t}}}
\newcommand{\IntGammau}{\ensuremath{\int_{\Gamma_u}}}
\newcommand{\IntGammaq}{\ensuremath{\int_{\Gamma_q}}}
\newcommand{\IntGammaT}{\ensuremath{\int_{\Gamma_T}}}
\newcommand{\IntGamma}{\ensuremath{\int_{\Gamma}}}
\newcommand{\IntOmegat}{\ensuremath{\int_{\Omega(t)}}}
\newcommand{\IntDomegat}{\ensuremath{\int_{\Domega(t)}}}
\newcommand{\IntOmegatDelt}{\ensuremath{\int_{\Omega(t+\Delt)}}}
\newcommand{\IntOmegar}{\ensuremath{\int_{\Omega_0}}}
\newcommand{\IntDomegar}{\ensuremath{\int_{\Domega_0}}}
\newcommand{\Bart}{\ensuremath{\bar{\mathbf{t}}}}
\newcommand{\BWs}{\ensuremath{\boldsymbol{W}^{*}}}
\newcommand{\BWss}{\ensuremath{\boldsymbol{W}^{**}}}
\newcommand{\BWssp}{\ensuremath{\boldsymbol{W}_p^{**}}}
\newcommand{\BWssq}{\ensuremath{\boldsymbol{W}_q^{**}}}
\newcommand{\BWsDev}{\ensuremath{\text{dev}(\BWs)}}
\newcommand{\BWsVol}{\ensuremath{\Third\text{tr}(\BWs)~\Bone}}
\newcommand{\What}{\ensuremath{\widehat{w}}}
\newcommand{\ThetaDot}{\ensuremath{\dot{\theta}}}
\newcommand{\Sump}{\ensuremath{\sum_{p=1}^{n_p}}}
\newcommand{\Sumpq}{\ensuremath{\sum_{q=1}^{n_p}}}
\newcommand{\Sumpc}{\ensuremath{\sum_{p=1}^{n_p^c}}}
\newcommand{\Sumpcq}{\ensuremath{\sum_{q=1}^{n_p^c}}}
\newcommand{\Sumg}{\ensuremath{\sum_{g=1}^{n_g}}}
\newcommand{\Sumgc}{\ensuremath{\sum_{g=1}^{n_g^c}}}
\newcommand{\Sumgh}{\ensuremath{\sum_{h=1}^{n_g}}}
\newcommand{\Bvdot}{\ensuremath{\dot{\mathbf{v}}}}
\newcommand{\Tdot}{\ensuremath{\dot{T}}}
\newcommand{\LBr}{\ensuremath{\left(}}
\newcommand{\LBc}{\ensuremath{\left\{}}
\newcommand{\LBs}{\ensuremath{\left[}}
\newcommand{\LBn}{\ensuremath{\left.}}
\newcommand{\RBr}{\ensuremath{\right)}}
\newcommand{\RBc}{\ensuremath{\right\}}}
\newcommand{\RBs}{\ensuremath{\right]}}
\newcommand{\RBn}{\ensuremath{\right.}}
\newcommand{\BsHat}{\ensuremath{\widehat{\boldsymbol{s}}}}
\newcommand{\BnHat}{\ensuremath{\widehat{\boldsymbol{n}}}}
\newcommand{\Ss}{\ensuremath{\sigma_s}}
\newcommand{\Sm}{\ensuremath{\sigma_m}}
\newcommand{\Sy}{\ensuremath{\sigma_y}}
\newcommand{\SqrtTT}{\ensuremath{\sqrt{\cfrac{3}{2}}}}
\newcommand{\Ssdot}{\ensuremath{\dot{\sigma}_s}}
\newcommand{\Smdot}{\ensuremath{\dot{\sigma}_m}}
\newcommand{\Lambdadot}{\ensuremath{\dot{\lambda}}}
\newcommand{\OSthr}{\ensuremath{\cfrac{1}{\Sthr}}}
\newcommand{\DotMbT}{\ensuremath{\dot{\MbT}}}
\newcommand{\TildeMbT}{\ensuremath{\widetilde{\MbT}}}
\newcommand{\BarT}{\ensuremath{\overline{T}}}
\newcommand{\Barq}{\ensuremath{\overline{q}}}
\newcommand{\Domega}{\ensuremath{\partial{\Omega}}}
\newcommand{\Av}[1]{\ensuremath{\langle#1\rangle}}
\newcommand{\AvSig}{\ensuremath{\langle\Bsig\rangle}}
\newcommand{\AvTau}{\ensuremath{\langle\Btau\rangle}}
\newcommand{\AvP}{\ensuremath{\langle\BP\rangle}}
\newcommand{\AvEps}{\ensuremath{\langle\Beps\rangle}}
\newcommand{\AvEpsdot}{\ensuremath{\langle\dot{\Beps}\rangle}}
\newcommand{\AvDisp}{\ensuremath{\langle\Bu\rangle}}
\newcommand{\AvF}{\ensuremath{\langle\BF\rangle}}
\newcommand{\AvFdot}{\ensuremath{\langle\dot{\BF}\rangle}}
\newcommand{\Avl}{\ensuremath{\overline{\Bl}}}
\newcommand{\AvSigBar}{\ensuremath{\overline{\Bsig}}}
\newcommand{\AvTauBar}{\ensuremath{\overline{\Btau}}}
\newcommand{\AvOmega}{\ensuremath{\langle\Bomega\rangle}}
\newcommand{\AvGradu}{\ensuremath{\langle\Gradu\rangle}}
\newcommand{\AvGradudot}{\ensuremath{\langle\Grad{\dot{\Bu}}\rangle}}
\newcommand{\AvGradv}{\ensuremath{\langle\Gradv\rangle}}
\newcommand{\AvPower}{\ensuremath{\langle\Bsig:\Gradv\rangle}}
\newcommand{\AvPowerInf}{\ensuremath{\langle\Bsig:\dot{\Beps}\rangle}}
\newcommand{\AvWorkInf}{\ensuremath{\langle\Bsig:\Beps\rangle}}
\newcommand{\AvPowerPF}{\ensuremath{\langle\BP^T:\dot{\BF}\rangle}}
\newcommand{\DA}{\ensuremath{\text{dA}}}
\newcommand{\DAvec}{\ensuremath{\text{d}\mathbf{A}}}
\newcommand{\Da}{\ensuremath{\text{da}}}
\newcommand{\Davec}{\ensuremath{\text{d}\mathbf{a}}}
\newcommand{\DV}{\ensuremath{\text{dV}}}
\newcommand{\BCe}{\ensuremath{\mathcal{E}}}
\newcommand{\GradX}[1]{\ensuremath{\Bnabla_0~#1}}
\newcommand{\DivX}[1]{\ensuremath{\Bdot{\Bnabla_0}{#1}}}
\newcommand{\Edot}{\ensuremath{\dot{e}}}
\newcommand{\Gdot}{\ensuremath{\dot{g}}}
\newcommand{\Jdot}{\ensuremath{\dot{J}}}
\newcommand{\Qdot}{\ensuremath{\dot{q}}}
\newcommand{\Etadot}{\ensuremath{\dot{\eta}}}
\newcommand{\Bxdot}{\ensuremath{\dot{\Bx}}}
\newcommand{\BEdot}{\ensuremath{\dot{\BE}}}
\newcommand{\BFdot}{\ensuremath{\dot{\BF}}}
\newcommand{\BQdot}{\ensuremath{\dot{\BQ}}}
\newcommand{\BSdot}{\ensuremath{\dot{\BS}}}
\newcommand{\BUdot}{\ensuremath{\dot{\BU}}}
\newcommand{\BVdot}{\ensuremath{\dot{\BV}}}
\newcommand{\Beq}{\begin{equation}}
\newcommand{\Eeq}{\end{equation}}
\newcommand{\Bal}{\begin{aligned}}
\newcommand{\Eal}{\end{aligned}}
\newcommand{\phihat}{\ensuremath{\widehat{\phi}}}
\newcommand{\phibar}{\ensuremath{\overline{\phi}}}
\newcommand{\lambdadot}{\ensuremath{\dot{\lambda}}}
\newcommand{\gammadot}{\ensuremath{\dot{\gamma}}}
\newcommand{\rhodot}{\ensuremath{\dot{\rho}}}
\newcommand{\dg}{\ensuremath{\text{d}g}}
\newcommand{\dE}{\ensuremath{\text{d}\BE}}
\newcommand{\dS}{\ensuremath{\text{d}\BS}}
\newcommand{\dT}{\ensuremath{\text{d}T}}
\newcommand{\dq}{\ensuremath{\text{d}q}}
\newcommand{\de}{\ensuremath{\text{d}e}}
\newcommand{\dpsi}{\ensuremath{\text{d}\psi}}
\newcommand{\deta}{\ensuremath{\text{d}\eta}}
\newcommand{\Bdhat}{\ensuremath{\widehat{\Bd}}}
\newcommand{\BDhat}{\ensuremath{\widehat{\BD}}}
\newcommand{\BFhat}{\ensuremath{\widehat{\BF}}}
\newcommand{\BFhatdot}{\ensuremath{\dot{\BFhat}}}
\newcommand{\BVhat}{\ensuremath{\widehat{\BV}}}
\newcommand{\qhat}{\ensuremath{\widehat{q}}}
\newcommand{\qtilde}{\ensuremath{\widetilde{q}}}
\newcommand{\Jhat}{\ensuremath{\widehat{J}}}




%=======================================================================
\begin{document}

  % =======================================================================
  % TITLE  FRONTPAGE 1
  % ----------------------------------------------------------------------
  \setbeamertemplate{footline}{\vspace{-20pt}\hspace{12pt}Biswajit.Banerjee@callaghaninnovation.govt.nz}
  \begin{frame}
    \titlepage
  \end{frame}

  \setbeamertemplate{background}{
    \includegraphics[width=\paperwidth,height=\paperheight,keepaspectratio]%
                    {FIGS/CallaghanBackground.png}
  }
  \setbeamertemplate{footline}{}

  % =======================================================================
  % What we are trying to do 
  % ----------------------------------------------------------------------
  \section{Fracture Effects}

    \begin{frame}
      \frametitle{Fracture simulation}
      \begin{center}
      \includegraphics[width=110mm]{./FIGS/DestructionVisualEffect.png} 
      \end{center}
    \end{frame}

    \begin{frame}
      \frametitle{MPM simulations with Uintah}
      \begin{columns}
        \begin{column}{0.3\textwidth}
          \begin{center}
            \includegraphics[width=30mm]{./FIGS/cylPeneJC_med_ep.jpg} 
          \end{center}
        \end{column}
        \begin{column}{0.25\textwidth}
          \begin{center}
            \includegraphics[width=30mm]{./FIGS/poolFire_300.jpg} 
          \end{center}
        \end{column}
        \begin{column}{0.45\textwidth}
          \begin{center}
            \rotatebox{-90}{\includegraphics[width=60mm]{./FIGS/BucketLiner_AreniscaBB_JWL.png}} \\
          \end{center}
        \end{column}
      \end{columns}
    \end{frame}

    \begin{frame}
      \frametitle{MPM simulations with Vaango}
      {\Grey Notice that large regions remain relatively rigid.} 
      \begin{columns}
        \begin{column}{0.5\textwidth}
          \begin{center}
            \includegraphics[width=50mm]{./FIGS/CamClayBunny.png} 
          \end{center}
        \end{column}
        \begin{column}{0.5\textwidth}
          \begin{center}
            \includegraphics[width=50mm]{./FIGS/DamageBunny.png} 
          \end{center}
        \end{column}
      \end{columns}
    \end{frame}

    \begin{frame}
      \frametitle{Peridynamics simulations}
      \begin{columns}
        \begin{column}{0.5\textwidth}
          \begin{center}
            \includegraphics[width=50mm]{./FIGS/PeriCrack2D.png} \\
            {\Grey EMUNE}
          \end{center}
        \end{column}
        \begin{column}{0.5\textwidth}
          \begin{center}
            \includegraphics[width=50mm]{./FIGS/PeriCylinder3D.png} \\
            {\Grey Peridigm}
          \end{center}
        \end{column}
      \end{columns}
    \end{frame}

  % =======================================================================
  % Plan and tasks
  % ----------------------------------------------------------------------
  \section{Plan and Tasks}

    \begin{frame}
      \frametitle{Project plan}
      \begin{columns}
        \begin{column}{\textwidth}
          \centering
          \includegraphics[width=90mm]{./FIGS/Weta_SmartIdea_Status_March013.pdf} \\
          %\def\svgwidth{\columnwidth}
          %\input{FIGS/Weta_SmartIdea_Status_March013.pdf_tex}
        \end{column}
      \end{columns}
    \end{frame}

    \begin{frame}
      \frametitle{Short-term Tasks}
      \begin{itemize}
        \item {\Grey Biswajit:} Serial and Parallel implementations of Peridynamics.
        \item {\Grey Bryan:} Rigid-body dynamics with MPM.
        \item {\Grey Kumar:} Contact algorithms and anisotropic material models.
        \item {\Grey Florin:} Anisotropic peridynamic fracture for wood/plasterboard.
        \item {\Grey Andreas:} Extraction of fracture surfaces from particle simulations for rendering.
        \item {\Grey Rojan:} Model-order reduction approaches for fracture.
        \item {\Grey Hooman:} Hybrid MPM-Peridynamics approaches.
      \end{itemize}
    \end{frame}

  % =======================================================================
  % What have I been up to
  % ----------------------------------------------------------------------
  \section{Recent progress}

    \begin{frame}
      \frametitle{Recent progress}
      \begin{itemize}
        \item Serial multibody peridynamics code being developed
        \item Several approaches tried (e.g., Matiti/Matiti2D/EMU2DC)
        \item Currently developing EMU2DC 
        \item The serial version will be used by Hooman for his work
      \end{itemize}
    \end{frame}

    \defverbatim[colored]\ListA {
      \begin{lstlisting}[language=XML,frame=single,basicstyle=\scriptsize,keywordstyle=\color{red}]
        <?xml version="1.0" encoding="iso-8859-1"?>
        <!-- <!DOCTYPE Vaango SYSTEM "input.dtd"> -->
        <!-- @version: -->
        <Vaango>
          <Meta>
            <title> Test input file for peridynamics </title>
          </Meta>
          <Time>
            <max_time> 1.0 </max_time>
            <max_iterations> 500 </max_iterations>
            <delt> 0.00000002 </delt>
          </Time>
          <Output>
            <output_file> test_output.dat </output_file>
            <output_iteration_interval> 25 </output_iteration_interval>
          </Output>
          <Peridynamics>
            <simulation_type> dynamic  </simulation_type>
            <modulus_type>    constant </modulus_type>
            <horizon_factor>  4.01     </horizon_factor>
          </Peridynamics>
          .............................
        </Vaango>
      \end{lstlisting}
    }
    \begin{frame}
      \frametitle{The input file}
      \ListA
    \end{frame}

    \defverbatim[colored]\ListDomain {
      \begin{lstlisting}[language=XML,frame=single,breaklines=true,basicstyle=\scriptsize,keywordstyle=\color{red}]
         <Domain>
           <min> [0.0, -2.0, 0.0] </min>
           <max> [4.0, 2.0, 0.0] </max>
           <num_cells> [10, 10, 1] </num_cells>
           <BoundaryConditions>
             <VelocityBC>
               <velocity> [0.0, 1.4e7, 0.0] </velocity>
               <Area>
                 <point> [0.0, -0.2, 0.0] </point>
                 <point> [1.0, -0.2, 0.0] </point>
               </Area>
             </VelocityBC>
             <VelocityBC>
               <velocity> [0.0, -1.4e7, 0.0] </velocity>
               <Area>
                 <point> [0.0, 0.2, 0.0] </point>
                 <point> [1.0, 0.2, 0.0] </point>
               </Area>
             </VelocityBC>
           </BoundaryConditions>
         </Domain>
      \end{lstlisting}
    }
    \begin{frame}
      \frametitle{Domain}
      \begin{columns}
        \begin{column}{\textwidth}
          \ListDomain
        \end{column}
      \end{columns}
    \end{frame}

\defverbatim[colored]\ListMaterial {
\begin{lstlisting}[language=XML,frame=single, linewidth=5cm,breaklines=true,basicstyle=\tiny,keywordstyle=\color{red}]
  <Material name="material 1">
    <young_modulus> 72.0e9 </young_modulus>
    <density> 2440.0 </density>
    <fracture_energy> 135.0 </fracture_energy>
    <DamageModel>
      <damage_viscosity> [0.0, 0.05, 0.0] </damage_viscosity>
      <damage_index> 0.35 </damage_index>
      <damage_stretch> [0.0, 0.0, 1.0] </damage_stretch>
    </DamageModel>
  </Material>

  <Material name="material 2">
    <young_modulus> 72.0e8 </young_modulus>
    <density> 244.0 </density>
    <fracture_energy> 13.5 </fracture_energy>
    <DamageModel>
      <damage_viscosity> [0.0, 0.005, 0.0] </damage_viscosity>
      <damage_index> 0.035 </damage_index>
      <damage_stretch> [0.0, 0.0, 0.1] </damage_stretch>
    </DamageModel>
  </Material>
\end{lstlisting}
}
\defverbatim[colored]\ListMaterialC {
\begin{lstlisting}[language=C++,frame=single,breaklines=true,linewidth=5cm,basicstyle=\tiny,keywordstyle=\color{red}]
MaterialSPArray mat_list;
int count = 0;
for (Uintah::ProblemSpecP mat_ps = ps->findBlock("Material"); mat_ps != 0;
     mat_ps = mat_ps->findNextBlock("Material")) {
  MaterialSP mat = std::make_shared<Material>();
  mat->initialize(mat_ps);
  mat->id(count);
  mat_list.emplace_back(mat);
  ++count;
  std::cout << *mat << std::endl;
}
\end{lstlisting}
}
    \begin{frame}
      \frametitle{Material}
      \begin{columns}
        \begin{column}{0.5\textwidth}
          \begin{minipage}{5cm}
            \ListMaterial
          \end{minipage}
        \end{column}
        \begin{column}{0.5\textwidth}
          \begin{minipage}{5cm}
            \ListMaterialC
          \end{minipage}
        \end{column}
      \end{columns}
    \end{frame}

\defverbatim[colored]\ListBody {
\begin{lstlisting}[language=XML,frame=single, linewidth=8cm,breaklines=true,basicstyle=\tiny,keywordstyle=\color{red}]
  <Body name="body 1">
    <material name="material 1"/>
    <Geometry>
      <input_node_file>    nodes_test_103by42.txt   </input_node_file>
      <input_element_file> element_test_103by42.txt </input_element_file>
    </Geometry>
    <InitialConditions>
      <velocity> [0.0, 0.0, 0.0] </velocity>
      <Crack>
        <LineString>
          <point> [-0.05, 0.0, 0.0] </point>
          <point> [-0.04, 0.0, 0.0] </point>
          <point> [-0.03, 0.0, 0.0] </point>
        </LineString>
      </Crack>
      <Crack>
        <LineString>
          <point> [0.04, 0.0, 0.0] </point>
          <point> [0.05, 0.0, 0.0] </point>
        </LineString>
      </Crack>
    </InitialConditions>
    <BoundaryConditions>
      <ExtForce>
        <force> [0.0, 1.4e7, 0.0] </force>
        <min> [0.0, -0.2, 0.0] </min>
        <max> [1.0, -0.2, 0.0] </max>
      </ExtForce>
      .......
    </BoundaryConditions>
  </Body>
\end{lstlisting}
}
    \begin{frame}
      \frametitle{Body}
      \begin{columns}
        \begin{column}{\textwidth}
          \ListBody
        \end{column}
      \end{columns}
    \end{frame}

\defverbatim[colored]\ListFamily {
\begin{lstlisting}[language=C++,frame=single, linewidth=5cm,breaklines=true,basicstyle=\tiny,keywordstyle=\color{red}]
namespace Emu2DC {

  class FamilyComputer {
  public:
    /**
     * Create an empty BondfamilyComputer object
     */
    FamilyComputer();
    ~FamilyComputer();
    /**
     *  Find which cells the nodes sit in and create a unordered map that maps nodes to cells
     *
     * @param domain Reference to the domain object
     * @param nodeList Reference to the vector of NodeP objects inside the domain
     */
    void createCellNodeMap(const Domain& domain,
                           const NodePArray& nodeList);
    ...
  };
}
\end{lstlisting}
}
\defverbatim[colored]\ListCellNodeMap {
\begin{lstlisting}[language=C++,frame=single, linewidth=5cm,breaklines=true,basicstyle=\tiny,keywordstyle=\color{red}]
  typedef std::tr1::unordered_multimap<long64, NodeP, Hash64> CellNodePMap;
  typedef CellNodePMap::iterator CellNodePMapIterator;
  typedef std::pair<long64, NodeP> CellNodePPair;
\end{lstlisting}
}
\defverbatim[colored]\ListNodeP {
\begin{lstlisting}[language=C++,frame=single, linewidth=5cm,breaklines=true,basicstyle=\tiny,keywordstyle=\color{red}]
  // using stdlib shared_ptr instead of SCIRun::Handle
  class Node;
  typedef std::shared_ptr<Node> NodeP;
\end{lstlisting}
}
\defverbatim[colored]\ListHash {
\begin{lstlisting}[language=C++,frame=single, linewidth=5cm,breaklines=true,basicstyle=\tiny,keywordstyle=\color{red}]
 // function object class for Hashing with lookup3
  struct Hash64 {
    std::size_t operator() (const long64& cellID) const {
      const u8* key = (const u8*) &cellID;
      u32 len = sizeof(cellID);
      u32 seed = 13;

      return lookup3((const u8*) &key, sizeof(key), 13 );
    }
\end{lstlisting}
}
    \begin{frame}
      \frametitle{FamilyComputer}
      \begin{columns}
        \begin{column}{0.5\textwidth}
          \begin{minipage}{5cm}
            \ListFamily
          \end{minipage}
        \end{column}
        \begin{column}{0.5\textwidth}
          \begin{minipage}{5cm}
            \ListCellNodeMap
            \ListNodeP
            \ListHash
          \end{minipage}
        \end{column}
      \end{columns}
    \end{frame}

  % =======================================================================
  % Work continues
  % ----------------------------------------------------------------------
  \section{And work continues}

    \begin{frame}
      \frametitle{Lessons so far}
      \begin{itemize}
        \item Learning recent developments in C++ 
        \item Keeping code simple but general is not easy
        \item ....
      \end{itemize}
    \end{frame}

    \begin{frame}
      \frametitle{Questions?}
    \end{frame}
\end{document}
%%%%%%%%%%%%%%%%%%%%%%%%%%%%%%%%%%%%%%%%%%%%%%%%%%%%%%%%%%%%%%%%%%%%%%%%%


