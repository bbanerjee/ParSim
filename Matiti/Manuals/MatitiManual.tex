\documentclass[11pt,fleqn]{book} % Default font size and left-justified equations

%--------------------------------------------------------------------------
% Document geometry % Page margins
%--------------------------------------------------------------------------
\usepackage[margin=1in,headsep=10pt,a4paper]{geometry}

%--------------------------------------------------------------------------
% MinionPro fonts
%--------------------------------------------------------------------------
\IfFileExists{MinionPro.sty}{%
  \usepackage{MnSymbol}
  \usepackage{MinionPro}
  \usepackage{MyriadPro}
}{%
  \usepackage[T1]{fontenc}
  \usepackage{newtxtext}
  %\usepackage{eulervm}
  \usepackage{amssymb}%needed for \mathbb
}

%--------------------------------------------------------------------------
% Color scheme used in this book
%--------------------------------------------------------------------------
\usepackage{xcolor} % Required for specifying colors by name
\definecolor{ocre}{RGB}{243,102,25} % Define the orange color used for highlighting throughout the book
\definecolor{dkgreen}{rgb}{0,0.6,0}
\definecolor{delim}{RGB}{20,105,176}
\definecolor{background}{HTML}{EEEEEE}
\definecolor{Red}{rgb}{0.8666,0.03137,0.02352}
\definecolor{Blue}{rgb}{0.00784,0.67059,0.91764}
\definecolor{Darkgreen}{rgb}{0,0.68235,0}
\definecolor{Green}{rgb}{0,0.8,0}
\definecolor{Royalblue}{rgb}{0,0.2,0.91764}
\definecolor{Brickred}{rgb}{0.644541,0.164065,0.164065}
\definecolor{Brown}{rgb}{0.6,0.4,0.4}
\definecolor{Orange}{rgb}{1,0.647059,0}
\definecolor{Indigo}{rgb}{0.746105,0,0.996109}
\definecolor{Violet}{rgb}{0.308598,0.183597,0.308598}
\definecolor{Lightgrey}{rgb}{0.762951,0.762951,0.762951}
\definecolor{Darkgrey}{rgb}{0.503548,0.503548,0.503548}
\definecolor{Pink}{rgb}{1,0.6,0.6}
\definecolor{DarkBlue}{rgb}{0,0.08,0.45}
\definecolor{OliveDrab}{rgb}{0.41961,0.55686,0.13725}
\definecolor{LightMagenta}{cmyk}{0.1,0.8,0,0.1}
\definecolor{OliveD}{HTML}{6B8E23}
\definecolor{authorNoteColor}{rgb}{.8,0,0}
\newcommand{\Ochre}{\color{ocre}}
\newcommand{\Red}{\color{Brickred}}
\newcommand{\Blue}{\color{Royalblue}}
\newcommand{\Green}{\color{Darkgreen}}
\newcommand{\DarkGreen}{\color{Darkgreen}}
\newcommand{\Violet}{\color{Violet}}
\newcommand{\Indigo}{\color{Indigo}}
\newcommand{\Orange}{\color{Orange}}
\newcommand{\Brown}{\color{Brown}}
\newcommand{\OliveD}{\color{OliveD}}
\newcommand{\Pink}{\color{Pink}}


%--------------------------------------------------------------------------
% Bibliography : biblatex does not work with tufte-book
%--------------------------------------------------------------------------
\usepackage[style=numeric,
            citestyle=numeric-comp,
            sorting=none,
            sortcites=true,
            autopunct=true,
            babel=hyphen,
            hyperref=true,
            abbreviate=false,
            backref=true,
            backend=biber]{biblatex}
\addbibresource{../Bibliography.bib} % BibTeX bibliography file
\defbibheading{bibempty}{}

%--------------------------------------------------------------------------
% Graphics
%--------------------------------------------------------------------------
\usepackage[pdftex]{graphicx}
\DeclareGraphicsExtensions{{.png},{.pdf},{.jpg},{jpeg}}
\graphicspath{ {../figures/}}
\setkeys{Gin}{width=\linewidth,totalheight=\textheight,keepaspectratio}

%--------------------------------------------------------------------------
% Links
%--------------------------------------------------------------------------
\usepackage{hyperref}
\hypersetup{pdftex, colorlinks=true, linkcolor=blue, citecolor=blue, filecolor=blue, urlcolor=blue, pdftitle=, pdfauthor= , pdfsubject=, pdfkeywords=}

%--------------------------------------------------------------------------
% Code listing
%--------------------------------------------------------------------------
\usepackage{listings}

\lstset{ %
  basicstyle=\scriptsize\ttfamily,           % the size of the fonts that are used for the code
  backgroundcolor=\color{white},      % choose the background color. You must add \usepackage{color}
  showspaces=false,               % show spaces adding particular underscores
  showstringspaces=false,         % underline spaces within strings
  showtabs=false,                 % show tabs within strings adding particular underscores
  tabsize=2,                      % sets default tabsize to 2 spaces
  captionpos=b,                   % sets the caption-position to bottom
  breaklines=true,                % sets automatic line breaking
  breakatwhitespace=false,        % sets if automatic breaks should only happen at whitespace
  commentstyle=\color{dkgreen}\upshape,       % comment style
  escapeinside={\%*}{*)},            % if you want to add LaTeX within your code
  morekeywords={*,MPM,ICE,MPMICE}               % if you want to add more keywords to the set
}

\colorlet{punct}{red!60!black}
\colorlet{numb}{magenta!60!black}

\lstdefinelanguage{XML}
{
  morestring=[b]",
  morestring=[s]{>}{<},
  morecomment=[s]{<?}{?>},
  morestring=[s]{"}{"},
  morecomment=[s]{?}{?},
  morecomment=[s]{!--}{--},
  stringstyle=\color{black},
  identifierstyle=\color{DarkBlue},
  keywordstyle=\color{Brickred},
  backgroundcolor=\color{background},
  frame=lines,
  morekeywords={xmlns,version,type}% list your attributes here
}

\lstdefinelanguage{C++}
{
  keywordstyle=\color{Brickred},
  stringstyle=\color{red},
  identifierstyle=\color{DarkBlue},
  backgroundcolor=\color{background},
  frame=lines,
  morecomment=[l][\color{magenta}]{\#}
}

\lstdefinelanguage{Cpp}
{
  language=C++,
  commentstyle=\color{dkgreen}\upshape,   
  stringstyle=\color{red},
  identifierstyle=\color{DarkBlue},
  backgroundcolor=\color{background},
  frame=lines,
  deletekeywords={...},
  escapeinside={\%*}{*)},
  keywordstyle=\color{Brickred},
  morekeywords={ProblemSpecP, ProcessorGroup, SimulationController,%
                AMRSimulationController, RegridderCommon, SolverInterface,%
                SolverFactory, UintahParallelComponent, SimulationInterface,%
                ComponentFactory, LoadBalancerCommon, LoadBalancerFactory,%
                DataArchiver, Output, SchedulerCommon, SchedulerFactory,%
                ProblemSetupException, Exception,%
                Task, Level, Patch, Ghost, Example,%
                GridP, SimulationStateP, LevelP, SchedulerP,%
                PatchSubset, MaterialSubset, DataWarehouse,% 
                SFCXVariable, PerPatch, IntVector, NodeIterator,%
                VarLabel, MPMMaterial, PatchSet, ParticleSubset,%
                MPMFlags, FlowModel, ConstitutiveModel, MyModel, MyFlow, PlasticityState,%
                particleIndex, TangentModulusTensor, Matrix3, Vector,
                ElasticModuli, TabularPlasticity, ModelState_Tabular, ParticleVariable,
                constParticleVariable, delt_vartype, ElasticModuliModelFactory,%
                YieldConditionFactory, ElasticModuliModel, YieldCondition, CMData,
                ParticleLabelVariableMap, ParameterDict, TabularData, IndependentVar,
                DependentVar, IndexKey, json, DoubleVec1D, DoubleVec2D,
                int,char,double,float,unsigned, size_t,%
                string, istringstream, cerr, exit}, 
  morecomment=[l][\color{magenta}]{\#}
}


\lstdefinelanguage{JSON}{
    numbers=left,
    numberstyle=\scriptsize,
    stepnumber=1,
    numbersep=8pt,
    showstringspaces=false,
    breaklines=true,
    frame=lines,
    backgroundcolor=\color{background},
    literate=
     *{0}{{{\color{numb}0}}}{1}
      {1}{{{\color{numb}1}}}{1}
      {2}{{{\color{numb}2}}}{1}
      {3}{{{\color{numb}3}}}{1}
      {4}{{{\color{numb}4}}}{1}
      {5}{{{\color{numb}5}}}{1}
      {6}{{{\color{numb}6}}}{1}
      {7}{{{\color{numb}7}}}{1}
      {8}{{{\color{numb}8}}}{1}
      {9}{{{\color{numb}9}}}{1}
      {:}{{{\color{punct}{:}}}}{1}
      {,}{{{\color{punct}{,}}}}{1}
      {\{}{{{\color{delim}{\{}}}}{1}
      {\}}{{{\color{delim}{\}}}}}{1}
      {[}{{{\color{delim}{[}}}}{1}
      {]}{{{\color{delim}{]}}}}{1},
}


%--------------------------------------------------------------------------
% Colored boxes:  Must come after graphicx and verbatim
% and Tikz
%--------------------------------------------------------------------------
\RequirePackage{tikz}
\usetikzlibrary{shadings,shadows}
\usetikzlibrary{decorations.pathmorphing}
\usetikzlibrary{patterns}
\usetikzlibrary{intersections}
\usetikzlibrary{calc}

\usepackage{tcolorbox}
\tcbuselibrary{most}
\newtcolorbox{NoteBox}[1][]{%
  colback=yellow!50,
  colframe=yellow!20!black,
  before skip=2mm,after skip=3mm,
  boxrule=0.4pt,left=5mm,right=2mm,top=1mm,bottom=1mm,
  sharp corners,
  rounded corners=southeast,arc is angular,arc=3mm,
  underlay={%
    \path[fill=tcbcol@back!80!black] ([yshift=3mm]interior.south east)--++(-0.4,-0.1)--++(0.1,-0.2);
    \path[draw=tcbcol@frame,shorten <=-0.05mm,shorten >=-0.05mm] ([yshift=3mm]interior.south east)--++(-0.4,-0.1)--++(0.1,-0.2);
    \path[fill=yellow!50!black,draw=none] (interior.south west) rectangle node[white]{\Huge\bfseries !} ([xshift=4mm]interior.north west);
    },
  drop fuzzy shadow,%
  #1}

\newtcolorbox{WarningBox}[1][]{%
  colback=pink!50,
  colframe=pink!20!black,
  before skip=2mm,after skip=3mm,
  boxrule=0.4pt,left=5mm,right=2mm,top=1mm,bottom=1mm,
  sharp corners,
  rounded corners=southeast,arc is angular,arc=3mm,
  underlay={%
    \path[fill=tcbcol@back!80!black] ([yshift=3mm]interior.south east)--++(-0.4,-0.1)--++(0.1,-0.2);
    \path[draw=tcbcol@frame,shorten <=-0.05mm,shorten >=-0.05mm] ([yshift=3mm]interior.south east)--++(-0.4,-0.1)--++(0.1,-0.2);
    \path[fill=pink!50!black,draw=none] (interior.south west) rectangle node[white]{\Huge\bfseries !} ([xshift=4mm]interior.north west);
    },
  drop fuzzy shadow,%
  #1}

\tcbuselibrary{skins}
\newtcolorbox[auto counter,number within=chapter]{ExampleBox}[1][]{%
  enhanced,
  colback=white,
  colframe=green!65!black,
  enlarge top by=10mm,
  overlay={%
    \path[fill=blue!65,line width=.4mm] (frame.north west)--++(17mm,0)coordinate(n2)--++(0,8mm)--++(-20mm,0) arc (-90:90:-4mm)--cycle;
    \node at ([shift={(5mm,4mm)}]frame.north west){\color{white}{\textbf{\sffamily EXAMPLE}}};
    \path[fill=green!65!blue] ([xshift=.4mm]n2)--++(0,8mm)--++(7mm,0)--++(0,-8mm)--cycle;
    \node at ([shift={(4mm,4mm)}]n2){\color{white}{\textbf{\sffamily \thetcbcounter}}};
    %\node at ([shift={(18mm,4mm)}]n2){\itshape\textbf{\sffamily Solution}};
  },
  #1}

\usetikzlibrary{calc}
\usetikzlibrary{positioning}
\tcbuselibrary{skins}
\newtcolorbox[auto counter,number within=section]{SummaryBox}[2][]{%
  enhanced,
  colback=white,
  colframe=ocre,
  enlarge top by=10mm,
  overlay={%
    \path[fill=ocre!65,line width=.4mm] (frame.north west)--++(17mm,0)coordinate(n2)--++(0,8mm)--++(-20mm,0) arc (-90:90:-4mm)--cycle;
    \node at ([shift={(5mm,4mm)}]frame.north west){\color{white}{\textbf{\sffamily Summary}}};
    \path[fill=ocre] ([xshift=.4mm]n2)--++(0,8mm)--++(10mm,0)--++(0,-8mm)--cycle;
    \node (A) at ([shift={(5mm,4mm)}]n2){\color{white}{\textbf{\sffamily \thetcbcounter}}};
    \node [right=0.5cm of A] {\itshape\textbf{\sffamily #2}};
  },
  #1}


%--------------------------------------------------------------------------
% Structure of the document
%--------------------------------------------------------------------------
%\input{00LayoutStructure}

%--------------------------------------------------------------------------
% Other packges
%--------------------------------------------------------------------------
%\usepackage{bm}%to get bold math symbols
%\usepackage[pdftex]{graphicx}
%\usepackage{verbatim}
%\usepackage[tt]{titlepic}%package i downloaded to put a pic in the titlepage
%\usepackage{helvet}
%\usepackage[usenames,dvipsnames]{xcolor}%options to use names like redviolet and others
%\usepackage{xspace}
%\usepackage[mathscr]{eucal}%Defines which font to use with \mathscr
%\usepackage{enumerate}
%\usepackage{comment}
%\usepackage{mathrsfs}
%\usepackage{floatrow}%automatically centers graphics without needing a \center command on each and every figure.

% Index
%\usepackage{calc} % For simpler calculation - used for spacing the index letter headings correctly
%\usepackage{makeidx} % Required to make an index

%-------------------------------------------------------
% For appendix
%-------------------------------------------------------
%\usepackage{esint}
\usepackage[toc,page,title,titletoc]{appendix}

%\usepackage{makeidx} % Used to generate the index

%\usepackage{xspace} % Used for printing a trailing space better than using a tilde (~) using the \xspace command

%\usepackage{epsf}
%\usepackage{epsfig}
%\usepackage{boxedminipage}
%\usepackage{stmaryrd}
%\usepackage{cancel}

%-------------------------------------------------------
% For text wrap around figures and captions for subfigures
%-------------------------------------------------------
\usepackage{wrapfig}
\usepackage{subcaption}

%-------------------------------------------------------
% For underlines
%-------------------------------------------------------
\usepackage[normalem]{ulem}

%-------------------------------------------------------
% For algorithms
%-------------------------------------------------------
\usepackage{algorithm, algpseudocode, float}
%\usepackage{algorithmic}

\makeatletter
\newenvironment{breakablealgorithm}
  {% \begin{breakablealgorithm}
   \begin{center}
     \refstepcounter{algorithm}% New algorithm
     \hrule height.8pt depth0pt \kern2pt% \@fs@pre for \@fs@ruled
     \renewcommand{\caption}[2][\relax]{% Make a new \caption
       {\raggedright\textbf{\ALG@name~\thealgorithm} ##2\par}%
       \ifx\relax##1\relax % #1 is \relax
         \addcontentsline{loa}{algorithm}{\protect\numberline{\thealgorithm}##2}%
       \else % #1 is not \relax
         \addcontentsline{loa}{algorithm}{\protect\numberline{\thealgorithm}##1}%
       \fi
       \kern2pt\hrule\kern2pt
     }
  }{% \end{breakablealgorithm}
     \kern2pt\hrule\relax% \@fs@post for \@fs@ruled
   \end{center}
  }
\makeatother
\renewcommand{\algorithmiccomment}[1]{\hfill{\color{ocre}$\triangleright$\textit{#1}}}



%\input{00LatexMacros.tex}

%-------------------------------------------------------
% For index
%-------------------------------------------------------
\makeindex 

%-------------------------------------------------------
% Background picture
%-------------------------------------------------------
\newcommand\BackgroundPic{%
\put(0,0){%
\parbox[b][\paperheight]{\paperwidth}{%
\vfill
\centering
\includegraphics[width=0.5\paperwidth,height=0.5\paperheight,%
keepaspectratio]{periodic_ellipsoids_tighter_fit.png}%
\vfill
}}}

%-------------------------------------------------------
% Versioning
%-------------------------------------------------------
\makeatletter
\def\MyYear#1{%
  \def\yy@##1##2##3##4;{##3##4}%
  \expandafter\yy@#1;
}
\def\MyMonth#1{%
  \def\yy@##1;{0##1}%
  \def\yy@##1##2;{##1##2}%
  \expandafter\yy@#1;
}
\makeatother
\newcommand{\version}{Version \MyYear{\the\year}.\MyMonth{\the\month}}



\input{00LayoutStructure.tex}
\input{00LatexMacros.tex}

\begin{document}

  %----------------------------------------------------------------------------------------
  %       TITLE PAGE
  %----------------------------------------------------------------------------------------
  \begingroup
    \thispagestyle{empty}
    \AddToShipoutPicture*{\BackgroundPic} % Image background
    \centering
    \vspace*{1cm}
    \par\normalfont\fontsize{35}{35}\sffamily\selectfont
    \textcolor{white}{Understanding the Matiti Peridynamics programs}\par % Book title
    \vspace*{0.5cm}
    {\Huge \textcolor{yellow}{Matiti version 14.10}}\par
    {\Huge \textcolor{yellow}{October 2014}}\par
    \vspace*{1cm}
    {\Huge \textcolor{yellow}{Hooman Zarreh}}\par % Author name
    {\Huge \textcolor{yellow}{and}}\par % Author name
    {\Huge \textcolor{yellow}{Biswajit Banerjee}}\par % Author name
  \endgroup


   %----------------------------------------------------------------------------------------
  %       COPYRIGHT PAGE
  %----------------------------------------------------------------------------------------

  \newpage
  ~\vfill
  \thispagestyle{empty}

  \noindent Copyright \copyright\ 2013 Callaghan Innovation\\ % Copyright notice
  \noindent Copyright \copyright\ 2015-2016 Parresia Research Limited\\ % Copyright notice

  %\noindent \textsc{Published by Publisher}\\ % Publisher

  %\noindent \textsc{book-website.com}\\ % URL

  %\noindent Licensed under the Creative Commons Attribution-NonCommercial 3.0 Unported License (the ``License''). You may not use this file except in compliance with the License. You may obtain a copy of the License at \url{http://creativecommons.org/licenses/by-nc/3.0}. Unless required by applicable law or agreed to in writing, software distributed under the License is distributed on an \textsc{``AS IS'' BASIS, WITHOUT WARRANTIES OR CONDITIONS OF ANY KIND}, either express or implied. See the License for the specific language governing permissions and limitations under the License.\\ % License information

  \noindent The contents of this manual can and will change significantly over time.  Please make sure that all the information is up to date.
  %\noindent \textit{First printing, March 2013} % Printing/edition date

  %----------------------------------------------------------------------------------------
  %       TABLE OF CONTENTS
  %----------------------------------------------------------------------------------------
  \chapterimage{fire-container-explosion-scirun-cut.png} % Table of contents heading image
  \pagestyle{empty} % No headers
  \tableofcontents % Print the table of contents itself
  \cleardoublepage % Forces the first chapter to start on an odd page so it's on the right
  \pagestyle{fancy} % Print headers again

  \setlength{\parindent}{0pt}
  \setlength{\parskip}{5pt}





\chapter{Matiti Structure} 
\label{sec:MatitiStructure} 
In the \emph{ParticleFracturing} directory you can find a directory named as \emph{Matiti}. \emph{matiti} itself consists of other folders like: src, opt, runs and triangle as it can be seen in figure~\ref{fig:MatitiDirectoryGraph}.
\begin{figure}
  \includegraphics{MatitiDirectoryGraph.png}
  \caption{Directory Graph for Matiti}
  \label{fig:MatitiDirectoryGraph}
\end{figure}
\begin{itemize}
\item From the \emph{Triangle} directory you need a library with which you can discretize your surface into triangle shaped parts.    
\item \emph{opt} is a directory that you make in order to make the needed files such as the executable files from your source directory so you can run the program. 
\item \emph{runs} is where you run your program and save the out-put data (as vtk file format) to be visualised using a proper package like visIt. 
\item \emph{src} is the source folder where all the input files and programs are gathered in it. This directory again includes other folders and files which are categorized in figure~\ref{fig:MatitiDirectoryGraph}. \\
First group which are shown by the blue boxes are:
 \begin{itemize}
 \item \emph{CMake} finds the runtimes for cmake so you can make all the needed files out of the src directory into the opt directory.
 \item \emph{Geometry} is where simple geometries such as point, vector, box and polygon are made, figure~\ref{fig:Geometry}:
\begin{figure}
  \includegraphics{Geometry.png}
  \caption{Graph for Geometry directory}
  \label{fig:Geometry}
\end{figure}
 \item \emph{GeometryPiece} is a directory in which geometry pieces can be created using two input nodes (as lower and upper) and the elementary geometries introduces in the \emph{Geometry} directory. This directory consists of files as shown in figure~\ref{fig:GeometryPiece}
\begin{figure}
  \includegraphics{GeometryPiece.png}
  \caption{Graph for GeometryPiece directory}
  \label{fig:GeometryPiece}
\end{figure}
 \item \emph{input-files} is a directory in which all your input files have been gathered. These are the files that you need to copy them into the \emph{runs} directory, whenever you want to run them, and from there run it using the executable file in the \emph{opt} directory (--> \emph{test\_peri}.) Organization and the skeleton of these input files are described in the next chapter (--> \textbf{Chapter}~\ref{chap:Organization})
 \end{itemize}
In addition to these directories, there are so many files in the src. We have made some groups shown by the violet color. These files are devided and arranged in each group based on their properties and applications:
\begin{itemize}
 \item \emph{containers} as shown in figure~\ref{fig:containers}
\begin{figure}
  \includegraphics{containers.png}
  \caption{Graph for Containers Group}
  \label{fig:containers}
\end{figure}
 \item \emph{Pointers} as shown in figure~\ref{fig:Pointers}
\begin{figure}
  \includegraphics{Pointers.png}
  \caption{Graph for Pointers Group}
  \label{fig:Pointers}
\end{figure}
 \item \emph{Types} as shown in figure~\ref{fig:Types}
\begin{figure}
  \includegraphics{Types.png}
  \caption{Graph for Types Group}
  \label{fig:Types}
\end{figure}
 \item \emph{Exceptions} as shown in figure~\ref{fig:Exceptions}
\begin{figure}
  \includegraphics{Exceptions.png}
  \caption{Graph for Exceptions Group}
  \label{fig:Exceptions}
\end{figure}
 \item \emph{Tests} as shown in figure~\ref{fig:Tests}
\begin{figure}
  \includegraphics{Tests.png}
  \caption{Graph for Tests Group}
  \label{fig:Tests}
\end{figure}
 \item \emph{Src \& header} as shown in figure~\ref{fig:SrcAndHeader}
\begin{figure}
  \includegraphics{SrcAndHeader.png}
  \caption{Graph for Src \& header Group}
  \label{fig:SrcAndHeader}
\end{figure}
 \end{itemize}
\end{itemize}







\chapter{Organization}
\label{chap:Organization}

The Matiti Peridynamics programs work as described below:
\begin{itemize}
\item First it gets an input using \emph{ProblemSpecReader} and \emph{ProblemSpecUtil}
\item All the calculations on the data would be done in a mainbody of the program (saved in the input files directory.) This part is called the \textbf{Domain}.
\item and the out put data would be saved as VTK files using \emph{Output} and \emph{OutputVTK} programs. 
\end{itemize}

All the above items are gathered in figure~\ref{fig:Perischem}
\begin{figure}
  \includegraphics{Perischem.png}
  \caption{Graph for the Peridynamics schematic job}
  \label{fig:Perischem}
\end{figure}

Domain is where you can make your bodies or objects. Before the domain part, in the main body, we set the \emph{time} and \emph{Simulation State} and in the domain part, having the domain points we set the \emph{VelocityBC}. Now in the domain we can make our object(s) or the Body(s) but before that we need to make the \emph{Nodes}. \\
Actually, the \emph{GeometryPiece} makes \emph{Elements} with which we can have our \emph{Nodes}. Moreover \emph{Material} should be made using the \emph{Damagemodel} and be passed to the \emph{Nodes} through the \emph{Bond}. For making the nodes we also need \emph{FamilyComputer} and \emph{HorizonComputer} to find the neighbours of each node. (See figure~\ref{fig:Nodes})
\begin{figure}
  \includegraphics{Nodes.png}
  \caption{Graph for the Nodes}
  \label{fig:Nodes}
\end{figure}

Now that we have the nodes we just need to give the \emph{Initial Conditions} and \emph{Force Boundary Conditions} so we can have our body(s). The \emph{Initial Condition} is made of \emph{Cracks}, \emph{initial Velocity} and the \emph{Body Forces} like the gravity.
All the information regarding the domain structure and organization is gathered in figure~\ref{fig:Domain})
\begin{figure}
  \includegraphics{Domain.png}
  \caption{Graph for the Domain}
  \label{fig:Domain}
\end{figure}

  
%----------------------------------------------------------------------------------------
%       BIBLIOGRAPHY
%----------------------------------------------------------------------------------------

\chapter*{Bibliography}
\addcontentsline{toc}{chapter}{\textcolor{ocre}{Bibliography}}

%\bibliographystyle{plain}
%\bibliography{Bibliography}
\printbibliography[heading=bibempty]




\end{document}
