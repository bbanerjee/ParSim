\chapter{Isotropic hardening models}
Several flow stress models have been implemented in \Vaango.  These are described in this chapter.

\section{Linear hardening model}
The linear hardening model in \Vaango has the form
\Beq
  \sigma_y(\Ep) = \sigma_0 + K \Ep
\Eeq
where $\sigma_0$ is the initial yield stress, $K$ is a hardening modulus, and
$\Ep$ is the equivalent plastic strain.

The linear hardening model can be invoked using
\lstset{language=XML}
\begin{lstlisting}
<flow_model type="linear">
  <sigma_0> 700.0e6 </sigma_0>
  <K>1.5e6</K>
</flow_model>
\end{lstlisting}

\section{Johnson-Cook model}
The Johnson-Cook (JC) model (\cite{Johnson1983}) has
the following relation for the flow stress ($\sigma_y$) 
\begin{equation}
  \sigma_y(\Ep,\Edot{\Teq},T) = 
  \left[A + B (\Ep)^n\right]\left[1 + C \ln(\Edot{\star})\right]
  \left[1 - (T^*)^m\right]
\end{equation}
where $\Ep$ is the equivalent plastic strain, 
$A$, $B$, $C$, $n$, $m$ are material constants, and
\begin{equation}
  \Edot{^\star} = \cfrac{\Edot{\Teq}}{\Edot{0}}; \quad
  T^* = \cfrac{(T-T_0)}{(T_m-T_0)}\,.
\end{equation}
In the above equations, $\Edot{\Teq}$ is the 
equivalent strain rate, 
$\Edot{0}$ is a reference strain rate, 
$T_0$ is a reference temperature, and $T_m$ is the melt temperature.  
For conditions where $T^\star < 0$, we assume that $m = 1$.

The inputs for this model have the form
\lstset{language=XML}
\begin{lstlisting}
<flow_model type="johnson_cook">
  <A>792.0e6</A>
  <B>510.0e6</B>
  <C>0.014</C>
  <n>0.26</n>
  <m>1.03</m>
  <T_r>298.0</T_r>
  <T_m>1793.0</T_m>
  <epdot_0>1.0</epdot_0>
</flow_model>
\end{lstlisting}

\section{Steinberg-Guinan model}
The Steinberg-Cochran-Guinan-Lund (SCG) model is a semi-empirical model
that was developed by \cite{Steinberg1980} for high strain rate 
situations and extended to low strain rates and bcc materials by
\cite{Steinberg1989}.  The flow stress in this model is given by
\begin{equation}\label{eq:SCGL}
  \sigma_y(\Ep,\Epdoteq,T) = 
   \left[\sigma_a f(\Ep) + \sigma_t (\Epdoteq, T)\right]
   \frac{\mu(p,T)}{\mu_0} 
\end{equation}
where $\sigma_a$ is the athermal component of the flow stress,
$f(\Ep)$ is a function that represents strain hardening,
$\sigma_t$ is the thermally activated component of the flow stress,
$\mu(p,T)$ is the shear modulus, and $\mu_0$ is the shear modulus 
at standard temperature and pressure.  The strain hardening function
has the form
\begin{equation}
  f(\Ep) = [1 + \beta(\Ep + \Epi)]^n ; \quad
  \sigma_a f(\Ep) \le \sigma_{\text{max}}
\end{equation}
where $\beta, n$ are work hardening parameters, and $\Epi$ is the 
initial equivalent plastic strain.  The thermal component $\sigma_t$
is computed using a bisection algorithm from the following equation (based 
on the work of \cite{Hoge1977})
\begin{equation}
  \Epdoteq = \left[\frac{1}{C_1}\exp\left[\frac{2U_k}{k_b~T}
    \left(1 - \frac{\sigma_t}{\sigma_p}\right)^2\right] + 
    \frac{C_2}{\sigma_t}\right]^{-1}; \quad
  \sigma_t \le \sigma_p
\end{equation}
where $2 U_k$ is the energy to form a kink-pair in a dislocation segment
of length $L_d$, $k_b$ is the Boltzmann constant, $\sigma_p$ is the Peierls
stress. The constants $C_1, C_2$ are given by the relations
\begin{equation}
  C_1 := \frac{\rho_d L_d a b^2 \nu}{2 w^2}; \quad
  C_2 := \frac{D}{\rho_d b^2}
\end{equation}
where $\rho_d$ is the dislocation density, $L_d$ is the length of a 
dislocation segment, $a$ is the distance between Peierls valleys, 
$b$ is the magnitude of the Burgers' vector, $\nu$ is the Debye frequency,
$w$ is the width of a kink loop, and $D$ is the drag coefficient.

The inputs for this model are of the form
\lstset{language=XML}
\begin{lstlisting}
  <flow_model type="steinberg_cochran_guinan">
    <mu_0> 81.8e9 </mu_0>
    <sigma_0> 1.15e9 </sigma_0>
    <Y_max> 0.25e9 </Y_max>
    <beta> 2.0 </beta>
    <n> 0.50 </n>
    <A> 20.6e-12 </A>
    <B> 0.16e-3 </B>
    <T_m0> 2310.0 </T_m0>
    <Gamma_0> 3.0 </Gamma_0>
    <a> 1.67 </a>
    <epsilon_p0> 0.0 </epsilon_p0>
  </flow_model>
\end{lstlisting}

\section{Zerilli-Armstrong model}
The Zerilli-Armstrong (ZA) model (\cite{Zerilli1987,Zerilli1993,Zerilli2004}) 
is based on simplified dislocation mechanics.  The general form of the
equation for the flow stress is
\begin{equation}
  \sigma_y(\Ep,\Epdoteq,T) = 
    \sigma_a + B\exp(-\beta(\Epdoteq) T) + 
                         B_0\sqrt{\Ep}\exp(-\alpha(\Epdoteq) T)
\end{equation}
where $\sigma_a$ is the athermal component of the flow stress given by
\begin{equation}
  \sigma_a := \sigma_g + \frac{k_h}{\sqrt{l}} + K(\Ep)^n,
\end{equation}
$\sigma_g$ is the contribution due to solutes and initial dislocation
density, $k_h$ is the microstructural stress intensity, $l$ is the 
average grain diameter, $K$ is zero for fcc materials, 
$B, B_0$ are material constants.  The functional forms of the exponents 
$\alpha$ and $\beta$ are 
\begin{equation}
  \alpha = \alpha_0 - \alpha_1 \ln(\Epdoteq); \quad
  \beta = \beta_0 - \beta_1 \ln(\Epdoteq); 
\end{equation}
where $\alpha_0, \alpha_1, \beta_0, \beta_1$ are material parameters that
depend on the type of material (fcc, bcc, hcp, alloys).  The Zerilli-Armstrong
model has been modified by \cite{Abed2005} for better performance at high 
temperatures.  However, we have not used the modified equations in our
computations.

The input for this model is of the form
\lstset{language=XML}
\begin{lstlisting}
  <flow_model type="zerilli_armstrong">
    <sigma_g>     46.5e6   </sigma_g>
    <k_H>         5.0e6    </k_H>
    <sqrt_l_inv>  3.7      </sqrt_l_inv>
    <B>           0.0      </B>
    <beta_0>      0.0      </beta_0>
    <beta_1>      0.0      </beta_1>
    <B_0>         890.0e6  </B_0>
    <alpha_0>     0.0028   </alpha_0>
    <alpha_1>     0.000115 </alpha_1>
    <K>           0.0      </K>
    <n>           0.0      </n>
  </flow_model>
\end{lstlisting}

\section{Polymer Zerilli-Armstrong model}
The Zerilli-Armstrong model for polymers has the form:
\begin{equation}
  \sigma_y(\Ep,\Epdoteq,T) = \sigma_g + B\exp(-\beta T^\star) + 
                             B_0\sqrt{\omega \Ep}\exp(-\alpha T^\star)
\end{equation}
where $\sigma_g$ is the athermal component of the flow stress and
\Beq
  \omega = \omega_a + \omega_b \ln(\Epdoteq) + \omega_p \sqrt{\pbar}
\Eeq
where $\omega_a$, $\omega_b$, $\omega_p$ are material parameters
and $\pbar = -p$ is the pressure (positive in compression).
The functional forms of the exponents $\alpha$ and $\beta$ are 
\begin{equation}
  \alpha = \alpha_0 - \alpha_1 \ln(\Epdoteq); \quad
  \beta = \beta_0 - \beta_1 \ln(\Epdoteq); 
\end{equation}
where $\alpha_0, \alpha_1, \beta_0, \beta_1$ are material parameters.
The factors $B$ and $B_0$ are defined as
\Beq
  B = B_{\text{pa}} \left(1 + B_\text{pb} \sqrt{\pbar}\right)^{B_\text{pn}} ~,~~
  B_0 = B_{\text{0pa}} \left(1 + B_\text{0pb} \sqrt{\pbar}\right)^{B_\text{0pn}} 
\Eeq
where $B_\text{pa}$, $B_\text{0pa}$, $B_\text{pb}$, $B_\text{0pb}$,
$B_\text{pn}$, and $B_\text{0pn}$ are material parameters.  Also,
\Beq
  T^\star = \frac{T}{T_0}
\Eeq
where $T_0$ is a reference temperature.

The input tags for the polymer ZA model are:
\lstset{language=XML}
\begin{lstlisting}
  <flow_model type="zerilli_armstrong_polymer">
    <sigma_g>     46.5e6   </sigma_g>
    <B_pa>        0.0      </B_pa>
    <B_pb>        0.0      </B_pb>
    <B_pn>        1.0      </B_pn>
    <beta_0>      0.0      </beta_0>
    <beta_1>      0.0      </beta_1>
    <T_0>         300.0    </T_0>
    <B_0pa>       890.0e6  </B_0pa>
    <B_0pb>       0.0      </B_0pb>
    <B_0pn>       1.0      </B_0pn>
    <omega_a>     0.0      </omega_a>
    <omega_b>     0.0      </omega_b>
    <omega_p>     0.0      </omega_p>
  </flow_model>
\end{lstlisting}

\section{Mechanical thresold stress model}
The Mechanical Threshold Stress (MTS) model 
(\cite{Follans1988,Goto2000a,Kocks2001})  
gives the following form for the flow stress
\begin{equation}
  \sigma_y(\Ep,\Epdoteq,T) = 
    \sigma_a + (S_i \sigma_i + S_e \sigma_e)\frac{\mu(p,T)}{\mu_0} 
\end{equation}
where $\sigma_a$ is the athermal component of mechanical threshold stress,
$\mu_0$ is the shear modulus at 0 K and ambient pressure, 
$\sigma_i$ is the component of the flow stress due to intrinsic barriers 
to thermally activated dislocation motion and dislocation-dislocation 
interactions, $\sigma_e$ is the component of the flow stress due to 
microstructural evolution with increasing deformation (strain hardening), 
($S_i, S_e$) are temperature and strain rate dependent scaling factors.  The
scaling factors take the Arrhenius form
\begin{align}
  S_i & = \left[1 - \left(\frac{k_b~T}{g_{0i}b^3\mu(p,T)}
  \ln\frac{\Epdot{0i}}{\Epdoteq}\right)^{1/q_i}
  \right]^{1/p_i} \\
  S_e & = \left[1 - \left(\frac{k_b~T}{g_{0e}b^3\mu(p,T)}
  \ln\frac{\Epdot{0e}}{\Epdoteq}\right)^{1/q_e}
  \right]^{1/p_e}
\end{align}
where $k_b$ is the Boltzmann constant, $b$ is the magnitude of the Burgers' 
vector, ($g_{0i}, g_{0e}$) are normalized activation energies, 
($\Epdot{0i}, \Epdot{0e}$) are constant reference strain rates, and
($q_i, p_i, q_e, p_e$) are constants.  The strain hardening component
of the mechanical threshold stress ($\sigma_e$) is given by a
modified Voce law
\begin{equation}\label{eq:MTSsige}
  \frac{d\sigma_e}{d\Ep} = \theta(\sigma_e)
\end{equation}
where
\begin{align}
  \theta(\sigma_e) & = 
     \theta_0 [ 1 - F(\sigma_e)] + \theta_{IV} F(\sigma_e) \\
  \theta_0 & = a_0 + a_1 \ln \Epdoteq + a_2 \sqrt{\Epdoteq} - a_3 T \\
  F(\sigma_e) & = 
    \cfrac{\tanh\left(\alpha \cfrac{\sigma_e}{\sigma_{es}}\right)}
    {\tanh(\alpha)}\\
  \ln(\cfrac{\sigma_{es}}{\sigma_{0es}}) & =
  \left(\frac{kT}{g_{0es} b^3 \mu(p,T)}\right)
  \ln\left(\cfrac{\Epdoteq}{\Epdot{0es}}\right)
\end{align}
and $\theta_0$ is the hardening due to dislocation accumulation, 
$\theta_{IV}$ is the contribution due to stage-IV hardening,
($a_0, a_1, a_2, a_3, \alpha$) are constants,
$\sigma_{es}$ is the stress at zero strain hardening rate, 
$\sigma_{0es}$ is the saturation threshold stress for deformation at 0 K,
$g_{0es}$ is a constant, and $\Epdot{0es}$ is the maximum strain rate.  Note
that the maximum strain rate is usually limited to about $10^7$/s.

The inputs for this model are of the form
\lstset{language=XML}
\begin{lstlisting}
  <flow_model type="mechanical_threshold_stress">
    <sigma_a>363.7e6</sigma_a>
    <mu_0>28.0e9</mu_0>
    <D>4.50e9</D>
    <T_0>294</T_0>
    <koverbcubed>0.823e6</koverbcubed>
    <g_0i>0.0</g_0i>
    <g_0e>0.71</g_0e>
    <edot_0i>0.0</edot_0i>
    <edot_0e>2.79e9</edot_0e>
    <p_i>0.0</p_i>
    <q_i>0.0</q_i>
    <p_e>1.0</p_e>
    <q_e>2.0</q_e>
    <sigma_i>0.0</sigma_i>
    <a_0>211.8e6</a_0>
    <a_1>0.0</a_1>
    <a_2>0.0</a_2>
    <a_3>0.0</a_3>
    <theta_IV>0.0</theta_IV>
    <alpha>2</alpha>
    <edot_es0>3.42e8</edot_es0>
    <g_0es>0.15</g_0es>
    <sigma_es0>1679.3e6</sigma_es0>
  </flow_model>
\end{lstlisting}

\section{Preston-Tonks-Wallace model}
The Preston-Tonks-Wallace (PTW) model (\cite{Preston2003}) attempts to 
provide a model for the flow stress for extreme strain rates 
(up to $10^{11}$/s) and temperatures up to melt.  The flow stress is
given by
\begin{equation}
  \sigma_y(\Ep,\Epdoteq,T) = 
     \begin{cases}
       2\left[\tau_s + \alpha\ln\left[1 - \varphi
        \exp\left(-\beta-\cfrac{\theta\Ep}{\alpha\varphi}\right)\right]\right]
       \mu(p,T) & \text{thermal regime} \\
       2\tau_s\mu(p,T) & \text{shock regime}
     \end{cases}
\end{equation}
with 
\begin{equation}
  \alpha := \frac{s_0 - \tau_y}{d}; \quad
  \beta := \frac{\tau_s - \tau_y}{\alpha}; \quad
  \varphi := \exp(\beta) - 1
\end{equation}
where $\tau_s$ is a normalized work-hardening saturation stress,
$s_0$ is the value of $\tau_s$ at 0K,
$\tau_y$ is a normalized yield stress, $\theta$ is the hardening constant
in the Voce hardening law, and $d$ is a dimensionless material
parameter that modifies the Voce hardening law.  The saturation stress
and the yield stress are given by
\begin{align}
  \tau_s & = \max\left\{s_0 - (s_0 - s_{\infty})
     \erf\left[\kappa
       \That\ln\left(\cfrac{\gamma\Xidot}{\Epdoteq}\right)\right],
     s_0\left(\cfrac{\Epdoteq}{\gamma\Xidot}\right)^{s_1}\right\} \\
  \tau_y & = \max\left\{y_0 - (y_0 - y_{\infty})
     \erf\left[\kappa
       \That\ln\left(\cfrac{\gamma\Xidot}{\Epdoteq}\right)\right],
     \min\left\{
       y_1\left(\cfrac{\Epdoteq}{\gamma\Xidot}\right)^{y_2}, 
       s_0\left(\cfrac{\Epdoteq}{\gamma\Xidot}\right)^{s_1}\right\}\right\} 
\end{align}
where $s_{\infty}$ is the value of $\tau_s$ close to the melt temperature,
($y_0, y_{\infty}$) are the values of $\tau_y$ at 0K and close to melt,
respectively, $(\kappa, \gamma)$ are material constants, $\That = T/T_m$,
($s_1, y_1, y_2$) are material parameters for the high strain rate
regime, and
\begin{equation}
  \Xidot = \frac{1}{2}\left(\cfrac{4\pi\rho}{3M}\right)^{1/3}
           \left(\cfrac{\mu(p,T)}{\rho}\right)^{1/2}
\end{equation}
where $\rho$ is the density, and $M$ is the atomic mass.

The inputs for this model are of the form
\lstset{language=XML}
\begin{lstlisting}
  <flow_model type="preston_tonks_wallace">
    <theta> 0.025 </theta>
    <p> 2.0 </p>
    <s0> 0.0085 </s0>
    <sinf> 0.00055 </sinf>
    <kappa> 0.11 </kappa>
    <gamma> 0.00001 </gamma>
    <y0> 0.0001 </y0>
    <yinf> 0.0001 </yinf>
    <y1> 0.094 </y1>
    <y2> 0.575 </y2>
    <beta> 0.25 </beta>
    <M> 63.54 </M>
    <G0> 518e8 </G0>
    <alpha> 0.20 </alpha>
    <alphap> 0.20 </alphap>
  </flow_model>
\end{lstlisting}

