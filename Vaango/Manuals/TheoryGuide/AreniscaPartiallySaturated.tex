\chapter{Arenisca: Partially Saturated Soils}

The partially saturated soil model uses Michael Homel's ``consistency bisection'' algorithm 
to find the plastic strain direction and to update the internal state variables.  A closest-point
return algorithm in transformed stress space is used to project the trial stress state on to the
yield surface.

\section{The consistency bisection algorithm}
\subsection{Fixed (nonhardening) yield surface}
Let the stress at the beginning of the load step be $\Bsig^\Told$ and let the trial stress be 
$\Bsig^\Trial$.  Assume the yield surface is fixed and let the correct projection of the trial 
stress on to the fixed yield surface be $\Bsig^{\Tnew, 0}$.

The increment of stress for the load step ($\Delta\Bsig^0$) is related to the 
elastic strain increment ($\Delta\Bveps^{\Te,0}) $by
\Beq
  \Delta\Bsig^0 = \Bsig^{\Tnew,0} - \Bsig^\Told = \SfC:\Delta\Bveps^{\Te,0}
\Eeq
where $\SfC$ is a constant elastic modulus tensor.  The elastic modulus tensor can be assumed
to be an average value of the nonlinear tangent modulus for the load step.

If we know $\SfC$, we can compute the elastic strain increment using
\Beq
  \Delta\Bveps^{\Te,0} = \SfC^{-1}:\Delta\Bsig^0 \,.
\Eeq

For a strain driven update algorithm, the total strain increment $\Delta\Bveps$ is known.
Assuming that the total strain increment can be additively decomposed into an elastic and a 
plastic part, we can find the plastic strain increment ($\Delta\Bveps^{\Tp,0}$) using
\Beq
  \Delta\Bveps^{\Tp,0} = \Delta\Bveps - \Delta\Bveps^{\Te,0} \,.
\Eeq

\subsection{Hardening yield surface}
Now, if we allow the yield surface to harden, the distance between the trial stress point and its 
projection on to the yield surface decreases compared to that for a fixed yield surface.  If
$\Delta\Bveps^\Tp$ is the plastic strain increment for a hardening yield surface, we have
\Beq
  \Delta\Bveps^\Tp > \Delta\Bveps^{\Tp,0}   
\Eeq
where the inequality can be evaluated using an appropriate Euclidean norm.
Note that this distance is proportional to the consistency parameter $\dot{\lambda}$.

\subsubsection{Fully saturated model}
In the fully saturated version of the Arenisca model, the internal variables are the hydrostatic
compressive strength ($X$) and the scalar isotropic backstress ($\zeta$).  These depend only on the
{\bf volumetric} plastic strain increment 
\Beq
  \Delta\Veps^\Tp_v = \Tr(\Delta\Bveps^\Tp) \,.
\Eeq
Because
\Beq
  \Delta\Bveps^\Tp_v > \Delta\Bveps^{\Tp,0}_v   
\Eeq
we can define a parameter, $\eta \in (0, 1)$, such that
\Beq
  \eta := \frac{\Delta\Bveps^\Tp_v}{\Delta\Bveps^{\Tp,0}_v} \,.
\Eeq
Because the solution is bounded by the fixed yield surface, a bisection algorithm can be used
to find the parameter $\eta$.

\subsubsection{Partially saturated model}
{\Red TODO}

\subsection{Bisection algorithm: Fully saturated}
\begin{algorithm}
  \begin{algorithmic}[1]
    \REQUIRE $\Veps^{\Tp,\Told}_v$, $\Delta\Veps^{\Tp,0}_v$, $\zeta^\Told$
    \STATE $\eta^\Tin \leftarrow 0$,  $\eta^\Tout \leftarrow 1$ 
    \STATE ELASTIC $\leftarrow$ \TRUE
    \REPEAT
    \STATE $\eta^\Tmid \leftarrow \Half(\eta^\Tout + \eta^\Tin)$
    \STATE $X^\Tnew \leftarrow X(\Veps^{\Tp,\Told}_v + \eta^\Tmid \,\Delta\Veps^{\Tp,0}_v)$
    \COMMENT {Update the hydrostatic compressive strength}
    \STATE $\zeta^\Tnew \leftarrow \zeta^\Told + 
        \left(\Partial{\zeta}{\Veps^{\Tp}_v}\right)\times(\eta^\Tmid \,\Delta\Veps^{\Tp,0}_v)$
    \COMMENT {Update the isotropic backstress}
    \STATE ELASTIC = Compute updated yield condition
    \IF {ELASTIC}
    \STATE $\eta^\Tout \leftarrow \eta^\Tmid$ 
    \ENDIF
    \UNTIL {ELASTIC = \FALSE}
    \STATE RETURN\_DIRECTION = Compute return to updated yield surface (no hardening)
    \IF {sign(RETURN\_DIRECTION) $\neq$ sign(??)}
    \STATE GO BACK 
    \ENDIF
    \STATE Compute 
    \WHILE {$\Mr(\Mx^k) \ne 0$}
      \STATE $\Mx^{k+1} \Leftarrow \Mx^k - \left[\left(\Partial{\Mr}{\Mx}\right)^{-1}\right]_{\Mx^k}\cdot
              \Mr(\Mx^k)$
      \STATE $k \leftarrow k+1$
    \ENDWHILE
  \end{algorithmic}
\end{algorithm}


