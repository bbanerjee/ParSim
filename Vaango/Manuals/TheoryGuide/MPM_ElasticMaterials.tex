\chapter{Elastic material models}
\section{Hypoelastic material}
\Textmag{Applicable to:} \Textsfc{explicit} and \Textsfc{implicit} \MPM

Hypoelastic materials have stress-deformation relationships of the form
\Beq
  \dot{\Bsig}(\BF) = \SfC(\BF) : \BdT(\BF)
\Eeq
where $\SfC$ is an elastic stiffness tensor and $\BdT$ is the symmetric
part of the velocity gradient.

The base hypoelastic material implemented in Vaango is linear and isotropic:
\Beq
  \dot{\Bsig} = \left(\kappa - \tfrac{2}{3}\mu\right) \Tr(\BdT)\,\BI + 2\mu \BdT
\Eeq
where $\mu$ is the shear modulus and $\kappa$ is the bulk modulus.

\begin{NoteBox}
  To ensure frame indifference, both $\Bsig$ and $\BdT$ are unrotated 
  using the beginning of the timestep deformation gradient polar decomposition before any
  constitutive relations are evaluated.  The updated stress is rotated back using the deformation 
  gradient decomposition at the end of the time step.
\end{NoteBox}

\section{Hyperelastic Material Models}
Several hyperelastic material models have been implemented in \Vaango.  Other models
can be easily implemented using the available infrastructure.  The general
model has the form
\Beq
  \Bsig = \frac{1}{J} \Partial{W}{\BF} \cdot\BF^T
\Eeq
where $W$ is a strain energy function and $J = \det\BF$.  
For isotropic hyperelastic functions that
are expessed in terms of the invariants ($I_1$, $I_2$, $J$) of the right Cauchy-Green deformation 
($\BC = \BF^T\cdot\BF$), the Cauchy stress is given by
\Beq
  \Bsig = \cfrac{2}{J}\left[\cfrac{1}{J^{2/3}}\left(\cfrac{\partial{W}}{\partial \bar{I}_1} + \bar{I}_1~\cfrac{\partial{W}}{\partial \bar{I}_2}\right)\BB -
   \cfrac{1}{J^{4/3}}~\cfrac{\partial{W}}{\partial \bar{I}_2}~\BB \cdot\BB \right]  + \left[\cfrac{\partial{W}}{\partial J} -
\cfrac{2}{3J}\left(\bar{I}_1~\cfrac{\partial{W}}{\partial \bar{I}_1} + 2~\bar{I}_2~\cfrac{\partial{W}}{\partial \bar{I}_2}\right)\right]~\BI
\Eeq
where $\BB = \BF\cdot\BF^T$, and
\Beq
  J = \det\BF ~,~~ \bar{I}_1 = J^{-2/3} I_1 ~,~~ \bar{I}_2 = J^{-4/3} I_2 ~,~~
  I_1 = \Tr\,\BC ~,~~ I_2 = \Half\left[(\Tr\,\BC)^2 - \Tr(\BC\cdot\BC)\right]
\Eeq
Note that $I_1$ and $I_2$ are identical for $\BC$ and $\BB$. Alternatively,
\Beq
  \Bsig  = \cfrac{2}{J}\left[\left(\cfrac{\partial W}{\partial I_1} + 
           I_1~\cfrac{\partial W}{\partial I_2}\right)\BB - 
           \cfrac{\partial W}{\partial I_2}~\BB \cdot\BB \right] + 
           2J~\cfrac{\partial W}{\partial I_3}~\BI
\Eeq
where $I_3 = J^2$.

The P-wave speed ($c$) needed to estimate the timestep can be computed using
\Beq
  c_i^2 = \frac{1}{\rho J} \PPartial{W}{\lambda_i}
        = \frac{1}{\rho J} \left[ \Partial{W}{I_1}\PPartial{I_1}{\lambda_i} + 
                                  \Partial{W}{I_2}\PPartial{I_2}{\lambda_i} + 
                                  \Partial{W}{I_3}\PPartial{I_3}{\lambda_i} \right] 
\Eeq
where $\lambda_i$ are the principal stretches, i.e., $I_1 = \sum_i \lambda_i^2$,
$I_2 = \lambda_1^2\lambda_2^2 + \lambda_2^2\lambda_3^2 + \lambda_1^2\lambda_3^2$ and
$I_3 = \lambda_1^2\lambda_2^2\lambda_3^2$.

\subsection{Compressible neo-Hookean material}
\Textmag{Applicable to:} \Textsfc{explicit} and \Textsfc{implicit} \MPM

The default strain energy function for the compressible neo-Hookean material model
implemented in \Vaango is (\cite{Simo1998}, p.307):
\Beq
  W = \frac{\kappa}{2}\left[\frac{1}{2}(J^2-1) - \ln J\right] + 
      \frac{\mu}{2}\left[\bar{I}_1 - 3\right]
\Eeq
The Cauchy stress corresponding to this function is
\Beq
  \Bsig = \frac{\kappa}{2}\left(J - \frac{1}{J}\right)\BI + \frac{\mu}{J}(\bar{\BB} - \Third\bar{I}_1\BI)
\Eeq
where $\bar{\BB} = J^{-2/3} \BB = J^{-2/3} \BF\cdot\BF^T$ and $J = \det\BF$.  
Consistency with linear elasticity requires that $\kappa = K$ and $\mu = G$ 
where $K$ and $G$ are the linear elastic bulk and shear moduli, respectively.

Alternative expressions for the bulk modulus factor are allowed and defined in
the equation-of-state submodels.

\subsection{Compressible Mooney-Rivlin material}
\Textmag{Applicable to:} \Textsfc{explicit} \MPM only

The compressible Mooney-Rivlin material implemented in \Vaango has the form
\Beq
  W = C_1(I_1 - 3) + C_2(I_2 - 3) + C_3\left(\frac{1}{I_3^2} - 1\right) +
      C_4(I_3 - 1)^2
\Eeq
where $C_1$, $C_2$ and $\nu$ are parameters and
\Beq
  C_3 = \Half(C_1 + 2C_2) ~,~~ 
  C_4 = \Half\left[\frac{C_1(5\nu-2) + C_2(11\nu-5)}{1-2\nu}\right] \,.
\Eeq
The corresponding Cauchy stress is
\Beq
  \Bsig = \frac{2}{J}\left[(C_1 + C_2 I_1)\BB - C_2 \BB\cdot\BB +
                           J^2 \left[-\frac{2C_3}{I_3^3} + 2C_4(I_3-1)\right]\right] \,.
\Eeq

\subsection{Transversely isotropic hyperelastic material}
\Textmag{Applicable to:} \Textsfc{explicit} and \Textsfc{implicit} \MPM

The transversely isotropic material model implemented in \Vaango is based on \cite{Weiss1994}.
The model asssumes a stiffer, ``fiber'', direction denoted $\Bfhat$ and isotropy ortogonal
to that direction.

The strain energy density function for the model has the form
\Beq
  W = W_v + W_d
\Eeq
where $W_v$ is the volumetric part and $W_d$ is the deviatoric (volume preserving) part. The
volumetric part of the strain energy is given by
\Beq
  W_v = \Half \kappa (\ln J)^2
\Eeq
where $\kappa$ is the bulk modulus and $J = \det\BF$.  The deviatoric part, $W_d$, is 
given by
\Beq
  W_d = \begin{cases}
          C_1(\bar{I_1} - 3) + C_2(\bar{I_2} - 3) + C_3\left[\exp\left(C_4(\lambar-1)\right) - 1\right]
          & \quad \text{for} \quad \lambar < \lambda^\star \\
          C_1(\bar{I_1} - 3) + C_2(\bar{I_2} - 3) + C_5\lambar + C_6\ln\lambar
          & \quad \text{for} \quad \lambar \ge \lambda^\star
        \end{cases}
\Eeq
where $C_1$, $C_2$, $C_3$, $C_4$, $C_5$, $\lambda^\star$ are model parameters, and
\Beq
  \Bal
  C_6 &= C_3\left[\exp\left(C_4(\lambda^\star-1)\right) - 1\right] - C_5\lambda^\star \\
  \lambar &= \sqrt{\bar{I_4}} ~,~~
  \bar{I_4} = \Bfhat \cdot (\bar{\BC} \cdot \Bfhat) ~,~~ \bar{\BC} = J^{-2/3} \BC  \,.
  \Eal
\Eeq
The fiber direction is updated using
\Beq
  \Bfhat_{n+1} = \frac{J^{-1/3}}{\lambar}\,\BF\cdot\Bfhat_n \,.
\Eeq
The Cauchy stress is given by
\Beq
  \Bsig = p\BI + \Bsig_d + \Bsig_f
\Eeq
where
\Beq
  \Bal
    p &= \kappa \frac{\ln(J)}{J} \\
    \Bsig_d & = \frac{2}{J}\left[(C_1 + C_2\bar{I_1})\bar{\BB} - C_2\bar{\BB}\cdot\bar{\BB}
                -\Third (C_1\bar{I_1} + 2C_2\bar{I_2})\BI\right] \\
    \Bsig_f &= \frac{\lambar}{J} \,\Partial{W_d}{\lambar} 
               \left(\Bfhat_{n+1}\otimes\Bfhat_{n+1} - \Third \BI\right)
  \Eal
\Eeq

The model also contains a failure feature that sets $\Bsig_d = 0$ when the maximum
shear strain, defined as the difference between the maximum and mimum eigenvalues
of $\BC$, exceeds a critical shear strain value.  Also, a fiber stretch
failure criterion can be used that compares $\sqrt{I_4}$ with a critical stretch value
and sets $\Bsig_f = 0$ is this value is exceeded.
