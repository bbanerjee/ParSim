\chapter{Elastic material models}
\section{Hyperelastic Material Models}

The subject of modeling the response of materials to deformation 
is a subject that has filled numerous textbooks.  Therefore, 
rather than attempt to condense these volumes, here the reader 
will be simply be given a simple material response model.  Other 
more complex material response models can be interchanged in the 
framework discussed above quite readily.

The author has come to prefer a class of models known as 
hyperelastic models.  What this means is that the stress 
response of these materials is derived from a strain energy 
function.  A strain energy function gives a relationship between 
the state of deformation that a material is in, and the amount 
of stored strain energy that this material has.  This is akin to 
the familiar relationship for the stored energy in a spring, 
$W=\frac{1}{2} k dx^2$ where k is the spring constant, and $dx$ is
the distance that the spring has been compressed or extended.

One such strain energy function is given by:

\begin{equation}
	W = \frac{\lambda}{4}(J^2-1)
 - (\frac{\lambda}{2}+\mu) \ln J
+ \frac{\mu}{2}\,\Tr(\tn{F}^T\tn{F} - 3)
\end{equation}

from which the following relationship for the stress can be 
derived:

\begin{equation}
       \sig = \frac{\lambda}{2}(J-\frac{1}{J})\tn{I} 
  + \mu (\tn{F}\tn{F}^T) - \tn{I})
 \label{stress}
\end{equation}

where $\lambda$ and $\mu$ are material constants, while $J$ and 
$\tn{F}$ describe the state of deformation.  These will be 
defined shortly.

In the Algorithm section, the calculation of the velocity 
gradient, $\nabla \tn{v}_p$ is given in Equation \ref{velgrad}.
Starting from there, we can then compute an increment in the 
deformation gradient, $\tn{F}(dt)$ by:
\begin{equation}
	\tn{F}(dt) = \nabla \tn{v}_p dt + \tn{I}.
\end{equation}
This increment in the deformation gradient can then be used to 
compute a new total deformation gradient using:
\begin{equation}
	\tn{F}(t+dt) = \tn{F}(dt) \tn{F}(t).
\end{equation}
Note that the initial (t=0) deformation gradient is simply the 
identity, i.e. $\tn{F}(0) = \tn{I}$.  Now with the deformation 
gradient, one can compute $J$ by:
\begin{equation}
	J = \det(\tn{F}(t+dt)).
\end{equation}

Note that $J$ represents the volumetric part of the deformation.  
Specifically, it is the ratio of the current volume of an 
element of material to it's original volume.  Similarly, we can
define an increment in $J$ as:
\begin{equation}
	J_{inc} = \det(\tn{F}(dt))
\end{equation}
which  is the ratio of the current volume of an element of 
material to it's volume at the previous timestep.  Thus we can 
write:
\begin{equation}
	v_p(t+dt) = J_{inc} v_p(t).
\end{equation}

Elastic material properties are frequently given in terms of 
bulk and shear moduli, or $\kappa$ and $\mu$.  The shear is 
sometimes denoted by $G$.  The shear modulus $\mu$ appears in 
Equation \ref{stress} above.  $\lambda$ can be computed from
$\kappa$ and $\mu$ by:
\begin{equation}
	\lambda = \kappa - \frac{2}{3}\mu.
\end{equation}

Lastly, based on material properties $\lambda$ and $\mu$, a material 
wavespeed can be computed:

\begin{equation}
	c^2 = (\lambda + 3 \mu)\frac{m_p}{v_p}.
\end{equation}

This wavespeed can be used in computing the timestep size as 
described above.

