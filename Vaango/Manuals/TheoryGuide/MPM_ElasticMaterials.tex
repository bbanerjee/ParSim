\chapter{Elastic material models}
\section{Hypoelastic material}
\Textmag{Applicable to:} \Textsfc{explicit} and \Textsfc{implicit} \MPM

Hypoelastic materials have stress-deformation relationships of the form
\Beq
  \dot{\Bsig}(\BF) = \SfC(\BF) : \BdT(\BF)
\Eeq
where $\SfC$ is an elastic stiffness tensor and $\BdT$ is the symmetric
part of the velocity gradient.

The base hypoelastic material implemented in Vaango is linear and isotropic:
\Beq
  \dot{\Bsig} = \left(\kappa - \tfrac{2}{3}\mu\right) \Tr(\BdT)\,\BI + 2\mu \BdT
\Eeq
where $\mu$ is the shear modulus and $\kappa$ is the bulk modulus.

\begin{NoteBox}
  To ensure frame indifference, both $\Bsig$ and $\BdT$ are unrotated 
  using the beginning of the timestep deformation gradient polar decomposition before any
  constitutive relations are evaluated.  The updated stress is rotated back using the deformation 
  gradient decomposition at the end of the time step.
\end{NoteBox}

\section{Hyperelastic Material Models}
Several hyperelastic material models have been implemented in \Vaango.  Other models
can be easily implemented using the available infrastructure.  The general
model has the form
\Beq
  \Bsig = \frac{1}{J} \Partial{W}{\BF} \cdot\BF^T
\Eeq
where $W$ is a strain energy function and $J = \det\BF$.  
For isotropic hyperelastic functions that
are expessed in terms of the invariants ($I_1$, $I_2$, $J$) of the right Cauchy-Green deformation 
($\BC = \BF^T\cdot\BF$), the Cauchy stress is given by
\Beq
  \Bsig = \cfrac{2}{J}\left[\cfrac{1}{J^{2/3}}\left(\cfrac{\partial{W}}{\partial \bar{I}_1} + \bar{I}_1~\cfrac{\partial{W}}{\partial \bar{I}_2}\right)\BB -
   \cfrac{1}{J^{4/3}}~\cfrac{\partial{W}}{\partial \bar{I}_2}~\BB \cdot\BB \right]  + \left[\cfrac{\partial{W}}{\partial J} -
\cfrac{2}{3J}\left(\bar{I}_1~\cfrac{\partial{W}}{\partial \bar{I}_1} + 2~\bar{I}_2~\cfrac{\partial{W}}{\partial \bar{I}_2}\right)\right]~\BI
\Eeq
where $\BB = \BF\cdot\BF^T$, and
\Beq
  J = \det\BF ~,~~ \bar{I}_1 = J^{-2/3} I_1 ~,~~ \bar{I}_2 = J^{-4/3} I_2 ~,~~
  I_1 = \Tr\,\BC ~,~~ I_2 = \Half\left[(\Tr\,\BC)^2 - \Tr(\BC\cdot\BC)\right]
\Eeq
Note that $I_1$ and $I_2$ are identical for $\BC$ and $\BB$. Alternatively,
\Beq
  \Bsig  = \cfrac{2}{J}\left[\left(\cfrac{\partial W}{\partial I_1} + 
           I_1~\cfrac{\partial W}{\partial I_2}\right)\BB - 
           \cfrac{\partial W}{\partial I_2}~\BB \cdot\BB \right] + 
           2J~\cfrac{\partial W}{\partial I_3}~\BI
\Eeq
where $I_3 = J^2$.

The P-wave speed ($c$) needed to estimate the timestep can be computed using
\Beq
  c_i^2 = \frac{1}{\rho J} \PPartial{W}{\lambda_i}
        = \frac{1}{\rho J} \left[ \Partial{W}{I_1}\PPartial{I_1}{\lambda_i} + 
                                  \Partial{W}{I_2}\PPartial{I_2}{\lambda_i} + 
                                  \Partial{W}{I_3}\PPartial{I_3}{\lambda_i} \right] 
\Eeq
where $\lambda_i$ are the principal stretches, i.e., $I_1 = \sum_i \lambda_i^2$,
$I_2 = \lambda_1^2\lambda_2^2 + \lambda_2^2\lambda_3^2 + \lambda_1^2\lambda_3^2$ and
$I_3 = \lambda_1^2\lambda_2^2\lambda_3^2$.

\subsection{Compressible neo-Hookean material}
\Textmag{Applicable to:} \Textsfc{explicit} and \Textsfc{implicit} \MPM

The default strain energy function for the compressible neo-Hookean material model
implemented in \Vaango is (\cite{Simo1998}, p.307):
\Beq
  W = \frac{\kappa}{2}\left[\frac{1}{2}(J^2-1) - \ln J\right] + 
      \frac{\mu}{2}\left[\bar{I}_1 - 3\right]
\Eeq
The Cauchy stress corresponding to this function is
\Beq
  \Bsig = \frac{\kappa}{2}\left(J - \frac{1}{J}\right)\BI + \frac{\mu}{J}(\bar{\BB} - \Third\bar{I}_1\BI)
\Eeq
where $\bar{\BB} = J^{-2/3} \BB = J^{-2/3} \BF\cdot\BF^T$ and $J = \det\BF$.  
Consistency with linear elasticity requires that $\kappa = K$ and $\mu = G$ 
where $K$ and $G$ are the linear elastic bulk and shear moduli, respectively.

Alternative expressions for the bulk modulus factor are allowed and defined in
the equation-of-state submodels.

\subsection{Compressible Mooney-Rivlin material}
\Textmag{Applicable to:} \Textsfc{explicit} \MPM only

The compressible Mooney-Rivlin material implemented in \Vaango has the form
\Beq
  W = C_1(I_1 - 3) + C_2(I_2 - 3) + C_3\left(\frac{1}{I_3^2} - 1\right) +
      C_4(I_3 - 1)^2
\Eeq
where $C_1$, $C_2$ and $\nu$ are parameters and
\Beq
  C_3 = \Half(C_1 + 2C_2) ~,~~ 
  C_4 = \Half\left[\frac{C_1(5\nu-2) + C_2(11\nu-5)}{1-2\nu}\right] \,.
\Eeq
The corresponding Cauchy stress is
\Beq
  \Bsig = \frac{2}{J}\left[(C_1 + C_2 I_1)\BB - C_2 \BB\cdot\BB +
                           J^2 \left[-\frac{2C_3}{I_3^3} + 2C_4(I_3-1)\right]\right] \,.
\Eeq

\subsection{Transversely isotropic hyperelastic material}
\Textmag{Applicable to:} \Textsfc{explicit} and \Textsfc{implicit} \MPM

The transversely isotropic material model implemented in \Vaango is based on \cite{Weiss1994}.
The model asssumes a stiffer, ``fiber'', direction denoted $\Bfhat$ and isotropy ortogonal
to that direction.

The strain energy density function for the model has the form
\Beq
  W = W_v + W_d
\Eeq
where $W_v$ is the volumetric part and $W_d$ is the deviatoric (volume preserving) part. The
volumetric part of the strain energy is given by
\Beq
  W_v = \Half \kappa (\ln J)^2
\Eeq
where $\kappa$ is the bulk modulus and $J = \det\BF$.  The deviatoric part, $W_d$, is 
given by
\Beq
  W_d = \begin{cases}
          C_1(\bar{I_1} - 3) + C_2(\bar{I_2} - 3) + C_3\left[\exp\left(C_4(\lambar-1)\right) - 1\right]
          & \quad \text{for} \quad \lambar < \lambda^\star \\
          C_1(\bar{I_1} - 3) + C_2(\bar{I_2} - 3) + C_5\lambar + C_6\ln\lambar
          & \quad \text{for} \quad \lambar \ge \lambda^\star
        \end{cases}
\Eeq
where $C_1$, $C_2$, $C_3$, $C_4$, $C_5$, $\lambda^\star$ are model parameters, and
\Beq
  \Bal
  C_6 &= C_3\left[\exp\left(C_4(\lambda^\star-1)\right) - 1\right] - C_5\lambda^\star \\
  \lambar &= \sqrt{\bar{I_4}} ~,~~
  \bar{I_4} = \Bfhat \cdot (\bar{\BC} \cdot \Bfhat) ~,~~ \bar{\BC} = J^{-2/3} \BC  \,.
  \Eal
\Eeq
The fiber direction is updated using
\Beq
  \Bfhat_{n+1} = \frac{J^{-1/3}}{\lambar}\,\BF\cdot\Bfhat_n \,.
\Eeq
The Cauchy stress is given by
\Beq
  \Bsig = p\BI + \Bsig_d + \Bsig_f
\Eeq
where
\Beq
  \Bal
    p &= \kappa \frac{\ln(J)}{J} \\
    \Bsig_d & = \frac{2}{J}\left[(C_1 + C_2\bar{I_1})\bar{\BB} - C_2\bar{\BB}\cdot\bar{\BB}
                -\Third (C_1\bar{I_1} + 2C_2\bar{I_2})\BI\right] \\
    \Bsig_f &= \frac{\lambar}{J} \,\Partial{W_d}{\lambar} 
               \left(\Bfhat_{n+1}\otimes\Bfhat_{n+1} - \Third \BI\right)
  \Eal
\Eeq

The model also contains a failure feature that sets $\Bsig_d = 0$ when the maximum
shear strain, defined as the difference between the maximum and mimum eigenvalues
of $\BC$, exceeds a critical shear strain value.  Also, a fiber stretch
failure criterion can be used that compares $\sqrt{I_4}$ with a critical stretch value
and sets $\Bsig_f = 0$ is this value is exceeded.

\section{Elastic modulus models}
\Textmag{Applicable to:} \Textsfc{Hypoelastic} \Textsfc{Tabular} material models

For selected material models that use isotropic hypoelasticity models and require bulk 
and shear moduli, specialized elastic moduli models can be used.  Some of these
models are discussed in this section.

\subsection{Support vector regression model}
The support vector regression (SVR) approach~\cite{Scholkopf2000,Smola2004} can be used to fit 
bulk modulus models to data without the need for closed form expressions.  The advantage of this approach is
that the resulting model requires few function evaluations and can, in principle,
be computed as fast as a closed-form model.

For the purpose of fitting a bulk modulus model we assume that the input (training) data are of the form
$\{(\BVeps_1, p_1), (\BVeps_2, p_2), \dots, (\BVeps_m, p_m)\} \subset \mathbb{R}^2 \times \mathbb{R}$.
Here $\BVeps_i = (\Veps_i, \Veps^p_i)$ where $\Veps$ is the total volumetric strain and
$\Veps^p$ is the plastic volumetric strain and $p_i$ is the mean stress (assumed positive in 
compression).
The aim of SVR is to find a function $p = f(\BVeps)$ that fits the data such that the function is 
as flat as possible (in $d+1$-dimensional space), and deviates from $p_i$ by at most $\epsilon$
(a small quantity).

In nonlinear support vector regression we fit functions of the form
\Beq
  p = f(\BVeps) =  \Bw \cdot \Bphi(\BVeps) + b
\Eeq
where $\Bw$ is a vector of parameters, $\Bphi(\BVeps)$ are vector-valued basis functions, $(\cdot)$ is an
inner product, and $b$ is a scalar offset.  The fitting process can be posed as
the following primal convex optimization problem~\cite{Vapnik1998}:
\Beq
  \begin{aligned}
    \underset{\Bw, b, \Bxi, \Bxi^\star}{\text{minimize}}\quad &
  \tfrac{1}{2} \Bw \cdot \Bw + C \sum_{i=1}^m (\xi_i + \xi^\star_i) \\
  \text{subject to}\quad &
     \begin{cases}
       -(\xi_i + \epsilon) \le p_i -  \Bw \cdot \Bphi(\BVeps_i) - b \le \xi_i^\star + \epsilon \\
       \xi_i, \xi_i^\star \ge 0 ~,~~ i = 1 \dots m
     \end{cases}
  \end{aligned}
\Eeq
where $C$ is a constraint multiplier, $m$ is the number of data points,
and $\xi_i, \xi_i^\star$ are constraints.

In practice, it is easier to solve the dual problem for which the expansion for 
\(f(\BVeps)\) becomes
\Beq \label{eq:svr}
  p = f(\BVeps) = \sum_{i=1}^m (\lambda_i^\star - \lambda_i) K(\BVeps_i, \BVeps) + b ~,~~
  K(\BVeps_i,\BVeps) = \Bphi(\BVeps_i) \cdot \Bphi(\BVeps)
\Eeq
where $\BVeps_i$ are the sample vectors, $\lambda_i$ and $\lambda_i^\star$ are dual coefficients, and
$K(\BVeps_i,\BVeps)$ is a kernel function.
The dual convex optimization problem has the form
\Beq
  \begin{aligned}
    \underset{\Blambda,\Blambda^\star}{\text{minimize}}\quad &
    \tfrac{1}{2}
    \sum_{i,j=1}^m (\lambda_i - \lambda_i^\star) K(\BVeps_i, \BVeps_j) (\lambda_j - \lambda_j^\star)
     + \epsilon \sum_{i=1}^m (\lambda_i + \lambda_i^\star)
     + \sum_{i=1}^m p_i(\lambda_i - \lambda_i^\star) \\
    \text{subject to}\quad &
       \begin{cases}
         \sum_{i=1}^m (\lambda_i - \lambda_i^\star) = 0 \\
         \lambda_i, \lambda_i^\star \in [0, C] ~,~~ i = 1 \dots m \,.
       \end{cases}
  \end{aligned}
\Eeq
The free parameters for the fitting process are the quantities $\epsilon$ and $C$.
SVR accuracy also depends strongly on the choice of kernel function.  In this paper,
we use the Gaussian radial basis function:
\Beq \label{eq:svr_rbf}
  K(\BVeps_i, \BVeps_j) = \exp\left[-\frac{(\BVeps_i-\BVeps_j)\cdot(\BVeps_i-\BVeps_j)}{\sigma^2\,d}\right]
   = \exp\left[-\gamma \Norm{\BVeps_i-\BVeps_j}{}^2\right] ~,~~ \gamma := \frac{1}{\sigma^2\,d}
\Eeq
where $d$ is the dimension of $\BVeps$ and $\sigma^2$ is the width of the support of the 
kernel (assumed to be equal to the norm of the covariance matrix of the training data in this paper).  

The minimization problem solves for the difference in the dual coefficients ($\Blambda - \Blambda^\star$) 
and the intercept ($b$), and outputs a reduced set ($m_{\text{SV}} < m$) of values of $\BVeps_i$ 
called ``support vectors''.  Given these quantities, the function~\eqref{eq:svr} can be evaluated
quite efficiently, particularly if the number of support vectors is small.  SVR fits to data can be 
computed using software such as the \Textsfc{LIBSVM} library~\cite{Chang2011}.  A variation
of the above approach, called $\nu$-SVR~\cite{Scholkopf2000} can also be used if sufficient
computational resources are available.

The bulk modulus can be computed from \eqref{eq:svr} using
\Beq
  \kappa(\BVeps) = \Partial{p}{\Veps^e} = \Partial{p}{\Veps}
   = \sum_{i=1}^{m_\text{SV}} (\lambda_i^\star - \lambda_i) \Partial{K_i}{\Veps} ~,~~ K_i := K(\BVeps_i, \BVeps) 
\Eeq
From \eqref{eq:svr_rbf},
\Beq
  \Partial{K_i}{\Veps} = 2\gamma (\Veps_i - \Veps) \exp\left[-\gamma \Norm{\BVeps_i-\BVeps}{}^2\right]
\Eeq
Therefore, the bulk modulus is given by,
\Beq
  \kappa(\BVeps)  
   = \sum_{i=1}^{m_\text{SV}} 2\gamma (\lambda_i^\star - \lambda_i) (\Veps_i - \Veps) \exp\left[-\gamma \Norm{\BVeps_i-\BVeps}{}^2\right]
\Eeq
If we need to account for elastic-plastic coupling, we may also need the derivative
\Beq
  \Partial{\kappa}{\Veps^p} 
   = \sum_{i=1}^{m_\text{SV}} 4\gamma^2 (\lambda_i^\star - \lambda_i) (\Veps_i - \Veps) (\Veps^p_i - \Veps^p) \exp\left[-\gamma \Norm{\BVeps_i-\BVeps}{}^2\right]
\Eeq
The bulk modulus model is also associated with a shear modulus model that computes the value of $\mu$ using
a Poisson's ratio ($\nu$) based on the value of $\kappa$.  The \Vaango implementation can be accessed
in the tabular plasticity models, using the tag \Textsfc{\textless elastic\_moduli\_model type="support\_vector"\textgreater}.


