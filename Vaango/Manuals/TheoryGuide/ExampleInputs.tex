\chapter{Example Input Files}
  \section{Hypoelastic-plastic model}
  An example of the portion of an input file that specifies a copper body
  with a hypoelastic stress update, Johnson-Cook plasticity model,
  Johnson-Cook Damage Model and Mie-Gruneisen Equation of State is shown 
  below.
  \lstset{language=XML}
  \begin{lstlisting}
  <material>

    <include href="inputs/MPM/MaterialData/MaterialConstAnnCopper.xml"/>
    <constitutive_model type="hypoelastic_plastic">
      <tolerance>5.0e-10</tolerance>
      <include href="inputs/MPM/MaterialData/IsotropicElasticAnnCopper.xml"/>
      <include href="inputs/MPM/MaterialData/JohnsonCookPlasticAnnCopper.xml"/>
      <include href="inputs/MPM/MaterialData/JohnsonCookDamageAnnCopper.xml"/>
      <include href="inputs/MPM/MaterialData/MieGruneisenEOSAnnCopper.xml"/>
    </constitutive_model>

    <burn type = "null" />
    <velocity_field>1</velocity_field>

    <geom_object>
      <cylinder label = "Cylinder">
        <bottom>[0.0,0.0,0.0]</bottom>
        <top>[0.0,2.54e-2,0.0]</top>
        <radius>0.762e-2</radius>
      </cylinder>
      <res>[3,3,3]</res>
      <velocity>[0.0,-208.0,0.0]</velocity>
      <temperature>294</temperature>
    </geom_object>

  </material>
  \end{lstlisting}

  The general material constants for copper are in the file 
  \verb+MaterialConstAnnCopper.xml+.  The contents are shown below
  \lstset{language=XML}
  \begin{lstlisting}
  <?xml version='1.0' encoding='ISO-8859-1' ?>
  <Uintah_Include>
    <density>8930.0</density>
    <toughness>10.e6</toughness>
    <thermal_conductivity>1.0</thermal_conductivity>
    <specific_heat>383</specific_heat>
    <room_temp>294.0</room_temp>
    <melt_temp>1356.0</melt_temp>
  </Uintah_Include>
  \end{lstlisting}

  The elastic properties are in the file \verb+IsotropicElasticAnnCopper.xml+.
  The contents of this file are shown below.
  \lstset{language=XML}
  \begin{lstlisting}
  <?xml version='1.0' encoding='ISO-8859-1' ?>
  <Uintah_Include>
    <shear_modulus>45.45e9</shear_modulus>
    <bulk_modulus>136.35e9</bulk_modulus>
  </Uintah_Include>
  \end{lstlisting}
  
  The constants for the Johnson-Cook plasticity model are in the file
  \verb+JohnsonCookPlasticAnnCopper.xml+.  The contents of this file are
  shown below.
  \lstset{language=XML}
  \begin{lstlisting}
  <?xml version='1.0' encoding='ISO-8859-1' ?>
  <Uintah_Include>
    <plasticity_model type="johnson_cook">
      <A>89.6e6</A>
      <B>292.0e6</B>
      <C>0.025</C>
      <n>0.31</n>
      <m>1.09</m>
    </plasticity_model>
  </Uintah_Include>
  \end{lstlisting}

  The constants for the Johnson-Cook damage model are in the file
  \verb+JohnsonCookDamageAnnCopper.xml+.  The contents of this file are
  shown below.
  \lstset{language=XML}
  \begin{lstlisting}
  <?xml version='1.0' encoding='ISO-8859-1' ?>
  <Uintah_Include>
    <damage_model type="johnson_cook">
      <D1>0.54</D1>
      <D2>4.89</D2>
      <D3>-3.03</D3>
      <D4>0.014</D4>
      <D5>1.12</D5>
    </damage_model>
  </Uintah_Include>
  \end{lstlisting}

  The constants for the Mie-Gruneisen model (as implemented in the 
  Uintah-Vaango Computational Framework) are in the file
  \verb+MieGruneisenEOSAnnCopper.xml+.  The contents of this file are
  shown below.
  \lstset{language=XML}
  \begin{lstlisting}
  <?xml version='1.0' encoding='ISO-8859-1' ?>
  <Uintah_Include>
    <equation_of_state type="mie_gruneisen">
      <C_0>3940</C_0>
      <Gamma_0>2.02</Gamma_0>
      <S_alpha>1.489</S_alpha>
    </equation_of_state>
  </Uintah_Include>
  \end{lstlisting}

  As can be seen from the input file, any other plasticity model, damage
  model and equation of state can be used to replace the Johnson-Cook
  and Mie-Gruneisen models without any extra effort (provided the models
  have been implemented and the data exist).

  The material data can easily be taken from a material database or specified
  for a new material in an input file kept at a centralized location.  At this
  stage material data for a range of materials is kept in the directory
  \verb|.../Vaango/StandAlone/inputs/MPM/MaterialData|.

  \section{Elastic-plastic model}
  The \verb|<constitutive_model type="elastic_plastic">| model is more stable
  (and also more general)
  than the \verb|<constitutive_model type="hypoelastic_plastic">| model.  A
  sample input file for this model is shown below.
  \lstset{language=XML}
  \begin{lstlisting}
  <MPM>
    <do_grid_reset> false </do_grid_reset>
    <time_integrator>explicit</time_integrator>
    <boundary_traction_faces>[zminus,zplus]</boundary_traction_faces> 
    <dynamic>true</dynamic>
    <solver>simple</solver>
    <convergence_criteria_disp>1.e-10</convergence_criteria_disp>
    <convergence_criteria_energy>4.e-10</convergence_criteria_energy>
    <DoImplicitHeatConduction>true</DoImplicitHeatConduction>
    <interpolator>linear</interpolator>
    <minimum_particle_mass> 1.0e-8</minimum_particle_mass>
    <maximum_particle_velocity> 1.0e8</maximum_particle_velocity>
    <artificial_damping_coeff> 0.0 </artificial_damping_coeff>
    <artificial_viscosity> true </artificial_viscosity>
    <accumulate_strain_energy> true </accumulate_strain_energy>
    <use_load_curves> false </use_load_curves>
    <turn_on_adiabatic_heating>   false    </turn_on_adiabatic_heating>
    <do_contact_friction_heating> false    </do_contact_friction_heating>
    <create_new_particles>        false    </create_new_particles>
    <erosion algorithm = "none"/>
  </MPM>

  <MaterialProperties>
    <MPM>
      <material name = "OFHCCu">
        <density>              8930.0 </density>
        <thermal_conductivity> 386.0  </thermal_conductivity>
        <specific_heat>        414.0  </specific_heat>
        <room_temp>            294.0  </room_temp>
        <melt_temp>            1356.0 </melt_temp>
        <constitutive_model type="elastic_plastic">
          <isothermal>                    false   </isothermal>
          <tolerance>                     1.0e-12 </tolerance>
          <do_melting>                    false   </do_melting>
          <evolve_porosity>               false   </evolve_porosity>
          <evolve_damage>                 false   </evolve_damage>
          <check_TEPLA_failure_criterion> false   </check_TEPLA_failure_criterion>
          <check_max_stress_failure>      false   </check_max_stress_failure>
          <initial_material_temperature>  696.0   </initial_material_temperature>
          
          <shear_modulus>           46.0e9  </shear_modulus>
          <bulk_modulus>            129.0e9 </bulk_modulus>
          <coeff_thermal_expansion> 1.76e-5 </coeff_thermal_expansion>
          <taylor_quinney_coeff>    0.9     </taylor_quinney_coeff>
          <critical_stress>         129.0e9 </critical_stress>

          <equation_of_state type = "mie_gruneisen">
            <C_0>     3940  </C_0>
            <Gamma_0> 2.02  </Gamma_0>
            <S_alpha> 1.489 </S_alpha>
          </equation_of_state>
          
          <plasticity_model type="mts_model">
            <sigma_a>40.0e6</sigma_a>
            <mu_0>47.7e9</mu_0>
            <D>3.0e9</D>
            <T_0>180</T_0>
            <koverbcubed>0.823e6</koverbcubed>
            <g_0i>0.0</g_0i>
            <g_0e>1.6</g_0e>
            <edot_0i>0.0</edot_0i>
            <edot_0e>1.0e7</edot_0e>
            <p_i>0.0</p_i>
            <q_i>0.0</q_i>
            <p_e>0.666667</p_e>
            <q_e>1.0</q_e>
            <sigma_i>0.0</sigma_i>
            <a_0>2390.0e6</a_0>
            <a_1>12.0e6</a_1>
            <a_2>1.696e6</a_2>
            <a_3>0.0</a_3>
            <theta_IV>0.0</theta_IV>
            <alpha>2</alpha>
            <edot_es0>1.0e7</edot_es0>
            <g_0es>0.2625</g_0es>
            <sigma_es0>770.0e6</sigma_es0>
          </plasticity_model>

          <shear_modulus_model type="mts_shear">
            <mu_0>47.7e9</mu_0>
            <D>3.0e9</D>
            <T_0>180</T_0>
          </shear_modulus_model>

          <melting_temp_model type = "constant_Tm">
          </melting_temp_model>

          <yield_condition type = "vonMises">
          </yield_condition>

          <stability_check type = "none">
          </stability_check>

          <damage_model type = "hancock_mackenzie">
            <D0> 0.0001 </D0>
            <Dc> 0.7    </Dc>
          </damage_model>
          
          <compute_specfic_heat> false </compute_specfic_heat>
          <specific_heat_model type="constant_Cp">
          </specific_heat_model>
          
        </constitutive_model>
        <geom_object>
          <box label = "box">
            <min>[0.0,    0.0,    0.0]</min>
            <max>[1.0e-2, 1.0e-2, 1.0e-2]</max>
          </box>
          <res>[1,1,1]</res>
          <velocity>[0.0, 0.0, 0.0]</velocity>
          <temperature>696</temperature>
        </geom_object>
      </material>

    </MPM>
  </MaterialProperties>
  \end{lstlisting}

  \subsection{An exploding ring experiment}
  The follwing shows the complete input file for an expanding ring test.
  \lstset{language=XML}
  \begin{lstlisting}

<?xml version='1.0' encoding='ISO-8859-1' ?>
<Uintah_specification> 
<!--Please use a consistent set of units, (mks, cgs,...)-->
    <!-- First crack at the tuna can problem -->

   <Meta>
       <title>Pressurization of a container via burning w/o fracture</title>
   </Meta>&gt;    
   <SimulationComponent>
     <type> mpmice </type>
   </SimulationComponent>
    <!--____________________________________________________________________-->
    <!--   T  I  M  E     V  A  R  I  A  B  L  E  S                         -->
    <!--____________________________________________________________________-->
   <Time>
       <max_Timesteps>   99999          </max_Timesteps>>
       <maxTime>             2.00e-2    </maxTime>
       <initTime>            0.0        </initTime>
       <delt_min>            1.0e-12    </delt_min>
       <delt_max>            1.0        </delt_max>
       <delt_init>           2.1e-8     </delt_init>
       <timestep_multiplier> 0.5        </timestep_multiplier>
   </Time>    
    <!--____________________________________________________________________-->
    <!--   G  R  I  D     V  A  R  I  A  B  L  E  S                         -->
    <!--____________________________________________________________________-->
    <Grid>
    <BoundaryConditions>
      <Face side = "x-">
        <BCType id = "all" label = "Symmetric" var = "symmetry">
        </BCType>
      </Face>
      <Face side = "x+">
        <BCType id = "0" label = "Pressure" var = "Neumann">
                              <value>  0.0 </value>
        </BCType>
        <BCType id = "all" label = "Velocity" var = "Dirichlet">
                              <value> [0.,0.,0.] </value>
        </BCType>
        <BCType id = "all" label = "Temperature" var = "Neumann">
                              <value> 0.0  </value>
        </BCType>
        <BCType id = "all" label = "Density" var = "Neumann">
                              <value> 0.0  </value>
        </BCType>
      </Face>
      <Face side = "y-">
        <BCType id = "all" label = "Symmetric" var = "symmetry">
        </BCType>
      </Face>                  
      <Face side = "y+">
        <BCType id = "0" label = "Pressure" var = "Neumann">
                              <value> 0.0   </value>
        </BCType>
        <BCType id = "all" label = "Velocity" var = "Dirichlet">
                              <value> [0.,0.,0.] </value>
        </BCType>
        <BCType id = "all" label = "Temperature" var = "Neumann">
                              <value> 0.0  </value>
        </BCType>
        <BCType id = "all" label = "Density" var = "Neumann">
                              <value> 0.0  </value>
        </BCType>
      </Face>
      <Face side = "z-">
        <BCType id = "all" label = "Symmetric" var = "symmetry">
        </BCType>
      </Face>                  
      <Face side = "z+">
        <BCType id = "all" label = "Symmetric" var = "symmetry">
        </BCType>
      </Face>
    </BoundaryConditions>
       <Level>
           <Box label = "1">
              <lower>        [ -0.08636, -0.08636, -0.0016933] </lower>
              <upper>        [ 0.08636, 0.08636, 0.0016933]    </upper>
              <extraCells>   [1,1,1]    </extraCells>
              <patches>      [2,2,1]    </patches>
              <resolution>   [102, 102, 1]                 </resolution>
           </Box>
       </Level>
    </Grid>
   
    <!--____________________________________________________________________-->
    <!--   O  U  P  U  T     V  A  R  I  A  B  L  E  S                      -->
    <!--____________________________________________________________________-->
   <DataArchiver>
      <filebase>explodeRFull.uda</filebase>
      <outputTimestepInterval> 20 </outputTimestepInterval>
      <save label = "rho_CC"/>
      <save label = "press_CC"/>
      <save label = "temp_CC"/>
      <save label = "vol_frac_CC"/>
      <save label = "vel_CC"/>
      <save label = "g.mass"/>
      <save label = "p.x"/>
      <save label = "p.mass"/>
      <save label = "p.temperature"/>
      <save label = "p.porosity"/>
      <save label = "p.particleID"/>
      <save label = "p.velocity"/>
      <save label = "p.stress"/>
      <save label = "p.damage" material = "0"/>
      <save label = "p.plasticStrain" material = "0"/>
      <save label = "p.strainRate" material = "0"/>
      <save label = "g.stressFS"/>
      <save label = "delP_Dilatate"/>
      <save label = "delP_MassX"/>
      <save label = "p.localized"/>
      <checkpoint cycle = "2" timestepInterval = "20"/>
   </DataArchiver>

   <Debug>
   </Debug>
    <!--____________________________________________________________________-->
    <!--    I  C  E     P  A  R  A  M  E  T  E  R  S                        -->
    <!--____________________________________________________________________-->
    <CFD>
       <cfl>0.5</cfl>
       <CanAddICEMaterial>true</CanAddICEMaterial>
       <ICE>
        <advection type = "SecondOrder"/>
        <ClampSpecificVolume>true</ClampSpecificVolume>
      </ICE>        
    </CFD>

    <!--____________________________________________________________________-->
    <!--     P  H  Y  S  I  C  A  L     C  O  N  S  T  A  N  T  S           -->
    <!--____________________________________________________________________-->   
    <PhysicalConstants>
       <gravity>            [0,0,0]   </gravity>
       <reference_pressure> 101325.0  </reference_pressure>
    </PhysicalConstants>

    <MPM>
      <time_integrator>                explicit   </time_integrator>
      <nodes8or27>                     27         </nodes8or27>
      <minimum_particle_mass>          3.e-12     </minimum_particle_mass>
      <maximum_particle_velocity>      1.e3       </maximum_particle_velocity>
      <artificial_damping_coeff>       0.0        </artificial_damping_coeff>
      <artificial_viscosity>           true       </artificial_viscosity>
      <artificial_viscosity_coeff1>    0.07       </artificial_viscosity_coeff1>
      <artificial_viscosity_coeff2>    1.6        </artificial_viscosity_coeff2>
      <turn_on_adiabatic_heating>      false      </turn_on_adiabatic_heating>
      <accumulate_strain_energy>       false      </accumulate_strain_energy>
      <use_load_curves>                false      </use_load_curves>
      <create_new_particles>           false      </create_new_particles>
      <manual_new_material>            false      </manual_new_material>
      <DoThermalExpansion>             false      </DoThermalExpansion>
      <testForNegTemps_mpm>            false      </testForNegTemps_mpm>
      <erosion algorithm = "ZeroStress"/>
    </MPM>

    <!--____________________________________________________________________-->
    <!--    MATERIAL PROPERTIES INITIAL CONDITIONS                          -->
    <!--____________________________________________________________________-->
    <MaterialProperties>
       <MPM>
         <material name = "Steel Ring">
           <include href="inputs/MPM/MaterialData/MatConst4340St.xml"/>
           <constitutive_model type="elastic_plastic">
             <isothermal>                    false </isothermal>
             <tolerance>                  1.0e-10  </tolerance>
             <evolve_porosity>               true  </evolve_porosity>
             <evolve_damage>                 true  </evolve_damage>
             <compute_specific_heat>         true  </compute_specific_heat>
             <do_melting>                    true  </do_melting>
             <useModifiedEOS>                true  </useModifiedEOS>
             <check_TEPLA_failure_criterion> true  </check_TEPLA_failure_criterion>
             <initial_material_temperature>  600.0 </initial_material_temperature>
             <taylor_quinney_coeff>          0.9   </taylor_quinney_coeff>
             <check_max_stress_failure>      false </check_max_stress_failure>
             <critical_stress>              12.0e9 </critical_stress>

             <!-- Warning: you must copy link this input file into your -->
             <!-- sus directory or these paths won't work.              -->

             <include href="inputs/MPM/MaterialData/IsoElastic4340St.xml"/>
             <include href="inputs/MPM/MaterialData/MieGrunEOS4340St.xml"/>
             <include href="inputs/MPM/MaterialData/ConstantShear.xml"/>
             <include href="inputs/MPM/MaterialData/ConstantTm.xml"/>
             <include href="inputs/MPM/MaterialData/JCPlastic4340St.xml"/>
             <include href="inputs/MPM/MaterialData/VonMisesYield.xml"/>
             <include href="inputs/MPM/MaterialData/DruckerBeckerStabilityCheck.xml"/>
             <include href="inputs/MPM/MaterialData/JCDamage4340St.xml"/>
             <specific_heat_model type="steel_Cp"> </specific_heat_model>

             <initial_mean_porosity>         0.005 </initial_mean_porosity>
             <initial_std_porosity>          0.001 </initial_std_porosity>
             <critical_porosity>             0.3   </critical_porosity>
             <frac_nucleation>               0.1   </frac_nucleation>
             <meanstrain_nucleation>         0.3   </meanstrain_nucleation>
             <stddevstrain_nucleation>       0.1   </stddevstrain_nucleation>
             <initial_porosity_distrib>      gauss </initial_porosity_distrib>

             <initial_mean_scalar_damage>    0.005  </initial_mean_scalar_damage>
             <initial_std_scalar_damage>     0.001 </initial_std_scalar_damage>
             <critical_scalar_damage>        1.0   </critical_scalar_damage>
             <initial_scalar_damage_distrib> gauss </initial_scalar_damage_distrib>
           </constitutive_model>                                      
       
                <geom_object>
                   <difference>
                    <cylinder label = "outer cylinder">
                     <bottom>           [0.0,0.0,-.05715]   </bottom>
                     <top>              [0.0,0.0, .05715]   </top>
                     <radius>           0.05715            </radius>
                   </cylinder>
                   <cylinder label = "inner cylinder">
                     <bottom>           [0.0,0.0,-.0508]   </bottom>
                     <top>              [0.0,0.0, .0508]   </top>
                     <radius>           0.0508             </radius>
                   </cylinder>
                   </difference>
                 <res>                 [2,2,2]         </res>
                 <velocity>            [0.0,0.0,0.0]   </velocity>
                 <temperature>         600             </temperature>
                </geom_object>
         </material>
         <material name = "reactant">
             <include href="inputs/MPM/MaterialData/MatConstPBX9501.xml"/>
             <constitutive_model type = "visco_scram">
               <include href="inputs/MPM/MaterialData/ViscoSCRAMPBX9501.xml"/>
               <include href="inputs/MPM/MaterialData/TimeTempPBX9501.xml"/>
               <randomize_parameters>          false </randomize_parameters>
               <use_time_temperature_equation> true  </use_time_temperature_equation>
               <useObjectiveRate>              true  </useObjectiveRate>
               <useModifiedEOS>                true  </useModifiedEOS>
             </constitutive_model>
                <geom_object>
                   <difference>
                     <cylinder label = "inner cylinder"> </cylinder>
                     <cylinder label = "inner hole">
                       <bottom>           [0.0,0.0,-.0508]   </bottom>
                       <top>              [0.0,0.0, .0508]   </top>
                       <radius>           0.01             </radius>
                     </cylinder>
                  </difference>
                  <res>                 [2,2,2]         </res>
                  <velocity>            [0.0,0.0,0.0]   </velocity>
                  <temperature>         440.0           </temperature>
                </geom_object>
         </material>

            <contact>
              <type>approach</type>
              <materials>              [0,1]         </materials>
              <mu> 0.0 </mu>
            </contact>
            <thermal_contact>
            </thermal_contact>
      </MPM>

       <ICE>
         <material>
           <EOS type = "ideal_gas">
           </EOS>
           <dynamic_viscosity>          0.0            </dynamic_viscosity>
           <thermal_conductivity>       0.0            </thermal_conductivity>
           <specific_heat>              716.0          </specific_heat>
           <gamma>                      1.4            </gamma>
           <geom_object>
             <difference>
                  <box>
                    <min>           [-0.254,-0.254,-0.254] </min>
                    <max>           [ 0.254, 0.254, 0.254] </max>
                  </box>
                 <cylinder label = "outer cylinder"> </cylinder>
               </difference>
             <cylinder label="inner hole"> </cylinder>
             <res>                      [2,2,2]        </res>
             <velocity>           [0.0,0.0,0.0]       </velocity>
             <!--
             <temperature>        300.0               </temperature>
             <density>    1.1792946927374306000e+00   </density>
             -->
             <temperature>        400.0               </temperature>
             <density>            0.884471019553073   </density>
             <pressure>           101325.0             </pressure>
           </geom_object>
         </material>
      </ICE>       

      <exchange_properties>  
         <exchange_coefficients>
              <momentum>  [0, 1e15, 1e15]     </momentum>
              <heat>      [0, 1e10, 1e10]     </heat>
        </exchange_coefficients>
      </exchange_properties> 
    </MaterialProperties>

    <AddMaterialProperties>  
      <ICE>
         <material name = "product">
           <EOS type = "ideal_gas">
           </EOS>
           <dynamic_viscosity>          0.0            </dynamic_viscosity>
           <thermal_conductivity>       0.0            </thermal_conductivity>
           <specific_heat>              716.0          </specific_heat>
           <gamma>                      1.4            </gamma>
           <geom_object>
                  <box>
                    <min>           [ 1.0, 1.0, 1.0] </min>
                    <max>           [ 2.0, 2.0, 2.0] </max>
                  </box>
             <res>                      [2,2,2]        </res>
             <velocity>           [0.0,0.0,0.0]       </velocity>
             <temperature>        300.0               </temperature>
             <density>    1.1792946927374306000e+00   </density>
             <pressure>           101325.0             </pressure>
           </geom_object>
         </material>
      </ICE>

      <exchange_properties>
         <exchange_coefficients>
              <momentum>  [0, 1e15, 1e15, 1e15, 1e15, 1e15]     </momentum>
              <heat>      [0, 1e10, 1e10, 1e10, 1e10, 1e10]     </heat>
              <!--
              <heat>      [0, 1, 1, 1, 1, 1]     </heat>
              -->
        </exchange_coefficients>
      </exchange_properties>
    </AddMaterialProperties>


    <Models>
      <Model type="Simple_Burn">
        <Active>       false        </Active>
        <fromMaterial> reactant     </fromMaterial>
        <toMaterial>   product      </toMaterial>
        <ThresholdTemp>       450.0 </ThresholdTemp>
        <ThresholdPressure> 50000.0 </ThresholdPressure>
        <Enthalpy>        2000000.0 </Enthalpy>
        <BurnCoeff>            75.3 </BurnCoeff>
        <refPressure>      101325.0 </refPressure>
      </Model>
    </Models>


</Uintah_specification>
  \end{lstlisting}

The PBS script used to run this test  is
\begin{lstlisting}

#
# ASK PBS TO SEND YOU AN EMAIL ON CERTAIN EVENTS: (a)bort (b)egin (e)nd (n)ever
#
# (User May Change)

#PBS -m abe

#
# SET THE NAME OF THE JOB:
#
# (User May Change)

#PBS -N ExplodeRing

#
# SET THE QUEUE IN WHICH TO RUN THE JOB.
#      (Note, there is currently only one queue, so you should never change this field.)

#PBS -q defaultq

#
# SET THE RESOURCES (# NODES, TIME) REQESTED FROM THE BATCH SCHEDULER:
#   - select: <# nodes>,ncpus=2,walltime=<time>
#     - walltime: walltime before PBS kills our job.
#               [[hours:]minutes:]seconds[.milliseconds]
#               Examples:
#                 walltime=60      (60 seconds)
#                 walltime=10:00   (10 minutes)
#                 walltime=5:00:00 (5 hours)
#
# (User May Change)

#PBS -l select=2:ncpus=2,walltime=24:00

#
# START UP LAM

cd $PBS_O_WORKDIR
lamboot

# [place your command here] >& ${PBS_O_WORKDIR}/output.${PBS_JOBID}
mpirun -np 4 ../sus_opt explodeRFull.ups >& output.${PBS_JOBID}

#
# REMEMBER, IF YOU ARE RUNNING TWO SERIAL JOBS, YOU NEED A:
# wait

#
# STOP LAM

lamhalt -v
exit
\end{lstlisting}

%  \bibliographystyle{unsrt}
%  \bibliography{Bibliography}

%\end{document}

