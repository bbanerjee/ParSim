\chapter{Fluid material models} \label{ch:ICEMaterials}
ICE use standard Newtonian fluid models for the simulation of fluids.  However,
since it was designed for shock-compression applications, we focus on some
of the high energy material models that are used by ICE in \Vaango.

\section{High Energy Material Reaction Models}\label{sec:HEReaction}

Two types of High Energy (HE) reaction models were considered here.  The first 
is a model for detonation, in which the reaction front proceeds as a shock 
wave through the solid reactant, leaving highly pressurized product gases 
behind the shock.  The second is a deflagration model, in which the 
reaction proceeds more slowly through the reactant in the form of a thermal
burn.  Each is described here.

\subsection{The JWL++ Detonation Model}\label{sec:JWLPP}

The detonation model used in two of the calculations discussed in 
Section~\ref{sec:numerical_results} is a reactive flow model known as 
JWL++\cite{JWLpp}.  JWL++ consists of equations of state for the
reactant and the products of reaction as well as a rate equation governing 
the transformation from product to reactant.  In addition, the model consists 
of a ``mixer" which is a rule for determining the pressure in a mixture of 
product and reactant, as found in a partially reacted cell.  Because pressure 
equilibration among materials is already part of the  multi-material CFD 
formulation described in Section~\ref{sec:numerical_algorithm}, the mixer was 
not part of the current implementation.  Lastly, two additional rules 
apply.  The first is that reaction begins in a cell when the pressure in that
cell exceeds 200 MPa.  Finally, no more than 20\% of the explosive in a cell 
is allowed to react in a given timestep.

The Murnaghan equation of state~\cite{Murnaghan1944} used for the solid reactant 
material is given by:
\Beq
  p= \frac{1}{n \kappa}\left(\frac{1}{v^n}-1\right)
  \label{Murnaghan}
\Eeq
where $v=\rho_0/\rho$, and $n$ and $\kappa$ are material dependent model 
parameters.  Note that while the reactants are solid materials, they are 
assumed to not support deviatoric stress.  Since a detonation propagates 
faster than shear waves, the strength in shear of the reactants can be 
neglected.  Since it is not necessary to track the deformation history 
of a particular material element, in this case, the reactant material was 
tracked only in the Eulerian frame, \textit{i.e.} not represented by 
particles within MPM.

The JWL C-term form is the equation of state used for products, and is 
given by:
\Beq
  P=A \exp(-R_1 v) + B \exp(-R_2 v) + \cfrac{C}{\rho_0 \kappa v^{n-1}}
  \label{JWLC}
\Eeq
where $A$, $B$, $C$, $R_1$, $R_2$, $\rho_0$ and $\kappa$ are all 
material dependent model parameters.

The rate equation governing the transformation of reactant to product is 
given by:
\Beq
  \Deriv{F}{t}=G(p+q)^b(1-F)
  \label{JWL++rate}
\Eeq
where $G$ is a rate constant, and 
$b$ indicates the power dependence on pressure.  $q$ is an artificial 
viscosity, but was not included in the current implementation of
the model.  Lastly:
\Beq
  F= \frac{\rho_{\Tproduct}}{\rho_{\Treactant}+\rho_{\Tproduct}}
\Eeq
is the burn fraction in a cell.  This can be differentiated and solved for 
a mass burn rate in terms of $dF$:
\Beq
  \Gamma = \Deriv{F}{t} \left(\rho_{\Treactant}+\rho_{\Tproduct}\right)
\Eeq

\subsection{Deflagration Model}\label{sec:deflagration}

The rate of thermal burning, or deflagration, of a monopropellant 
solid explosive is typically assumed to behave as:
\Beq
  D=Ap^n
  \label{deflagration_rate}
\Eeq
where $D$ can be thought of as the velocity at which the burn front 
propagates through the reactant (with units of length/time) and $p$ is the 
local pressure~\cite{Son2000}.  $A$ and $n$ are parameters that are 
empirically determined for particular explosives.  
Because deflagration is a 
surface phenomenon, our implementation requires the identification of the 
surface of the explosive.  The surface is assumed to lie within those cells 
which have the highest gradient of mass density of the reactant 
material.  Within each surface cell, an estimate of the surface area $a$
is made based on the direction of the gradient, and the rate $D$ above 
is converted to a mass burn rate by:
\Beq
  \Gamma=a D \rho_{\Treactant}
  \label{deflagration_mass_rate}
\Eeq
where $\rho_{\Treactant}$ is the local density of the explosive.  While the reaction 
rate is independent of temperature, initiation of the burn depends 
on reaching a threshold temperature at the surface.

Since the rate at which a deflagration propagates is much slower than the 
shear wave speed in the reactant, it is important to track its deformation 
as pressure builds up within the container.  This deformation may lead to 
the formation of more surface area upon which the reaction can take place, and 
the change to the shape of the explosive can affect the eventual violence of the
explosion.  Because of this, for deflagration cases, the explosive is 
represented by particles in the Lagrangian frame.  The stress response is usually
treated by an implementation of \Textsfc{ViscoSCRAM} \cite{Hackett2000Viscoscram}, which includes 
representation of the material's viscoelastic response, and considers effects 
of micro-crack growth within the granular composite material.

