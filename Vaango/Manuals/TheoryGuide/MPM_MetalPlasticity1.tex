\chapter{Isotropic metal plasticity}

This section describes the current implementation of the hypoelastic-
plastic model.  The stress update algorithm is a slightly modified 
version of the approach taken by 
Nemat-Nasser et al. (1991,1992)~\cite{Nemat1991,Nemat1992}, Wang (1994)~\cite{Wang1994}, 
Maudlin (1996)~\cite{Maudlin1996}, and Zocher et al. (2000)~\cite{Zocher2000}.  

\subsection{Simplified theory for hypoelastic-plasticity}
A simplified version of the theory behind the stress update algorithm
(in the contex of von Mises plasticity) is given below.

  Following ~\cite{Maudlin1996}, the rotated spatial rate of deformation 
  tensor ($\Bd$) is decomposed into an elastic part ($\Bd^e$) and a 
  plastic part ($\Bd^p$)
  \begin{equation}
     \Bd = \Bd^e + \Bd^p
  \end{equation}
  If we assume plastic incompressibility ($\Tr{(\Bd^p)} = 0$), we get
  \begin{equation}
     \Beta = \Beta^e + \Beta^p
  \end{equation}
  where $\Beta$, $\Beta^e$, and $\Beta^p$ are the deviatoric parts of $\Bd$,
  $\Bd^e$, and $\Bd^p$, respectively.  For isotropic materials, the hypoelastic
  constitutive equation for deviatoric stress is
  \begin{equation}
    \dot{\Bs} = 2\mu(\Beta - \Beta^p)
  \end{equation}
  where $\Bs$ is the deviatoric part of the stress tensor and $\mu$ is the
  shear modulus.  We assume that the flow stress obeys the Huber-von Mises
  yield condition
  \begin{equation}
    f := \sqrt{\frac{3}{2}}\norm{\Bs} - \sigma_y \le 0  ~~\text{or,}~~
    F := \frac{3}{2} \Bs:\Bs - \sigma_y^2 \le 0 
  \end{equation}
  where $\sigma_y$ is the flow stress.  Assuming an associated flow rule,
  and noting that $\Bd^p = \Beta^p$, we have
  \begin{equation}
    \Beta^p = \Bd^p = \lambda\Partial{f}{\Bsig} 
                    = \Lambda\Partial{F}{\Bsig} = 3\Lambda\Bs
  \end{equation}
  where $\Bsig$ is the stress.  Let $\Bu$ be a tensor proportional to the 
  plastic straining direction, and define $\gamma$ as
  \begin{equation}
    \Bu = \sqrt{3} \frac{\Bs}{\norm{\Bs}}; \quad
    \gamma := \sqrt{3}\Lambda\norm{\Bs}  \quad \Longrightarrow
    \gamma\Bu = 3\Lambda\Bs
  \end{equation}
  Therefore, we have
  \begin{equation} \label{eq:stresseqn}
    \Beta^p = \gamma\Bu; \quad  
    \dot{\Bs} = 2\mu(\Beta - \gamma\Bu)
  \end{equation}
  From the consistency condition, if we assume that the deviatoric stress
  remains constant over a timestep, we get 
  \begin{equation}
    \gamma = \frac{\Bs:\Beta}{\Bs:\Bu}
  \end{equation}
  which provides an initial estimate of the plastic strain-rate.  To obtain
  a semi-implicit update of the stress using equation (\ref{eq:stresseqn}), we
  define
  \begin{equation}\label{eq:taueqn}
    \tau^2 := \frac{3}{2} \Bs:\Bs = \sigma_y^2
  \end{equation}
  Taking a time derivative of equation (\ref{eq:taueqn}) gives us
  \begin{equation}\label{eq:taudot}
    \sqrt{2} \dot{\tau} = \sqrt{3} \frac{\Bs:\dot{\Bs}}{\norm{\Bs}}
  \end{equation}
  Plugging equation (\ref{eq:taudot}) into equation (\ref{eq:stresseqn})$_2$
  we get
  \begin{equation}\label{eq:tau}
    \dot{\tau} = \sqrt{2}\mu(\Bu:\Beta - \gamma\Bu:\Bu)
               = \sqrt{2} \mu (d - 3\gamma)
  \end{equation} 
  where $d = \Bu:\Beta$.  If the initial estimate of the plastic strain-rate
  is that all of the deviatoric strain-rate is plastic, then we get an 
  approximation to $\gamma$, and the corresponding error 
  ($\gamma_{\text{er}}$) given by
  \begin{equation}\label{eq:gammaer}
    \gamma_{\text{approx}} = \frac{d}{3}; \quad
    \gamma_{\text{er}} = \gamma_{\text{approx}} - \gamma = \frac{d}{3} - \gamma
  \end{equation}
  The incremental form of the above equation is
  \begin{equation}\label{eq:delgamma}
    \Delta\gamma = \frac{d^*\Delta t}{3} - \Delta\gamma_{\text{er}}
  \end{equation}
  Integrating equation (\ref{eq:tau}) from time $t_n$ to time $t_{n+1} = 
  t_n + \Delta t$, and using equation (\ref{eq:delgamma}) we get
  \begin{equation}\label{eq:taun}
    \tau_{n+1} = \tau_n + \sqrt{2}\mu(d^*\Delta t - 3\Delta\gamma)
               = \tau_n + 3\sqrt{2}\mu\Delta\gamma_{\text{er}}
  \end{equation}
  where $d^*$ is the average value of $d$ over the timestep.
  Solving for $\Delta\gamma_{\text{er}}$ gives
  \begin{equation}\label{eq:delgammaer}
    \Delta\gamma_{\text{er}} = \cfrac{\tau_{n+1} - \tau_n}{3\sqrt{2}\mu}
      = \cfrac{\sqrt{2}\sigma_y - \sqrt{3}\norm{\Bs_n}}{6\mu}
  \end{equation}
  The direction of the total strain-rate ($\Bu^{\eta}$) and the
  direction of the plastic strain-rate ($\Bu^s$) are given by 
  \begin{equation}
    \Bu^{\eta} = \frac{\Beta}{\norm{\Beta}} ; \quad
    \Bu^{s} = \frac{\Bs}{\norm{\Bs}} 
  \end{equation}
  Let $\theta$ be the fraction of the time increment that sees elastic
  straining.  Then
  \begin{equation}\label{eq:theta}
    \theta = \frac{d^* - 3\gamma_n}{d^*}
  \end{equation}
  where $\gamma_n = d_n/3$ is the value of $\gamma$ at the beginning of the 
  timestep.  We also assume that 
  \begin{equation}\label{eq:dstar}
    d^* = \sqrt{3}\Beta:[(1-\theta)\Bu^{\eta} + \frac{\theta}{2}
              (\Bu^{\eta}+\Bu^{s})]
  \end{equation}
  Plugging equation (\ref{eq:theta}) into equation (\ref{eq:dstar}) we get
  a quadratic equation that can be solved for $d^*$ as follows
  \begin{equation}
    \frac{2}{\sqrt{3}} (d^*)^2 - (\Beta:\Bu^s + \norm{\Beta}) d^*
       + 3\gamma_n (\Beta:\Bu^s - \norm{\Beta}) = 0
  \end{equation}
  The real positive root of the above quadratic equation is taken as the
  estimate for $d$.  The value of $\Delta\gamma$ can now be calculated
  using equations (\ref{eq:delgamma}) and (\ref{eq:delgammaer}).  A
  semi-implicit estimate of the deviatoric stress can be obtained at this
  stage by integrating equation (\ref{eq:stresseqn})$_2$
  \begin{align}
    \tilde{\Bs}_{n+1} & = \Bs_n + 2\mu\left(\eta\Delta t  - \sqrt{3}\Delta\gamma
         \cfrac{\tilde{\Bs}_{n+1}}{\norm{\Bs_{n+1}}}\right) \\
     & = \Bs_n + 2\mu\left(\eta\Delta t  - \frac{3}{\sqrt{2}}\Delta\gamma
         \cfrac{\tilde{\Bs}_{n+1}}{\sigma_y}\right)
  \end{align}
  Solving for $\tilde{\Bs}_{n+1}$, we get
  \begin{equation}
    \tilde{\Bs}_{n+1} = \cfrac{\Bs_{n+1}^{\text{trial}}}
           {1 + 3\sqrt{2}\mu\cfrac{\Delta\gamma}{\sigma_y}}
  \end{equation}
  where $\Bs_{n+1}^{\text{trial}} = \Bs_n + 2\mu\Delta t\Beta$.
  A final radial return adjustment is used to move the stress to the yield
  surface 
  \begin{equation}
    \Bs_{n+1} = \sqrt{\frac{2}{3}}\sigma_y \cfrac{\tilde{\Bs}_{n+1}}
                {\norm{\tilde{\Bs}_{n+1}}}
  \end{equation}
  A pathological situation arises if $\gamma_n = \Bu_n:\Beta_n$ is less
  than or equal to zero or 
  $\Delta\gamma_{\text{er}} \ge \frac{d^*}{3} \Delta t $.  
  This can occur is the rate of plastic deformation
  is small compared to the rate of elastic deformation or if the timestep
  size is too small (see~\cite{Nemat1992}).  In such situations, we use a
  locally implicit stress update that uses Newton iterations (as
  discussed in \cite{Simo1998}, page 124) to compute $\tilde{\Bs}$.

\section{Models}
  Below are some of the strain-rate, strain, and temperature dependent models 
  for metals that are implemented in Vaango.


  \subsection{Porosity model}
  The evolution of porosity is calculated as the sum of the rate of growth 
  and the rate of nucleation~\cite{Ramaswamy1998a}.  The rate of growth of
  porosity and the void nucleation rate are given by the following equations
  ~\cite{Chu1980}
  \begin{align}
    \dot{f} &= \dot{f}_{\text{nucl}} + \dot{f}_{\text{grow}} \\
    \dot{f}_{\text{grow}} & = (1-f) \text{Tr}(\BD_p) \\
    \dot{f}_{\text{nucl}} & = \cfrac{f_n}{(s_n \sqrt{2\pi})}
            \exp\left[-\Half \cfrac{(\epsilon_p - \epsilon_n)^2}{s_n^2}\right]
            \dot{\epsilon}_p
  \end{align}
  where $\BD_p$ is the rate of plastic deformation tensor, $f_n$ is the volume 
  fraction of void nucleating particles , $\epsilon_n$ is the mean of the 
  distribution of nucleation strains, and $s_n$ is the standard 
  deviation of the distribution.

  The inputs tags for porosity are of the form
  \lstset{language=XML}
  \begin{lstlisting}
    <evolve_porosity> true </evolve_porosity>
    <initial_mean_porosity>         0.005 </initial_mean_porosity>
    <initial_std_porosity>          0.001 </initial_std_porosity>
    <critical_porosity>             0.3   </critical_porosity>
    <frac_nucleation>               0.1   </frac_nucleation>
    <meanstrain_nucleation>         0.3   </meanstrain_nucleation>
    <stddevstrain_nucleation>       0.1   </stddevstrain_nucleation>
    <initial_porosity_distrib>      gauss </initial_porosity_distrib>
  \end{lstlisting}

