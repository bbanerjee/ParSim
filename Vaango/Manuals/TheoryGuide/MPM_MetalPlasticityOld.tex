\chapter{Metal plasticity : traditional approach}

\section{Preamble}
Let $\boldsymbol{F}$ be the deformation gradient, $\boldsymbol{\sigma}$ be the 
Cauchy stress, and $\boldsymbol{d}$ be the rate of deformation tensor.  We
first decompose the deformation gradient into a stretch and a rotations using
$\BF = \BR\cdot\BU$.  The rotation $\BR$ is then used to rotate the stress and
the rate of deformation into the material configuration to give us
\Beq
  \widehat{\Bsig} = \BR^T\cdot\Bsig\cdot\BR ~;~~
  \widehat{\BdT} = \BR^T\cdot\BdT\cdot\BR 
\Eeq
This is equivalent to using a Green-Naghdi objective stress rate.  In the 
following all equations are with respect to the hatted quantities and we 
drop the hats for convenience.

Let us split the Cauchy stress into a mean (isotropic) and a deviatoric part
\Beq
  \Bsig = \Bsig_\Tiso + \Bsig_\Tdev = \sigma_m~\Bone + \Bsig_{\Tdev} ~;~~ \sigma_m = \Third~\Tr(\Bsig) ~.
\Eeq
Taking the time derivative gives us the rate-form of the stress
\Beq
  \dot{\Bsig} = \dot{\sigma}_m~\Bone + \dot{\Bsig}_{\Tdev} ~.
\Eeq
Similarly, we can split the rate of deformation (symmetric part of the spatial
velocity gradient) into volumetric ($d_v$) and isochoric distortional ($\BdT_{\Tdis}$) parts:
\Beq
  \BdT = \BdT_\Tvol + \BdT_\Tdis = \Third\,d_v\,\Bone + \BdT_\Tdis ~.
\Eeq
The rate of deformation is additively decomposed into an elastic ($\BdT^e$) and a plastic 
part ($\BdT^p$):
\[
  \BdT = \BdT^e + \BdT^p \,.
\]
We may also split $\BdT^e$ into volumetric ($d_v^e$) and isochoric ($\BdT_\Tdis^e$) parts as
\Beq
  \BdT^e = \Third d_v^e~\Bone + \BdT_\Tdis^e~;~~ d_v^e = \Tr(\BdT^e) \,.
\Eeq
Similarly, we can split $\BdT^p$ into a volumetric part ($d_v^p$) and a
trace-free part ($\BdT_\Tdis^p$) , i.e.,
\Beq
  \BdT^p = \Third~d_v^p~\Bone + \BdT_\Tdis^p ~;~~ d_v^p = \Tr(\BdT^p) \,.
\Eeq
Therefore,
\Beq
  d_v = d_v^e + d_v^p ~,~~ \BdT_\Tdis = \BdT_\Tdis^e + \BdT_\Tdis^p \,.
\Eeq

\section{Elastic relation}
We assume that the elastic response of the material is isotropic.  The constitutive 
relation for a hypoelastic material of grade 0 can be expressed as
\Beq
  \dot{\Bsig}^e = 
    \left[\lambda~\Tr(\BdT^e) - 3~\kappa~\alpha~\Deriv{}{t}(T-T_0)\right]~\Bone + 2~\mu~\BdT^e = 
    \left[\lambda~\Tr(\BdT^e) - 3~\kappa~\alpha~\dot{T}\right]~\Bone + 2~\mu~\BdT^e  
\Eeq
where $\BdT^e$ is the elastic part of the rate of deformation 
tensor, $\lambda, \mu$ are the Lame tangent moduli, $\kappa$ is the bulk modulus, $\alpha$
is the coefficient of thermal expansion, $T_0$ is the reference temperature, and $T$ is the
current temperature. In terms of the volumetric and deviatoric parts of the rate of deformation,
we can write
\Beq
  \dot{\Bsig}^e = \left[\left(\lambda + \cfrac{2}{3}~\mu\right)~d_v^e 
     - 3~\kappa~\alpha~\dot{T}\right]~\Bone + 2~\mu~\BdT_\Tdis^e  =
  \kappa~\left[d_v^e - 3~\alpha~\dot{T}\right]~\Bone + 2~\mu~\BdT_\Tdis^e  
\Eeq
Therefore, we have
\Beq
  \dot{\Bsig}^e_\Tdev = 2~\mu~\BdT_\Tdis^e ~.
\Eeq
and
\Beq
  \dot{\sigma}_m^e = \kappa~\left[d_v^e - 3~\alpha~\dot{T}\right] ~.
\Eeq
We will use a standard elastic-plastic stress update algorithm to integrate the rate 
equation for the deviatoric stress.  However, we will assume that the volumetric part 
of the Cauchy stress of the intact metal (excluding voids) can be computed using 
an equation of state.  Then the final
Cauchy stress will be given by
\Beq
  \Bsig^e = \left[\sigma_m^e(J^e) - 3~\kappa~\alpha~(T-T_0)\right]~\Bone + \Bsig^e_\Tdev ~;~~ 
  J^e = \det(\BF^e) \,; \kappa := J^e~\Deriv{\sigma_m^e(J^e)}{J^e}~.
\Eeq
We assume that the plastic part of the deformation of the intact metal is volume preserving.
For Gurson-type models, a separate algorithm is needed, and the use of the algorithm discussed
in this chapter will lead to a small error in the computed value of $\Bsig$.

The moduli $\kappa$ and $\mu$ may depend on the deformation gradient,
temperature, and strain rate.  Since the temperature and porosity depend on the
amount of plastic deformation, this implies that the elastic and plastic
responses are coupled.  However, in this algorithm we ignore that coupling with
the assumption that the errors will be small.  The consequence of this
assumption is that the moduli are computed only at the beginning of each time
step on the basis of the mean stress and temperature at that point in the
simulation.

\section{Flow rule}
We assume that the flow rule is given by
\Beq
  \BdT^p = \dot{\gamma}~\hat{\BM}
\Eeq
where $\hat{\BM}$ is a \textbf{unit} tensor ($\hat{\BM}:\hat{\BM} = 1$) in the direction of the plastic flow
rate, and $\dot{\gamma}$ is the plastic multiplier.
Then, using the flow rule, we have the volumetric part:
\[
  d_v^p = \Tr(\BdT^p) = \dot{\gamma}\,\Tr(\hat{\BM}) =: \dot{\gamma}\,M_v
\]
or,
\Beq
 \BdT_\Tvol^p = \Third\,\dot{\gamma}\,M_v\,\Bone =: \dot{\gamma} \BM_\Tiso\,.
\Eeq
In the same way, we can write the deviatoric part of the flow rule as
\Beq
  \BdT_\Tdis^p = \dot{\gamma}~\left(\hat{\BM} - \Third~\Tr(\hat{\BM})~\Bone\right)
          =: \dot{\gamma}\,\BM_\Tdev ~.
\Eeq
Note that $\BM_\Tdev$ is \textit{not} a unit tensor and an
appropriate normalizing factor is needed to make it a unit tensor.

\section{Hardening, porosity evolution, and plastic dissipation}
We could assume that the strain rate, temperature, and porosity can be fixed at the
beginning of a time step and consider only the evolution of plastic strain and
the back stress while calculating the current stress.  However, that causes
problems of convergence and it is probably more appropriate to simultaneously
consider all the ordinary differential equations that contain rates of change.

We assume that the equivalent plastic strain evolves according to the relation
\Beq
  \dot{\Ve^p} = \dot{\gamma}~h^{\alpha}
\Eeq

We also assume that the back stress evolves according to the relation
\Beq
  \dot{\Bbeta} = \dot{\gamma}~\Bh^{\beta}
\Eeq
where $\Bbeta$ is the back stress.  If $\Bbeta_\Tiso$ and $\Bbeta_\Tdev$ are the 
volumetric and deviatoric parts of $\Bbeta$, respectively, we can write
\Beq
  \dot{\Bbeta}_\Tiso = \dot{\gamma}~\Bh^{\beta}_\Tiso  \qquad
  \dot{\Bbeta}_\Tdev = \dot{\gamma}~\Bh^{\beta}_\Tdev ~.
\Eeq

The porosity $\phi$ is assumed to evolve according to the relation
\Beq
  \dot{\phi} = \dot{\gamma}~h^{\phi} ~.
\Eeq

The temperature change due to plastic dissipation is assumed to be
given by the rate equation
\Beq
  \dot{T} = \cfrac{\chi}{\rho~C_p}~\sigma_y~\dot{\Ve^p} = \dot{\gamma}\,h^T \,.
\Eeq

\section{Yield condition}
The yield condition is assumed to be of the form
\Beq
  f(\Bsig_\Tiso, \Bsig_\Tdev, \Bbeta_\Tiso, \Bbeta_\Tdev, \Ve^p, \phi, T, \dot{\Ve}, \dots) 
    = f(\Bxi_\Tiso, \Bxi_\Tdev, \Ve^p, \phi, T, \dot{\Ve}, \dots) = 0
\Eeq
where $\Bxi_\Tiso := \Bsig_\Tiso - \Bbeta_\Tiso$ and $\Bxi_\Tdev := \Bsig_\Tdev - \Bbeta_\Tdev$.
The Kuhn-Tucker loading-unloading conditions are
\Beq
  \dot{\gamma} \ge 0 ~;~~  f \le 0 ~;~~ \dot{\gamma}~f = 0
\Eeq
and the consistency condition is $\dot{f} = 0$.

%\section{Temperature increase due to plastic dissipation}
%The temperature is updated using
%\Beq
%  T_{n+1} = T_n + 
%   \cfrac{\chi_{n+1}~\Delta t}{\rho_{n+1}~C_p}~\sigma^{n+1}_y~\dot{\Ve^p}_{n+1} ~.
%\Eeq

\section{Continuum elastic-plastic tangent modulus}
To determine whether the material has undergone a loss of stability we need to compute
the acoustic tensor which needs the computation of the continuum elastic-plastic tangent
modulus.

Recall that
\Beq
  \Bsig = \Bsig_\Tiso + \Bsig_\Tdev = \sigma_m~\Bone + \Bsig_\Tdev \quad \implies \quad \dot{\Bsig} =
     \dot{\Bsig}_\Tiso + \dot{\Bsig}_\Tdev = \dot{\sigma}_m~\Bone + \dot{\Bsig}_\Tdev~.
\Eeq
We assume that 
\Beq
  \dot{\sigma}_m^e = J^e~\Partial{\sigma_m^e}{J^e}~\Tr(\BdT^e) = \kappa\,\Tr(\BdT^e) \qquad \Tand \quad
  \dot{\Bsig}^e_\Tdev = 2~\mu~\BdT_\Tdis^e ~.
\Eeq
The Kuhn-Tucker and consistency conditions require that the stress state is on the boundary
of the elastic domain and therefore is directly related to the elastic part of
the rate of deformation tensor.  However, we don't know the fraction of the
total deformation tensor that is elastic, and that is what we have to determine.
Therefore, the consistency condition can be expressed as
\Beq
  \dot{f}(\Bsig^e_\Tiso, \Bsig^e_\Tdev, \Bbeta_\Tiso, \Bbeta_\Tdev, \Ve^p, \phi, T, \dot{\Ve}, \dots) = 0 ~.
\Eeq
Keeping $\dot{\Ve}$ fixed over the time interval, we can use the chain rule
to get
\Beq
  \dot{f} = \Partial{f}{\Bsig^e_\Tiso}:\dot{\Bsig}^e_\Tiso +
            \Partial{f}{\Bsig^e_\Tdev}:\dot{\Bsig}^e_\Tdev +
            \Partial{f}{\Bbeta_\Tiso}:\dot{\Bbeta}_\Tiso + 
            \Partial{f}{\Bbeta_\Tdev}:\dot{\Bbeta}_\Tdev + 
            \Partial{f}{\Ve^p}~\dot{\Ve^p} + \Partial{f}{\phi}~\dot{\phi} 
            + \Partial{f}{T}~\dot{T} = 0~.
\Eeq
The associated rate equations are
\Beq
  \Bal
    \dot{\Bsig}^e_\Tiso & = \kappa~\BdT_\Tvol^e  = \kappa~(\BdT_\Tvol - \BdT_\Tvol^p)   
      = \kappa\,[\BdT_\Tvol - \dot{\gamma}~\BM_\Tiso] \\
    \dot{\Bsig}^e_\Tdev & = 2~\mu~\BdT_\Tdis^e = 2~\mu~(\BdT_\Tdis - \BdT_\Tdis^p) = 
      2\mu~[\BdT_\Tdis - \dot{\gamma}~\BM_\Tdev] \\
    \dot{\Bbeta}_\Tiso & = \dot{\gamma}~\Bh^{\beta}_\Tiso \\
    \dot{\Bbeta}_\Tdev & = \dot{\gamma}~\Bh^{\beta}_\Tdev \\
    \dot{\Ve^p} & = \dot{\gamma}~h^{\alpha} \\
    \dot{\phi} & = \dot{\gamma}~h^{\phi} \\
    \dot{T} & = \dot{\gamma}~h^{T} \\
  \Eal
\Eeq
Plugging these into the expression for $\dot{f}$ gives
\Beq
  \kappa\,\Partial{f}{\Bsig^e_\Tiso}: [\BdT_\Tvol - \dot{\gamma}~\BM_\Tiso] +
  2~\mu~\Partial{f}{\Bsig^e_\Tdev}:[\BdT_\Tdis - \dot{\gamma}~\BM_\Tdev] 
    + \dot{\gamma}~\Partial{f}{\Bbeta_\Tiso}:\Bh^{\beta}_\Tiso 
    + \dot{\gamma}~\Partial{f}{\Bbeta_\Tdev}:\Bh^{\beta}_\Tdev 
    + \dot{\gamma}~\Partial{f}{\Ve^p}~h^{\alpha} 
    + \dot{\gamma}~\Partial{f}{\phi}~h^{\phi}
    + \dot{\gamma}~\Partial{f}{T}~h^{T}  = 0
\Eeq
or,
\Beq
  \dot{\gamma} = \frac{\kappa\,\BF_{\sigma_v} : \BdT_\Tvol + 2\mu\, \BF_{\sigma_d} : \BdT_\Tdis}{H}
\Eeq
where 
\[
  \BF_{\sigma_v} := \Partial{f}{\Bsig^e_\Tiso} ~;~~
  \BF_{\sigma_d} := \Partial{f}{\Bsig^e_\Tdev} ~;~~
  H := \kappa\,\Partial{f}{\Bsig^e_\Tiso}:\BM_\Tiso + 2~\mu~\Partial{f}{\Bsig^e_\Tdev}:\BM_\Tdev 
     - \Partial{f}{\Bbeta_\Tiso}:\Bh^{\beta}_\Tiso
     - \Partial{f}{\Bbeta_\Tdev}:\Bh^{\beta}_\Tdev
     - \Partial{f}{\Ve^p}~h^{\alpha} - \Partial{f}{\phi}~h^{\phi}
     - \Partial{f}{T}~h^{T} \,.
\]
Plugging this expression for $\dot{\gamma}$ into the equations for
$\dot{\Bsig}_\Tiso$ and $\dot{\Bsig}_\Tdev$, we get
\Beq
  \dot{\Bsig}^e_\Tiso = \kappa~\left[\BdT_\Tvol - 
     \left(\frac{\kappa\,\BF_{\sigma_v} : \BdT_\Tvol + 2\mu\, \BF_{\sigma_d} : \BdT_\Tdis}{H} \right)~\BM_\Tiso\right] 
  \quad
  \dot{\Bsig}^e_\Tdev = 2~\mu~\left[\BdT_\Tdis - 
     \left(\frac{\kappa\,\BF_{\sigma_v} : \BdT_\Tvol + 2\mu\, \BF_{\sigma_d} : \BdT_\Tdis}{H} \right)~\BM_\Tdev\right]~.
\Eeq

\textit{These equations clearly indicate that the volumetric and deviatoric responses cannot be
uncoupled unless the yield function depends only on the deviatoric stress. Since that assumption
limits us to J2-plasticity and does not allow us to include Gurson-type models, we will not separate
out the deviatoric and volumetric components of stress (or volumetric/isochoric deformation rates) 
in this model even though plastic deformation in the metal matrix is volume-preserving.}

\subsection{J2 plasticity}
At this stage, note that a symmetric $\Bsig$ implies a symmetric $\Bsig_\Tdev$ and hence a 
symmetric $\BdT_\Tdis$.  Also, $\BM_\Tdev$ is symmetric, since we assume that the
the flow rule is associated, i.e.,
\Beq
  \hat{\BM} = \frac{\BN}{\Norm{\BN}{}} ~,~~ \BN := \Partial{f}{\Bsig} \,.
\Eeq  
Then we can write,
\Beq
  \BdT_\Tdis = \SfI^{4s}:\BdT_\Tdis \qquad \Tand \qquad \BM_\Tdev =
     \SfI^{4s}:\BM_\Tdev
\Eeq
where $\SfI^{4s}$ is the fourth-order symmetric identity tensor.  Also note that
if $\BA$, $\BC$, $\BD$ are second order tensors and $\SfB$ is a fourth order tensor,
then
\Beq
  (\BA:\SfB:\BC)~(\SfB:\BD) \equiv A_{ij}~B_{ijkl}~C_{kl}~B_{mnpq}~D_{pq}
     = (B_{mnpq}~D_{pq})~(A_{ij}~B_{ijkl})~C_{kl} \equiv [(\SfB:\BD)\otimes(\BA:\SfB)]:\BC ~.
\Eeq
Therefore, we have,
\Beq
  \dot{\Bsig}^e_\Tdev = 2~\mu~\left[\SfI^{4s}:\BdT_\Tdis - \left( 
    \cfrac{2~\mu~[\SfI^{4s}:\BM_\Tdev]\otimes[\BF_\sigma:\SfI^{4s}]}
    {H}\right):\BdT_\Tdis\right]~.
\Eeq
Also,
\Beq
  \SfI^{4s}:\BM_\Tdev = \BM_\Tdev \qquad \Tand \qquad
  \BF_\sigma:\SfI^{4s} = \BF_\sigma ~.
\Eeq
Hence we can write
\Beq
  \dot{\Bsig}^e_\Tdev = 2~\mu~\left[\SfI^{4s} - \left( 
    \cfrac{2~\mu~\BM_\Tdev\otimes\BF_\sigma}{H}\right)\right]:\BdT_\Tdis
\Eeq
or,
\Beq
  \Bal
   \dot{\Bsig}^e_\Tdev & = \SfB^{ep}:\BdT_\Tdis = \SfB^{ep}:\left[\BdT - \Third~\Tr(\BdT)~\Bone\right]
     = \SfB^{ep}:\BdT - \Third~\left[\SfB^{ep}:
          (\Bone\otimes\Bone)\right]:\BdT\\
     & =: \SfB^{ep}_\Tdev:\BdT
  \Eal
\Eeq
where
\Beq
  \SfB^{ep} := 
    2~\mu~\left[\SfI^{4s} - \left( 
    \cfrac{2~\mu~\BM_\Tdev\otimes\BF_\sigma}{H}\right)\right] ~,~~
     \SfB_\Tdev^{ep} := \SfB^{ep} - \Third~\left[\SfB^{ep}:
          (\Bone\otimes\Bone)\right]
\Eeq
For the mean stress component, we have
\[
  \Bal
  \dot{\Bsig}^e_\Tiso  & = \dot{\sigma}^e_m~\Bone
    = J^e~\Partial{\sigma_m^e}{J^e}~\left[\Tr(\BdT) -\dot{\gamma} \Tr(\hat{\BM})\right] ~\Bone \\
    & = \kappa ~\left[\Tr(\BdT) -\frac{2\mu\, \BF_\sigma : \BdT_\Tdis}{H}
           \Tr(\hat{\BM})\right] ~\Bone  \\
    & = \kappa ~\left[\Tr(\BdT) - \frac{2\mu\, \Tr(\Hat{\BM})}{H} \BF_\sigma : \left(\BdT -
           \Third\Tr(\BdT)\Bone\right)\right] ~\Bone  \\
    & = \kappa (\Bone\otimes\Bone):\BdT - \frac{2\mu\kappa\,
          \Tr(\Hat{\BM})}{H} \left[(\Bone \otimes \BF_\sigma) : \BdT -
            \Third \Tr(\BF_\sigma) (\Bone\otimes\Bone) : \BdT\right]  \\
    & =: \SfB_\Tiso^{ep} : \BdT
  \Eal
\]
where,
\[
  \SfB_\Tiso^{ep} = \kappa (\Bone\otimes\Bone) - \frac{2\mu\kappa\,
          \Tr(\Hat{\BM})}{H} \left[(\Bone \otimes \BF_\sigma) -
            \Third \Tr(\BF_\sigma) (\Bone\otimes\Bone)\right]  \,.
\]

Adding the volumetric and deviatoric  components gives
\Beq
  \Bal
  \dot{\Bsig}^e & = \dot{\Bsig}^e_\Tiso + \dot{\Bsig}^e_\Tdev \\
   & = \SfB_\Tiso^{ep} : \BdT  
       + \SfB_\Tdev^{ep}:\BdT =: \SfC^{ep}:\BdT \,.
  \Eal
\Eeq
The quantity $\SfC^{ep}$ is the continuum elastic-plastic tangent modulus.  We also use the 
continuum elastic-plastic tangent modulus in the implicit version of the code.  However,
for improved accuracy and faster convergence, an algorithmically  consistent tangent modulus 
should be used instead.  That tangent modulus can be calculated in the usual manner and 
is left for development and implementation as an additional feature in the future.

\subsection{J2-I1 plasticity}
For J2-I1 plasticity, there is little advantage to be gained if we split the stress into 
volumetric and deviatoric components.  Therefore, we use the total stress and the total rate
of deformation:
\[
  \dot{\Bsig}^e = \left(\kappa-\tfrac{2}{3}\mu\right)\,\Tr(\BdT^e)\,\Bone + 2\mu\,\BdT^e 
\]
and
\[
  \BdT^e = \BdT - \BdT^p = \BdT - \dot{\gamma}\,\hat{\BM} \,.
\]
Then,
\[
  \dot{f} = \Partial{f}{\Bsig^e}:\dot{\Bsig}^e +
            \Partial{f}{\Bbeta}:\dot{\Bbeta} + 
            \Partial{f}{\Ve^p}~\dot{\Ve^p} + \Partial{f}{\phi}~\dot{\phi} 
            + \Partial{f}{T}~\dot{T} = 0
\]
and we have
\[
  \dot{\gamma} = \frac{[(\kappa-\tfrac{2}{3}\mu)\,\Tr(\BF_\sigma)\,\Bone + 2\mu\BF_\sigma]:\BdT}{H}
\]
where
\[
  H = (\kappa-\tfrac{2}{3}\mu)\,\Tr(\BF_\sigma)\,\Tr(\hat{\BM}) + 2\mu\BF_\sigma:\hat{\BM}
     - \Partial{f}{\Bbeta}:\Bh^{\beta}
     - \Partial{f}{\Ve^p}~h^{\alpha} - \Partial{f}{\phi}~h^{\phi}
     - \Partial{f}{T}~h^{T} \,.
\]
Therefore,
\[
  \BdT^e =  \left[\SfI^{4s} - \frac{(\kappa-\tfrac{2}{3}\mu)\,\Tr(\BF_\sigma)\,\hat{\BM}\otimes\Bone +
      2\mu\hat{\BM}\otimes\BF_\sigma}{H}\right]:\BdT =: \SfD^e :\BdT 
\]
and
\[
  \dot{\Bsig}^e = \left[\left(\kappa-\tfrac{2}{3}\mu\right)\,[(\Bone\otimes\Bone):\SfD^e +
2\mu\,\SfI^{4s}\right]:\BdT  =: \SfC^{ep}:\BdT \,.
\]

\section{Stress update}
A standard return algorithm is used to compute the updated Cauchy stress.  

\subsection{Integrating the rate equations}
\subsubsection{J2 plasticity}
Recall that the rate equation for the deviatoric stress is given by
\Beq
  \dot{\Bsig}^e_\Tdev = 2~\mu~\BdT_\Tdis^e ~.
\Eeq
Integration of the rate equation using a Backward Euler scheme gives
\Beq
  (\Bsig^e_\Tdev)_{n+1} - (\Bsig^e_\Tdev)_n = 2~\mu_{n+1}~\Delta t~(\BdT_\Tdis^e)_{n+1}
    = 2~\mu_{n+1}~\Delta t~[(\BdT_\Tdis)_{n+1} - (\BdT_\Tdis^p)_{n+1}]
\Eeq
Now, from the flow rule, we have
\Beq
  \BdT_\Tdis^p = \dot{\gamma}~\left(\hat{\BM} -\Third~\Tr(\hat{\BM})~\Bone\right) 
    = \dot{\gamma} \BM_\Tdev~.
\Eeq
Identifying $\dot{\BVeps}_\Tdis^p = \BdT_\Tdis^p$, we can write the integrated strain as
\Beq
  (\BVeps_\Tdis^p)_{n+1} = (\BVeps_\Tdis^p)_{n} + \Delta\gamma_{n+1} (\BM_\Tdev)_{n+1}\,.
\Eeq
Therefore, 
\Beq
  (\Bsig^e_\Tdev)_{n+1} - (\Bsig^e_\Tdev)_n 
    = 2~\mu_{n+1}~\Delta t~(\BdT_\Tdis)_{n+1} - 2~\mu_{n+1}~\Delta\gamma_{n+1}~(\BM_\Tdev)_{n+1} ~.
\Eeq
where $\Delta\gamma := \dot{\gamma}~\Delta t$.  At this stage we ignore 
elastic-plastic coupling and assume that $\mu_{n+1} = \mu_n$.  Define the trial stress
\Beq
  \Bsig_\Tdev^{\Trial} := (\Bsig^e_\Tdev)_n + 2~\mu_n~\Delta t~(\BdT_\Tdis)_{n+1} ~.
\Eeq
Then
\Beq
  (\Bsig^e_\Tdev)_{n+1} = \Bsig_\Tdev^{\Trial} - 2~\mu_n~\Delta\gamma_{n+1}~(\BM_\Tdev)_{n+1} ~.
\Eeq
Also recall that the back stress is given by
\Beq
  \dot{\Bbeta}_\Tdev = \dot{\gamma}~\Bh_\Tdev^{\beta}
\Eeq
The evolution equation for the back stress can be integrated to get
\Beq
  (\Bbeta_\Tdev)_{n+1} - (\Bbeta_\Tdev)_n = \Delta\gamma_{n+1}~(\Bh_\Tdev^{\beta})_{n+1} ~.
\Eeq
Define
\Beq
  \Bxi_{n+1} := (\Bsig^e_\Tdev)_{n+1} - (\Bbeta_\Tdev)_{n+1} ~.
\Eeq
Plugging in the expressions for $(\Bsig^e_\Tdev)_{n+1}$ and $(\Bbeta_\Tdev)_{n+1}$, we get
\Beq
  \Bxi_{n+1} = \Bsig_\Tdev^{\Trial} - 2~\mu_n~\Delta\gamma_{n+1}~(\BM_\Tdev)_{n+1} 
     - (\Bbeta_\Tdev)_n - \Delta\gamma_{n+1}~(\Bh_\Tdev^{\beta})_{n+1} ~.
\Eeq
Define
\Beq
  \Bxi^{\Trial} := \Bsig_\Tdev^{\Trial} - (\Bbeta_\Tdev)_n ~.
\Eeq
Then
\Beq
  \Bxi_{n+1} = \Bxi^{\Trial} - \Delta\gamma_{n+1}\left[2~\mu_n~(\BM_\Tdev)_{n+1} +
     (\Bh_\Tdev^{\beta})_{n+1}\right] ~.
\Eeq
Similarly, the evolution of the plastic strain is given by
\Beq
  \Ve^p_{n+1} = \Ve^p_{n} + \Delta\gamma_{n+1}~h^{\alpha}_{n+1} \,.
\Eeq
The porosity evolves as
\Beq
  \phi_{n+1} = \phi_n + \Delta\gamma_{n+1}~h^{\phi}_{n+1} ~.
\Eeq
The temperature evolves as
\Beq
  T_{n+1} = T_n + \Delta\gamma_{n+1}~h^T_{n+1} ~.
\Eeq
The yield condition is discretized as
\Beq
  f((\Bsig_\Tdev)_{n+1}, (\Bbeta_\Tdev)_{n+1}, \Ve^p_{n+1}, \phi_{n+1}, T_{n+1}, \dot{\Ve}_{n+1}, \dots) = 
  f(\Bxi_{n+1}, \Ve^p_{n+1}, \phi_{n+1}, T_{n+1}, \dot{\Ve}_{n+1}, \dots) = 0 ~.
\Eeq

\subsubsection{J2-I1 plasticity}
In this case, we use the rate equation for the total stress is given by
\Beq
  \dot{\Bsig}^e = \left(k - \tfrac{2}{3}\mu\right) \BdT^e:\Bone + 2~\mu~\BdT^e ~.
\Eeq
Backward Euler gives
\Beq
  \Bal
  \Bsig^e_{n+1} - \Bsig^e_n & = 
    \Delta t\,\left(\kappa_{n+1} - \tfrac{2}{3}\mu_{n+1}\right) \BdT^e_{n+1}:\Bone +
      2~\Delta t\,\mu_{n+1}~\BdT^e_{n+1} \\
    & = \left(\kappa_{n+1} - \tfrac{2}{3}\mu_{n+1}\right) (\Delta t\,\BdT_{n+1}-\Delta t\,\BdT^p_{n+1}):\Bone +
      2\mu_{n+1}~(\Delta t\,\BdT_{n+1}-\Delta t\,\BdT^p_{n+1})
  \Eal
\Eeq
Flow rule:
\Beq
  \BdT^p = \dot{\gamma}~\hat{\BM}
\Eeq
Set $\dot{\BVeps}^p := \BdT^p$, to write the integrated strain:
\Beq
  \BVeps^p_{n+1} = \BVeps^p_{n} + \Delta\gamma_{n+1} \hat{\BM}_{n+1}
  \quad \leftrightarrow \quad \Delta t\,\BdT^p_{n+1} = \Delta\gamma_{n+1} \hat{\BM}_{n+1}\,.
\Eeq
Then, 
\Beq
  \Bsig^e_{n+1} = \Bsig^e_n  +
    \left(\kappa_{n+1} - \tfrac{2}{3}\mu_{n+1}\right) (\Delta
t\,\BdT_{n+1}-\Delta\gamma_{n+1}\,\hat{\BM}_{n+1}):\Bone +
      2\mu_{n+1}~(\Delta t\,\BdT_{n+1}-\Delta\gamma_{n+1}\,\hat{\BM}_{n+1}) 
\Eeq
where $\Delta\gamma := \dot{\gamma}~\Delta t$.  Ignore 
elastic-plastic coupling and assume $\kappa_{n+1} = \kappa_n$ and $\mu_{n+1} = \mu_n$.  The trial stress is
\Beq
  \Bsig^{\Trial} := \Bsig^e_n + \left(\kappa_n - \tfrac{2}{3}\mu_n\right)\Delta t\,\BdT_{n+1}:\Bone
    + 2~\mu_n~\Delta t~\BdT_{n+1} ~.
\Eeq
Then
\Beq
  \Bsig^e_{n+1} = \Bsig^{\Trial} - \left(\kappa_{n+1} - \tfrac{2}{3}\mu_{n+1}\right) 
    \Delta\gamma_{n+1}\,\hat{\BM}_{n+1}:\Bone -
      2\mu_{n+1} \,\Delta\gamma_{n+1}\,\hat{\BM}_{n+1}  \,.
\Eeq
For the back stress, 
\Beq
  \Bbeta_{n+1} - \Bbeta_n = \Delta\gamma_{n+1}~\Bh^{\beta}_{n+1} ~.
\Eeq
Define
\Beq
  \Bxi_{n+1} := \Bsig^e_{n+1} - \Bbeta_{n+1} 
\Eeq
to arrive at the combined stress equation
\Beq
  \Bxi_{n+1}  = \Bsig^{\Trial} - \left(\kappa_{n+1} - \tfrac{2}{3}\mu_{n+1}\right) 
    \Delta\gamma_{n+1}\,\hat{\BM}_{n+1}:\Bone -
      2\mu_{n+1} \,\Delta\gamma_{n+1}\,\hat{\BM}_{n+1}  
     - \Bbeta_n - \Delta\gamma_{n+1}~\Bh^{\beta}_{n+1} ~.
\Eeq
Define
\Beq
  \Bxi^{\Trial} := \Bsig^{\Trial} - \Bbeta_n 
\Eeq
to get
\Beq
  \Bxi_{n+1}  = \Bxi^{\Trial} - \Delta\gamma_{n+1}\left[\left(\kappa_{n+1} - \tfrac{2}{3}\mu_{n+1}\right) 
    \hat{\BM}_{n+1}:\Bone - 2\mu_{n+1} \,\hat{\BM}_{n+1}  - \Bh^{\beta}_{n+1}\right] ~.
\Eeq
The other rate equations are integrated as before:
\Beq
  \Bal
  \Ve^p_{n+1} & = \Ve^p_{n} + \Delta\gamma_{n+1}~h^{\alpha}_{n+1} \\
  \phi_{n+1} &= \phi_n + \Delta\gamma_{n+1}~h^{\phi}_{n+1} \\
  T_{n+1} &= T_n + \Delta\gamma_{n+1}~h^T_{n+1} ~.
  \Eal
\Eeq
The yield condition is discretized as
\Beq
  f(\Bsig^e_{n+1}, \Bbeta_{n+1}, \Ve^p_{n+1}, \phi_{n+1}, T_{n+1}, \dot{\Ve}_{n+1}, \dots) = 
  f(\Bxi_{n+1}, \Ve^p_{n+1}, \phi_{n+1}, T_{n+1}, \dot{\Ve}_{n+1}, \dots) = 0 ~.
\Eeq

\subsection{Newton iterations}
\subsubsection{J2 plasticity}
The following coupled equations have to be solved for the unknowns at $t_{n+1}$:
\Beq
  \Bal
  & (\Bsig^e_\Tdev)_{n+1} = \Bsig_\Tdev^{\Trial} - 2~\mu_n~\Delta\gamma_{n+1}~(\BM_\Tdev)_{n+1} \\
  & (\BVeps_\Tdis^p)_{n+1} = (\BVeps_\Tdis^p)_{n} + \Delta\gamma_{n+1} (\BM_\Tdev)_{n+1} \\
  & (\Bbeta_\Tdev)_{n+1} = (\Bbeta_\Tdev)_n + \Delta\gamma_{n+1}~(\Bh_\Tdev^{\beta})_{n+1} \\
  & \Ve^p_{n+1}  = \Ve^p_{n} + \Delta\gamma_{n+1}~h^{\alpha}_{n+1} \\
  & \phi_{n+1}  = \phi_n + \Delta\gamma_{n+1}~h^{\phi}_{n+1}  \\
  & T_{n+1}  = T_n + \Delta\gamma_{n+1}~h^{T}_{n+1}  \\
  & f_{n+1}(\Bxi_{n+1}, \Ve^p_{n+1}, \phi_{n+1}, T_{n+1}, \dot{\Ve}_{n+1}, \dots)  = 0 ~.
  \Eal
\Eeq
Note that, for associated plasticity, the first two equations depend on $f_{n+1}$.  
Recognizing that the second equation is alreday incorporated into the first, and combining the
stress and backstress equations, we can express the above equations in terms of residuals as,
\Beq
  \Bal
  \Br_\xi  &= \Bxi_{n+1} - \Bxi^{\Trial} + \Delta\gamma_{n+1}[2\mu_n\,(\BM_\Tdev)_{n+1} +
    (\Bh_\Tdev^{\beta})_{n+1}] = \Bzero \\
  r_{\Ve^p} & =  \Ve^p_{n+1}  - \Ve^p_{n} - \Delta\gamma_{n+1}~h^{\alpha}_{n+1} = 0 \\
  r_{\phi} & =  \phi_{n+1}  - \phi_n - \Delta\gamma_{n+1}~h^{\phi}_{n+1} = 0  \\
  r_T & =  T_{n+1}  - T_n - \Delta\gamma_{n+1}~h^{T}_{n+1} = 0  \\
  r_f & = f_{n+1}(\Bxi_{n+1}, \Ve^p_{n+1}, \phi_{n+1}, T_{n+1}, \dot{\Ve}_{n+1}, \dots)  = 0 ~.
  \Eal
\Eeq
or, if $\Bx = (\Bxi_{n+1}, \Ve^p_{n+1}, \phi_{n+1}, T_{n+1}, \Delta\gamma_{n+1})$ is 
the vector of \textit{independent} unknowns in the above equations, 
\Beq
  \Br(\Bx) = \Bzero \,.
\Eeq
An iterative Newton method can be used to find the solution:
\Beq
  \Bx^{(k+1)} = \Bx^{(k)} - \left[\Partial{\Br}{\Bx}\right]^{-1}_{(k)}~\Br^{(k)} 
  \quad \Tor \quad
  -\BJv^{k)} \, \delta\Bx^{(k)} = \Br^{(k)} \,.
\Eeq
To simplify the notation, we drop the subscripts $(n+1)$ and ``$\Tdev$'', and 
set $\Bx \equiv (\BsT, \BbT, \Ve^p, \phi, T, \Delta\gamma)$. 
Also, for metal plasticity, we assume that $\hat{\BM} = f_{\Bsig}/\Norm{f_{\Bsig}}{}$ where
$f_{\Bsig} := \Partial{f}{\Bsig}$.  Now, with the isotropic-deviatoric split, and since
$f$ does not depend on the isotropic component of stress, 
\[
  f_{\Bsig} = f_{\BsT}:\Partial{\BsT}{\Bsig} + f_{p} \Partial{p}{\Bsig}
   = f_{\BsT}:\Partial{\BsT}{\Bsig} = f_{\BsT} - \Third \Tr(f_{\BsT})\,\Bone = f_{\BsT}
\]
(with $p := \Bsig_\Tiso : \Bone = \Tr(\Bsig)$ and $\BsT := \Bsig_\Tdev = \Bsig - \Third\Tr(\Bsig)\Bone$).  
Therefore, 
\[
  \BM := \BM_\Tdev = \hat{\BM} \,.
\]

Let us now compute the derivatives.
\[
  \Bal
  \Partial{\Br_\xi}{\Bxi} &= \SfI^{4s} +
     \Delta\gamma\left[2\mu_n\,\Partial{\BM}{\Bxi} + \Partial{\Bh^\beta}{\Bxi}\right] \,,
  \Partial{\Br_\xi}{\Ve^p} = 
     \Delta\gamma\left[2\mu_n\,\Partial{\BM}{\Ve^p} + \Partial{\Bh^\beta}{\Ve^p}\right] \,,
  \Partial{\Br_\xi}{\phi} = 
     \Delta\gamma\left[2\mu_n\,\Partial{\BM}{\phi} + \Partial{\Bh^\beta}{\phi}\right] \\
  \Partial{\Br_\xi}{T} &= 
     \Delta\gamma\left[2\mu_n\,\Partial{\BM}{T} + \Partial{\Bh^\beta}{T}\right] \,,
  \Partial{\Br_\xi}{\Delta\gamma} = 2\mu_n\,\BM + \Bh^{\beta}
  \Eal
\]
\[
  \Bal
  \Partial{r_{\Ve^p}}{\Bxi} &=  -\Delta\gamma\,\Partial{h^\alpha}{\Bxi} \,,
  \Partial{r_{\Ve^p}}{\Ve^p} = 1 -\Delta\gamma\,\Partial{h^\alpha}{\Ve^p} \,,
  \Partial{r_{\Ve^p}}{\phi} = -\Delta\gamma\,\Partial{h^\alpha}{\phi} \,, 
  \Partial{r_{\Ve^p}}{T} = -\Delta\gamma\,\Partial{h^\alpha}{T} \,,
  \Partial{r_{\Ve^p}}{\Delta\gamma} = -h^\alpha
  \Eal
\]
\[
  \Bal
  \Partial{r_{\phi}}{\Bxi} &=  -\Delta\gamma\,\Partial{h^\phi}{\Bxi} \,,
  \Partial{r_{\phi}}{\Ve^p} = -\Delta\gamma\,\Partial{h^\phi}{\Ve^p} \,,
  \Partial{r_{\phi}}{\phi} = 1 -\Delta\gamma\,\Partial{h^\phi}{\phi} \,, 
  \Partial{r_{\phi}}{T} = -\Delta\gamma\,\Partial{h^\phi}{T} \,,
  \Partial{r_{\phi}}{\Delta\gamma} = -h^\phi
  \Eal
\]
\[
  \Bal
  \Partial{r_{T}}{\Bxi} &=  -\Delta\gamma\,\Partial{h^T}{\Bxi} \,,
  \Partial{r_{T}}{\Ve^p} = -\Delta\gamma\,\Partial{h^T}{\Ve^p} \,,
  \Partial{r_{T}}{\phi} = -\Delta\gamma\,\Partial{h^T}{\phi} \,, 
  \Partial{r_{T}}{T} = 1 -\Delta\gamma\,\Partial{h^T}{T} \,,
  \Partial{r_{T}}{\Delta\gamma} = -h^T
  \Eal
\]
\[
  \Bal
  \Partial{r_{f}}{\Bxi} &=  \Partial{f}{\Bxi} =: f_{\Bxi} \,,
  \Partial{r_{f}}{\Ve^p} = \Partial{f}{\Ve^p} =: f_{\Ve} \,,
  \Partial{r_{f}}{\phi} = \Partial{f}{\phi} =: f_{\phi} \,, 
  \Partial{r_{f}}{T} = \Partial{f}{T} =: f_{T} \,,
  \Partial{r_{f}}{\Delta\gamma} = 0
  \Eal
\]

The system of equations can be expressed as
\[
  -\begin{bmatrix}
    \Partial{\Br_\xi}{\Bxi} & \Partial{\Br_\xi}{\Ve^p} & 
      \Partial{\Br_\xi}{\phi} & \Partial{\Br_\xi}{T} & \Partial{\Br_\xi}{\Delta\gamma} \\
    \Partial{r_{\Ve^p}}{\Bxi} & \Partial{r_{\Ve^p}}{\Ve^p} & 
      \Partial{r_{\Ve^p}}{\phi} & \Partial{r_{\Ve^p}}{T} & \Partial{r_{\Ve^p}}{\Delta\gamma} \\
    \Partial{r_{\phi}}{\Bxi} & \Partial{r_{\phi}}{\Ve^p} & 
      \Partial{r_{\phi}}{\phi} & \Partial{r_{\phi}}{T} & \Partial{r_\phi}{\Delta\gamma} \\
    \Partial{r_{T}}{\Bxi} & \Partial{r_{T}}{\Ve^p} & 
      \Partial{r_{T}}{\phi} & \Partial{r_{T}}{T} & \Partial{r_T}{\Delta\gamma} \\
    f_{\Bxi} & f_{\Ve} & f_\phi & f_T & 0 
  \end{bmatrix}^{(k)}
  \begin{bmatrix}
    \delta\Bxi \\ \delta\Ve^p \\ \delta\phi \\ \delta T \\ \delta\Delta\gamma  
  \end{bmatrix}^{(k)} = 
  \begin{bmatrix}
    \Br_\xi \\ r_{\Ve^p} \\ r_\phi \\ r_T \\ r_f  
  \end{bmatrix}^{(k)}  
\]

Define 
\Beq
  \delta\Bx := \Bx^{(k+1)} - \Bx^{(k)} ~.
\Eeq
Therefore,
\Beq
  \Bal
  \Deriv{\Ba}{\Delta\gamma} & = 
   -\Partial{\Bxi}{\Delta\gamma}  - (2~\mu~\BM_\Tdev + \Bh_\Tdev^{\beta})
   - \Delta\gamma~\left(2~\mu~\Partial{\BM_\Tdev}{\Delta\gamma} + 
        \Partial{\Bh_\Tdev^{\beta}}{\Delta\gamma}\right) \\
   & =
   -\Partial{\Bxi}{\Delta\gamma}  - (2~\mu~\BM_\Tdev + \Bh_\Tdev^{\beta})
   - \Delta\gamma~\left(
      2~\mu~\Partial{\BM_\Tdev}{\Bxi}:\Partial{\Bxi}{\Delta\gamma} + 
      2~\mu~\Partial{\BM_\Tdev}{\Ve^p}~\Partial{\Ve^p}{\Delta\gamma} + 
      2~\mu~\Partial{\BM_\Tdev}{\phi}~\Partial{\phi}{\Delta\gamma} + 
      \right. \\
   & \qquad \qquad
      \left.
      \Partial{\Bh_\Tdev^{\beta}}{\Bxi}:\Partial{\Bxi}{\Delta\gamma} + 
      \Partial{\Bh_\Tdev^{\beta}}{\Ve^p}~\Partial{\Ve^p}{\Delta\gamma} +
      \Partial{\Bh_\Tdev^{\beta}}{\phi}~\Partial{\phi}{\Delta\gamma} 
      \right) \\
  \Deriv{b}{\Delta\gamma} & = -\Partial{\Ve^p}{\Delta\gamma} +  h^{\alpha} 
    + \Delta\gamma~\left(\Partial{h^{\alpha}}{\Bxi}:\Partial{\Bxi}{\Delta\gamma} + 
                        \Partial{h^{\alpha}}{\Ve^p}~\Partial{\Ve^p}{\Delta\gamma} + 
                        \Partial{h^{\alpha}}{\phi}~\Partial{\phi}{\Delta\gamma}\right) \\
  \Deriv{c}{\Delta\gamma} & = -\Partial{\phi}{\Delta\gamma} +  h^{\phi} 
    + \Delta\gamma~\left(\Partial{h^{\phi}}{\Bxi}:\Partial{\Bxi}{\Delta\gamma} + 
                        \Partial{h^{\phi}}{\Ve^p}~\Partial{\Ve^p}{\Delta\gamma} + 
                        \Partial{h^{\phi}}{\phi}~\Partial{\phi}{\Delta\gamma}\right) \\
  \Deriv{f}{\Delta\gamma} & 
     = \Partial{f}{\Bxi}:\Partial{\Bxi}{\Delta\gamma} + 
          \Partial{f}{\Ve^p}~\Partial{\Ve^p}{\Delta\gamma} +
          \Partial{f}{\phi}~\Partial{\phi}{\Delta\gamma} ~.
  \Eal
\Eeq
Now, define
\Beq
   \Delta\Bxi := \Partial{\Bxi}{\Delta\gamma}~\delta\gamma ~;~~
   \Delta\Ve^p := \Partial{\Ve^p}{\Delta\gamma}~\delta\gamma ~;~~
   \Delta\phi := \Partial{\phi}{\Delta\gamma}~\delta\gamma ~.
\Eeq
Then
\Beq
  \Bal
  \Ba^{(k)} & - \Delta\Bxi - [2~\mu~\Dev(\Br^{(k)}) + \Bh_\Tdev^{\beta (k)}]~\delta\gamma \\
   & \qquad \qquad
    - 2~\mu~\Delta\gamma~\left(
      \Partial{\Dev(\Br^{(k)})}{\Bxi}:\Delta\Bxi + 
      \Partial{\Dev(\Br^{(k)})}{\Ve^p}~\Delta\Ve^p + 
      \Partial{\Dev(\Br^{(k)})}{\phi}~\Delta\phi 
      \right) \\
   & \qquad \qquad
    - \Delta\gamma~\left(
      \Partial{\Bh_\Tdev^{\beta (k)}}{\Bxi}:\Delta\Bxi + 
      \Partial{\Bh_\Tdev^{\beta (k)}}{\Ve^p}~\Delta\Ve^p +
      \Partial{\Bh_\Tdev^{\beta (k)}}{\phi}~\Delta\phi
      \right)  = 0\\
  b^{(k)} & - \Delta\Ve^p + h^{\alpha}~\delta\gamma 
    + \Delta\gamma~\left(\Partial{h^{\alpha (k)}}{\Bxi}:\Delta\Bxi + 
                        \Partial{h^{\alpha (k)}}{\Ve^p}~\Delta\Ve^p +
                        \Partial{h^{\alpha (k)}}{\phi}~\Delta\phi\right)
     = 0 \\
  c^{(k)} & - \Delta\phi + h^{\phi}~\delta\gamma 
    + \Delta\gamma~\left(\Partial{h^{\phi (k)}}{\Bxi}:\Delta\Bxi + 
                        \Partial{h^{\phi (k)}}{\Ve^p}~\Delta\Ve^p +
                        \Partial{h^{\phi (k)}}{\phi}~\Delta\phi\right)
     = 0 \\
  f^{(k)} & + \Partial{f^{(k)}}{\Bxi}:\Delta\Bxi + 
          \Partial{f^{(k)}}{\Ve^p}~\Delta\Ve^p +
          \Partial{f^{(k)}}{\phi}~\Delta\phi 
      = 0
  \Eal
\Eeq
Because the derivatives of $\Br^{(k)}, \Bh^{\alpha (k)}, \Bh^{\beta (k)}, \Bh^{\phi (k)}$ with respect 
to $\Bxi, \Ve^p, \phi$ may be difficult to calculate, we instead use a semi-implicit scheme in our 
implementation where the quantities $\Br$, $h^{\alpha}$, $\Bh^{\beta}$, and $h^{\phi}$ are evaluated 
at $t_n$.  Then the problematic derivatives disappear and we are left with
\Beq
  \Bal
  \Ba^{(k)} - \Delta\Bxi - [2~\mu~(\BM_\Tdev)_n + (\Bh_\Tdev^\beta)_n]~\delta\gamma & = 0\\
  b^{(k)} - \Delta\Ve^p + h^{\alpha}_n~\delta\gamma & = 0 \\
  c^{(k)} - \Delta\phi + h^{\phi}_n~\delta\gamma & = 0 \\
  f^{(k)} + \Partial{f^{(k)}}{\Bxi}:\Delta\Bxi + 
          \Partial{f^{(k)}}{\Ve^p}~\Delta\Ve^p +
          \Partial{f^{(k)}}{\phi}~\Delta\phi & = 0
  \Eal
\Eeq
We now force $\Ba^{(k)}$, $b^{(k)}$, and $c^{(k)}$ to be zero at all times, leading
to the expressions
\Beq
  \Bal
  \Delta\Bxi & = - [2~\mu~(\BM_\Tdev)_n + (\Bh_\Tdev^\beta)_n]~\delta\gamma \\
  \Delta\Ve^p & =  h^{\alpha}_n~\delta\gamma \\
  \Delta\phi & =  h^{\phi}_n~\delta\gamma \\
  f^{(k)} + \Partial{f^{(k)}}{\Bxi}:\Delta\Bxi + 
          \Partial{f^{(k)}}{\Ve^p}~\Delta\Ve^p +
          \Partial{f^{(k)}}{\phi}~\Delta\phi & = 0
  \Eal
\Eeq
Plugging the expressions for $\Delta\Bxi$, $\Delta\Ve^p$, $\Delta\phi$ from the 
first three equations into the fourth gives us
\Beq
  f^{(k)} -\Partial{f^{(k)}}{\Bxi}:[2~\mu~(\BM_\Tdev)_n + (\Bh_\Tdev^\beta)_n]~\delta\gamma +
          h^{\alpha}_n~\Partial{f^{(k)}}{\Ve^p}~\delta\gamma  + 
          h^{\phi}_n~\Partial{f^{(k)}}{\phi}~\delta\gamma  = 0 
\Eeq
or
\Beq
  \Delta\gamma^{(k+1)} - \Delta\gamma^{(k)} = \delta\gamma = 
   \cfrac{f^{(k)}}
   {\Partial{f^{(k)}}{\Bxi}:[2~\mu~(\BM_\Tdev)_n + (\Bh_\Tdev^\beta)_n] - 
   h^{\alpha}_n~\Partial{f^{(k)}}{\Ve^p} -
   h^{\phi}_n~\Partial{f^{(k)}}{\phi} } ~.
\Eeq

\subsection{Algorithm}
The following stress update algorithm is used for each (plastic) time step:
\begin{enumerate}
  \item Initialize:
  \Beq
    k = 0 ~;~~ (\Ve^p)^{(k)} = \Ve^p_n ~;~~ \phi^{(k)} = \phi_n ~;~~
    \Bbeta^{(k)} = \Bbeta_n ~;~~ \Delta\gamma^{(k)} = 0 ~;~~
    \Bxi^{(k)} = \Bxi^{\Trial}~.
  \Eeq
  \item Check yield condition:
  \Beq
    f^{(k)} := f(\Bxi^{(k)}, (\Ve^p)^{(k)}, \phi^{(k)}, \dot{\Ve}_n, T_n, \dots)
  \Eeq
  If $f^{(k)} < \text{tolerance}$ then 
  go to step 5 else go to step 3.
  \item Compute updated $\delta\gamma^{(k)}$ using
  \Beq
    \delta\gamma^{(k)} = 
     \cfrac{f^{(k)}}
     {\Partial{f^{(k)}}{\Bxi}:[2~\mu~(\BM_\Tdev)_n + (\Bh_\Tdev^\beta)_n] - 
     h^{\alpha}_n~\Partial{f^{(k)}}{\Ve^p} - 
     h^{\phi}_n~\Partial{f^{(k)}}{\phi}} ~.
  \Eeq
  Compute
  \Beq
  \Bal
    \Delta\Bxi^{(k)} & = -[2~\mu~(\BM_\Tdev)_n + (\Bh_\Tdev^\beta)_n]~\delta\gamma^{(k)} \\
    (\Delta\Ve^p)^{(k)} & =  h^{\alpha}_n~\delta\gamma^{(k)}   \\
    \Delta\phi^{(k)} & =  h^{\phi}_n~\delta\gamma^{(k)}   
  \Eal
  \Eeq
  \item Update variables:
  \Beq
    \Bal
    (\Ve^p)^{(k+1)} & = (\Ve^p)^{(k)} + (\Delta\Ve^p)^{(k)} \\
    \phi^{(k+1)} & = \phi^{(k)} + \Delta\phi^{(k)} \\
    \Bxi^{(k+1)} & = \Bxi^{(k)} + \Delta\Bxi^{(k)} \\
    \Delta\gamma^{(k+1)} & = \Delta\gamma^{(k)} + \delta\gamma^{(k)}
    \Eal
  \Eeq
  Set $k \leftarrow k+1$ and go to step 2.
  \item Update and calculate back stress and the deviatoric part of Cauchy stress:
  \Beq
    \Ve^p_{n+1} = (\Ve^p)^{(k)} ~;~~
    \phi_{n+1} = \phi^{(k)} ~;~~
    \Bxi_{n+1} = \Bxi^{(k)} ~;~~
    \Delta\gamma_{n+1} = \Delta\gamma^{(k)}
  \Eeq
  and
  \Beq
    \Bal
    \widehat{\Bbeta}_{n+1} & = \widehat{\Bbeta}_n + \Delta\gamma_{n+1}~\Bh^{\beta}(\Bxi_{n+1}, \Ve^p_{n+1}, \phi_{n+1}) \\
    \Bbeta_{n+1} & = \widehat{\Bbeta}_{n+1} - \Third~\Tr(\widehat{\Bbeta}_{n+1})~\Bone \\
    (\Bsig_\Tdev)_{n+1} & = \Bxi_{n+1} + \Bbeta_{n+1}
    \Eal
  \Eeq
  \item Update the temperature and the Cauchy stress
  \Beq
    \Bal
    T_{n+1} & = T_n + 
     \cfrac{\chi_{n+1}~\Delta t}{\rho_{n+1}~C_p}~\sigma^{n+1}_y~\dot{\Ve^p}_{n+1} 
     = T_n + 
     \cfrac{\chi_{n+1}~\Delta\gamma_{n+1}}{\rho_{n+1}~C_p}~\sigma^{n+1}_y~h^{\alpha}_{n+1} \\
    p_{n+1} & = p(J_{n+1}) \\ 
    \kappa_{n+1} & = J_{n+1}~\left[\Deriv{p(J)}{J}\right]_{n+1} \\
    \Bsig_{n+1} & = \left[p_{n+1} -
3~\kappa_{n+1}~\alpha~(T_{n+1}-T_0)\right]~\Bone + (\Bsig_\Tdev)_{n+1}
    \Eal
  \Eeq
\end{enumerate}

\section{Examples}
Let us now look at a few examples.
\subsection{Example 1}
Consider the case of $J_2$ plasticity with the yield condition
\Beq
  f := \sqrt{\frac{3}{2}} \Norm{\Bsig_\Tdev-\Bbeta}{} - \sigma_y(\Ve^p, \dot{\Ve}, T, \dots) = 
       \sqrt{\frac{3}{2}} \Norm{\Bxi}{} - \sigma_y(\Ve^p, \dot{\Ve}, T, \dots) \le 0 
\Eeq
where $\Norm{\Bxi} = \sqrt{\Bxi:\Bxi}$. Assume the associated flow rule
\Beq
  \BdT^p = \dot{\gamma}~\Br = \dot{\gamma}~\Partial{f}{\Bsig} = \dot{\gamma}~\Partial{f}{\Bxi} ~.
\Eeq
Then
\Beq
  \Br = \Partial{f}{\Bxi} = \sqrt{\frac{3}{2}}~\cfrac{\Bxi}{\Norm{\Bxi}{}} 
\Eeq
and
\Beq
  \BdT^p = \sqrt{\frac{3}{2}}~\dot{\gamma}~\cfrac{\Bxi}{\Norm{\Bxi}{}} ~;~~
  \Norm{\BdT^p}{} = \sqrt{\frac{3}{2}}~\dot\gamma ~.
\Eeq
The evolution of the equivalent plastic strain is given by
\Beq
  \dot{\Ve^p} = \dot{\gamma}~h^{\alpha} = \sqrt{\cfrac{2}{3}}~\Norm{\BdT^p}{} = \dot{\gamma}~.
\Eeq
This definition is consistent with the definition of equivalent plastic strain
\Beq
  \Ve^p = \int_0^t \dot{\Ve}^p~d\tau = 
   \int_0^t \sqrt{\cfrac{2}{3}}~\Norm{\BdT^p}{}~d\tau ~.
\Eeq
The evolution of porosity is given by (there is no evolution of porosity)
\Beq
  \dot{\phi} = \dot{\gamma}~h^{\phi} = 0
\Eeq
The evolution of the back stress is given by the Prager kinematic hardening rule
\Beq
  \dot{\widehat{\Bbeta}} = \dot{\gamma}~\Bh^{\beta} = \frac{2}{3}~H'~\BdT^p 
\Eeq
where $\widehat{\Bbeta}$ is the back stress and
$H'$ is a constant hardening modulus.  Also, the trace of $\BdT^p$ is 
\Beq
  \Tr(\BdT^p) = \sqrt{\frac{3}{2}}~\dot{\gamma}~\cfrac{\Tr(\Bxi)}{\Norm{\Bxi}{}}~.
\Eeq
Since $\Bxi$ is deviatoric, $\Tr(\Bxi) = 0$ and hence $\BdT^p = \BdT_\Tdis^p$.
Hence, $\widehat{\Bbeta} = \Bbeta$ (where $\Bbeta$ is the deviatoric part of $\widehat{\Bbeta}$), and
\Beq
  \dot{\Bbeta} = \sqrt{\frac{2}{3}}~H'~\dot{\gamma}~\cfrac{\Bxi}{\Norm{\Bxi}{}} ~.
\Eeq

These relation imply that
\Beq
  \boxed{
  \Bal
    \Br & = \sqrt{\frac{3}{2}}~\cfrac{\Bxi}{\Norm{\Bxi}{}} \\
     h^{\alpha} & = 1 \\
     h^{\phi} & = 0 \\
    \Bh^{\beta} & = \sqrt{\frac{2}{3}}~H'~\cfrac{\Bxi}{\Norm{\Bxi}{}} ~.
  \Eal
  }
\Eeq
We also need some derivatives of the yield function.  These are
\Beq
  \Bal
  \Partial{f}{\Bxi} & = \Br \\
  \Partial{f}{\Ve^p} & = -\Partial{\sigma_y}{\Ve^p} \\
  \Partial{f}{\phi} & = 0 ~.
  \Eal
\Eeq

Let us change the kinematic hardening model and use the Armstrong-Frederick
model instead, i.e.,
\Beq
  \dot{\Bbeta} = \dot{\gamma}~\Bh^{\beta} = \frac{2}{3}~H_1~\BdT^p - H_2~\Bbeta~\Norm{\BdT^p}{} ~.
\Eeq
Since
\Beq
  \BdT^p = \sqrt{\frac{3}{2}}~\dot{\gamma}~\cfrac{\Bxi}{\Norm{\Bxi}{}}
\Eeq
we have
\Beq
  \Norm{\BdT^p}{} = 
   \sqrt{\frac{3}{2}}~\dot{\gamma}~\cfrac{\Norm{\Bxi}{}}{\Norm{\Bxi}{}} = 
   \sqrt{\frac{3}{2}}~\dot{\gamma} ~.
\Eeq
Therefore,
\Beq
  \dot{\Bbeta} = \sqrt{\frac{2}{3}}~H_1~\dot{\gamma}~\cfrac{\Bxi}{\Norm{\Bxi}{}} 
    - \sqrt{\frac{3}{2}}~H_2~\dot{\gamma}~\Bbeta ~.
\Eeq
Hence we have
\Beq
  \boxed{
  \Bh^{\beta} = \sqrt{\frac{2}{3}}~H_1~\cfrac{\Bxi}{\Norm{\Bxi}{}} 
    - \sqrt{\frac{3}{2}}~H_2~\Bbeta ~.
   }
\Eeq

\subsection{Example 2}
Let us now consider a Gurson type yield condition with kinematic hardening.  In this
case the yield condition can be written as
\Beq
  f := \cfrac{3~\Bxi:\Bxi}{2~\sigma_y^2} + 
     2~q_1~\phi^{*}~\cosh\left(\cfrac{q_2~\Tr(\Bsig)}{2~\sigma_y}\right)
     - [1 + q_3~(\phi^*)^2]
\Eeq
where $\phi$ is the porosity and
\Beq
  \phi^* = \begin{cases}
             \phi & \text{for}~ \phi \le \phi_c \\
             \phi_c - \cfrac{\phi_u^* - \phi_c}{\phi_f - \phi_c}~(\phi - \phi_c) & 
              \text{for}~ \phi > \phi_c
           \end{cases}
\Eeq
Final fracture occurs for $\phi = \phi_f$ or when $\phi_u^* = 1/q_1$.  

Let us use an associated flow rule
\Beq
  \BdT^p = \dot{\gamma}~\Br = \dot{\gamma}~\Partial{f}{\Bsig} ~.
\Eeq
Then
\Beq
  \Br = \Partial{f}{\Bsig} = \cfrac{3~\Bxi}{\sigma_y^2} + \cfrac{q_1~q_2~\phi^{*}}{\sigma_y}~
   \sinh\left(\cfrac{q_2~\Tr(\Bsig)}{2~\sigma_y}\right)~\Bone ~.
\Eeq
In this case
\Beq
  \Tr(\Br) = \cfrac{3~q_1~q_2~\phi^{*}}{\sigma_y}~\sinh\left(\cfrac{q_2~\Tr(\Bsig)}{2~\sigma_y}\right)
  \ne 0 
\Eeq
Therefore,
\Beq
  \BdT^p \ne \BdT_\Tdis^p ~.
\Eeq

For the evolution equation for the plastic strain we use
\Beq
  (\Bsig-\widehat{\Bbeta}):\BdT^p = (1 - \phi)~\sigma_y~\dot{\Ve}^p
\Eeq
where $\dot{\Ve}^p$ is the effective plastic strain rate in the matrix material.  Hence,
\Beq
  \dot{\Ve}^p = \dot{\gamma}~h^{\alpha}
    = \dot{\gamma}~\cfrac{(\Bsig - \widehat{\Bbeta}):\Br}{(1 - \phi)~\sigma_y} ~.
\Eeq

The evolution equation for the porosity is given by
\Beq
  \dot{\phi} = (1 - \phi)~\Tr(\BdT^p) + A~\dot{\Ve^p}
\Eeq
where
\Beq
A = \cfrac{f_n}{s_n \sqrt{2\pi}} \exp [-1/2 (\Ve^p - \Ve_n)^2/s_n^2]
\Eeq
and $ f_n $ is the volume fraction of void nucleating particles, 
$ \Ve_n $ is the mean of the normal distribution of nucleation strains, and 
$ s_n $ is the standard deviation of the distribution.

Therefore,
\Beq
  \dot{\phi} = \dot{\gamma}~h^{\phi} =
    \dot{\gamma}~\left[(1 - \phi)~\Tr(\Br) + A~
    \cfrac{(\Bsig - \widehat{\Bbeta}):\Br}{(1 - \phi)~\sigma_y}\right] ~.
\Eeq

If the evolution of the back stress is given by the Prager kinematic hardening rule
\Beq
  \dot{\widehat{\Bbeta}} = \dot{\gamma}~\Bh^{\beta} = \frac{2}{3}~H'~\BdT^p 
\Eeq
where $\widehat{\Bbeta}$ is the back stress, then
\Beq
  \dot{\widehat{\Bbeta}} = \frac{2}{3}~H'~\dot{\gamma}~\Br ~.
\Eeq
Alternatively, if we use the Armstrong-Frederick model, then
\Beq
  \dot{\widehat{\Bbeta}} = \dot{\gamma}~\Bh^{\beta} = 
   \frac{2}{3}~H_1~\BdT^p - H_2~\widehat{\Bbeta}~\Norm{\BdT^p}{} ~.
\Eeq
Plugging in the expression for $\BdT^p$, we have
\Beq
  \dot{\widehat{\Bbeta}} = \dot{\gamma}~
  \left[\frac{2}{3}~H_1~\Br - H_2~\widehat{\Bbeta}~\Norm{\Br}{}\right] ~.
\Eeq
Therefore, for this model,
\Beq
  \boxed{
  \Bal
  \Br & = \cfrac{3~\Bxi}{\sigma_y^2} + \cfrac{q_1~q_2~\phi^{*}}{\sigma_y}~
   \sinh\left(\cfrac{q_2~\Tr(\Bsig)}{2~\sigma_y}\right)~\Bone  \\
  h^{\alpha} &  
    = \cfrac{(\Bsig - \Bbeta):\Br}{(1 - \phi)~\sigma_y} \\
  h^{\phi} & = 
    (1 - \phi)~\Tr(\Br) + A~
    \cfrac{(\Bsig - \widehat{\Bbeta}):\Br}{(1 - \phi)~\sigma_y}  \\
  \Bh^{\beta} & = 
   \frac{2}{3}~H_1~\Br - H_2~\widehat{\Bbeta}~\Norm{\Br}{}
  \Eal
  }
\Eeq
The other derivatives of the yield function that we need are
\Beq
  \Bal
  \Partial{f}{\Bxi} & = \cfrac{3~\Bxi}{\sigma_y^2} \\
  \Partial{f}{\Ve^p} & = \Partial{f}{\sigma_y}~\Partial{\sigma_y}{\Ve^p} 
   = -\left[\cfrac{3~\Bxi:\Bxi}{\sigma_y^3} +
     \cfrac{q_1~q_2~\phi^*~\Tr(\Bsig)}{\sigma_y^2}~
     \sinh\left(\cfrac{q_2~\Tr(\Bsig)}{2~\sigma_y}\right)\right]~
     \Partial{\sigma_y}{\Ve^p}\\
  \Partial{f}{\phi} & = 2~q_1~\Deriv{\phi^*}{\phi}~
    \cosh\left(\cfrac{q_2~\Tr(\sigma)}{2~\sigma_y}\right) 
    - 2~q_3~\phi^*~\Deriv{\phi^*}{\phi} ~.
  \Eal
\Eeq



\end{document}







