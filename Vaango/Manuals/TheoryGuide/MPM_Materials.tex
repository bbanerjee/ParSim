
\chapter{MPM Material Models}
The MPM material models implemented in \Vaango we originally
chosen for three purposes:
\begin{itemize}
  \item To verify the accuracy of the material point method (MPM)
        and to validate the coupling between the computational fluid 
        dynamics code (ICE) and MPM.
  \item To model the elastic-plastic deformation of metals 
        and the consequent damage in the regimes of 
        both high and low strain rates and high and low temperatures.
  \item To model polymer bonded explosives 
        under various strain rates and temperatures.
  \item To model the explosive deformation of rocks and soils.
\end{itemize}
In this chapter we discuss some general features of the MPM material
models.  Individual models are complex and are discussed in separate
chapters.

\subsection{Material models for the validation of MPM}
The models that have been implemented for the verification of MPM
are:
\begin{itemize}
   \item Isotropic hypoelastic model using the Jaumann rate of 
         stress.
     \begin{enumerate}
        \item  MPM predictions have been compared with exact results 
               for thick cylinders under internal pressure for small
               strains, three-point beam bending, etc.
        \item  MPM predictions for the strain/stress contours for
               a set of disks in contact have been found to match
               experimental results.
     \end{enumerate}
   \item Isotropic hyperelastic material models for Mooney-Rivlin
         rubber and a modified compressible Neo-Hookean material. 
         Isotropic strain hardening plasticity for the Neo-Hookean
         material.
     \begin{enumerate}
        \item  A billet compression problem has been simulated using
               MPM and the results have been found to closely 
               match finite element simulations.
        \item  MPM simulations for a thick cylinder under internal
               pressure with plastic deformation (perfect plasticity)
               compare well with the exact solution.
     \end{enumerate}
\end{itemize}

\subsection{Material models for metals}
The material models for metals are used to determine
the state of stress  for an applied deformation
rate and deformation gradient at each material point.  The strain 
rates can vary from $10^{-3}$/s to $10^6$/s and temperatures in 
the container can vary from 250 K to 1000 K.  For example, in
an explosion inside a container, plasticity dominates
the deformation of the container during the expansion of the 
explosive gases inside.  At high strain rates the volumetric
response of the container is best obtained using an equation of 
state.  After the plastic strain in the container
has reached a threshold value a damage/erosion model is required to
rupture the container.

Two plasticity models with strain rate and temperature dependency 
are the Johnson-Cook and the Mechanical Threshold Stress (MTS) 
models.  The volumetric response is calculated using a modified
Mie-Gruneisen equation of state.  A damage model that ties in well 
with the Johnson-Cook plasticity model is the Johnson-Cook damage 
model.  The erosion algorithm either removes the contribution
of the mass of the material point or forces the material point
to undergo no tension or shear under further loading.

The stress update at each material point is performed using either
of the two methods discussed below.
\begin{itemize}
   \item Isotropic Hypoelastic-plastic material model using an 
         additive decomposition of the rate of deformation tensor.
     \begin{enumerate}
        \item The rate of deformation tensor a material point
              is calculated using the grid velocities.
        \item An incremental update of the left stretch and the
              rate of rotation tensors is calculated.
        \item The stress and the rate of deformation are rotated
              into the material coordinates.
        \item A trial elastic deviatoric stress state is calculated.
        \item The flow stress is calculated using the plasticity
              model and compared with the vonMises yield condition.
        \item If the stress state is elastic, an update of the 
              stress is computed using the Mie-Gruneisen equation
              of state or the isotropic hypoelastic constitutive
              equation.
        \item If the stress state is plastic, all the strain rate 
              is considered to the plastic and an elastic correction
              along with a radial return step move the stress state
              to the yield surface.  The hydrostatic part of the 
              stress is calculated using the equation of state or
              the hypoelastic constitutive equation.
        \item A scalar damage parameter is calculated and used
              to determine whether material points are to be eroded
              or not.
        \item Stresses and deformation rates are rotated back to the
              laboratory coordinates.
     \end{enumerate}
   \item Isotropic Hyperelastic-plastic material model using a 
         multiplicative decomposition of the deformation gradient.
     \begin{enumerate}
        \item The velocity gradient at a material point
              is calculated using the grid velocities.
        \item An incremental update of the deformation gradient and
              the left Cauchy-Green tensor is calculated.
        \item A trial elastic deviatoric stress state is calculated
              assuming a compressible Neo-Hookean elastic model.
        \item The flow stress is calculated using the plasticity
              model and compared with the vonMises yield condition.
        \item If the stress state is elastic, an update of the 
              stress is computed using the Mie-Gruneisen equation
              of state or the compressible Neo-Hookean constitutive
              equation.
        \item If the stress state is plastic, all the strain rate 
              is considered to the plastic and an elastic correction
              along with a radial return step move the stress state
              to the yield surface.  The hydrostatic part of the 
              stress state is calculated using the Mie-Gruneisen
              equation of state or the Neo-Hookean model.
        \item A scalar damage parameter is calculated and used
              to determine whether material points are to be eroded
              or not.
     \end{enumerate}
\end{itemize}

The implementations have been tested against Taylor impact test data
for 4340 steel and HY 100 steel as well as one-dimensional problems
which have been compared with experimental stress-strain data.  

\subsection{Material models for the explosive}
The explosive is modeled using the \Textsfc{ViscoSCRAM} constitutive 
model.  Since large deformations or strains are not expected in 
the explosive, a small strain formulation has been implemented into 
\Vaango.  The model consists of five generalized Maxwell elements
arranged in parallel, crack growth, friction at the crack 
interfaces and heating due to friction and reactions at the 
crack surfaces.  The implementation has been verified with 
experimental data and found to be accurate.


