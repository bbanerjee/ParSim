\chapter{MPM Material Models}
In this chapter we discuss some general features of the MPM material
models.  Individual models are complex and are discussed in separate
chapters.  Notation and definitions that are used frequently are
also elaborated upon here.

\section{Notation and definitions}
A primary assumption made in many of the material models in Vaango is that
stresses and (moderate) strains can be additively decomposed into volumetric and deviatoric parts.

\subsection{Volumetric-deviatoric decomposition}
The volumetric-deviatoric decomposition of stress ($\Bsig$) is expressed as
\Beq
  \Bsig = p\BI + \BsT
\Eeq
where the mean stress ($p$) and the deviatoric stress ($\BsT$) are given by
\Beq
  p = \Third\Tr(\Bsig) = \Third\Bsig:\BI = \tfrac{1}{3}(\sigma_{11} + \sigma_{22} + \sigma_{33})
  \quad \Tand \quad
  \BsT = \Bsig - p\BI = 
    \begin{bmatrix} 
       \sigma_{11} - p & \sigma_{12} & \sigma_{13} \\
       \sigma_{12} & \sigma_{22} - p & \sigma_{23} \\
       \sigma_{13} & \sigma_{23} & \sigma_{33} - p
    \end{bmatrix} \,.
\Eeq

Similarly, the volumeric-deviatoric split of the strain ($\BVeps$) is expressed as
\Beq
  \BVeps = \Third \Veps_v \BI + \BVeps_s
\Eeq
where the volumetric strain ($\Veps_v$) and the deviatoric strain ($\BVeps_s$) are defined as
\Beq
  \Veps_v = \Tr(\BVeps) = \BVeps:\BI = \Veps_{11} + \Veps_{22} + \Veps_{33}
  \quad \Tand \quad
  \BVeps_s = \BVeps - \Third \Veps_v\BI = 
    \begin{bmatrix} 
       \Veps_{11} - \Third \Veps_v & \Veps_{12} & \Veps_{13} \\
       \Veps_{12} & \Veps_{22} - \Third \Veps_v & \Veps_{23} \\
       \Veps_{13} & \Veps_{23} & \Veps_{33} - \Third \Veps_v
    \end{bmatrix} \,.
\Eeq

\subsection{Stress invariants}
The principal invariants and principal deviatoric invariants of the stress are used in 
several models.  Frequently used invariants are:
\Beq
  \Bal
    I_1 & = \Tr(\Bsig) = \sigma_{11} + \sigma_{22} + \sigma_{33} \\
    I_2 & = \Half\left[\Tr(\Bsig)^2 - \Tr(\Bsig^2)\right]
          = \sigma_{11} \sigma_{22} + \sigma_{22} \sigma_{33} + \sigma_{33} \sigma_{11}
            -  (\sigma_{12}^2 + \sigma_{23}^2 + \sigma_{13}^2) \\
    I_3 & = \det(\Bsig)
          = \sigma_{11}\sigma_{22}\sigma_{33} + 2\sigma_{12}\sigma_{23}\sigma_{13}
            - \sigma_{12}^2 \sigma_{33} - \sigma_{23}^2 \sigma_{11}
            - \sigma_{13}^2 \sigma_{22} \\
    J_2 & = \Half \BsT:\BsT 
          = \tfrac{1}{6}\left[(\sigma_{11} - \sigma_{22})^2 + 
             (\sigma_{22} - \sigma_{33})^2 + (\sigma_{33} - \sigma_{11})^2\right] +
             (\sigma_{12}^2 + \sigma_{23}^2 + \sigma_{31}^2) \\
    J_3 & = \det(\BsT) = \tfrac{2}{27} I_1^3 - \Third I_1 I_2 + I_3
  \Eal
\Eeq
Alternatives to $I_1$, $J_2$ and $j_3$ are $p$, $q$, and $\theta$, defined as
\Beq
  p := \frac{1}{3} I_1~,~~ q := \sqrt{3 J_2}~,~~
  \cos3\theta := \left(\cfrac{r}{q}\right)^3 = \frac{3\sqrt{3}}{2} \frac{J_3}{J_2^{3/2}} ~,~~
  r^3 = \frac{27}{2} J_3\,.
\Eeq
A geometric accurate view of the stress state and yield surfaces is obtained if 
the isomorphic cylindrical coordinates $z$, $\rho$, and $\theta$ are used instead, where
\Beq
  z := \frac{I_1}{\sqrt{3}} = \sqrt{3} p ~,~~
  \rho := \sqrt{2 J_2} = \sqrt{\frac{2}{3}} q ~,~~
  \cos3\theta := \frac{3\sqrt{3}}{2} \frac{J_3}{J_2^{3/2}} \,.
\Eeq

\subsection{Effective stress and strain}
The effective stress and strain (sometimes also referred to as the shear stress and shear strain
in the code) are defined such that the product is equal to the plastic work done.  These
measures are strictly applicable only to $J_2$ plasticity models but have also been used elsewhere.

The effective stress is defined as
\Beq \label{eq:eff_stress}
  \sigma_\Teff = q = \sqrt{3J_2} = \sqrt{\tfrac{3}{2} \BsT:\BsT}
    = \sqrt{\tfrac{1}{2}\left[(\sigma_{11} - \sigma_{22})^2 + 
             (\sigma_{22} - \sigma_{33})^2 + (\sigma_{33} - \sigma_{11})^2\right] +
             3 (\sigma_{12}^2 + \sigma_{23}^2 + \sigma_{31}^2) } \,.
\Eeq

The effective strain is defined as
\Beq \label{eq:eff_strain}
  \Veps_\Teff = \sqrt{\tfrac{2}{3} \BVeps_s:\BVeps_s}
\Eeq
so that
\Beq
  \sigma_\Teff \Veps_\Teff = \sqrt{(\BsT:\BsT) (\BVeps_s:\BVeps_s)} \,.
\Eeq
From the definition of $\BVeps_s$ we see that
\Beq
  \Bal
  \BVeps_s:\BVeps_s & = \BVeps:\BVeps - \tfrac{2}{3} \Tr(\BVeps)\BI:\BVeps + \tfrac{1}{9}\left[\Tr(\BVeps)\right]^2 \BI:\BI
     = \BVeps:\BVeps - \tfrac{2}{3} \left[\Tr(\BVeps)\right]^2 + \tfrac{1}{3}\left[\Tr(\BVeps)\right]^2
     = \BVeps:\BVeps - \Third \left[\Tr(\BVeps)\right]^2 \\
     &= \Veps_{11}^2 + \Veps_{22}^2 + \Veps_{33}^2 + 2\Veps_{12}^2 + 2\Veps_{23}^2 + 2\Veps_{13}^2
       - \Third\left[ \Veps_{11}^2 + \Veps_{22}^2 + \Veps_{23}^2 + 2\Veps_{11}\Veps_{22} + 
       2\Veps_{22}\Veps_{33} + 2\Veps_{11}\Veps_{33} \right] \\
     &= \Third\left[(\Veps_{11} - \Veps_{22})^2 + (\Veps_{22} - \Veps_{33})^2 + (\Veps_{33} - \Veps_{11})^2 \right] + 2(\Veps_{12}^2 + \Veps_{23}^2 + \Veps_{13}^2)
  \Eal
\Eeq
Therefore,
\Beq
  \Veps_\Teff = \sqrt{
      \tfrac{2}{3}\left[\Third\left[(\Veps_{11} - \Veps_{22})^2 + (\Veps_{22} - \Veps_{33})^2 + (\Veps_{33} - \Veps_{11})^2\right] + 2(\Veps_{12}^2 + \Veps_{23}^2 + \Veps_{13}^2)\right]}
\Eeq
For volume preserving plastic deformations, $\Tr(\BVeps) = 0$, and we have
\Beq
  \Veps_\Teff = \sqrt{
     \tfrac{2}{3}\left(\Veps_{11}^2 + \Veps_{22}^2 + \Veps_{33}^2\right) + 
     \tfrac{4}{3}\left(\Veps_{12}^2 + \Veps_{23}^2 + \Veps_{13}^2\right)} 
     = \sqrt{
     \tfrac{2}{3}\left(\Veps_{11}^2 + \Veps_{22}^2 + \Veps_{33}^2\right) + 
     \tfrac{1}{3}\left(\gamma_{12}^2 + \gamma_{23}^2 + \gamma_{13}^2\right)} \,.
\Eeq

\subsection{Equivalent plastic strain}
For models where an equivalent plastic strain is computed, we define a scalar equivalent plastic
strain rate as
\Beq \label{eq:eq_plastic_strain_rate}
  \dot{\Veps}_p^\Teq = \sqrt{\dot{\BVeps}^p:\dot{\BVeps}^p} 
\Eeq
where $\dot{\BVeps}^p(t)$ is the plastic strain rate tensor.

The definition of the scalar equivalent plastic strain is
\Beq \label{eq:eq_plastic_strain}
  \Veps_p^\Teq(t) = \int_0^t \dot{\Veps}_p^\Teq(\tau) d\tau \,.
\Eeq

\subsection{Velocity gradient, rate-of-deformation, deformation gradient}
The velocity gradient is represented by $\BlT$ and the defromation gradient by $\BF$.  The
rate-of-deformation is
\Beq
  \BdT = \Half(\BlT + \BlT^T) = \Half(\Grad{\Bv} + \Grad{\Bv}^T) \,.
\Eeq

\subsection{Eigenvectors and coordinate transformations}
For most situations, tensor components in \Vaango are expressed in terms of the basis vectors
$\Be_1 = (1,0,0)$, $\Be_2 = (0,1,0)$, and $\Be_3 = (0,0,1)$.  However, in some situations
tensor components have to be expressed in the eigenbasis of a second-order tensor.
Let these eigenvectors be $\Bv_1$, $\Bv_2$ and $\Bv_3$.  Then a vector $\Ba$
with components $(a_1, a_2, a_2)$ in the original basis has components $(a_1', a_2', a_3')$ in
the eigenbasis.  The two sets of components are related by
\Beq
  \begin{bmatrix} a_1' \\ a_2' \\a_3' \end{bmatrix} = 
  \begin{bmatrix} 
    \Be_1\cdot\Bv_1 & \Be_2\cdot\Bv_1 & \Be_3 \cdot \Bv_1 \\
    \Be_1\cdot\Bv_2 & \Be_2\cdot\Bv_2 & \Be_3 \cdot \Bv_2 \\
    \Be_1\cdot\Bv_3 & \Be_2\cdot\Bv_3 & \Be_3 \cdot \Bv_3 
  \end{bmatrix}  
  \begin{bmatrix} a_1 \\ a_2 \\a_3 \end{bmatrix} = 
  \begin{bmatrix} 
    v_{11} & v_{12} & v_{13} \\
    v_{21} & v_{22} & v_{23} \\
    v_{31} & v_{32} & v_{33} 
  \end{bmatrix}  
  \begin{bmatrix} a_1 \\ a_2 \\a_3 \end{bmatrix}  = 
  \BQ \cdot \Ba \,.
\Eeq
The matrix that is used for this coordinate transformation, $\BQ$, is given by
\Beq
  \BQ^T := \begin{bmatrix} \Bv_1 & \Bv_2 & \Bv_3 \end{bmatrix}
\Eeq
where $\Bv_1$, $\Bv_2$, and $\Bv_3$ are \Textbfc{column vectors} representing the components
of the eigenvectors in the reference basis.

The above coordinate transformation for vectors can be written in index notation as
\Beq
  a'_i = Q_{ij} a_j \,.
\Eeq
For transformations of second-order tensors, we have
\Beq
  T'_{ij} = Q_{ip} Q_{jq} T_{pq} \,.
\Eeq
For fourth-order tensors, the transformation relation is
\Beq
  C'_{ijk\ell} = Q_{im} Q_{jn} Q_{kp} Q_{\ell q} C_{mnpq} \,.
\Eeq

If the second-order tensor $\BT$ is symmetric, we can express it in Mandel notation 
as a six-dimensional vector $\hat{\Bt}$:
\Beq
  \hat{\Bt} = 
    \begin{bmatrix} T_{11} & T_{22} & T_{33} & \sqrt{2} T_{23} & \sqrt{2} T_{31} & \sqrt{2} T_{12} \end{bmatrix}^T
\Eeq
Then the transformation matrix is a $6\times 6$ matrix, $\hat{\BQ}$, such that
\Beq
  \hat{t}'_{i} = \hat{Q}_{ij} \hat{t}_{j} \,.
\Eeq
The matrix $\hat{\BQ}$ has components~\cite{Mehrabadi1990},
\Beq
  \hat{\BQ} = 
  \begin{bmatrix}
    Q_{11}^2 & Q_{12}^2 & Q_{13}^2 & \sqrt{2} Q_{12}Q_{13} & \sqrt{2} Q_{11}Q_{13} & \sqrt{2} Q_{11}Q_{12} \\
    Q_{21}^2 & Q_{22}^2 & Q_{23}^2 & \sqrt{2} Q_{22}Q_{23} & \sqrt{2} Q_{21}Q_{23} & \sqrt{2} Q_{21}Q_{22} \\
    Q_{31}^2 & Q_{32}^2 & Q_{33}^2 & \sqrt{2} Q_{32}Q_{33} & \sqrt{2} Q_{31}Q_{33} & \sqrt{2} Q_{31}Q_{32} \\
    \sqrt{2} Q_{21}Q_{31} & \sqrt{2} Q_{22}Q_{32} & \sqrt{2} Q_{23}Q_{33} & 
    Q_{22}Q_{33} + Q_{23}Q_{32} & Q_{21}Q_{33} + Q_{31}Q_{23} & Q_{21}Q_{32} + Q_{31}Q_{22} \\
    \sqrt{2} Q_{11}Q_{31} & \sqrt{2} Q_{12}Q_{32} & \sqrt{2} Q_{13}Q_{33} & 
    Q_{12}Q_{33} + Q_{32}Q_{13} & Q_{11}Q_{33} + Q_{13}Q_{31} & Q_{11}Q_{32} + Q_{31}Q_{12} \\
    \sqrt{2} Q_{11}Q_{31} & \sqrt{2} Q_{12}Q_{22} & \sqrt{2} Q_{13}Q_{23} & 
    Q_{12}Q_{23} + Q_{22}Q_{13} & Q_{11}Q_{23} + Q_{21}Q_{13} & Q_{11}Q_{22} + Q_{21}Q_{12} 
  \end{bmatrix}
\Eeq
Similarly, the transformation relation for fourth-order tensors simplifies to
\Beq
  \hat{C}'_{ij} = \hat{Q}_{ip} \hat{Q}_{jq} \hat{C}_{pq} 
\Eeq
where 
\Beq
  \hat{\BC} = \begin{bmatrix}
         C_{1111} & C_{1122} & C_{1133} & \sqrt{2} C_{1123} & \sqrt{2} C_{1131} & \sqrt{2} C_{1112} \\
         C_{2211} & C_{2222} & C_{2233} & \sqrt{2} C_{2223} & \sqrt{2} C_{2231} & \sqrt{2} C_{2212} \\
         C_{3311} & C_{3322} & C_{3333} & \sqrt{2} C_{3323} & \sqrt{2} C_{3331} & \sqrt{2} C_{3312} \\
         \sqrt{2} C_{2311} & \sqrt{2} C_{2322} & \sqrt{2} C_{2333} & 2 C_{2323} & 2 C_{2331} & 2 C_{2312} \\
         \sqrt{2} C_{3111} & \sqrt{2} C_{3122} & \sqrt{2} C_{3133} & 2 C_{3123} & 2 C_{3131} & 2 C_{3112} \\
         \sqrt{2} C_{1211} & \sqrt{2} C_{1222} & \sqrt{2} C_{1233} & 2 C_{1223} & 2 C_{1231} & 2 C_{1212} 
         \end{bmatrix}
\Eeq


\section{Material models available in Vaango}
The MPM material models implemented in \Vaango were originally
chosen for the following purposes:
\begin{itemize}
  \item To verify the \Textsfc{accuracy} of the material point method (MPM)
        and to validate the \Textsfc{coupling} between the computational fluid 
        dynamics code (ICE) and MPM.
  \item To model the elastic-plastic deformation of \Textsfc{metals} 
        and the consequent damage in the regimes of 
        both high and low strain rates and high and low temperatures.
  \item To model \Textsfc{polymer bonded explosives} and \Textsfc{polymers}
        under various strain rates and temperatures.
  \item To model the deformation of \Textsfc{biological tissues}.
  \item To model the explosive deformation of \Textsfc{rocks and soils}.
\end{itemize}

As of \Vaango \version, the material models that have been implemented are:
\begin{enumerate}
  \item Rigid material
  \item Ideal gas material
  \item Water material
  \item Membrane material
  \item Programmed burn material
  \item Tabular equation of state
  \item Murnaghan equation of state
  \item JWL++ equation of state
  \item Hypoelastic material
  \item Hypoelastic material with manufactured solutions
  \item Hypoelastic material implementation in FORTRAN
  \item Polar-orthotropic hypoelastic material
  \item Compressible neo-Hookean hyperelastic material
  \item Compressible neo-Hookean hyperelastic material with manufactured solutions
  \item Compressible neo-Hookean hyperelastic material with damage
  \item Unified explicit/implicit compressible Neo-Hookean hyperelastic material with damage
  \item Compressible Neo-Hookean hyperelastic-J$_2$ plastic material with damage
  \item Compressible Mooney-Rivlin hyperelastic material
  \item Compressible neo-Hookean material for shells
  \item Transversely isotropic hyperelastic material
  \item The $p$-$\alpha$ model for porous materials
  \item Viscoelastic material written in FORTRAN for damping
  \item Simplified Maxwell viscoelastic material
  \item Visco-SCRAM model for viscoelastic materials with cracks
  \item Visco-SCRAM hotspot model
  \item Tabular plasticity model
  \item Tabular plasticity model with cap
  \item Hypoelastic J$_2$ plasticity model with damage for high-rates
  \item Viscoplastic J$_2$ plasticity model
  \item Mohr-Coulomb material
  \item Drucker-Prager material with deformation induced elastic anisotropy
  \item CAM-Clay model for soils
  \item Nonlocal Drucker-Prager material
  \item Arenisca material for rocks and soils
  \item Arenisca3 material for rocks and soils
  \item Arena material for partially saturated soils
  \item Arena-mixture material for mixes of partially saturated sand and clay
  \item Brannon's soil model
  \item Soil foam model
\end{enumerate}
A small subset of these models also have implementations that can be used with
Implicit \MPM.

Some of these models can work with multiple sub-models such as elasticity model
or yield condition.  As of \Vaango \version, the implemented sub-models are:
\begin{enumerate}
  \item Equations of state:
  \begin{enumerate}
    \item Pressure model for Air
    \item Pressure model for Borja's CAMClay
    \item Pressure model for Granite
    \item Pressure model for hyperelastic materials
    \item Pressure model for Hypoelastic materials
    \item Pressure model for Mie-Gruneisen equation of state
    \item Mie-Gruneisen energy-based equation of state for pressure
    \item Pressure model for Water
  \end{enumerate}
  \item Shear modulus models:
  \begin{enumerate}
    \item Constant shear modulus model
    \item Shear modulus model for Borja's CAMClay
    \item Shear modulus model by Nadal and LePoac
    \item Mechanical Threshold Stress shear modulus model
    \item Preston-Tonks-Wallace shear modulus model
    \item Steinberg-Guinan shear modulus model
  \end{enumerate}
  \item Combined elastic modulus models:
  \begin{enumerate}
    \item Constant elastic modulus model
    \item Tabular elastic modulus model
    \item Neural net elastic modulus model
    \item Arena elastic modulus model
    \item Arena mixture elastic modulus model
    \item Arenisca elastic modulus model
  \end{enumerate}
  \item Yield condition models:
  \begin{enumerate}
    \item Yield condition for Arena model
    \item Yield condition for Arena mixture model
    \item Yield condition for Arenisca3 model
    \item Yield condition for CamClay model
    \item Yield condition for Gurson model
    \item Yield condition for Tabular plasticity with Cap
    \item Yield condition for Tabular plasticity with
    \item Yield condition for vonMises J$_2$ plasticity
    \item Classic Mohr-Coulomb model
    \item Sheng's Mohr-Coulomb model
  \end{enumerate}
  \item Plastic flow stress models 
  \begin{enumerate}
    \item Isotropic hardening plastic flow model
    \item Johnson-Cook plastic flow model
    \item Mechanical Threshold Stress plastic flow model
    \item Preston-Tonks-Wallace plastic flow model
    \item Steinberg-Guinan plastic flow model
    \item SuvicI viscoplastic flow model
    \item Zerilli-Armstrong metal plastic flow model
    \item Zerilli-Armstrong polymer plastic flow model
  \end{enumerate}
  \item Plastic internal variable models:
  \begin{enumerate}
    \item Arena internal variable model
    \item Borja internal variable model
    \item Brannan's soil model internal variable model
    \item Tabular plasticity with cap internal variable model
  \end{enumerate}
  \item Kinematic hardening models:
  \begin{enumerate}
    \item Prager kinematic hardening model
    \item Armstrong-Frederick kinematic hardening model
    \item Arena kinematic hardening model
  \end{enumerate}
  \item Damage models
  \begin{enumerate}
    \item Becker's damage model
    \item Drucker and Becker combined damage model
    \item Drucker loss of stability model
    \item Johnson-Cook damage model
    \item Hancock-MacKenzie damage model
  \end{enumerate}
  \item Melting model
  \begin{enumerate}
    \item Constant melting temperature model
    \item Linear melting temperature model
    \item BPS melting model
    \item Steinberg-Guinan melting temperature model
  \end{enumerate}
  \item Specific heat model 
  \begin{enumerate}
    \item Constant specific heat
    \item Cubic specific heat model
    \item Copper specific heat model
    \item Steel specific heat model
  \end{enumerate}
\end{enumerate}



