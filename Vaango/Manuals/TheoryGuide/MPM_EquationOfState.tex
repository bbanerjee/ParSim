\chapter{Equation of state models} \label{ch:EOS}
In the isotropic metal plasticity models implemented in \Vaango, the volumetric 
part of the Cauchy stress can be calculated using an equation of state.  The 
equations of state that are implemented in \Vaango are described below.

\section{Hypoelastic equation of state}
In this case we assume that the stress rate is given by
\Beq
    \dot{\Bsig} = \lambda~\Tr(\BdT^e)~\BI + 2~\mu~\BdT^e
\Eeq
where $\Bsig$ is the Cauchy stress, $\BdT^e$ is the elastic part of
the rate of deformation, and $\lambda, \mu$ are constants.

If $\Dev(\BdT^e)$ is the deviatoric part of $\BdT^e$ then we can write
\Beq \label{eq:sigdot_hypo}
    \dot{\Bsig} = \left(\lambda + \frac{2}{3}~\mu\right)~\Tr(\BdT^e)~\BI + 
        2~\mu~\Dev(\BdT^e) = \kappa~\Tr(\BdT^e)~\BI + 2~\mu~\Dev(\BdT^e) ~.
\Eeq
If we split $\Bsig$ into a volumetric and a deviatoric part, i.e.,
$\Bsig = p~\BI + \Bs$, take the time derivative to get
$\dot{\Bsig} = \dot{p}~\BI + \dot{\Bs}$, and compare the result with
\eqref{eq:sigdot_hypo}, we see that
\Beq
    \dot{p} = \kappa~\Tr(\BdT^e) ~.
\Eeq
In addition we assume that $\BdT = \BdT^e + \BdT^p$.  If we also assume that 
the plastic volume change is negligible ($\Tr(\BdT^p) \approx 0$), 
which is reasonable for a void-free metal matrix, we have
\Beq
    \dot{p} = \kappa~\Tr(\BdT) ~.
\Eeq
This is the equation that is used to calculate the pressure $p$ in the 
default hypoelastic equation of state.  For a forward Euler integration step,
\Beq
    \boxed{
    p_{n+1} = p_n + \kappa~\Tr(\BdT_{n+1})~\Delta t ~.
    }
\Eeq
To get the derivative of $p$ with respect to $J$, where $J = \det(\BF)$,
we note that
\Beq
    \dot{p} = \Partial{p}{J}~\dot{J} = \Partial{p}{J}~J~\Tr(\BdT) ~.
\Eeq
Therefore,
\Beq
    \boxed{
    \Partial{p}{J} = \cfrac{\kappa}{J} ~.
    }
\Eeq

This model is invoked in \Vaango using
\lstset{language=XML}
\begin{lstlisting}
  <equation_of_state type="default_hypo">
  </equation_of_state>
\end{lstlisting}

\section{Default hyperelastic equation of state}
In this model the pressure is computed using the relation
\begin{equation}
  p = \Half~\kappa~\left(J^e - \cfrac{1}{J^e}\right)
\end{equation}
where $\kappa$ is the bulk modulus and $J^e$ is determinant of the elastic 
part of the deformation gradient.

We can also compute
\begin{equation}
  \Deriv{p}{J} = \Half~\kappa~\left(1 + \cfrac{1}{(J^e)^2}\right) ~.
\end{equation}

The metal plasticity implementations in \Vaango assume that the volume
change of the matrix during plastic deformation can be neglected, i.e., $J^e = J$.

This model is invoked using
\lstset{language=XML}
\begin{lstlisting}
  <equation_of_state type="default_hyper">
  </equation_of_state>
\end{lstlisting}

\section{Mie-Gruneisen equation of state}
The pressure ($p$) is calculated using a Mie-Gr{\"u}neisen equation of state 
of the form (\cite{Wilkins1999,Zocher2000})
\begin{equation} \label{eq:EOSMG_upd}
  p =  - \frac{\rho_0~C_0^2~(1 - J^e)
           [1 - \Gamma_0 (1 - J^e)/2]}
           {[1 - S_{\alpha}(1 - J^e)]^2} - \Gamma_0~E 
  ~;~~~ J^e := \det{\BF^e} 
\end{equation}
where $C_0$ is the bulk speed of sound, $\rho_0$ is the initial mass density,
$\Gamma_0$ is the Gr{\"u}neisen's gamma at the reference state,
$S_{\alpha} = dU_s/dU_p$ is a linear Hugoniot slope coefficient,
$U_s$ is the shock wave velocity, $U_p$ is the particle velocity, and
$E$ is the internal energy density (per unit reference volume), $\BF^e$ is
the elastic part of the deformation gradient.  For isochoric plasticity,
\begin{equation*}
  J^e = J = \det(\BF) = \cfrac{\rho_0}{\rho} ~.
\end{equation*}
  The internal energy is computed using
  \begin{equation}
    E = \frac{1}{V_0} \int C_v dT \approx \frac{C_v (T-T_0)}{V_0}
  \end{equation}
  where $V_0 = 1/\rho_0$ is the reference specific volume at temperature 
  $T = T_0$, and $C_v$ is the specific heat at constant volume.

Also,
\Beq
  \boxed{
  \Partial{p}{J^e} = 
  \cfrac{\rho_0~C_0^2~[1 + (S_{\alpha}-\Gamma_0)~(1-J^e)]}
        {[1-S_{\alpha}~(1-J^e)]^3} - \Gamma_0~\Partial{E}{J^e}.
  }
\Eeq
We neglect the $\Partial{E}{J^e}$ term in our calculations.

This model is invoked in Vaango using
\lstset{language=XML}
\begin{lstlisting}
  <equation_of_state type="mie_gruneisen">
    <C_0>5386</C_0>
    <Gamma_0>1.99</Gamma_0>
    <S_alpha>1.339</S_alpha>
    <rho_0> 7200 </rho_0>
  </equation_of_state>
\end{lstlisting}

An alternative formulation is also available that can be used for models where a 
linear Hugoniot is not accurate enough.  A cubic model can be used in that
formulation.
\Beq
  p_{n+1} =  - \frac{\rho_0~C_0^2~(1 - J^e_{n+1})
           [1 - \Gamma_0 (1 - J^e_{n+1})/2]}
           {[1 - S_{\alpha}(1 - J^e_{n+1}) - S_2 (1 - J^e_{n+1})^2 - S_3 (1 - J^e_{n+1})^3]^2} - \Gamma_0~e_{n+1} 
  ~;~~~ J^e := \det{\BF^e} 
\Eeq
This model is invoked using the label \Textbfc{mie\_gruneisen\_energy}.

\section{Equations of state used in the ARENA model}
In many models, a tangent bulk modulus is computed using the equation of state and
the pressure is updated using an integration step.  While this approach less accurate 
than directly evaluating the equation of state, it is useful when a composite 
material is being simulated that does not have well-characterized equations of state 
at all states.

The equations of state used by the ARENA model for soils are described below.  The bars above
quantities indicate negation.

\subsection{Solid matrix material}
The pressure in the solid matrix is expressed as
\Beq \label{eq:eos_matrix}
  \pbar_s = K_s \bar{\Veps_v^s} ~;~~ \bar{\Veps_v^s} := \ln\left(\frac{V_{s0}}{V_s}\right)
\Eeq
where $\pbar_s = -p_s$ is the solid matrix pressure, $K_s$ is the solid bulk modulus,
$\bar{\Veps_v^s}$ is the volumetric strain, $V_{s0}$ is the
initial volume of the solid, and $V_s$ is the current volume of the solid.  The solid
bulk modulus is assumed to modeled by the Murnaghan equation:
\Beq
  K_s(\pbar_s) = K_{s0} + n_s\,(\pbar_s - \pbar_{s0})
\Eeq
where $K_{s0}$ and $n_s$ are material properties, and $\pbar_{s0}$ is a reference pressure.

\subsection{Pore water}
The equation of state of the pore water is
\Beq \label{eq:eos_water}
  \pbar_w = K_w \bar{\Veps_v^w} + \pbar_0 ~;~~ \bar{\Veps_v^w} := \ln\left(\frac{V_{w0}}{V_w}\right)
\Eeq
where $\pbar_w = -p_w$ is the water pressure, $K_w$ is the water bulk modulus, $V_{w0}$ is the
initial volume of water, $V_w$ is the current volume of water, $\pbar_0$ is the
initial water pressure, and $\bar{\Veps}_v^w$ is the volumetric strain in the water.  We use the
isothermal Murnaghan bulk modulus model for water:
\Beq
  K_w(\pbar_w) = K_{w0} + n_w\,(\pbar_w - \pbar_{w0})
\Eeq
where $K_{w0}$ and $n_w$ are material properties, and $\pbar_{w0}$ is a reference pressure.

\subsection{Pore air}
The isentropic ideal gas equation of state for the pore air is
\Beq\label{eq:eos_air}
  \pbar_a = \pbar_r\left[\exp(\gamma\,\bar{\Veps_v^a}) - 1 \right]
  ~;~~ \bar{\Veps_v^a} := \ln\left(\frac{V_{a0}}{V_a}\right)
\Eeq
where the quantities with subscript $a$ represent quantities for the air model analogous to those
for the water model in~\eqref{eq:pw}, $\pbar_r$ is a reference pressure (101325 Pa) and $\gamma = 1.4$.
The bulk modulus of air ($K_a$) varies with the volumetric strain in the air:
\Beq
  K_a = \Deriv{\pbar_a}{\bar{\Veps_v^a}} = \gamma\,\pbar_r\,\exp(\gamma\,\bar{\Veps_v^a})
      = \gamma\,(\pbar_a + \pbar_r) \,.
\Eeq

