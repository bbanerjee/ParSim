\section{Compressible Neo-Hookean Model} There are implementations of several
hyperelastic-plastic model described by Simo and Hughes\cite{Simo1998} (pp. 307 -- 321). 
 The model is dubbed "Unified Compressible Neo-Nookean Model" or UCNH for short.  Models can 
still be specified with old input file specifications, (i.e. comp\_neo\_hook, comp\_neo\_hook\_plastic,
cnh\_damage, cnhp\_damage) however these are merely wrappers for the underyling UCNH model.
 Plastic flow and failure can be modelled in addition to elasticity by  specifying 
several additional options with input flags. This models is very robust, and relatively 
straightforward because hyperelastic models don't require rotation back and forth 
between laboratory and material frames of reference.

NOTE: Support for Implicit CNH and CNH with specified solver is still lacking (for a short time).

The basic input section for UCNH:

\begin{lstlisting}
            <constitutive_model type="UCNH"> 
              <!-- Necessary flags for all CNH models -->
               <bulk_modulus>   8.9e9  </bulk_modulus>
               <shear_modulus>  3.52e9 </shear_modulus>
               <useModifiedEOS> true   </useModifiedEOS>
                
              <!-- Plasticity Parameters -->
               <usePlasticity>     true  </usePlasticity>
               <yield_stress>      100.0 </yield_stress>
               <hardening_modulus> 500.0 </hardening_modulus>
               <alpha>             1.0   </alpha>

            </constitutive_model>
\end{lstlisting}

A fairly sophisticated means of seeding explicit material heterogeneity is also provided for. 
To use these features the following four steps are required: 

1. To allow for failure (by material point erosion) the following must be set: 

In the \tt <MPM> \normalfont block, the erosion algorithm must be set to one of the following: 

\begin{lstlisting}
<erosion algorithm="AllowNoTension"/>
<erosion algorithm="AllowNoShear"/>
<erosion algorithm="ZeroStress"/>
\end{lstlisting}

In the \tt <constitutive\_model> \normalfont block:
\begin{lstlisting}
<useDamage>true</useDamage>
\end{lstlisting}

2. The failure surface type must be specified.  This is also in the \tt <constitutive\_model> \normalfont 
block.  One of the following must be specified:

\begin{lstlisting}
<failure_criteria>  MohrCoulomb </failure_criteria>
<failure_criteria>  MaximumPrincipalStress </failure_criteria>
<failure_criteria>  MaximumPrincipalStrain </failure_criteria>
\end{lstlisting}

The cohesion, $c$, is assigned using a distribution, as described below.  For the maximum 
principal stress and strain failure criteria, the cohesion is the maximum value of principal 
stress or strain that may be obtained (must be positive).  The MohrCoulomb failure criteria is
given by 
\begin{equation}
\frac{\sigma_3-\sigma_1}{2}=c\cos(\phi)-\frac{\sigma_3+\sigma_1}{2}\sin(\phi)
\end{equation}
where $\sigma_i$ are the ordered principal stresses, positive in tension 
($\sigma_3 > \sigma_2 > \sigma_1$).  Note, the MohrCoulomb failure 
surface also requires a friction angle, $\phi$, (in degrees):

\begin{lstlisting}
<friction_angle> friction angle </friction_angle>
\end{lstlisting}

A tensile cutoff failure surface may be added for MohrCoulomb.
The tensile cutoff is taken to be a fraction of the cohesion.  
This parameter is specified using:

\begin{lstlisting}
<tensile_cutoff_fraction> 0.1 </tensile_cutoff_fraction> 
\end{lstlisting}

Setting this to a large number effectively removes this failure surface, leaving just Mohr-Coulomb.

3. Material heterogeneity type must be specified.  For MohrCoulomb the cohesion is distributed 
spatially (an independent assignment for each material point).  For MaximumPrincipalStress and 
MaximumPrincipalStrain, the threshold stress or strain for failure, respectively, is distributed 
spatially (an independent assignment for each material point).  Material heterogeneity is 
distributed spatially by assigning values consistent with a distribution function.  Three different 
distributions may be used.  All parameters are in the \tt <constitutive\_model> \normalfont block:

\begin{lstlisting}
<failure_distrib> gauss </failure_distrib>
<failure_distrib> weibull </failure_distrib>
<failure_distrib> constant </failure_distrib>
\end{lstlisting}

A Gaussian (gauss) distribution requires the following parameters:

\begin{lstlisting}
<failure_mean> Gaussian mean value of cohesion </failure_mean>
<failure_std> Gaussian standard deviation of cohesion </failure_std>
<failure_seed> random number generator seed </failure_seed>
\end{lstlisting}

A Weibull (weibull) distribution requires the following parameters:
\begin{lstlisting}
<failure_mean> Weibull mean value of cohesion </failure_mean>
<failure_std> Weibull modulus </failure_std>
<failure_seed> random number generator seed </failure_seed>
\end{lstlisting}

A homogeneous (constant) assignment requires the following parameters:
\begin{lstlisting}
<failure_mean> value (all particles assigned one value) </failure_mean>
\end{lstlisting}

4. Distribution scaling with numerical resolution may optionally be specified.  This is only 
available for Gaussian and Weibull distributions.  All parameters are in the 
\tt <constitutive\_model> \normalfont block:

\begin{lstlisting}
<scaling> kayenta </scaling>
<scaling> none (default) </scaling>
\end{lstlisting}

For kayenta scaling, the mean value of the distribution is scaled by the factor
\begin{equation}
\biggl(\frac{\bar V}{V}\biggr)^{1/n}
\end{equation}
where $V$ is the particle volume, a function of numerical resolution.  The reference volume, 
$\bar V$ and exponent, $n$, both must be specified
\begin{lstlisting}
<reference_volume> $\bar V$ </reference_volume>
<exponent> n </exponent>
\end{lstlisting}
The exponent defaults to the Weibull modulus if the Weibull distribution is used.  This physically
motivated scaling provides for an increase in mean cohesion with decreasing particle size, generallly
consistent with the observation that smaller quantities of material contain fewer critical flaws.

\tt <comp\_neo\_hook> \normalfont is a basic elastic model, which calls the underlying \tt <UCNH> \normalfont .
The specifications for CNH are:

\begin{lstlisting}
          <constitutive_model type="comp_neo_hook">
               <bulk_modulus> 8.9e9  </bulk_modulus>
               <shear_modulus>3.52e9  </shear_modulus>
               <useModifiedEOS> true  </useModifiedEOS>
          </constitutive_model>
\end{lstlisting}

\tt <comp\_neo\_hook\_plastic> \normalfont as the constitutive model type,
 tells Uintah to use the basic elastic model extended
to include plasticity with isotropic linear hardening, 
which is equivalent to \tt <usePlasticity> \normalfont in UCNH.
The specifications for CNHP are:

\begin{lstlisting}
          <constitutive_model type="comp_neo_hook_plastic">
               <bulk_modulus> 8.9e9     </bulk_modulus>
               <shear_modulus>3.52e9    </shear_modulus>
               <useModifiedEOS> true    </useModifiedEOS>
               <yield_stress>100.0      </yield_stress>
               <hardening_modulus>500.0 </hardening_modulus>
               <alpha>     1.0          </alpha>
          </constitutive_model>
\end{lstlisting}


\tt <cnh\_damage> \normalfont as the constiutive model or \tt <useDamage> \normalfont 
tells Uintah to use a basic elastic model, with an extension
to failure based on a stress or strain as given below, thus yielding an
elastic-brittle failure model.  This model also allows a distribution
of failure strain (or stress) based on normal or Weibull distributions.
Note that the post-failure behaviour of simulations is not always robust.

The specification for CNHD are:

\begin{lstlisting}
            <constitutive_model type="cnh_damage"> 
               <bulk_modulus>   8.9e9  </bulk_modulus>
               <shear_modulus>  3.52e9 </shear_modulus>
               <useModifiedEOS> true   </useModifiedEOS>
                
            </constitutive_model>
\end{lstlisting}

When specifying \tt <cnh\_damage> \normalfont, the material heterogeneity and damage 
specification described for the general model (UCNH) may also be specified.

\tt <cnhp\_damage> \normalfont as the constitutive model (or both damage and plasticity
 flags discussed above) uses an extension
to failure based on a stress or strain as given below, thus yielding an
elastic-plastic model with failure.  Note that the post-failure behaviour of
simulations is not always robust.  The input section for damage and plasticity is
similar to that for UCNH without \tt <useDamage> \normalfont and \tt <usePlasticity>. \normalfont

When a particle has failed, the value of the particle variable \tt p.localized \normalfont
will be larger than one (0 means the particle has not failed) and can be output in the \tt DataArchiver
\normalfont section of the input file. In addition, the total number of failed particles as
a function of time \tt TotalLocalizedParticle \normalfont can be output. 

Another damage model that can be used with \tt <cnh\_damage> \normalfont and
\tt <cnhp\_damage> \normalfont is a subset of the brittle damage model of LS-DYNA's Concrete
 Model 159 (FHWA-HRT-057-062, 2007). The model is invoked by the following MPMFlag
\begin{lstlisting}
     <erosion algorithm="BrittleDamage"/>
\end{lstlisting}
in the \tt <MPM> \normalfont section of the input file. Two key features of the model are the use of
progressive (as opposed to sudden) damage due to softening to improve numerical stability, 
and the reduction of mesh size sensitivity via the specification of fracture energy. 

Brittle damage occurs when the mean stress $\sigma_{kk}/3$ is tensile
and the energy $\tau_b$, related to the maximum principal 
strain $\epsilon_{max}$, has exceeded a threshold value $r_0^b$

\begin{equation}
\sigma_{kk}>0, \phantom{ijkl}
\tau_b = \sqrt{E \epsilon_{max}^2} \geq r_0^b
\end{equation}
where $E$ is the Young's modulus. If at the next time step the mean stress is less than
zero (compressive), the damage mechanism can be optionally inactivated such that the current stress 
is set temporarily to a fraction of the undamaged stress to 
model stiffness recovery due to crack closing. When the mean stress becomes tensile again,
the value of the previous maximum damage $d$ can be restored; recovery is a user option in Uintah
but should be used with caution since stiffening is more prone to instability.
The softening function for brittle damage is assumed to be
\begin{equation}
d(\tau_b)= \frac{0.999}{D} \left(\frac{1+D}{1+D \exp^{-C(\tau_b-r_0^b)}} \right)
\end{equation}
where $C$ and $D$ are constants that define the shape of the softening
stress-strain curve.
  
To regulate mesh size sensitivity, the fracture energy
($G_f$), defined as the area under the stress-displacement
curve for displacement larger than $x_0$ (the displacement at peak strength), is to be maintained constant.
The user needs to input $G_f$ and $D$; $C$ is calculated internally. 

The maximum increment of damage that can accumulate over a single time
step is a user-defined input to avoid excessive damage accumulation over
a single time step to reduce numerical instability.

The brittle damage model is applicable to \tt <cnh\_damage> \normalfont and 
\tt <cnhp\_damage>. \normalfont For \tt cnh\_damage, \normalfont 
the parameters for brittle damage can be specified as

\begin{lstlisting}
                <constitutive_model type="cnh_damage">
               <shear_modulus>3.52e9</shear_modulus>
                 <bulk_modulus>8.9e9</bulk_modulus>
   <brittle_damage_initial_threshold>57.0 </brittle_damage_initial_threshold>
  <brittle_damage_fracture_energy>11.2</brittle_damage_fracture_energy>
  <brittle_damage_constant_D>0.1</brittle_damage_constant_D>
 <brittle_damage_max_damage_increment>0.1</brittle_damage_max_damage_increment>
<brittle_damage_allowRecovery> false </brittle_damage_allowRecovery>
<brittle_damage_recoveryCoeff> 1.0 </brittle_damage_recoveryCoeff>
   <brittle_damage_printDamage> false </brittle_damage_printDamage>
                </constitutive_model>

\end{lstlisting}

The tags in the input file for brittle damage are shown in the following table.

\begin{table}[ht]
\centering
\begin{tabular} {c c l}
\hline
Tag & Symbol & Description \\
\hline
brittle\_damage\_initial\_threshold & $r_0^b$ &  material property \\
brittle\_damage\_fracture\_energy & $G_f$ &  material property \\
brittle\_damage\_constant\_D & $D$ & material property \\
brittle\_damage\_max\_damage\_increment & & optional, default=0.1 \\
brittle\_damage\_allowRecovery & & allow crack closing (stiffening) \\
& & optional, default=false \\
brittle\_damage\_recoveryCoeff & & fraction of undamaged stress to recover\\
& & (between 0 and 1), optional\\
& & default=1.0 (full recovery) used only\\
& & when brittle\_damage\_allowRecovery \\
& & is set to true  \\
brittle\_damage\_printDamage & & print the state of damage \\
& & of damaged particles, default=false \\
& & (to reduce large amounts of output) \\
\hline
\end{tabular}
\end{table}

When a particle is damaged, the value of the particle variable \tt p.damage \normalfont
can be output in the \tt DataArchiver \normalfont section of the input file.

