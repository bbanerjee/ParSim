

\subsection{Implementation}
The elastic response is assumed to be isotropic.  The material
constants that are taken as input for the elastic response are the
bulk and shear modulus.  The flow rule is determined from the input
and the appropriate plasticity model is created using the
\TT{PlasticityModelFactory} class.  The damage evolution rule
is determined from the input and a damage model is created using
the \TT{DamageModelFactory} class.  The equation of state
that is used to determine the pressure is also determined from the
input.  The equation of state model is created using the
\TT{MPMEquationOfStateFactory} class.

In addition, a damage evolution variable ($D$) is stored at each time
step (this need not be the case and will be transfered to the
damage models in the future).  The left stretch and rotation are
updated incrementally at each
time step (instead of performing a polar decomposition) and the
rotation tensor is used to rotate the Cauchy stress and rate of deformation
to the material coordinates at each time step (instead of using a
objective stress rate formulation).

Any evolution variables for the plasticity model, damage model or the
equation of state are specified in the class that encapsulates the
particular model.

The flow stress is calculated from the plasticity model using a
function call of the form
\begin{lstlisting}
    double flowStress = d_plasticity->computeFlowStress(tensorEta, tensorS, 
                                                        pTemperature[idx],
                                                        delT, d_tol, matl, idx);
\end{lstlisting}
A number of plasticity models can be evaluated using the inputs in the
\TT{computeFlowStress} call.  The variable \TT{d_plasticity} is
polymorphic and can represent any of the plasticity models that can be
created by the plasticity model factory.  The plastic evolution variables
are updated using a polymorphic function along the lines of
\TT{computeFlowStress}.

The equation of state is used to calculate the hydrostatic stress using
a function call of the form
\begin{lstlisting}
    Matrix3 tensorHy = d_eos->computePressure(matl, bulk, shear, 
                                              tensorF_new, tensorD, 
                                              tensorP, pTemperature[idx], 
                                              rho_cur, delT);
\end{lstlisting}

Similarly, the damage model is called using a function of the type
\begin{lstlisting}
    double damage = d_damage->computeScalarDamage(tensorEta, tensorS, 
                                                  pTemperature[idx],
                                                  delT, matl, d_tol, 
                                                  pDamage[idx]);
\end{lstlisting}

Therefore, the plasticity, damage and equation of state models are
easily be inserted into any other type of stress update algorithm
without any change being needed in them as can be seen in the
hyperelastic-plastic stress update algorithm discussed below.


\subsubsection{Adding new models}
  In the parallel implementation of the stress update algorithm, sockets have
  been added to allow for the incorporation of a variety of plasticity, damage,
  yield, and bifurcation models without requiring any change in the stress
  update code.  The algorithm is shown in Algorithm~\ref{algo1}.  The
  equation of state, plasticity model, yield condition, damage model, and
  the stability criterion are all polymorphic objects created using a
  factory idiom in C++~(\cite{Coplien92}).
  \begin{table}[p]
    \caption{Stress Update Algorithm} \label{algo1}
    \vspace{12pt}
    \begin{tabbing}
    \quad \=\quad \=\quad \=\quad \=\quad \kill
    {\bf Persistent}:Initial moduli, temperature, porosity, \\
      \>\>        scalar damage, equation of state, plasticity model, \\
      \>\>        yield condition, stability criterion, damage model\\
    {\bf Temporary}:Particle state at time $t$ \\
    {\bf Output:} Particle state at time $t+\Delta t$\\ \\

    {\bf For} {\it all the patches in the domain}\\
      \> Read the particle data and initialize updated data storage\\
      \> {\bf For} {\it all the particles in the patch}\\
      \>\>   Compute the velocity gradient and the rate of deformation tensor\\
      \>\>   Compute the deformation gradient and the rotation tensor\\
      \>\>   Rotate the Cauchy stress and the rate of deformation tensor \\
      \>\>\> to the material configuration\\
      \>\>   Compute the current shear modulus and melting temperature\\
      \>\>   Compute the pressure using the equation of state,  \\
      \>\>\>  update the hydrostatic stress, and  \\
      \>\>\>  compute the trial deviatoric stress\\
      \>\>   Compute the flow stress using the plasticity model\\
      \>\>   Evaluate the yield function\\
      \>\>   {\bf If} {\it particle is elastic} \\
      \>\>\>     Update the elastic deviatoric stress from the trial stress\\
      \>\>\>     Rotate the stress back to laboratory coordinates\\
      \>\>\>     Update the particle state\\
      \>\>   {\bf Else} \\
      \>\>\>     Compute the elastic-plastic deviatoric stress\\
      \>\>\>     Compute updated porosity, scalar damage, and \\
      \>\>\>\>       temperature increase due to plastic work\\
      \>\>\>     Compute elastic-plastic tangent modulus and
                     evaluate stability condition\\
      \>\>\>     Rotate the stress back to laboratory coordinates\\
      \>\>\>     Update the particle state\\
      \>\>   {\bf End If} \\
      \>\>  {\bf If}
             {\it Temperature $>$ Melt Temperature} or
             {\it Porosity $>$ Critical Porosity} or
             {\it Unstable}\\
      \>\>\>       Tag particle as failed\\
      \>\>  {\bf End If} \\
      \>\> Convert failed particles into a material with a different
           velocity field \\
      \> {\bf End For} \\
    {\bf End For}
    \end{tabbing}
  \end{table}

Addition of a new model requires the following steps (the example below is only
for the flow stress model but the same idea applies to other models) :
\begin{enumerate}
    \item Creation of a new class that encapsulates the plasticity
    model.  The template for this class can be copied from the
    existing plasticity models.  The data that is unique to
    the new model are specified in the form of
    \begin{itemize}
      \item A structure containing the constants for the plasticity
            model.
      \item Particle variables that specify the variables that
            evolve in the plasticity model.
    \end{itemize}
    \item The implementation of the plasticity model involves the
    following steps.
    \begin{itemize}
      \item Reading the input file for the model constants in the
            constructor.
      \item Adding the variables that evolve in the plasticity model
            appropriately to the task graph.
      \item Adding the appropriate flow stress calculation method.
    \end{itemize}
    \item The \TT{PlasticityModelFactory} is then modified so that
          it recognizes the added plasticity model.
\end{enumerate}

