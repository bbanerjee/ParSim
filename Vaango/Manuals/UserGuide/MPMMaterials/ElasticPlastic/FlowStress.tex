\subsection{Flow Stress}
  We have explored seven temperature and strain rate dependent
  models that can be used to compute the flow stress.  Some of these are also
  pressure dependent (note that plastic flow does is non-associative for
  pressure dependent models):
  \begin{enumerate}
    \item the Isotropic Hardening model
    \item the Johnson-Cook (JC) model
    \item the Steinberg-Cochran-Guinan-Lund (SCG) model.
    \item the Zerilli-Armstrong (ZA) model.
    \item the Zerilli-Armstrong for polymers model.
    \item the Mechanical Threshold Stress (MTS) model.
    \item the Preston-Tonks-Wallace (PTW) model.
  \end{enumerate}

  \subsubsection{Isotropic Hardening Flow Stress Model}
  The Isotropic Hardening model is a simple linear relationship for the flow stress
  \begin{equation}
    \sigma_y(\Ep) = \sigma_y + K (\Ep)
  \end{equation}
  where $\Ep$ is the equivalent plastic strain, $\sigma_y$ and $k$ are material constants.

  The inputs for this model are
  \begin{lstlisting}[language=XML]
  <flow_model type="isotropic_hardening">
    <sigma_Y>792.0e6</sigma_y>
    <K>510.0e6</K>
  </flow_model>
  \end{lstlisting}

  \subsubsection{JC Flow Stress Model}
  The Johnson-Cook (JC) model (\cite{Johnson83}) is purely empirical and gives
  the following relation for the flow stress ($\sigma_y$)
  \begin{equation}
    \sigma_y(\Ep,\Epdot{},T) = 
    \left[A + B (\Ep)^n\right]\left[1 + C \ln(\Epdot{}^{*})\right]
    \left[1 - (T^*)^m\right]
  \end{equation}
  where $\Ep$ is the equivalent plastic strain, $\Epdot{}$ is the
  plastic strain rate, A, B, C, n, m are material constants,
  \begin{equation}
    \Epdot{}^{*} = \cfrac{\Epdot{}}{\Epdot{0}}; \quad
    T^* = \cfrac{(T-T_0)}{(T_m-T_0)}~,
  \end{equation}
  $\Epdot{0}$ is a user defined plastic strain rate,
  $T_0$ is a reference temperature, and $T_m$ is the melt temperature.
  For conditions where $T^* < 0$, we assume that $m = 1$.

  The inputs for this model are
  \begin{lstlisting}[language=XML]
  <flow_model type="johnson_cook">
    <A>792.0e6</A>
    <B>510.0e6</B>
    <C>0.014</C>
    <n>0.26</n>
    <m>1.03</m>
    <T_r>298.0</T_r>
    <T_m>1793.0</T_m>
    <epdot_0>1.0</epdot_0>
  </flow_model>
  \end{lstlisting}

  \subsubsection{SCG Flow Stress Model}
  The Steinberg-Cochran-Guinan-Lund (SCG) model is a semi-empirical model
  that was developed by \cite{Steinberg80} for high strain rate
  situations and extended to low strain rates and bcc materials by
  \cite{Steinberg89}.  The flow stress in this model is given by
  \begin{equation}\label{eq:SCGL}
    \sigma_y(\Ep,\Epdot{},T) = 
     \left[\sigma_a f(\Ep) + \sigma_t (\Epdot{}, T)\right]
     \frac{\mu(p,T)}{\mu_0} 
  \end{equation}
  where $\sigma_a$ is the athermal component of the flow stress,
  $f(\Ep)$ is a function that represents strain hardening,
  $\sigma_t$ is the thermally activated component of the flow stress,
  $\mu(p,T)$ is the shear modulus, and $\mu_0$ is the shear modulus
  at standard temperature and pressure.  The strain hardening function
  has the form
  \begin{equation}
    f(\Ep) = [1 + \beta(\Ep + \Epi)]^n ; \quad
    \sigma_a f(\Ep) \le \sigma_{\text{max}}
  \end{equation}
  where $\beta, n$ are work hardening parameters, and $\Epi$ is the
  initial equivalent plastic strain.  The thermal component $\sigma_t$
  is computed using a bisection algorithm from the following equation (based
  on the work of \cite{Hoge77})
  \begin{equation}
    \Epdot{} = \left[\frac{1}{C_1}\exp\left[\frac{2U_k}{k_b~T}
      \left(1 - \frac{\sigma_t}{\sigma_p}\right)^2\right] + 
      \frac{C_2}{\sigma_t}\right]^{-1}; \quad
    \sigma_t \le \sigma_p
  \end{equation}
  where $2 U_k$ is the energy to form a kink-pair in a dislocation segment
  of length $L_d$, $k_b$ is the Boltzmann constant, $\sigma_p$ is the Peierls
  stress. The constants $C_1, C_2$ are given by the relations
  \begin{equation}
    C_1 := \frac{\rho_d L_d a b^2 \nu}{2 w^2}; \quad
    C_2 := \frac{D}{\rho_d b^2}
  \end{equation}
  where $\rho_d$ is the dislocation density, $L_d$ is the length of a
  dislocation segment, $a$ is the distance between Peierls valleys,
  $b$ is the magnitude of the Burgers' vector, $\nu$ is the Debye frequency,
  $w$ is the width of a kink loop, and $D$ is the drag coefficient.

  The inputs for this model are of the form
  \begin{lstlisting}[language=XML]
    <flow_model type="steinberg_cochran_guinan">
      <mu_0> 81.8e9 </mu_0>
      <sigma_0> 1.15e9 </sigma_0>
      <Y_max> 0.25e9 </Y_max>
      <beta> 2.0 </beta>
      <n> 0.50 </n>
      <A> 20.6e-12 </A>
      <B> 0.16e-3 </B>
      <T_m0> 2310.0 </T_m0>
      <Gamma_0> 3.0 </Gamma_0>
      <a> 1.67 </a>
      <epsilon_p0> 0.0 </epsilon_p0>
    </flow_model>
  \end{lstlisting}

  \subsubsection{ZA Flow Stress Model}
  The Zerilli-Armstrong (ZA) model (\cite{Zerilli87,Zerilli93,Zerilli04})
  is based on simplified dislocation mechanics.  The general form of the
  equation for the flow stress is
  \begin{equation}
    \sigma_y(\Ep,\Epdot{},T) = 
      \sigma_a + B\exp(-\beta(\Epdot{}) T) + 
                           B_0\sqrt{\Ep}\exp(-\alpha(\Epdot{}) T)
  \end{equation}
  where $\sigma_a$ is the athermal component of the flow stress given by
  \begin{equation}
    \sigma_a := \sigma_g + \frac{k_h}{\sqrt{l}} + K\Ep^n,
  \end{equation}
  $\sigma_g$ is the contribution due to solutes and initial dislocation
  density, $k_h$ is the microstructural stress intensity, $l$ is the
  average grain diameter, $K$ is zero for fcc materials,
  $B, B_0$ are material constants.  The functional forms of the exponents
  $\alpha$ and $\beta$ are
  \begin{equation}
    \alpha = \alpha_0 - \alpha_1 \ln(\Epdot{}); \quad
    \beta = \beta_0 - \beta_1 \ln(\Epdot{}); 
  \end{equation}
  where $\alpha_0, \alpha_1, \beta_0, \beta_1$ are material parameters that
  depend on the type of material (fcc, bcc, hcp, alloys).  The Zerilli-Armstrong
  model has been modified by \cite{Abed05} for better performance at high
  temperatures.  However, we have not used the modified equations in our
  computations.

  The inputs for this model are of the form
  \begin{lstlisting}[language=XML]
    
    <flow_model type="zerilli_armstrong">
       <sigma_g>     50.0e6   </sigma_g>
       <k_H>         5.0e6    </k_H>
       <sqrt_l_inv>  5.0      </sqrt_l_inv>
       <B>           25.0e6   </B>
       <beta_0>      0.0      </beta_0>
       <beta_1>      0.0      </beta_1>
       <B_0>         0.0      </B_0>
       <alpha_0>     0.0      </alpha_0>
       <alpha_1>     0.0      </alpha_1>
       <K>           5.0e9    </K>
       <n>           1.0      </n>
     </flow_model>
  \end{lstlisting}

  \subsubsection{ZA for Polymers Flow Stress Model}
  The Zerilli-Armstrong flow stress model for polymers(\cite{Zerilli07})
  is a modification to the ZA flow stress model for metals motivated by considering
  thermally activated processess appropriate to polymers, in place of
  dislocations.  The ZA flow stress function for polymers has three terms.  
  The first term accounts for a saturation of the flow stress to finite stress 
  at higher temperatures (Although such stress component is specified as ``athermal'' 
  it should follow the generally weaker temperature dependence of the elastic 
  shear modulus, hence the subscript ``g''). The second gives the yield stress as 
  a function of temperature and plastic strain rate. The third gives an increment 
  due to strain hardening, influenced by the pressure. The general form of the
  equation for the flow stress implemented in Uintah is
  \begin{equation}
    \sigma_y(\Ep,\Epdot{},p,T) = 
      \sigma_g + B\exp(-\beta(T-T_0)) + 
                           B_0\sqrt{\omega\Ep}\exp(-\alpha(T-T_0))
  \end{equation}
  where it should be noted that the equation is slightly modified from the
  original to include a reference temperature $T_0$ and an athermal stress, 
  $\sigma_g$, which is a constant.  The other terms are specified via
  \begin{equation}
    B=B_{pa}(1+B_{pb}\sqrt{p})^{B_{pn}};\quad
    B_0=B_{0pa}(1+B_{0pb}\sqrt{p})^{B_{0pn}};\quad
    \omega=\omega_a+\omega_b\ln(\Epdot{})+\omega_p \sqrt{p}
  \end{equation}
  where $B_{pa}$, $B_{pb}$, $B_{pn}$, $B_{0pa}$, $B_{0pb}$, $B_{0pn}$, $\omega_a$, $\omega_b$, 
  and $\omega_p$ are material parameters.  The functional forms of the exponents
  $\alpha$ and $\beta$ are (as in the original)
  \begin{equation}
    \alpha = \alpha_0 - \alpha_1 \ln(\Epdot{}); \quad
    \beta = \beta_0 - \beta_1 \ln(\Epdot{}); 
  \end{equation}
  where $\alpha_0, \alpha_1, \beta_0, \beta_1$ are material parameters.  Note that the
  pressure is taken to be min(p,0), eliminating pressure dependence in tension.

  The inputs for this model are of the form
  \begin{lstlisting}[language=XML]
    
         <flow_model type="zerilli_armstrong_polymer">
           <sigma_g>     50.0e6      </sigma_g>
           <B_pa>        0.0      </B_pa>
           <B_pb>        0.0      </B_pb>
           <B_pn>        0.0      </B_pn>
           <beta_0>      0.0      </beta_0>
           <beta_1>      0.0      </beta_1>
           <T_0>         0.0      </T_0>
           <B_0pa>       500.0e6      </B_0pa>
           <B_0pb>       0.0      </B_0pb>
           <B_0pn>       0.0      </B_0pn>
           <omega_a>     1.0      </omega_a>
           <omega_b>     0.0      </omega_b>
           <omega_p>     0.0      </omega_p>
           <alpha_0>     0.0      </alpha_0>
           <alpha_1>     0.0      </alpha_1>
         </flow_model>

  \end{lstlisting}

  \subsubsection{MTS Flow Stress Model}
  The Mechanical Threshold Stress (MTS) model
  (\cite{Follans88,Goto00a,Kocks01})
  gives the following form for the flow stress
  \begin{equation}
    \sigma_y(\Ep,\Epdot{},T) = 
      \sigma_a + (S_i \sigma_i + S_e \sigma_e)\frac{\mu(p,T)}{\mu_0} 
  \end{equation}
  where $\sigma_a$ is the athermal component of mechanical threshold stress,
  $\mu_0$ is the shear modulus at 0 K and ambient pressure,
  $\sigma_i$ is the component of the flow stress due to intrinsic barriers
  to thermally activated dislocation motion and dislocation-dislocation
  interactions, $\sigma_e$ is the component of the flow stress due to
  microstructural evolution with increasing deformation (strain hardening),
  ($S_i, S_e$) are temperature and strain rate dependent scaling factors.  The
  scaling factors take the Arrhenius form
  \begin{align}
    S_i & = \left[1 - \left(\frac{k_b~T}{g_{0i}b^3\mu(p,T)}
    \ln\frac{\Epdot{0i}}{\Epdot{}}\right)^{1/q_i}
    \right]^{1/p_i} \\
    S_e & = \left[1 - \left(\frac{k_b~T}{g_{0e}b^3\mu(p,T)}
    \ln\frac{\Epdot{0e}}{\Epdot{}}\right)^{1/q_e}
    \right]^{1/p_e}
  \end{align}
  where $k_b$ is the Boltzmann constant, $b$ is the magnitude of the Burgers'
  vector, ($g_{0i}, g_{0e}$) are normalized activation energies,
  ($\Epdot{0i}, \Epdot{0e}$) are constant reference strain rates, and
  ($q_i, p_i, q_e, p_e$) are constants.  The strain hardening component
  of the mechanical threshold stress ($\sigma_e$) is given by a
  modified Voce law
  \begin{equation}\label{eq:MTSsige}
    \frac{d\sigma_e}{d\Ep} = \theta(\sigma_e)
  \end{equation}
  where
  \begin{align}
    \theta(\sigma_e) & = 
       \theta_0 [ 1 - F(\sigma_e)] + \theta_{IV} F(\sigma_e) \\
    \theta_0 & = a_0 + a_1 \ln \Epdot{} + a_2 \sqrt{\Epdot{}} - a_3 T \\
    F(\sigma_e) & = 
      \cfrac{\tanh\left(\alpha \cfrac{\sigma_e}{\sigma_{es}}\right)}
      {\tanh(\alpha)}\\
    \ln(\cfrac{\sigma_{es}}{\sigma_{0es}}) & =
    \left(\frac{kT}{g_{0es} b^3 \mu(p,T)}\right)
    \ln\left(\cfrac{\Epdot{}}{\Epdot{0es}}\right)
  \end{align}
  and $\theta_0$ is the hardening due to dislocation accumulation,
  $\theta_{IV}$ is the contribution due to stage-IV hardening,
  ($a_0, a_1, a_2, a_3, \alpha$) are constants,
  $\sigma_{es}$ is the stress at zero strain hardening rate,
  $\sigma_{0es}$ is the saturation threshold stress for deformation at 0 K,
  $g_{0es}$ is a constant, and $\Epdot{0es}$ is the maximum strain rate.  Note
  that the maximum strain rate is usually limited to about $10^7$/s.

  The inputs for this model are of the form
  \begin{lstlisting}[language=XML]
    <flow_model type="mts_model">
      <sigma_a>363.7e6</sigma_a>
      <mu_0>28.0e9</mu_0>
      <D>4.50e9</D>
      <T_0>294</T_0>
      <koverbcubed>0.823e6</koverbcubed>
      <g_0i>0.0</g_0i>
      <g_0e>0.71</g_0e>
      <edot_0i>0.0</edot_0i>
      <edot_0e>2.79e9</edot_0e>
      <p_i>0.0</p_i>
      <q_i>0.0</q_i>
      <p_e>1.0</p_e>
      <q_e>2.0</q_e>
      <sigma_i>0.0</sigma_i>
      <a_0>211.8e6</a_0>
      <a_1>0.0</a_1>
      <a_2>0.0</a_2>
      <a_3>0.0</a_3>
      <theta_IV>0.0</theta_IV>
      <alpha>2</alpha>
      <edot_es0>3.42e8</edot_es0>
      <g_0es>0.15</g_0es>
      <sigma_es0>1679.3e6</sigma_es0>
    </flow_model>
  \end{lstlisting}

  \subsubsection{PTW Flow Stress Model}
  The Preston-Tonks-Wallace (PTW) model (\cite{Preston03}) attempts to
  provide a model for the flow stress for extreme strain rates
  (up to $10^{11}$/s) and temperatures up to melt.  The flow stress is
  given by
  \begin{equation}
    \sigma_y(\Ep,\Epdot{},T) = 
       \begin{cases}
         2\left[\tau_s + \alpha\ln\left[1 - \varphi
          \exp\left(-\beta-\cfrac{\theta\Ep}{\alpha\varphi}\right)\right]\right]
         \mu(p,T) & \text{thermal regime} \\
         2\tau_s\mu(p,T) & \text{shock regime}
       \end{cases}
  \end{equation}
  with
  \begin{equation}
    \alpha := \frac{s_0 - \tau_y}{d}; \quad
    \beta := \frac{\tau_s - \tau_y}{\alpha}; \quad
    \varphi := \exp(\beta) - 1
  \end{equation}
  where $\tau_s$ is a normalized work-hardening saturation stress,
  $s_0$ is the value of $\tau_s$ at 0K,
  $\tau_y$ is a normalized yield stress, $\theta$ is the hardening constant
  in the Voce hardening law, and $d$ is a dimensionless material
  parameter that modifies the Voce hardening law.  The saturation stress
  and the yield stress are given by
  \begin{align}
    \tau_s & = \max\left\{s_0 - (s_0 - s_{\infty})
       \erf\left[\kappa
         \That\ln\left(\cfrac{\gamma\Xidot}{\Epdot{}}\right)\right],
       s_0\left(\cfrac{\Epdot{}}{\gamma\Xidot}\right)^{s_1}\right\} \\
    \tau_y & = \max\left\{y_0 - (y_0 - y_{\infty})
       \erf\left[\kappa
         \That\ln\left(\cfrac{\gamma\Xidot}{\Epdot{}}\right)\right],
       \min\left\{
         y_1\left(\cfrac{\Epdot{}}{\gamma\Xidot}\right)^{y_2}, 
         s_0\left(\cfrac{\Epdot{}}{\gamma\Xidot}\right)^{s_1}\right\}\right\} 
  \end{align}
  where $s_{\infty}$ is the value of $\tau_s$ close to the melt temperature,
  ($y_0, y_{\infty}$) are the values of $\tau_y$ at 0K and close to melt,
  respectively, $(\kappa, \gamma)$ are material constants, $\That = T/T_m$,
  ($s_1, y_1, y_2$) are material parameters for the high strain rate
  regime, and
  \begin{equation}
    \Xidot = \frac{1}{2}\left(\cfrac{4\pi\rho}{3M}\right)^{1/3}
             \left(\cfrac{\mu(p,T)}{\rho}\right)^{1/2}
  \end{equation}
  where $\rho$ is the density, and $M$ is the atomic mass.

  The inputs for this model are of the form
  \begin{lstlisting}[language=XML]
    <flow_model type="preston_tonks_wallace">
      <theta> 0.025 </theta>
      <p> 2.0 </p>
      <s0> 0.0085 </s0>
      <sinf> 0.00055 </sinf>
      <kappa> 0.11 </kappa>
      <gamma> 0.00001 </gamma>
      <y0> 0.0001 </y0>
      <yinf> 0.0001 </yinf>
      <y1> 0.094 </y1>
      <y2> 0.575 </y2>
      <beta> 0.25 </beta>
      <M> 63.54 </M>
      <G0> 518e8 </G0>
      <alpha> 0.20 </alpha>
      <alphap> 0.20 </alphap>
    </flow_model>
  \end{lstlisting}
