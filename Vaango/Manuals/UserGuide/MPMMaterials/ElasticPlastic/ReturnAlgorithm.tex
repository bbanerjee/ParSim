\subsection{Return Algorithms}

Two return algorithms are presently available.  Both assume the direction
of plastic flow is proportional to the current deviatoric stress.
  \begin{enumerate}
    \item Radial Return ({\bf default}).
    \item Modified Nemat-Nasser/Maudlin Return Algorithm.
  \end{enumerate}

\subsubsection{Radial Return Algorithm}
The plastic state is obtained using an iterative radial return
procedure as described in \cite{Simo98}, page 124, except that the
Newton procedure has been generalized to allow flow stresses to be
functions of both equivalent plastic strain and equivalent plastic
strain rate.

The rotated spatial rate of deformation tensor ($\Bd$) is additively
decomposed into an elastic part, $\Bd^e$, and a plastic part, $\Bd^p$,
  \begin{equation}
     \Bd = \Bd^e + \Bd^p
  \end{equation}
  It is convenient to work with the deviatoric parts of $\Bd$,
  $\Bd^e$, and $\Bd^p$, denoted $\Beta$, $\Beta^e$, and
  $\Beta^p$ respectively. The same additive decomposition obtains
  \begin{equation}
     \Beta = \Beta^e + \Beta^p
  \end{equation}
  Presently these models are limited to the case of plastic
  incompressibility ($\Tr{(\Bd^p)} = 0$), so that $\Bd^p=\Beta^p$.

The radial return algorithm assumes a yield condition of the form 
  \begin{equation}\label{eq:yield_condition}
    f(\Bs,\Ep,\dot{\Ep}) = \sqrt{3J_2} - \sigma_y(\Ep,\dot{\Ep})
  \end{equation}
where $\sigma_y$ is the flow stress, $J_2 = {1\over2}\Bs:\Bs$ is the
second invariant of the deviatoric part of the Cauchy stress, $\Bs$,
and the equivalent plastic strain is defined as
  \begin{equation}\label{eq:ep}
    \Ep = \int_0^t\sqrt{{2\over3}\Bd^p:\Bd^p}dt = \int_0^t\sqrt{{2\over3}\Beta^p:\Beta^p}dt
  \end{equation}
Assuming a state at the end of the previous time step, time $t^n$,
satisfying $f(\Bs^n,\Ep^n,\Epdot{}^n)\le0$, a new state satisfying
Eqn. \ref{eq:yield_condition} at time $t^{n+1}=t^n+\Delta t$, where
$\Delta t$ is the time step size, is sought.

Attention is further restricted to plastic flow associated with the yield condition,
Eqn. \ref{eq:yield_condition}, i.e.
  \begin{equation}\label{eq:associated_flow}
    \Bd^p = \Beta^p \propto \Partial{f}{\Bsig} = \lambdadot{\Bs\over\norm{\Bs}} = \lambdadot\Bn
  \end{equation}
where $\Bsig$ is the Cauchy stress , $\Bn={\Bs/\norm{\Bs}}$ and
$\lambdadot>0$ is a proportionality constant to be determined.
Attention is also restricted to isotropic materials, for which the
deviatoric response may be separated from the volumetric response.
The linear hypoelastic/plastic constitutive equation for deviatoric
response is
  \begin{equation}\label{eq:linearhypoelasticity}
    \dot{\Bs} = 2\mu(\Beta - \Beta^p)
  \end{equation}
where $\mu$ is the shear modulus.  The shear modulus is required to be
constant over the time step.  This permits evolution of the shear
modulus based on the state at the beginning of the time step using the
various shear modulus models described later, which are pressure and
temperature dependent.  It also allows for a visoelastic deviatoric
stress response provided an instantaneous shear modulus may be
defined, as described later in this section for linear
hypoviscoelasticity.

A trial stress is calculated assuming no plastic deformation, i.e.
  \begin{equation}\label{eq:strial}
    {\Bs^{\text{trial}}} = \Bs^n + 2\mu\Beta\Delta t
  \end{equation}
If $f(\Bs^{\text{trial}},\Ep^n,0)\le0$, the deformation is purely
elastic and the solution at time $t^{n+1}$ is
$\Bs^{n+1}=\Bs^{\text{trial}}$, $\Ep^{n+1}=\Ep^n$.  If
$f(\Bs^{\text{trial}},\Ep^n,0)>0$, the deformation is at least
partially plastic.  In this case
  \begin{equation}\label{eq:stress_update_0}
    \Bs^{n+1} = \Bs^n + \dot{\Bs}\Delta t
  \end{equation}
where $\dot{\Bs}$ is given by Eqn. \ref{eq:linearhypoelasticity}.
Eqn. \ref{eq:stress_update_0} may be rewritten in terms of the trial
stress
  \begin{equation}\label{eq:stress_update_1}
    \Bs^{n+1} = \Bs^{\text{trial}} - 2\mu\Beta^p\Delta t = \Bs^{\text{trial}} - 2\mu\lambdadot\Delta t {\Bs^{n+1}\over\norm{\Bs^{n+1}}}
  \end{equation}
using Eqn.s \ref{eq:strial}, \ref{eq:linearhypoelasticity} and
\ref{eq:associated_flow}.  This equation may be rearranged to give
  \begin{equation}
    \Bs^{\text{trial}} = \Bs^{n+1} \left[ 1 + {2\mu\lambdadot\Delta t\over\norm{\Bs^{n+1}}}\right]
  \end{equation}
which gives the key result that $\Bs^{\text{trial}} \propto
\Bs^{n+1}$, i.e. the trial stress and the updated stress are in the
same direction.  Consequently the flow direction may be written
  \begin{equation}\label{eq:flow_direction}
    \Bn={\Bs^{n+1}\over\norm{\Bs^{n+1}}}={\Bs^{\text{trial}}\over\norm{\Bs^{\text{trial}}}}
  \end{equation}
and Eqn. \ref{eq:stress_update_1} may be rewritten
  \begin{equation}\label{eq:stress_update_2}
    \Bs^{n+1} = \Bs^{\text{trial}} - 2\mu\lambdadot\Delta t \Bn
  \end{equation}
or, contracting both sides with $\Bn$, and using Eqn. \ref{eq:flow_direction},
  \begin{equation}
    \norm{\Bs^{n+1}} = \norm{\Bs^{\text{trial}}} - 2\mu\lambdadot\Delta t
  \end{equation}
which is a scalar equation for the proportionality constant
$\lambdadot$.  Using the yield condition
$f(\Bs^{n+1},\Ep^{n+1},\Epdot{}^{n+1})=0$
(Eqn. \ref{eq:yield_condition}), this equation may be written in terms
of $\lambdadot$
  \begin{equation}\label{eq:stress_update_3}
     \sqrt{2\over3}\sigma_y(\Ep^{n+1},\Epdot{}^{n+1}) = \norm{\Bs^{\text{trial}}} - 2\mu\lambdadot\Delta t
  \end{equation}
where from Eqn.s \ref{eq:ep} and \ref{eq:associated_flow},
$\Ep^{n+1}=\Ep^n+\sqrt{2\over3}\lambdadot\Delta t$ and $\dot{\Ep}^{n+1}=\sqrt{2\over3}\lambdadot$.

For the special case of linear isotropic hardening,
$\sigma_y(\Ep,\Epdot{})=\sigma_{y0}+k\Ep$, where $\sigma_{y0}$ is the
initial yield stress and $k$ is the hardening modulus,
Eqn. \ref{eq:stress_update_3} may be solved exactly.  More generally
the solution may be found using Newton iteration.  Defining
$\Delta\lambda = \lambdadot\Delta t$, and letting $j$ denote the
iteration, define
  \begin{equation}
     g(\Delta\lambda_j) = \norm{\Bs^{\text{trial}}} - 2\mu\Delta\lambda_j - \sqrt{2\over3}\sigma_y(\Epj^{n+1},\dot{\epsilon}_{p,j}^{n+1})
  \end{equation}
and, using the chain rule, the derivative may be calculated
  \begin{equation}
     {dg\over d\Delta\lambda}(\Delta\lambda_j) = - 2\mu - {2\over3}\left[{\partial\sigma_y\over \partial\Ep}(\Epj^{n+1},\dot{\epsilon}_{p,j}^{n+1}) + {\partial\sigma_y\over \partial{\dot{\epsilon}_{p,j}^{n+1}}}(\Epj^{n+1},\dot{\epsilon}_{p,j}^{n+1}){1\over\Delta t} \right]
  \end{equation}
Then until $|g(\Delta\lambda_{j+1})| < \text{TOL}$, calculate
  \begin{equation}
     \Delta\lambda_{j+1}=\Delta\lambda_j-{g(\Delta\lambda_j)\over{dg\over d\Delta\lambda}(\Delta\lambda_j)}
  \end{equation}
where $\Delta\lambda_0=0$, $\Epo^{n+1}=\Ep^n$,
$\dot{\epsilon}_{p,0}^{n+1}=0$, and, once $\Delta\lambda$ has been
determined to a specified accuracy, the final values of plastic strain
and strain rate are given by
  \begin{equation}
    \Ep^{n+1}=\Ep^n+\sqrt{2\over3}\Delta\lambda_{j+1}
  \end{equation}
  \begin{equation}
    \dot{\Ep}^{n+1}=\sqrt{2\over3}{\Delta\lambda_{j+1}\over\Delta t}
  \end{equation}
and the final value of $\Bs^{n+1}$ is calculated from Eqn. \ref{eq:stress_update_2}.

While several of the allowed flow stresses are of the form
$\sigma_y(\Ep,\dot{\Ep})$ as given in Eqn. \ref{eq:yield_condition},
others include temperature and/or pressure dependence.  Similarly,
several of the shear moduli models are functions of temperature and/or
pressure.  Using the radial return algorithm in these cases amounts to
convergence to the yield surface neglecting temperature changes, and
assuming a non-associated flow rule of the form in
Eqn. \ref{eq:associated_flow} (which is non-associated because the
pressure dependence of the flow stress has been neglected,
i.e. Eqn. \ref{eq:associated_flow} no longer holds).  This non
associated flow rule results in zero plastic dilation (actually
dilitation).  The end result is convergence to the flow stress with
temperature and pressure held constant, i.e. to
$\sigma_y(\Ep^{n+1},\dot{\Ep}^{n+1},p^n,T^n)$ with $\mu(p^n,T^n)$ and
no plastic dilitation.  While holding pressure and temperature fixed
over a time step is probably a good approximation for most explicit
calculations, non-associated flow may not be.

Finally, it was found that this return algorithm worked equally well
for linear hypoviscoelastic deviatoric response, i.e.
  \begin{equation}\label{eq:linearhypoviscoelasticity}
    \dot{\Bs} = 2\mu(\Beta - \Beta^p) - \sum_{i=1}^N {\Bs_i\over \tau_i}
  \end{equation}
rather than Eqn. \ref{eq:linearhypoelasticity}.
Eqn. \ref{eq:linearhypoviscoelasticity} is the constitutive equation
for $N$ linear Maxwell elements in parallel, each with shear modulus
$\mu_i$, time constant $\tau_i$, and deviatoric stress $\Bs_i$, and
  \begin{equation}
    \Bs = \sum_{i=1}^N \Bs_i \quad\quad \mu = \sum_{i=1}^N \mu_i
  \end{equation}
This model is detailed in the Deviatoric Stress Models section.
Combined with a yield condition, the combination results in a
model for viscoplastic material response.

The radial return algorith is the {\bf default}, and also may be
explicitely invoked with the tag
  \begin{lstlisting}
    <plastic_convergence_algo>radialReturn</plastic_convergence_algo>
  \end{lstlisting}

\subsubsection{Modified Nemat-Nasser/Maudlin Return Algorithm}

This stress update algorithm is a slightly modified version of the
approach taken by Nemat-Nasser et
al. (1991,1992)~\cite{Nemat91,Nemat92}, Wang (1994)~\cite{Wang94},
Maudlin (1996)~\cite{Maudlin96}, and Zocher et
al. (2000)~\cite{Zocher00}.  It is presently only documented in the
code itself.  It is also known to give erroneous results under uniaxial
stress conditions.

The modified Nemat-Nasser/Maudlin return algorith is invoked with
the tag
  \begin{lstlisting}
    <plastic_convergence_algo>biswajit</plastic_convergence_algo>
  \end{lstlisting}
{\it This is an experts only algorithm!!!}

