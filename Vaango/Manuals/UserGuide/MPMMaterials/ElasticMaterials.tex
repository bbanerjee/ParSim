\section{Elastic materials}
Please refer to the \Vaango Theory Manual for brief descriptions of the theory used in 
these material models.

\subsection{Hypoelastic material}
A hypoelastic material model can be specified in the \Textsfc{UPS} input file using:
\begin{lstlisting}[language=XML]
  <constitutive_model type="hypo_elastic">
    <K> 32.0e6 </K>
    <G> 12.0e6 </G>
    <alpha> 1.0e-4 </alpha>
  </constitutive_model>
\end{lstlisting}

Here $K$ is the bulk modulus, $G$ is the shear modulus, and $\alpha$ is the coefficient
of thermal expansion.  If the Young's modulus ($E$) and Poisson's ratio ($\nu$) of the 
material are known, the bulk and shear modulus can be computed using
\Beq
  K = \frac{E}{3(1-2\nu)} \quad \Tand \quad
  G = \frac{E}{2(1+\nu)} \,.
\Eeq

\subsection{Elastic modulus models}
For selected hypoelasticity-based material models, the model used to compute the bulk and shear modulus 
can be chosen separately.  Some of these elastic moduli models are discussed below.

\subsubsection{Support vector model}
The support vector based elastic modulus model is specified using an input
description of the following form.
\begin{lstlisting}[language=XML]
  <constitutive_model type=".....">
    <elastic_moduli_model type="support_vector">
      <filename>SVR_fit.json</filename>
      <G0>3500</G0>
      <nu>0.189</nu>
    </elastic_moduli_model>
    ......
  </constitutive_model>
\end{lstlisting}

The JSON input file is required to contain a support vector regression fit to pressure data
as a function of the total volumetric strain and the plastic volumetric strain.  The format 
of the JSON file should be of the following form:
\begin{lstlisting}[language=JSON]
{
  "X_var": ["Total volumetric strain (\%)", "Plastic volumetric strain (\%)"],
  "y_var": "Pressure (MPa)",
  "X_conversion_factor": [0.01, 0.01],
  "y_conversion_factor": 1000000.0,
  "X_scale": [0.018115942028985508, 0.031249261081060988],
  "X_min": [-10.0, 0.0],
  "X_max": [45.2, 32.00075667088534],
  "y_scale": [0.00016356411948678396],
  "y_min": [-1112.690309606444],
  "y_max": [5001.12],
  "gamma": 4.7474110414223025,
  "support_vectors": [[0.9998641304347827, 1.0], [0.8832644927536233, 1.0],
    [0.9954365942028985, 1.0], [0.6534438405797102, 0.7060732743351001],
    [0.8108822463768116, 1.0], [0.7982101449275363, 0.943522446672424], ...],
  "dual_coeffs": [[10.0, 10.0, 10.0, -10.0, 10.0, 10.0, 2.5932059232147773, 10.0, ...],
  "intercept": [0.8428855972115061]
}
\end{lstlisting}

\subsection{Compressible Mooney-Rivlin Model} This model is generally parameterized
for rubber type materials.  Usage is as follows:
\begin{lstlisting}[language=XML]
  <constitutive_model type="comp_mooney_rivlin">
    <he_constant_1>100000.0</he_constant_1>
    <he_constant_2>20000.0</he_constant_2>
    <he_PR>.49</he_PR>
  </constitutive_model>
\end{lstlisting}
where \Textsfc{\textless he\_constant\_(1,2)\textgreater} are usually referred to
as $C1$ and $C2$ in the literature, and \Textsfc{he\_PR} is the Poisson's ratio ($\nu$).
The initial shear modlus, $G$, is related to the two Mooney-Rivlin constants and the 
initial bulk modulus can be computed from $G$ and $\nu$ using
\Beq
  G = 2 (C_1 + C_2) \quad \Tand \quad K = \frac{2G(1+\nu)}{3(1-2\nu)} \,.
\Eeq

\subsection{Compressible Neo-Hookean Model} There are implementations of several
hyperelastic-plastic model described by Simo and Hughes\cite{Simo1998} (pp. 307 -- 321). 
 The model is dubbed "Unified Compressible Neo-Nookean Model" or UCNH for short.  Models can 
still be specified with old input file specifications, (i.e. comp\_neo\_hook, comp\_neo\_hook\_plastic,
cnh\_damage, cnhp\_damage) however these are merely wrappers for the underyling UCNH model.
 Plastic flow and failure can be modelled in addition to elasticity by  specifying 
several additional options with input flags. This models is very robust, and relatively 
straightforward because hyperelastic models don't require rotation back and forth 
between laboratory and material frames of reference.

\begin{NoteBox}
NOTE: Support for Implicit CNH and CNH with specified solver does not exist yet.
\end{NoteBox}

\subsubsection{Purely elastic}
For purely-elastic compressible neo-Hookean material simulations, the input has the form:
\begin{lstlisting}[language=XML]
  <constitutive_model type="UCNH"> 
    <bulk_modulus>   8.9e9  </bulk_modulus>
    <shear_modulus>  3.52e9 </shear_modulus>
    <useModifiedEOS> true   </useModifiedEOS>
  </constitutive_model>
\end{lstlisting}
Alternatively, this model can be invoked usig the \Textxml{comp\_neo\_hook} tag:
\begin{lstlisting}[language=XML]
  <constitutive_model type="comp_neo_hook"> 
    <bulk_modulus>   8.9e9  </bulk_modulus>
    <shear_modulus>  3.52e9 </shear_modulus>
    <useModifiedEOS> true   </useModifiedEOS>
  </constitutive_model>
\end{lstlisting}

\subsubsection{Elastic with brittle damage}
The \Textsfc{cnh\_damage} tag or the \Textxml{useDamage} tag
tells \Vaango to use a basic elastic model, with an extension
to failure based on a stress or strain as given below, thus yielding an
elastic-brittle failure model.  This model also allows a distribution
of failure strain (or stress) based on normal or Weibull distributions.
Note that the post-failure behaviour of simulations is not always robust.
The specification is:
\begin{lstlisting}[language=XML]
  <constitutive_model type="cnh_damage"> 
    <bulk_modulus>   8.9e9  </bulk_modulus>
    <shear_modulus>  3.52e9 </shear_modulus>
    <useModifiedEOS> true   </useModifiedEOS>
  </constitutive_model>
\end{lstlisting}
When specifying \Textsfc{cnh\_damage}, the material heterogeneity and damage 
specification described for the general model (UCNH) may also be specified
as discussed below.

\subsubsection{Elastic-Plastic ($J_2$-plasticity)}
For simulations with plasticity enabled, use
\begin{lstlisting}[language=XML]
  <constitutive_model type="UCNH"> 
    <!-- Necessary flags for all CNH models -->
    <bulk_modulus>   8.9e9  </bulk_modulus>
    <shear_modulus>  3.52e9 </shear_modulus>
    <useModifiedEOS> true   </useModifiedEOS>
                
    <!-- Plasticity Parameters -->
    <usePlasticity>     true  </usePlasticity>
    <yield_stress>      100.0 </yield_stress>
    <hardening_modulus> 500.0 </hardening_modulus>
    <alpha>             1.0   </alpha>
  </constitutive_model>
\end{lstlisting}
This model includes $J_2$-plasticity with isotropic linear hardening, 
and can be alternatively invoked using the \Textsfc{comp\_neo\_hook\_plastic} tag:
\begin{lstlisting}[language=XML]
  <constitutive_model type="comp_neo_hook_plastic">
    <bulk_modulus> 8.9e9 </bulk_modulus>
    <shear_modulus> 3.52e9 </shear_modulus>
    <useModifiedEOS> true </useModifiedEOS>
    <yield_stress> 100.0 </yield_stress>
    <hardening_modulus> 500.0 </hardening_modulus>
    <alpha> 1.0 </alpha>
  </constitutive_model>
\end{lstlisting}

\subsubsection{Elastic-Plastic with damage}
The \Textsfc{UCNH} model with damage can alteratively be invoked with the \Textsfc{cnhp\_damage}
tag. This constitutive model is an extension of the hyperelastic-plastic neo-Hookean model
to failure based on a stress or strain, thus yielding an elastic-plastic model with failure.  
Note that the post-failure behaviour of simulations is not always robust.  

When the \Textsfc{cnhp\_damage} tag is used instead of \Textsfc{UCNH}, the input section for 
damage and plasticity is similar to that for UCNH without \Textxml{useDamage} and 
\Textxml{usePlasticity}.

A fairly sophisticated means of seeding explicit material heterogeneity is also provided for. 
To use these features the following four steps are required: 
\begin{enumerate}
\item \Textmag{Erosion algorithm:} 
  To allow for failure (by material point erosion), in the \Textxml{MPM} block, the 
  \Textsfc{erosion algorithm} must be set to one of the following: 
  \begin{lstlisting}[language=XML]
    <erosion algorithm="AllowNoTension"/>
    <erosion algorithm="AllowNoShear"/>
    <erosion algorithm="ZeroStress"/>
  \end{lstlisting}
  In the \Textxml{constitutive\_model} block:
  \begin{lstlisting}[language=XML]
    <useDamage>true</useDamage>
  \end{lstlisting}

\item \Textmag{Failure criterion:} 
  The failure criterion must be specified.  This is also in the \Textxml{constitutive\_model}
  block.  One of the following must be specified:
  \begin{lstlisting}[language=XML]
    <failure_criteria>  MohrCoulomb </failure_criteria>
    <failure_criteria>  MaximumPrincipalStress </failure_criteria>
    <failure_criteria>  MaximumPrincipalStrain </failure_criteria>
  \end{lstlisting}

  The \Textsfc{MohrCoulomb failure criterion} is given by 
  \begin{equation}
  \frac{\sigma_3-\sigma_1}{2}=c\cos(\phi)-\frac{\sigma_3+\sigma_1}{2}\sin(\phi)
  \end{equation}
  where $\sigma_i$ are the ordered principal stresses, positive in tension 
  ($\sigma_3 > \sigma_2 > \sigma_1$).  Note, the MohrCoulomb failure 
  surface requires a friction angle, $\phi$, (in degrees):
  \begin{lstlisting}[language=XML]
    <friction_angle> friction angle </friction_angle>
  \end{lstlisting}
  and the \Textsfc{cohesion} ($c$) which is assigned using a distribution, as described below.  

  \begin{NoteBox}
  For the maximum 
  principal stress and strain failure criteria, the cohesion is the maximum value of principal 
  stress or strain that may be obtained (must be positive).  
  \end{NoteBox}

  A tensile cutoff failure surface may be added for MohrCoulomb.
  The tensile cutoff is taken to be a fraction of the cohesion.  
  This parameter is specified using:
  \begin{lstlisting}[language=XML]
    <tensile_cutoff_fraction> 0.1 </tensile_cutoff_fraction> 
  \end{lstlisting}
  Setting this to a large number effectively removes this failure surface, leaving just Mohr-Coulomb.

\item \Textmag{Material heterogeneity:}
  Material heterogeneity type must be specified.  For MohrCoulomb the cohesion is distributed 
  spatially (an independent assignment for each material point).  For MaximumPrincipalStress and 
  MaximumPrincipalStrain, the threshold stress or strain for failure, respectively, is distributed 
  spatially (an independent assignment for each material point).  Material heterogeneity is 
  distributed spatially by assigning values consistent with a distribution function.  Three different 
  distributions may be used.  All parameters are in the \Textxml{constitutive\_model} block:
  \begin{lstlisting}[language=XML]
    <failure_distrib> gauss </failure_distrib>
    <failure_distrib> weibull </failure_distrib>
    <failure_distrib> constant </failure_distrib>
  \end{lstlisting}

  A Gaussian \Textsfc{gauss} distribution requires the following parameters:
  \begin{lstlisting}[language=XML]
    <failure_mean> Gaussian mean value of cohesion </failure_mean>
    <failure_std> Gaussian standard deviation of cohesion </failure_std>
    <failure_seed> random number generator seed </failure_seed>
  \end{lstlisting}

  A Weibull (weibull) distribution requires the following parameters:
  \begin{lstlisting}[language=XML]
    <failure_mean> Weibull mean value of cohesion </failure_mean>
    <failure_std> Weibull modulus </failure_std>
    <failure_seed> random number generator seed </failure_seed>
  \end{lstlisting}

  A homogeneous (constant) assignment requires the following parameters:
  \begin{lstlisting}[language=XML]
    <failure_mean> value (all particles assigned one value) </failure_mean>
  \end{lstlisting}

\item \Textmag{Distribution scaling:}
  Distribution scaling with numerical resolution may optionally be specified.  This is only 
  available for Gaussian and Weibull distributions.  All parameters are in the 
  \Textxml{constitutive\_model} block:
  \begin{lstlisting}[language=XML]
    <scaling> kayenta </scaling>
    <scaling> none (default) </scaling>
  \end{lstlisting}

  For kayenta scaling, the mean value of the distribution is scaled by the factor
  \begin{equation}
   \biggl(\frac{\bar V}{V}\biggr)^{1/n}
  \end{equation}
  where $V$ is the particle volume, a function of numerical resolution.  The reference volume, 
  $\bar V$ and exponent, $n$, both must be specified
  \begin{lstlisting}[language=XML]
    <reference_volume> $\bar V$ </reference_volume>
    <exponent> n </exponent>
  \end{lstlisting}
  The exponent defaults to the Weibull modulus if the Weibull distribution is used.  This physically
  motivated scaling provides for an increase in mean cohesion with decreasing particle size, generallly
  consistent with the observation that smaller quantities of material contain fewer critical flaws.
\end{enumerate}

\subsubsection{Post-failure behavior}
When a particle has failed, the value of the particle variable \Textsfc{p.localized}
will be larger than one (0 means the particle has not failed) and can be output in the 
\Textsfc{DataArchiver} section of the input file. In addition, the total number of failed particles as
a function of time \Textsfc{TotalLocalizedParticle} can be output. 

\subsubsection{Brittle damage}
Another damage model that can be used with \Textsfc{cnh\_damage} and
\Textsfc{cnhp\_damage} is a subset of the brittle damage model of LS-DYNA's Concrete
 Model 159 (FHWA-HRT-057-062, 2007). The model is invoked by the following MPMFlag
\begin{lstlisting}[language=XML]
  <erosion algorithm="BrittleDamage"/>
\end{lstlisting}
in the \Textxml{MPM} section of the input file. Two key features of the model are the use of
progressive (as opposed to sudden) damage due to softening to improve numerical stability, 
and the reduction of mesh size sensitivity via the specification of fracture energy. 

Brittle damage occurs when the mean stress $\sigma_{kk}/3$ is tensile
and the energy $\tau_b$, related to the maximum principal 
strain $\epsilon_{max}$, has exceeded a threshold value $r_0^b$
\begin{equation}
  \sigma_{kk}>0, \phantom{ijkl}
  \tau_b = \sqrt{E \epsilon_{max}^2} \geq r_0^b
\end{equation}
where $E$ is the Young's modulus. If at the next time step the mean stress is less than
zero (compressive), the damage mechanism can be optionally inactivated such that the current stress 
is set temporarily to a fraction of the undamaged stress to 
model stiffness recovery due to crack closing. When the mean stress becomes tensile again,
the value of the previous maximum damage $d$ can be restored; recovery is a user option in \Vaango
but should be used with caution since stiffening is more prone to instability.
The softening function for brittle damage is assumed to be
\begin{equation}
  d(\tau_b)= \frac{0.999}{D} \left(\frac{1+D}{1+D \exp^{-C(\tau_b-r_0^b)}} \right)
\end{equation}
where $C$ and $D$ are constants that define the shape of the softening
stress-strain curve.
  
To regulate mesh size sensitivity, the fracture energy
($G_f$), defined as the area under the stress-displacement
curve for displacement larger than $x_0$ (the displacement at peak strength), is to be 
maintained constant.
The user needs to input $G_f$ and $D$; $C$ is calculated internally. 

The maximum increment of damage that can accumulate over a single time
step is a user-defined input to avoid excessive damage accumulation over
a single time step to reduce numerical instability.

For \Textsfc{cnh\_damage}, the parameters for brittle damage can be specified as
\begin{lstlisting}[language=XML]
<constitutive_model type="cnh_damage">
  <shear_modulus>3.52e9</shear_modulus>
  <bulk_modulus>8.9e9</bulk_modulus>
  <brittle_damage_initial_threshold>57.0 </brittle_damage_initial_threshold>
  <brittle_damage_fracture_energy>11.2</brittle_damage_fracture_energy>
  <brittle_damage_constant_D>0.1</brittle_damage_constant_D>
  <brittle_damage_max_damage_increment>0.1</brittle_damage_max_damage_increment>
  <brittle_damage_allowRecovery> false </brittle_damage_allowRecovery>
  <brittle_damage_recoveryCoeff> 1.0 </brittle_damage_recoveryCoeff>
  <brittle_damage_printDamage> false </brittle_damage_printDamage>
</constitutive_model>
\end{lstlisting}

The tags in the input file for brittle damage are shown in the following table.
\begin{table}[ht]
\centering
\begin{tabular} {c c l}
\hline
Tag & Symbol & Description \\
\hline
brittle\_damage\_initial\_threshold & $r_0^b$ &  material property \\
brittle\_damage\_fracture\_energy & $G_f$ &  material property \\
brittle\_damage\_constant\_D & $D$ & material property \\
brittle\_damage\_max\_damage\_increment & & optional, default=0.1 \\
brittle\_damage\_allowRecovery & & allow crack closing (stiffening) \\
& & optional, default=false \\
brittle\_damage\_recoveryCoeff & & fraction of undamaged stress to recover\\
& & (between 0 and 1), optional\\
& & default=1.0 (full recovery) used only\\
& & when brittle\_damage\_allowRecovery \\
& & is set to true  \\
brittle\_damage\_printDamage & & print the state of damage \\
& & of damaged particles, default=false \\
& & (to reduce large amounts of output) \\
\hline
\end{tabular}
\end{table}

When a particle is damaged, the value of the particle variable \Textsfc{p.damage}
can be output in the \Textsfc{DataArchiver} section of the input file.

