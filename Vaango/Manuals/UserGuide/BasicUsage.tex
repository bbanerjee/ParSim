\chapter{Basic Vaango usage} \label{Chapter:UCF}
Several executable programs have been developed using the \Vaango
Computational Framework (UCF).  The primary code that drives the
components implemented in \Vaango is called \Textsfc{vaango}.

Although \Uintah was developed for complex fire-explosion simulations,
the general nature of the algorithms and the framework has allowed 
researchers to use the code to investigate a
wide range of problems.  The \Vaango framework is general purpose enough to
allow for the implementation of a variety of implicit and explicit
algorithms on structured grids but focuses on particle based
algorithms.

\section{Installing the Vaango software}
For information on downloading the \Vaango software package 
and how to setup and build it, please refer to the \Vaango Installation Guide.

\section{Running Vaango}
For single processor simulations, the \Textsfc{vaango} executable
is run from the command line prompt like this:
\begin{lstlisting}[backgroundcolor=\color{background}]
  vaango input.ups
\end{lstlisting}
where \Textsfc{input.ups} is an XML formatted input file.  The
\Vaango software contains numerous example input files located
in the \Textsfc{src/StandAlone/inputs} directory.

For multiprocessor runs, the user generally uses \Textsfc{mpirun}
to launch the code.  Depending on the environment, batch
scheduler, launch scripts, etc, \Textsfc{mpirun} may or may not
be used.  However, in general, something like the following is used:
\begin{lstlisting}[backgroundcolor=\color{background}]
  mpirun -np num_processors vaango input.ups
\end{lstlisting}

\Textbfc{num\_processors} is the number of processors that will
be used.  The input file must contain a patch layout that has at least
the same number (or greater) of patches as processors specified by a
number following the \Textbfc{-np} option shown above.

In addition, the \Textsfc{-mpi} flag is optional but sometimes
necessary if the mpi environment is not automatically detected from
within the \Textsfc{vaango} executable.

\Vaango provides for restarting from checkpoint as well.  For information on
this, see Section~\ref{Sec:DataArchiver}, which describes how to create
checkpoint data, and how to restart from it.

\section{Vaango Problem Specification} \label{Sec:UPS}
The \Vaango framework uses XML like input files to specify the various
parameters required by simulation components.  These are called
\Textsfc{Uintah Problem Specification} files and have the extension \Textbfc{.ups}
because they are directly based on the \Uintah input file format.
The \Textbfc{.ups} files are validated based on the specification
found in \Textbfc{src/StandAlone/inputs/UPS\_SPEC/ups\_spec.xml}
and its sibling files.  

The application developer is free to use any of the specified tags to
specify the data needed by the simulation.  The essential tags that
are required by \Vaango include the following:
\begin{lstlisting}[language=XML]
  <Uintah_specification>
  <SimulationComponent>
  <Time>
  <DataArchiver>
  <Grid>
\end{lstlisting}

Individual components have additional tags that specify properties,
algorithms, materials, etc. that are unique to that individual
components.  Within the individual sections on MPM, ICE, and MPMICE
the individual tags will be explained more fully.

The \Textsfc{vaango} executable verifies that the input file adheres to a consistent
specification and that all necessary tags are specified.  However, it
is up to the individual creating or modifying the input file to put in
physically reasonable set of consistent parameters.

\section{Simulation Components} \label{Sec:SimulationComponent}
The input file tag for \Textsfc{SimulationComponent} has the \Textsfc{type}
attribute that must be specified with either \Textsfc{mpm, mpmice, ice, peridynamics}
as in:
\begin{lstlisting}[language=XML]
<SimulationComponent type = "mpm" />
\end{lstlisting}

\section{Time Related Variables} \label{Sec:TimeRelatedVariables}
\Vaango components are time-dependent codes.  As such, one of the first
entries in each input file describes the time-stepping parameters.  An
input file segment is given below that encompasses all of the possible
parameters.  The function of each of these parameters is described below.

\begin{lstlisting}[language=XML]
<Time>
    <maxTime>            1.0         </maxTime>
    <initTime>           0.0         </initTime>
    <delt_min>           0.0         </delt_min>
    <delt_max>           1.0         </delt_max>
    <delt_init>          1.0e-9      </delt_init>
    <max_delt_increase>  2.0         </max_delt_increase>
    <timestep_multiplier>1.0         </timestep_multiplier>
    <max_timestep>       100         </max_timestep>
    <end_on_max_time_exactly>true    </end_on_max_time_exactly>
</Time>
\end{lstlisting}

The following fields are required:
\begin{itemize}
\item \Textbfc{maxTime} - how long in physical time to run the simulation for
\item \Textbfc{initTime} - what time to begin the simulation at
\item \Textbfc{delt\_min} - the smallest timestep the simulation will take
\item \Textbfc{delt\_max} - the largest timestep the simulation will take
\item \Textbfc{timestep\_multiplier} - multiplies the timestep by this number (before adjusting to min or max timestep)
\end{itemize}

The following fields are optional:
\begin{itemize}
\item \Textbfc{delt\_init} - The timestep to take initially (assuming it's less than the one computed by the simulation)
\item \Textbfc{initial\_delt\_range} - The period of time to use the \Textbfc{delt\_init} (default = 0)
\item \Textbfc{max\_delt\_increase} - Maximum amount to multiply the previous delt by (if the newly computed delt is greater than the previous one)
\item \Textbfc{max\_iterations} - The number of timesteps to run the simulation for (even on a restart)
\item \Textbfc{max\_timesteps} - The timestep number to end the simulation on (not usually used with \Textbfc{max\_iterations})
\item \Textbfc{override\_restart\_delt} - On a restart, use this delt instead of the most-recently-used delt.
\item \Textbfc{clamp\_timesteps\_to\_output} - Sync the delt with the DataArchiver - when an output timestep occurs, reduce the delt to have the time land on the timestep interval (default = false)
\item \Textbfc{end\_on\_max\_time\_exactly} - clamp the delt such that the last timesteps end on what was specified in \Textbfc{maxTime} (default = false)
\end{itemize}
\begin{NoteBox}
A word about timesteps: In general, the timestep (delt) is computed at various stages within a timestep, and the smallest one is used, unless it needs to raise the delt to the \Textbfc{delt\_min}.
\end{NoteBox}

\section{Data Archiver} \label{Sec:DataArchiver}
The Data Archiver section specifies the directory name where data will
be stored and what variables will be saved and how often data is saved
and how frequently the simulation is checkpointed.

\subsection{Saving data}
The \Textsfc{\textless filebase\textgreater} tag is used to specify the directory
name and by convention, the \Textsfc{.uda} suffix is attached denoting the
``\Vaango Data Archive".

Data can be saved based on a frequency setting that is either based on integer time
intervals:
\begin{lstlisting}[language=XML]
  <outputTimestepInterval> 100 </outputTimestepInterval>
\end{lstlisting}
or real-valued timestep intervals:
\begin{lstlisting}[language=XML]
  <outputInterval> 1.0e-3 </outputInterval>}
\end{lstlisting}

Each simulation component specifies variables with label names that
can be specified for data output.  By convention, particle data are
denoted by \Textsfc{p.} followed by a particular variable name
such as mass, velocity, stress, etc.  Whereas for node based data, the
convention is to use the \Textsfc{g.} followed by the variable
name, such as mass, stress, velocity, etc.  Similarly, cell-centered
and face-centered data typically end with the a trailing \Textsfc{CC}
or \Textsfc{FC,}  respectively.  Within the DataArchiver
section, variables are specified with the following format:

\begin{lstlisting}[language=XML]
   <save label = "p.mass" />
   <save label = "g.mass" />
\end{lstlisting}

To see a list of
variables available for saving for a given component, execute the following
command from the \Textsfc{StandAlone} directory:

\begin{lstlisting}[backgroundcolor=\color{background}]
inputs/labelNames component
\end{lstlisting}
where \Textbfc{component} is, e.g., \Textbfc{mpm}, \Textbfc{ice}, etc.

\subsection{Reduction variables}
While most variable data remains confined to a single patch and is transferred
to adjacent patches only if needed, there are certain variables that need to
be communicated to all the patches.  Such variables are typically \Textsfc{reduced}
before communication and are called \Textsfc{Reduction} variables.

\MPM reduction variables that can be saved in the data archive include the 
following possibilities:
\begin{lstlisting}[language=XML]
   <save label = "TotalMass" />
   <save label = "TotalMomentum" />
   <save label = "ThermalEnergy" />
   <save label = "KineticEnergy" />
   <save label = "StrainEnergy" />
   <save label = "AccStrainEnergy" />
   <save label = "TotalVolumeDeformed" />
   <save label = "CenterOfMassPosition" />
\end{lstlisting}
The accumulated strain energy, \Textsfc{AccStrainEnergy}, is useful for incremental
material models where only an incremental strain energy computed in each step.

\subsection{Check-pointing for restarts}
Check-pointing information can be created that provides a mechanism for
restarting a simulation at a later point in time.  The \Textbfc{\textless checkpoint\textgreater}
tag with the \Textbfc{cycle} and \Textbfc{ interval} attributes describe how many
copies of checkpoint data is stored (cycle) and how often it is generated
(interval).  You may also use the \Textbfc{walltimeStart} and \Textbfc{walltimeInterval}
options for specifying when and how offen a checkpoint will be output based
on wall-clock time.

As an example of checkpoint data that has two timesteps worth of
checkpoint data that is created every .01 seconds of simulation time
are shown below:
\begin{lstlisting}[language=XML]
<checkpoint cycle = "2" interval = "0.01"/>
\end{lstlisting}
An alternative checkpointing scheme that saves data every 25 timesteps can
be specified using:
\begin{lstlisting}[language=XML]
<checkpoint cycle = "2" timestepInterval = "25"/>
\end{lstlisting}

To restart from a checkpointed archive, simply put \Textsfc{-restart} in the
\Textsfc{vaango} command-line arguments and specify the .uda directory instead of
a ups file (\Textsfc{vaango} reads the copied \Textsfc{input.xml} from the
archive).  One can optionally specify a certain timestep to restart
from with \Textsfc{-t timestep} with multiple checkpoints, but the
last checkpointed timestep is the default.  When restarting, \Textsfc{vaango}
copies all of the appropriate information from the old uda directory to its
new uda directory.
% I'M NOT SURE ABOUT THIS, -nocopy REMOVES THE OLD UDA?  I THOUGHT IT
% JUST LEFT STUFF IN THE OLD UDA, WHERE AS -copy MADE A COPY OF OLD TIMESTEP
% DATA, AND -move MOVED THE OLD TIMESTEP DATA TO THE NEW UDA.  
%  If one doesn't want to keep the old uda directory
%around, they can specify \Textsfc{-nocopy} to have it be removed
%(e.g., if you are cramped for disk space).  Either way it creates a new uda
%directory for you as always.

Here are some examples:

\begin{lstlisting}[backgroundcolor=\color{background}]
./vaango -restart disks.uda.000 -nocopy
./vaango -restart disks.uda.000 -t 29
\end{lstlisting}
%
%__________________________________

\section{Simulation Options} \label{Sec:SimulationOptions}


There are many options available when running MPM simulations.  These
are generally specified in the \Textsfc{\textless MPM\textgreater} section of the input file.
A list of these options taken from 
\Textbfc{inputs/UPS\_SPEC/mpm\_spec.xml}  is given in the \Vaango
Developers Manual.

\section{Geometry creation} 
Within several of the components, the material is described by a
combination of physical parameters and the geometry.  Geometry objects
use the notion of constructive solid geometry operations to compose
the layout of the material from simple shapes such as boxes, spheres,
cylinders and cones, as well as operators which include the union,
intersections, differences of the simple shapes.  In addition to the
simple shapes, triangulated surfaces can be used in conjunction with
the simple shapes and the operations on these shapes.  

See Chapter~\ref{chap:GeometryCreation} for further detail.

\subsection{Resolution}
An additional input in the \Textbfc{\textless geom\_object\textgreater} field is the
\Textsfc{\textless res\textgreater} tag.  In MPM, this simply refers to how many particles
are placed in each cell in each coordinate direction.  For multi-material ICE
simulations, the \Textsfc{\textless res\textgreater} serves a similar purpose in that one
can specify the subgrid resolution of the initial material distribution
of mixed cells at the interface of geometry objects.

\section{Boundary conditions}\label{sec:ucf_bc}

Boundary conditions are specified within the \Textsfc{\textless Grid\textgreater}
but are described separately for clarity.  The essential idea is that
boundary conditions are specified on the domain of the grid.  Values
can be assigned either on the entire face, or parts of the face.
Combinations of various geometric descriptions are used to aid in the
assignment of values over specific regions of the grid.  Each of the
six faces of the grid is denoted by either the minus or plus side of
the domain.

The XML description of a particular boundary condition includes which
side of the domain, the material id, what type of boundary condition
(Dirichlet or Neumann) and which variable and the value assigned.  The
following is a an MPM specification of a Dirichlet boundary condition
assigned to the velocity component on the x minus face (the entire
side) with a vector value of [0.0,0.0,0.0] applied to all of the materials.

\begin{lstlisting}[language=XML]
 <Grid>
       <BoundaryConditions>
         <Face side = "x-">
             <BCType id = "all" var = "Dirichlet" label = "Velocity">
                   <value> [0.0,0.0,0.0] </value>
             </BCType>
         </Face>
         <Face side = "x+">
            <BCType id = "all" var = "Dirichlet" label = "Velocity">
                 <value> [0.0,0.0,0.0] </value>
            </BCType>
         </Face>
        . . . .
        <BoundaryCondition>
   . . . .
  <Grid>
\end{lstlisting}

The notation \Textsfc{\textless Face side = "x-"\textgreater} indicates that the
entire x minus face of the boundary will have the boundary condition
applied.  The \Textsfc{id = "all"} means that all the
materials will have this value.  To specify the boundary condition for
a particular material, specify an integer number instead of the
"all".  The \Textsfc{var = "Dirichlet"} is used to specify
whether it is a Dirichlet or Neumann or symmetry boundary conditions.
Different components may use the \Textsfc{var} to include a
variety of different boundary conditions and are explained more fully
in the following component sections.  The \Textsfc{label = "Velocity"}
specifies which variable is being assigned and again is
component dependent.  The \Textsfc{\textless value\textgreater [0.0,0.0,0.0] \textless/value\textgreater}
specifies the value.

An example of a more complicated boundary condition demonstrating a
hot jet of fluid issued into the domain is described.  The jet is
described by a circle on one side of the domain with boundary
conditions that are different in the circular jet compared to the rest
of the side.
\begin{lstlisting}[language=XML]
 <Face circle = "y-" origin = "0.0 0.0 0.0" radius = ".5">
        <BCType id = "0"   label = "Pressure" var = "Neumann">
                              <value> 0.0   </value>
        </BCType>
        <BCType id = "0" label = "Velocity" var = "Dirichlet">
                              <value> [0.,1.,0.] </value>
        </BCType>
        <BCType id = "0" label = "Temperature" var = "Dirichlet">
                              <value> 1000.0  </value>
        </BCType>
        <BCType id = "0" label = "Density" var = "Dirichlet">
                              <value> .35379  </value>
        </BCType>
        <BCType id = "0" label = "SpecificVol"  var = "computeFromDensity">
                              <value> 0.0  </value>
        </BCType>
      </Face>
      <Face side = "y-">
        <BCType id = "0"   label = "Pressure"     var = "Neumann">
                              <value> 0.0   </value>
        </BCType>
        <BCType id = "0" label = "Velocity"     var = "Dirichlet">
                              <value> [0.,0.,0.] </value>
        </BCType>
        <BCType id = "0" label = "Temperature"  var = "Neumann">
                              <value> 0.0  </value>
        </BCType>
        <BCType id = "0" label = "Density"      var = "Neumann">
                              <value> 0.0  </value>
        </BCType>
        <BCType id = "0" label = "SpecificVol"  var = "computeFromDensity">
                              <value> 0.0  </value>
        </BCType>
      </Face>
\end{lstlisting}
The jet is described by the circle on the y minus face with the origin
at $(0,0,0)$ and a radius of $0.5$.  For the region outside of the circle,
the boundary conditions are different.  Each side must have at least
the \Textsfc{side} specified, but additional circles and
rectangles can be specified on a given face.

An example of the \Textsfc{rectangle} is specified as with the
lower corner at $(0,0.181,0)$ and upper corner at $(0,0.5,0)$.
\begin{lstlisting}[language=XML]
 <Face rectangle = "x-" lower = "0.0 0.181 0.0" upper = "0.0 0.5 0.0">
\end{lstlisting}

\section{Grid specification} \label{Sec:Grid}
The \Textsfc{\textless Grid\textgreater} section specifies the domain of the
structured grid and includes tags which indicate the lower and upper
corners, the number of extra cells which can be used by various
components for the application of boundary conditions or interpolation
schemes.  

The grid is decomposed into a number of patches.  For single processor
problems, usually one patch is used for the entire domain.  For
multiple processor simulations, there must be at least one patch per
processor.  Patches are specified along the x,y,z directions of the
grid using the \Textsfc{\textless patches\textgreater [2,5,3] \textless/patches\textgreater} which
specifies two patches along the x direction, five patches along the y
direction and 3 patches along the z direction.  The maximum number of
processors that \Textsfc{vaango} could use is $2\times 5\times 3 = 30$.
Attempting to use more processors than patches
will cause a run time error during initialization.

Finally, the grid spacing can specified using either a fixed number of
cells along each x,y,z direction or by the size of the grid cell in
each direction.  To specify a fixed number of grid cells, use the \Textsfc{
\textless resolution\textgreater [20,20,3] \textless/resolution\textgreater}.  This specifies 20
grid cells in the x direction, 20 in the y direction and 3 in the z
direction.  To specify the grid cell size use the \Textsfc{\textless spacing\textgreater
[0.5,0.5,0.3] \textless /spacing\textgreater}.  This specifies the a grid cell
size of .5 in the x and y directions and .3 in the z direction.  The
\Textsfc{\textless resolution\textgreater} and \Textsfc{\textless spacing\textgreater} cannot be
specified together.  The following two examples would generate
identical grids:
\begin{lstlisting}[language=XML]
<Level>
    <Box label="1">
       <lower>        [0,0,0]          </lower>
       <upper>        [5,5,5]          </upper>
       <extraCells>   [1,1,1]          </extraCells>
       <patches>      [1,1,1]          </patches>
    </Box>
    <spacing>         [0.5,0.5,0.5]    </spacing>
</Level>
\end{lstlisting}

\begin{lstlisting}[language=XML]
<Level>
    <Box label="1">
       <lower>        [0,0,0]          </lower>
       <upper>        [5,5,5]          </upper>
       <resolution>   [10,10,10]       </resolution>
       <extraCells>   [1,1,1]          </extraCells>
       <patches>      [1,1,1]          </patches>
    </Box>
</Level>
\end{lstlisting}

The above examples indicate that the grid domain has a lower corner at
$(0,0,0)$ and an upper corner at $(5,5,5)$ with one extra cell in each
direction.  The domain is broken down into one patch covering the
entire domain with a grid spacing of $(.5,.5,.5)$.  Along each dimension
there are ten cells in the interior of the grid and one layer of
\Textsfc{extraCells} outside of the domain.  \Textsfc{extraCells}
are the \Uintah nomenclature for what are frequently referred to 
as \Textsfc{ghost-cells}.
