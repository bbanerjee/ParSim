\chapter{Regression testing in Vaango}
This chapter provides documentation for the \Vaango regression testers and the helper
scripts that they call.  The regression tester code is primarily in Python 
with a few interspersed shell scripts and is located at
\begin{lstlisting}[language=sh, backgroundcolor=\color{background}]
Vaango/src/R_tester
\end{lstlisting}
The files in this structure are listed below:
\begin{lstlisting}[language=sh, backgroundcolor=\color{background}]
|-- CMakeLists.txt
|-- Examples.py
|-- GPU.py
|-- ICE.py
|-- IMPM.py
|-- Models.py
|-- MPMF.py
|-- MPMICE.py
|-- MPM.py
|-- UCF.py
|-- helpers
|   |-- cleanup
|   |-- compare_dat_files.pl
|   |-- compare_dats
|   |-- compare_vaango_runs
|   |-- compare_vaango_runs_by_udas
|   |-- highwater_percent.pl
|   |-- mem_leak_check
|   |-- modUPS.py
|   |-- performance_check
|   |-- performance_percent.pl
|   |-- plotMemUsage.gp
|   |-- plotRunTimeHistory.gp
|   |-- replace_all_GS
|   |-- replace_gold_standard
|   |-- runComponentTests
|   |-- runTests
|   |-- runVaangoTests.py
|   |-- selectComponents.sh
|-- toplevel
|   |-- generateGoldStandards.py
\end{lstlisting}

The location of the place where the regression tests are actually run
is the \Textsfc{\textless build\_dir \textgreater} which is typically called \Textsfc{dbg|opt}
and the regression tester makefiles and other copied scripts are in
\begin{lstlisting}[language=sh, backgroundcolor=\color{background}]
Vaango/<build\_dir>/R_tester
\end{lstlisting}

\section{Basic operation}
\subsection{Creating gold standards}
The first step in setting up a local version of the regression tester
is to create gold standards based on a tested and working version of the code.
To do that, go to the \Textsfc{\textless build\_dir \textgreater} and run
\begin{lstlisting}[language=sh, backgroundcolor=\color{background}]
make gold_standard
\end{lstlisting}

The gold standards are saved in 
\begin{lstlisting}[language=sh, backgroundcolor=\color{background}]
Vaango/\textless build\_dir>/R_tester/gold_standards
\end{lstlisting}

\subsection{Running regression tests}
To run all the regression tests from the \Textsfc{\textless build\_dir \textgreater}, you can do 
\begin{lstlisting}[language=sh, backgroundcolor=\color{background}]
make localRT
\end{lstlisting}
will either be in relation to the tester root, which currently is 

\subsection{Viewing regression tester results}
Regression tester output is locally stored in the \Textsfc{\textless build\_dir \textgreater/R\_Tester/localRTData}
directory.  For example, MPM related test results will be stored in the 
\Textsfc{MPM-results} directory.

There will be many log files saved. They will show you the resulting output 
of the test, which (hopefully) explains clearly why the test failed.

\subsection{Adding new regression tests}
After a new model has been developed, it is expected that you will add
a regression test for the model to the tester.  To do that go to the 
\Textsfc{Vaango/src/R\_Tester} directory and select the component to 
which you want to add a new test.  For example, if you want to add a 
new test called \Textsfc{const\_test\_tabular} that uses two MPI processes
to \Textsfc{MPM.py}, edit the python script and add the following to 
the \Textbfc{NIGHTLYTESTS} array:
\begin{lstlisting}[language=Python]
NIGHTLYTESTS = [  
  .........
  ("const_test_viscoelastic_fortran.ups", 
   "ViscoElastic/const_test_viscoelastic_fortran.ups",                       
    1,  "Linux", ["exactComparison"] ), \
  .........
  ("const_test_tabular.ups", 
   "const_test_tabular.ups",                       
    4,  "Linux", ["exactComparison"] )
]
\end{lstlisting}

If you create a new component and want to add a set of tests for that
component, follow the procedure used in the other test files.  For
a better understanding of the details, notice that
NIGHTLYTESTS is a list of tests.  There can be as many tests
as you like, but each test must be a list of 5-6 items.  For example, in
MPM we find one of the tests:
\begin{lstlisting}[language=Python]
("heatcond2mat", "heatcond2mat.ups", "", 1, "ALL")
\end{lstlisting}
The first parameter is the name of the test.  It doesn't really matter
what the name is, most people just use the name of the ups file, but
doesn't need to be.

The second parameter is the name of the ups file.
The third is any additional flags that you want to pass to vaango.
The fourth is the number of processors to use.
The fifth is which operating system to run on.  Current choices are
"Linux", "IRIX64", or "ALL"  - these are case-sensitive.

Next, you'll notice the \Textsfc{exit(runVaangoTests(...))} call at the very
bottom.  All you need to do here is change the third argument to
\textless algo \textgreater, where algo is the algorithm directly passed into vaango,
like "ice" or "mpm", AND the uppercase version of algo must be the
name of a directory with ups files (ICE, MPM).

The callback function, if passed into runVaangoTests will be called before
executing every test.  It must have the same parameters as the runVaangoTest
function, found in runVaangoTests.py.  

Then add that file to the cvs repository and commit it, and the regression
tester will see it and take care of the rest!

\section{Regression test scripts}

The main script that is used in \Textsfc{runVaangoTests.py}.
Scripts that are further down the script hierarchy

\subsection{cleanup}
\begin{lstlisting}[language=sh, backgroundcolor=\color{background}]
cleanup - Vaango/R_Tester/helpers/cleanup (sh script)
\end{lstlisting}
-------
This script can be called from the command line in the \Textsfc{\textless build\_dir \textgreater}.
It tries to remove all Vaango.* in the directory that don't have a *.lock link
pointing to it.

  Previously, any directory that had a .lock file pointing to it had a file
in BUILDROOT/ called lock.  Cleanup removes those lock files, and creates one
for every folder that is pointed to by a .lock file.

  Moves all the (non-locked) \Vaango.* dirs to the "trash" dir and then
tries to remove them.  Tries to delete everything in trask.  If not
successful (due to not owning the files), send email to people to
clean it up.

\subsection{MPM.py|ICE.py etc.}
These are called from \Textsfc{runVaangoTests.py}.  Each test is defined in an array
\begin{lstlisting}[language=Python]
TESTS = [   (nameOfTests, upsInputFile, extraFlags, numProcs, whichOS),
             ...,
	]
\end{lstlisting}
Then it calls 
\begin{lstlisting}[language=Python]
runVaangoTests(argv, TESTS, algo)
\end{lstlisting}
You can create your own test file, and it will automatically run in the tester.
It will need 755 permissions.
Example:
\begin{lstlisting}[language=Python]
#!/usr/bin/python

from sys import argv,exit
from helpers.runVaangoTests import runVaangoTests

TESTS = [   ("advect", "advect.ups", "", 1, "ALL"),     \
            ("hotBlob2mat", "hotBlob2mat.ups", "", 1, "ALL"),     \
            ("hotBlob2mat_sym", "hotBlob2mat_sym.ups", "", 1, "ALL"),     \
            ("hotBlob8patch", "hotBlob2mat8patch.ups", "", 8, "ALL"),     \
        ]

exit(runVaangoTests(argv, TESTS, "ice"))
\end{lstlisting}

You can also call \Textsfc{modUPS} to use the same ups file with different settings:
\begin{lstlisting}[language=Python]
from helpers.modUPS import modUPS
test8patch_ups = modUPS("%s/MPM" % inputs_root(), \
                           "test_table.ups", \
                           ["<patches>[2,2,2]</patches>", \
                            "<maxTime>10.0</maxTime>"])
\end{lstlisting}
This will change the \Textsfc{test\_table.ups} patches from whatever they 
were to [2,2,2] and the maxTime to 10.0.
Make sure to put the entire field from the beginning of the tag to the close
of the tag.

\Textsfc{modUPS} will scan through the ups file and look for the pattern of the tag 
scecified.  I.e., in the example above, it will look for 
\begin{lstlisting}[language=XML]
<patches>...</patches>
\end{lstlisting}
and then replace it with 
\begin{lstlisting}[language=XML]
<patches>[2,2,2]</patches>. 
\end{lstlisting}
It can also replace tags in form 
\begin{lstlisting}[language=XML]
<varname=something />.
\end{lstlisting}

\subsection{runVaangoTests.py}
\Textsfc{runVaangoTests.py} is divided largely into 2 functions, \Textsfc{runVaangoTests}, 
and \Textsfc{runVaangoTest}.
\Textsfc{runVaangoTests} takes the tests passed into it from MPM, ICE, etc. 
and executes vaango on them.  It then compares the results of the output with the 
saved goldStandard.

usage:
\begin{lstlisting}[language=Python]
  runVaangoTests(argv, TESTS, testtype, callback)
\end{lstlisting}
where \Textbfc{argv} is the same as the arguments passed into MPM, ICE etc,
\Textbfc{vaangodir} inputs goldStandard mode maxParallelism test,
\Textbfc{TESTS} is a list of tests, 
\Textbfc{testtype} is the algorithm to run - mpm, ice, mpmice, examples, etc,
\Textbfc{callback} is an optional parameter.  If you pass a callback, you must define 
that function, which has parameters:
\Textbfc{test, vaangodir, inputsdir, compare\_root, algo, mode, max\_parallelism}
Callback is called before running each test.

Files created by \Textsfc{runVaangoTests} the the \Textbfc{R\_Tester/localRTData} 
directory include, the directories \Textsfc{\textless resultsdir \textgreater} (MPM-results, ICE-results, etc.).
In the \Textsfc{\textless resultsdir \textgreater}, the following files are created:
\begin{lstlisting}[language=sh, backgroundcolor=\color{background}]
  <resultsdir>/<testname>
  <resultsdir>/<testname>/<testname>.uda.000
  <resultsdir>/<testname>/replace_gold_standard
  <resultsdir>/<testname>/compare_vaango_runs.log.txt
  <resultsdir>/<testname>/timestamp
  <resultsdir>/<testname>/vaango.log.txt
  <resultsdir>/<testname>/restart/<testname>.uda.000
  <resultsdir>/<testname>/restart/compare_vaango_runs.log.txt
  <resultsdir>/<testname>/restart/timestamp
  <resultsdir>/<testname>/restart/vaango.log.txt
\end{lstlisting}
  
\Textsfc{runVaangoTests} basically does a lot of error checking.
It proceeds to iterate through the tests, and if there was a solotest arg
specified, it will check and see if that is the current test.  
Then it compares the OS specified of the current test, and if they are not 
compatible it continues with the next test.  Then it calls the callback if 
there is one, and calls \Textsfc{runVaangoTest}.  Then it repeats that process for the 
restart test (if the \Vaango test passed) AND the files created in the resultsdir
directory will then be created in the restart directory.

\Textsfc{runVaangoTest} is responsible for the running and testing of one individual test.
It first checks to see if the test requires more processors than the tester's
max paralellism setting (passed into startTester with the -j flag).  It 
proceeds to run the tests, noting that at any point if a test fails, it 
displays where the log message is that will be helpful, and then returns
immediately.  \Vaango is then run with the parameters as set by the test, and 
sends its output to \Textsfc{vaango.log.txt}.  If it passes, it proceeds to compare the udas
generated by vaango with the gold standard, with \Textsfc{compare\_vaango\_runs}, which saves
its output in \Textsfc{compare\_vaango\_runs.log.txt}.  If we are in debug mode, it tests 
for memory leaks with \Textsfc{mem\_leak\_check} which saves its output in 
\Textsfc{mem\_leak\_check.log.txt}.

\subsection{compare\_vaango\_runs - sh script}
This script compares output of vaango with gold standard and can be called using
\begin{lstlisting}[language=sh, backgroundcolor=\color{background}]
  compare_vaango_runs <testname> <testdir> <compare_root> <vaangodir>
\end{lstlisting}
where \Textsfc{testname} is the name of test, \Textsfc{testdir} is the
directory where test is, \Textsfc{compare\_root} is the folder where testdir/testname 
should be - i.e, OS/TestData/mode/ALGO, \Textsfc{vaangodir} is the folder where 
\Textsfc{vaango} and \Textsfc{compare\_uda} are located.

If there is no data for the specified test in \Textsfc{compare\_root}, \Textsfc{compare\_vaango\_runs}
creates the necessary directories and copies the output from vaango (in 
testdir/testname) over to be the new gold standard.  Otherwise, it calls
\Textsfc{compare\_dats} and \Textsfc{compare\_udas} on the input.

\subsection{mem\_leak\_check - sh script}
This script checks and compares memory usage and can be called with
\begin{lstlisting}[language=sh, backgroundcolor=\color{background}]
  mem_leak_check <testname> <stats_file> <compare_stats_file> <tmpdir>
\end{lstlisting}
where \Textsfc{stats\_file} is the name of malloc output file -i.e., \Textsfc{malloc\_stats},
\Textsfc{compare\_stats\_file} is the name/location of file to compare to.

Looks for memory leaks in \Textsfc{stats\_file}, and outputs them to 
\Textsfc{tmpdir/scinew\_malloc\_stats}.  Then calls \Textsfc{highwater\_percent.pl} to check
memory usage. If the percentage returned is greater than 10, an error is
posted, if it returns a percentage less that -5\%, then it replaces the stats
in the goldStandard.

\subsection{compare\_dats -  sh script}
It checks to see if there are any .dat files to compare.  If there are,
it calls \Textsfc{compare\_dat\_files.pl} to compare them.
\begin{lstlisting}[language=sh, backgroundcolor=\color{background}]
  compare_dats <testname> <testdir> <compare_udadir> <tmpdir>
\end{lstlisting}
where \Textsfc{compare\_uda\_dir} is the compare\_root/testname/udadir.

\subsection{compare\_udas  sh script}
Simply calls \Textsfc{compare\_uda} (executable in vaangodir), with the flags given
\begin{lstlisting}[language=sh, backgroundcolor=\color{background}]
  compare_udas <vaangodir> <test> <test_udadir> <compare_udadir> <tmp_dir> <flags>
\end{lstlisting}

\subsection{compare\_dat\_files.pl - perl script}
Checks to see if values in dat files and gold standard dat files are within
a tolerable percent of error.
\begin{lstlisting}[language=sh, backgroundcolor=\color{background}]
  compare_dat_files.pl <abs error> <rel error> <udadir> <compare_udadir> 
                       <dat filenames>
\end{lstlisting}
where \Textsfc{abs error} - Absolute error - the max allowable exact difference,
\Textsfc{rel error} - Relative error - the max allowable relative difference - 
based on significant figures,
\Textsfc{udadir}          - Uda directory with data files,
\Textsfc{compare\_udadur}  - Gold standard uda directory,
\Textsfc{dat filenames}   - list of .dat filenames.

\subsection{highwater\_percent.pl -  perl script}
Compares memory allocated by the two input files and calculates the 
percent difference.
\begin{lstlisting}[language=sh, backgroundcolor=\color{background}]
  highwater_percent.pl <malloc_stats_file> <compare_malloc_file>
\end{lstlisting}

\subsection{replace\_gold\_standard - sh script}
Script created by \Textsfc{runVaangoTests}.  To be run at the command line, when deemed
necessary.  It replaces the files in the gold standard with the ones created by this test.
\begin{lstlisting}[language=sh, backgroundcolor=\color{background}]
(A) calls (B)
(A)  replace_gold_standard
(B)  replace_gold_standard gold_standard_dir replacement_dir testname
\end{lstlisting}
where \Textsfc{gold\_standard\_dir} is the gold-standard directory,
\Textsfc{replacement\_dir} is the directory to replace gold standard with
\Textsfc{testname} is the name of the test

\subsection{Other files}
These are items that are not scripts themselves, but may give a little insight
to the tester.
\begin{lstlisting}[language=sh, backgroundcolor=\color{background}]
__init__.py - \Vaango/R_Tester/helpers/__init__.py
\end{lstlisting}
The existence of this file alone (it is virtually empty), helps to create a 
python package for helpers.  This makes it so the python scripts can call each
other's functions.




