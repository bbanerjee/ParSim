\documentclass[11pt,fleqn]{book} % Default font size and left-justified equations

%--------------------------------------------------------------------------
% Document geometry % Page margins
%--------------------------------------------------------------------------
\usepackage[margin=1in,headsep=10pt,a4paper]{geometry}

%--------------------------------------------------------------------------
% MinionPro fonts
%--------------------------------------------------------------------------
\IfFileExists{MinionPro.sty}{%
  \usepackage{MnSymbol}
  \usepackage{MinionPro}
  \usepackage{MyriadPro}
}{%
  \usepackage[T1]{fontenc}
  \usepackage{newtxtext}
  %\usepackage{eulervm}
  \usepackage{amssymb}%needed for \mathbb
}

%--------------------------------------------------------------------------
% Color scheme used in this book
%--------------------------------------------------------------------------
\usepackage{xcolor} % Required for specifying colors by name
\definecolor{ocre}{RGB}{243,102,25} % Define the orange color used for highlighting throughout the book
\definecolor{dkgreen}{rgb}{0,0.6,0}
\definecolor{delim}{RGB}{20,105,176}
\definecolor{background}{HTML}{EEEEEE}
\definecolor{Red}{rgb}{0.8666,0.03137,0.02352}
\definecolor{Blue}{rgb}{0.00784,0.67059,0.91764}
\definecolor{Darkgreen}{rgb}{0,0.68235,0}
\definecolor{Green}{rgb}{0,0.8,0}
\definecolor{Royalblue}{rgb}{0,0.2,0.91764}
\definecolor{Brickred}{rgb}{0.644541,0.164065,0.164065}
\definecolor{Brown}{rgb}{0.6,0.4,0.4}
\definecolor{Orange}{rgb}{1,0.647059,0}
\definecolor{Indigo}{rgb}{0.746105,0,0.996109}
\definecolor{Violet}{rgb}{0.308598,0.183597,0.308598}
\definecolor{Lightgrey}{rgb}{0.762951,0.762951,0.762951}
\definecolor{Darkgrey}{rgb}{0.503548,0.503548,0.503548}
\definecolor{Pink}{rgb}{1,0.6,0.6}
\definecolor{DarkBlue}{rgb}{0,0.08,0.45}
\definecolor{OliveDrab}{rgb}{0.41961,0.55686,0.13725}
\definecolor{LightMagenta}{cmyk}{0.1,0.8,0,0.1}
\definecolor{OliveD}{HTML}{6B8E23}
\definecolor{authorNoteColor}{rgb}{.8,0,0}
\newcommand{\Ochre}{\color{ocre}}
\newcommand{\Red}{\color{Brickred}}
\newcommand{\Blue}{\color{Royalblue}}
\newcommand{\Green}{\color{Darkgreen}}
\newcommand{\DarkGreen}{\color{Darkgreen}}
\newcommand{\Violet}{\color{Violet}}
\newcommand{\Indigo}{\color{Indigo}}
\newcommand{\Orange}{\color{Orange}}
\newcommand{\Brown}{\color{Brown}}
\newcommand{\OliveD}{\color{OliveD}}
\newcommand{\Pink}{\color{Pink}}


%--------------------------------------------------------------------------
% Bibliography : biblatex does not work with tufte-book
%--------------------------------------------------------------------------
\usepackage[style=numeric,
            citestyle=numeric-comp,
            sorting=none,
            sortcites=true,
            autopunct=true,
            babel=hyphen,
            hyperref=true,
            abbreviate=false,
            backref=true,
            backend=biber]{biblatex}
\addbibresource{../Bibliography.bib} % BibTeX bibliography file
\defbibheading{bibempty}{}

%--------------------------------------------------------------------------
% Graphics
%--------------------------------------------------------------------------
\usepackage[pdftex]{graphicx}
\DeclareGraphicsExtensions{{.png},{.pdf},{.jpg},{jpeg}}
\graphicspath{ {../figures/}}
\setkeys{Gin}{width=\linewidth,totalheight=\textheight,keepaspectratio}

%--------------------------------------------------------------------------
% Links
%--------------------------------------------------------------------------
\usepackage{hyperref}
\hypersetup{pdftex, colorlinks=true, linkcolor=blue, citecolor=blue, filecolor=blue, urlcolor=blue, pdftitle=, pdfauthor= , pdfsubject=, pdfkeywords=}

%--------------------------------------------------------------------------
% Code listing
%--------------------------------------------------------------------------
\usepackage{listings}

\lstset{ %
  basicstyle=\scriptsize\ttfamily,           % the size of the fonts that are used for the code
  backgroundcolor=\color{white},      % choose the background color. You must add \usepackage{color}
  showspaces=false,               % show spaces adding particular underscores
  showstringspaces=false,         % underline spaces within strings
  showtabs=false,                 % show tabs within strings adding particular underscores
  tabsize=2,                      % sets default tabsize to 2 spaces
  captionpos=b,                   % sets the caption-position to bottom
  breaklines=true,                % sets automatic line breaking
  breakatwhitespace=false,        % sets if automatic breaks should only happen at whitespace
  commentstyle=\color{dkgreen}\upshape,       % comment style
  escapeinside={\%*}{*)},            % if you want to add LaTeX within your code
  morekeywords={*,MPM,ICE,MPMICE}               % if you want to add more keywords to the set
}

\colorlet{punct}{red!60!black}
\colorlet{numb}{magenta!60!black}

\lstdefinelanguage{XML}
{
  morestring=[b]",
  morestring=[s]{>}{<},
  morecomment=[s]{<?}{?>},
  morestring=[s]{"}{"},
  morecomment=[s]{?}{?},
  morecomment=[s]{!--}{--},
  stringstyle=\color{black},
  identifierstyle=\color{DarkBlue},
  keywordstyle=\color{Brickred},
  backgroundcolor=\color{background},
  frame=lines,
  morekeywords={xmlns,version,type}% list your attributes here
}

\lstdefinelanguage{C++}
{
  keywordstyle=\color{Brickred},
  stringstyle=\color{red},
  identifierstyle=\color{DarkBlue},
  backgroundcolor=\color{background},
  frame=lines,
  morecomment=[l][\color{magenta}]{\#}
}

\lstdefinelanguage{Cpp}
{
  language=C++,
  commentstyle=\color{dkgreen}\upshape,   
  stringstyle=\color{red},
  identifierstyle=\color{DarkBlue},
  backgroundcolor=\color{background},
  frame=lines,
  deletekeywords={...},
  escapeinside={\%*}{*)},
  keywordstyle=\color{Brickred},
  morekeywords={ProblemSpecP, ProcessorGroup, SimulationController,%
                AMRSimulationController, RegridderCommon, SolverInterface,%
                SolverFactory, UintahParallelComponent, SimulationInterface,%
                ComponentFactory, LoadBalancerCommon, LoadBalancerFactory,%
                DataArchiver, Output, SchedulerCommon, SchedulerFactory,%
                ProblemSetupException, Exception,%
                Task, Level, Patch, Ghost, Example,%
                GridP, SimulationStateP, LevelP, SchedulerP,%
                PatchSubset, MaterialSubset, DataWarehouse,% 
                SFCXVariable, PerPatch, IntVector, NodeIterator,%
                VarLabel, MPMMaterial, PatchSet, ParticleSubset,%
                MPMFlags, FlowModel, ConstitutiveModel, MyModel, MyFlow, PlasticityState,%
                particleIndex, TangentModulusTensor, Matrix3, Vector,
                ElasticModuli, TabularPlasticity, ModelState_Tabular, ParticleVariable,
                constParticleVariable, delt_vartype, ElasticModuliModelFactory,%
                YieldConditionFactory, ElasticModuliModel, YieldCondition, CMData,
                ParticleLabelVariableMap, ParameterDict, TabularData, IndependentVar,
                DependentVar, IndexKey, json, DoubleVec1D, DoubleVec2D,
                int,char,double,float,unsigned, size_t,%
                string, istringstream, cerr, exit}, 
  morecomment=[l][\color{magenta}]{\#}
}


\lstdefinelanguage{JSON}{
    numbers=left,
    numberstyle=\scriptsize,
    stepnumber=1,
    numbersep=8pt,
    showstringspaces=false,
    breaklines=true,
    frame=lines,
    backgroundcolor=\color{background},
    literate=
     *{0}{{{\color{numb}0}}}{1}
      {1}{{{\color{numb}1}}}{1}
      {2}{{{\color{numb}2}}}{1}
      {3}{{{\color{numb}3}}}{1}
      {4}{{{\color{numb}4}}}{1}
      {5}{{{\color{numb}5}}}{1}
      {6}{{{\color{numb}6}}}{1}
      {7}{{{\color{numb}7}}}{1}
      {8}{{{\color{numb}8}}}{1}
      {9}{{{\color{numb}9}}}{1}
      {:}{{{\color{punct}{:}}}}{1}
      {,}{{{\color{punct}{,}}}}{1}
      {\{}{{{\color{delim}{\{}}}}{1}
      {\}}{{{\color{delim}{\}}}}}{1}
      {[}{{{\color{delim}{[}}}}{1}
      {]}{{{\color{delim}{]}}}}{1},
}


%--------------------------------------------------------------------------
% Colored boxes:  Must come after graphicx and verbatim
% and Tikz
%--------------------------------------------------------------------------
\RequirePackage{tikz}
\usetikzlibrary{shadings,shadows}
\usetikzlibrary{decorations.pathmorphing}
\usetikzlibrary{patterns}
\usetikzlibrary{intersections}
\usetikzlibrary{calc}

\usepackage{tcolorbox}
\tcbuselibrary{most}
\newtcolorbox{NoteBox}[1][]{%
  colback=yellow!50,
  colframe=yellow!20!black,
  before skip=2mm,after skip=3mm,
  boxrule=0.4pt,left=5mm,right=2mm,top=1mm,bottom=1mm,
  sharp corners,
  rounded corners=southeast,arc is angular,arc=3mm,
  underlay={%
    \path[fill=tcbcol@back!80!black] ([yshift=3mm]interior.south east)--++(-0.4,-0.1)--++(0.1,-0.2);
    \path[draw=tcbcol@frame,shorten <=-0.05mm,shorten >=-0.05mm] ([yshift=3mm]interior.south east)--++(-0.4,-0.1)--++(0.1,-0.2);
    \path[fill=yellow!50!black,draw=none] (interior.south west) rectangle node[white]{\Huge\bfseries !} ([xshift=4mm]interior.north west);
    },
  drop fuzzy shadow,%
  #1}

\newtcolorbox{WarningBox}[1][]{%
  colback=pink!50,
  colframe=pink!20!black,
  before skip=2mm,after skip=3mm,
  boxrule=0.4pt,left=5mm,right=2mm,top=1mm,bottom=1mm,
  sharp corners,
  rounded corners=southeast,arc is angular,arc=3mm,
  underlay={%
    \path[fill=tcbcol@back!80!black] ([yshift=3mm]interior.south east)--++(-0.4,-0.1)--++(0.1,-0.2);
    \path[draw=tcbcol@frame,shorten <=-0.05mm,shorten >=-0.05mm] ([yshift=3mm]interior.south east)--++(-0.4,-0.1)--++(0.1,-0.2);
    \path[fill=pink!50!black,draw=none] (interior.south west) rectangle node[white]{\Huge\bfseries !} ([xshift=4mm]interior.north west);
    },
  drop fuzzy shadow,%
  #1}

\tcbuselibrary{skins}
\newtcolorbox[auto counter,number within=chapter]{ExampleBox}[1][]{%
  enhanced,
  colback=white,
  colframe=green!65!black,
  enlarge top by=10mm,
  overlay={%
    \path[fill=blue!65,line width=.4mm] (frame.north west)--++(17mm,0)coordinate(n2)--++(0,8mm)--++(-20mm,0) arc (-90:90:-4mm)--cycle;
    \node at ([shift={(5mm,4mm)}]frame.north west){\color{white}{\textbf{\sffamily EXAMPLE}}};
    \path[fill=green!65!blue] ([xshift=.4mm]n2)--++(0,8mm)--++(7mm,0)--++(0,-8mm)--cycle;
    \node at ([shift={(4mm,4mm)}]n2){\color{white}{\textbf{\sffamily \thetcbcounter}}};
    %\node at ([shift={(18mm,4mm)}]n2){\itshape\textbf{\sffamily Solution}};
  },
  #1}

\usetikzlibrary{calc}
\usetikzlibrary{positioning}
\tcbuselibrary{skins}
\newtcolorbox[auto counter,number within=section]{SummaryBox}[2][]{%
  enhanced,
  colback=white,
  colframe=ocre,
  enlarge top by=10mm,
  overlay={%
    \path[fill=ocre!65,line width=.4mm] (frame.north west)--++(17mm,0)coordinate(n2)--++(0,8mm)--++(-20mm,0) arc (-90:90:-4mm)--cycle;
    \node at ([shift={(5mm,4mm)}]frame.north west){\color{white}{\textbf{\sffamily Summary}}};
    \path[fill=ocre] ([xshift=.4mm]n2)--++(0,8mm)--++(10mm,0)--++(0,-8mm)--cycle;
    \node (A) at ([shift={(5mm,4mm)}]n2){\color{white}{\textbf{\sffamily \thetcbcounter}}};
    \node [right=0.5cm of A] {\itshape\textbf{\sffamily #2}};
  },
  #1}


%--------------------------------------------------------------------------
% Structure of the document
%--------------------------------------------------------------------------
%\input{00LayoutStructure}

%--------------------------------------------------------------------------
% Other packges
%--------------------------------------------------------------------------
%\usepackage{bm}%to get bold math symbols
%\usepackage[pdftex]{graphicx}
%\usepackage{verbatim}
%\usepackage[tt]{titlepic}%package i downloaded to put a pic in the titlepage
%\usepackage{helvet}
%\usepackage[usenames,dvipsnames]{xcolor}%options to use names like redviolet and others
%\usepackage{xspace}
%\usepackage[mathscr]{eucal}%Defines which font to use with \mathscr
%\usepackage{enumerate}
%\usepackage{comment}
%\usepackage{mathrsfs}
%\usepackage{floatrow}%automatically centers graphics without needing a \center command on each and every figure.

% Index
%\usepackage{calc} % For simpler calculation - used for spacing the index letter headings correctly
%\usepackage{makeidx} % Required to make an index

%-------------------------------------------------------
% For appendix
%-------------------------------------------------------
%\usepackage{esint}
\usepackage[toc,page,title,titletoc]{appendix}

%\usepackage{makeidx} % Used to generate the index

%\usepackage{xspace} % Used for printing a trailing space better than using a tilde (~) using the \xspace command

%\usepackage{epsf}
%\usepackage{epsfig}
%\usepackage{boxedminipage}
%\usepackage{stmaryrd}
%\usepackage{cancel}

%-------------------------------------------------------
% For text wrap around figures and captions for subfigures
%-------------------------------------------------------
\usepackage{wrapfig}
\usepackage{subcaption}

%-------------------------------------------------------
% For underlines
%-------------------------------------------------------
\usepackage[normalem]{ulem}

%-------------------------------------------------------
% For algorithms
%-------------------------------------------------------
\usepackage{algorithm, algpseudocode, float}
%\usepackage{algorithmic}

\makeatletter
\newenvironment{breakablealgorithm}
  {% \begin{breakablealgorithm}
   \begin{center}
     \refstepcounter{algorithm}% New algorithm
     \hrule height.8pt depth0pt \kern2pt% \@fs@pre for \@fs@ruled
     \renewcommand{\caption}[2][\relax]{% Make a new \caption
       {\raggedright\textbf{\ALG@name~\thealgorithm} ##2\par}%
       \ifx\relax##1\relax % #1 is \relax
         \addcontentsline{loa}{algorithm}{\protect\numberline{\thealgorithm}##2}%
       \else % #1 is not \relax
         \addcontentsline{loa}{algorithm}{\protect\numberline{\thealgorithm}##1}%
       \fi
       \kern2pt\hrule\kern2pt
     }
  }{% \end{breakablealgorithm}
     \kern2pt\hrule\relax% \@fs@post for \@fs@ruled
   \end{center}
  }
\makeatother
\renewcommand{\algorithmiccomment}[1]{\hfill{\color{ocre}$\triangleright$\textit{#1}}}



%\input{00LatexMacros.tex}

%-------------------------------------------------------
% For index
%-------------------------------------------------------
\makeindex 

%-------------------------------------------------------
% Background picture
%-------------------------------------------------------
\newcommand\BackgroundPic{%
\put(0,0){%
\parbox[b][\paperheight]{\paperwidth}{%
\vfill
\centering
\includegraphics[width=0.5\paperwidth,height=0.5\paperheight,%
keepaspectratio]{periodic_ellipsoids_tighter_fit.png}%
\vfill
}}}

%-------------------------------------------------------
% Versioning
%-------------------------------------------------------
\makeatletter
\def\MyYear#1{%
  \def\yy@##1##2##3##4;{##3##4}%
  \expandafter\yy@#1;
}
\def\MyMonth#1{%
  \def\yy@##1;{0##1}%
  \def\yy@##1##2;{##1##2}%
  \expandafter\yy@#1;
}
\makeatother
\newcommand{\version}{Version \MyYear{\the\year}.\MyMonth{\the\month}}



\input{../00LayoutStructure.tex}
\input{../00LatexMacros.tex}


\begin{document}

  %----------------------------------------------------------------------------------------
  %       TITLE PAGE
  %----------------------------------------------------------------------------------------
  \begingroup
    \thispagestyle{empty}
    \AddToShipoutPicture*{\BackgroundPic} % Image background
    \centering
    \vspace*{1cm}
    \par\normalfont\fontsize{35}{35}\sffamily\selectfont
    \textcolor{white}{Vaango Installation Guide}\par % Book title
    \vspace*{0.5cm}
    {\Large \textcolor{white}{\version}}\par
    {\Large \textcolor{white}{\today}}\par
    \vspace*{1cm}
    {\Large \textcolor{white}{Biswajit Banerjee}}\par % Author name
    {\Large \textcolor{white}{and}}\par % Author name
    {\Large \textcolor{white}{The Uintah team}}\par % Author name
  \endgroup

    %----------------------------------------------------------------------------------------
  %       COPYRIGHT PAGE
  %----------------------------------------------------------------------------------------

  \newpage
  ~\vfill
  \thispagestyle{empty}

  \noindent Copyright \copyright\ 2013 Callaghan Innovation\\ % Copyright notice
  \noindent Copyright \copyright\ 2015-2016 Parresia Research Limited\\ % Copyright notice

  %\noindent \textsc{Published by Publisher}\\ % Publisher

  %\noindent \textsc{book-website.com}\\ % URL

  %\noindent Licensed under the Creative Commons Attribution-NonCommercial 3.0 Unported License (the ``License''). You may not use this file except in compliance with the License. You may obtain a copy of the License at \url{http://creativecommons.org/licenses/by-nc/3.0}. Unless required by applicable law or agreed to in writing, software distributed under the License is distributed on an \textsc{``AS IS'' BASIS, WITHOUT WARRANTIES OR CONDITIONS OF ANY KIND}, either express or implied. See the License for the specific language governing permissions and limitations under the License.\\ % License information

  \noindent The contents of this manual can and will change significantly over time.  Please make sure that all the information is up to date.
  %\noindent \textit{First printing, March 2013} % Printing/edition date

  %----------------------------------------------------------------------------------------
  %       TABLE OF CONTENTS
  %----------------------------------------------------------------------------------------
  \chapterimage{fire-container-explosion-scirun-cut.png} % Table of contents heading image
  \pagestyle{empty} % No headers
  \tableofcontents % Print the table of contents itself
  \cleardoublepage % Forces the first chapter to start on an odd page so it's on the right
  \pagestyle{fancy} % Print headers again

  \setlength{\parindent}{0pt}
  \setlength{\parskip}{5pt}




\chapter{Overview of Vaango} \label{sec:overview} 
The \Vaango library is a fork of \Uintah
(\url{http://uintah.utah.edu}) intended for solid mechanics and
coupled solid-fluid simulations.  

The main components of \Vaango
are a material point method (MPM) code, a compressible fluid dynamics
code (ICE), a coupled MPM-ICE code, and an experimental Peridynamics
code.  Efforts are underway to incorporate a Discrete Element Method (DEM)
code and a Smoothed Particle Hydrodynamics (SPH) code into the framework.

The library framework in \Uintah was designed for solving PDEs on 
structured AMR grids on large parallel supercomputers.  In \Vaango, we
mainly use the grid for searches and scratch-pad calculations.

\Visit (\url{https://wci.llnl.gov/simulation/computer-codes/visit/})
is used to visualize data generated from \Vaango.

\section{Obtaining the Code}
\Vaango can be obtained either as a zipped archive (approx. 550 Mb) from
\begin{lstlisting}[language=sh, backgroundcolor=\color{background}]
https://github.com/bbanerjee/ParSim/archive/master.zip
\end{lstlisting}
or cloned from Github using either HTTP
\begin{lstlisting}[language=sh, backgroundcolor=\color{background}]
git clone --recursive https://github.com/bbanerjee/ParSim.git
\end{lstlisting}
or SSH
\begin{lstlisting}[language=sh, backgroundcolor=\color{background}]
git clone --recursive git@github.com:bbanerjee/ParSim.git
\end{lstlisting}
The above commands check out the \Parsim source tree and
installs it into a directory called \Textsfc{ParSim} in the user's home
directory.  The \Parsim code contains several research codes.  The
main \Vaango code can be found in the \Textsfc{ParSim/Vaango/src}
directory.  Manuals for \Vaango are in \Textsfc{ParSim/Vaango/Manuals}.

You may also need to install \Textsfc{git} on your machine if you 
want to clone the repository from \Textsfc{Github}:
\begin{lstlisting}[backgroundcolor=\color{background}]
sudo apt-get install git-core
\end{lstlisting}

\begin{WarningBox}
The \Textsfc{--recursive} flag is needed in order to make sure that the
submodules \Textsfc{googletest} and \Textsfc{json} are also populated
during the download process.
\end{WarningBox}

\section{Developer version}
If you are a developer, we will suggest that you use the following longer process.
\begin{enumerate}
  \item  Create a \Textsfc{GitHub} account
\begin{lstlisting}[backgroundcolor=\color{background}]
   https://github.com/signup/free
\end{lstlisting}

  \item  Install \Textsfc{git} on your machine
\begin{lstlisting}[backgroundcolor=\color{background}]
   sudo apt-get install git-core
\end{lstlisting}

  \item  Setup git configuration
\begin{lstlisting}[backgroundcolor=\color{background}]
   git config --global user.email "your_email@your_address.com"
   git config --global user.name "your_github_username"
\end{lstlisting}

  \item Email your \Textsfc{github} username to the owner of \Parsim.

  \item Create a clone of the ParSim repository
\begin{lstlisting}[backgroundcolor=\color{background}]
   git clone https://github.com/bbanerjee/ParSim
\end{lstlisting}

  \item Check branches/master
\begin{lstlisting}[backgroundcolor=\color{background}]
  git branch -a
\end{lstlisting}

  \item Configure \Textsfc{ssh} for secure access (if needed)
\begin{lstlisting}[backgroundcolor=\color{background}]
  cd ~/.ssh
  mkdir backup
  cp * backup/
  rm id_rsa*
  ssh-keygen -t rsa -C "your_email@your_address.com"
  sudo apt-get install xclip
  cd ~/ParSim/
  xclip -sel clip < ~/.ssh/id_rsa.pub
  ssh -T git@github.com
  ssh-add
  ssh -T git@github.com
\end{lstlisting}

  \item For password-less access with \Textsfc{ssh}, use
\begin{lstlisting}[backgroundcolor=\color{background}]
  git remote set-url origin git@github.com:bbanerjee/ParSim.git
\end{lstlisting}

  \item To change the default message editor, do
\begin{lstlisting}[backgroundcolor=\color{background}]
  git config --global core.editor "vim"
\end{lstlisting}
\end{enumerate}

\begin{WarningBox}
If you are new to \Vaango, we strongly suggest that you fork the code from GitHub and submit 
your changes via pull requests.
\end{WarningBox}

\subsection{Getting the latest version, adding files etc.}
\begin{enumerate}
  \item Check status
\begin{lstlisting}[backgroundcolor=\color{background}]
  git status
  git diff origin/master
\end{lstlisting}

  \item Update local repository
\begin{lstlisting}[backgroundcolor=\color{background}]
  git pull
\end{lstlisting}

  \item Add file to your local repository
\begin{lstlisting}[backgroundcolor=\color{background}]
  git add README
  git commit -m 'Added README file for git'
\end{lstlisting}

  \item Update main repository with your changes
\begin{lstlisting}[backgroundcolor=\color{background}]
  git push origin master
\end{lstlisting}
\end{enumerate}

\begin{WarningBox}
If you are new to \Vaango, we strongly suggest that you fork the code from GitHub and submit 
your changes via pull requests.
\end{WarningBox}

\section{Installation Overview}
\Vaango can be installed individually and in conjunction with
the visualization tool, \Visit.  It is recommended for
first time users to install \Vaango first and then install
\Visit second after you  have a working build of
\Vaango.  

\begin{WarningBox}
\Visit recognizes files in the \Uintah format automatically.  \Vaango
uses an older version of the \Uintah data file format that may not 
be recognized by the latest version of \Visit.  We have found that
\Visit 2.10 works with the \Vaango code.
\end{WarningBox}

For older versions of \Visit, the \Textsfc{cmake} command used to
build \Vaango may have to be modified to include the location of 
the \Visit installation directory.  

\begin{WarningBox}
The \Visit versions that we have tested require \Textsfc{Python 2.7}.  However, most
modern Linux imstallation use \Textsfc{Python 3} by default.  For \Visit to work with
\Textsfc{Python 3} installed, you will have to create a virtual environment for 
\Textsfc{Python 2.7} using
\begin{lstlisting}[backgroundcolor=\color{background}]
virtualenv visit-env
\end{lstlisting}
To activate the environment, run
\begin{lstlisting}[backgroundcolor=\color{background}]
source visit-env/bin/activate
\end{lstlisting}
After that install packages that may be needed by \Visit.  To exit the environment use:
\begin{lstlisting}[backgroundcolor=\color{background}]
deactivate
\end{lstlisting}
\end{WarningBox}

\subsection{Explicit time integration only}
Most of the material models in \Vaango support only explicit time
integration.  You you want to use only these models, the installation 
process is relatively straightforward:
\begin{enumerate}
\item Install basic dependencies: MPI, etc..
\item Configure and compile \Vaango
\item Install visualization tools 
\end{enumerate}

\subsection{Implicit and explicit time integration}
A few \Vaango models need implicit time integration.  You you need to use these
models, the installation procedure follows the basic outline:
\begin{enumerate}
\item Install basic dependencies: MPI, Blas, Lapack, Hypre, PETSc, etc.
\item Configure and compile Vaango
\item Install visualization tools (optional if installing on a
  supercomputer or first time users) and reconfigure Vaango.
\end{enumerate}

\chapter{Prerequisites and library dependencies}
\Vaango depends on several tools and libraries that are
commonly available or easily installed on various Linux or Unix like
OS distributions.  

\begin{WarningBox}
The following instructions are tailored for installations on Ubuntu/Debian-like systems.
\end{WarningBox}

\section{Prerequsites for explicit time integration only}
\subsection{Cmake}
You will probably need to install \Textsfc{Cmake} to control the software compilation process. To do that:
\begin{lstlisting}[backgroundcolor=\color{background}]
     sudo apt-get install cmake
\end{lstlisting}

\subsection{Compilers}
\begin{itemize}
  \item You will need a C++14 compliant compiler, either \Textsfc{g++} or \Textsfc{clang}:
\begin{lstlisting}[backgroundcolor=\color{background}]
sudo apt-get install g++
\end{lstlisting}
\begin{lstlisting}[backgroundcolor=\color{background}]
sudo apt-get install clang-3.8
\end{lstlisting}

  \item and a C Compiler such as \Textsfc{gcc}:
\begin{lstlisting}[backgroundcolor=\color{background}]
sudo apt-get install gcc
\end{lstlisting}

  \item You also need to have \Textsfc{gfortran}:
\begin{lstlisting}[backgroundcolor=\color{background}]
sudo apt-get install gfortran
\end{lstlisting}

  \item You may need \Textsfc{Boost} for parts of the code (and also some of the unit tests) to compile.
\begin{lstlisting}[backgroundcolor=\color{background}]
sudo apt-get install libboost-all-dev
\end{lstlisting}
        \begin{NoteBox}
        We are trying to remove \Textsfc{Boost} dependencies from the code.  Try to build the code
        without \Textsfc{Boost}, and only install the library if the build fails.
        \end{NoteBox}
\end{itemize}

\subsection{MPI and XML libraries}
\begin{itemize}
  \item Install the \Textsfc{OpenMPI} libraries:
\begin{lstlisting}[backgroundcolor=\color{background}]
sudo apt-get install mpi-default-dev
\end{lstlisting}

  \item and \Textsfc{libxml2}:
\begin{lstlisting}[backgroundcolor=\color{background}]
sudo apt-get install libxml2
sudo apt-get install libxml2-dev
\end{lstlisting}
\end{itemize}

\subsection{Eigen3 library}
The Vaango MPM code has transitioned to \Textsfc{Eigen3} from version 18.01 onwards.  The peridynamics code 
also uses the \Textsfc{Eigen3} library. You will have to install this library if you don't have it in your system.
\begin{lstlisting}[backgroundcolor=\color{background}]
sudo apt-get install libeigen3-dev
\end{lstlisting}

\subsection{HDF5 library}
From version 19.03, the Vaango MPM code uses the \Textsfc{HDF5} library to read files generated 
with \Textsfc{Tensorflow}.  Install the devleopment library using
\begin{lstlisting}[backgroundcolor=\color{background}]
sudo apt-get install libhdf5-dev
\end{lstlisting}

\subsection{Other libraries}
\begin{itemize}
  \item You will also need to install the development version of zlib.
\begin{lstlisting}[backgroundcolor=\color{background}]
sudo apt-get install zlib1g zlib1g-dev
\end{lstlisting}
  \item The \Textsfc{googletest} libraries are cloned into the repository when you use
        the \Textsfc{--recursive} flag.  You don't need any further installation.

  \item The \Textsfc{json} header-only library is also cloned into the repository when you use
        the \Textsfc{--recursive} flag.  You don't need any further installation.
\end{itemize}

\subsection{Optional libraries}
\begin{itemize}
  \item In order for related code like \Textsfc{MPM3D\_xx} etc. to work, you will also need the 
        \Textsfc{VTK} libraries. Use
\begin{lstlisting}[backgroundcolor=\color{background}]
sudo apt-get install libvtk5-dev
sudo apt-get install python-vtk tcl-vtk libvtk-java libvtk5-qt4-dev
\end{lstlisting}
        \begin{NoteBox}
        This dependency is not needed for most installations.
        \end{NoteBox}

  \item In some older versions of \Vaango, \Textsfc{libpcl} is used read point data file. In order 
        to get PCL(The Point Cloud Library) you need to run these three commands:
\begin{lstlisting}[backgroundcolor=\color{background}]
sudo add-apt-repository ppa:v-launchpad-jochen-sprickerhof-de/pcl
sudo apt-get update
sudo apt-get install libpcl-all
\end{lstlisting}
        \begin{NoteBox}
        This dependency is not needed for most installations.
        \end{NoteBox}
\end{itemize}

\section{Extra prerequisites for implicit  time integration}
Required libraries:
\begin{itemize}
\item blas
\item lapack
\item Hypre v2.8.0b (\url{https://computation.llnl.gov/casc/hypre/software.html})
\item PETSc v3.3 (\url{http://www.mcs.anl.gov/petsc/petsc-as/download/index.html})
\end{itemize}
Note, we suggest using Hypre v2.8.0b and PETSc v3.3-p3. 

\subsection{PETSc Installation}
The PETSc library is needed to run implicit MPM.  Pre-compiled binaries 
are available for download.  However, they typically work only with MPICH and not
OpenMPI.  You will need to download PETSc 3.12.5 or newer.  As of version 20.05,
the recommended download page is
\url{https://www.mcs.anl.gov/petsc/download/index.html}
and the recommended version is 3.12.5.
Alternatively, you can download the latest release from GitLab.

We recommend building the debug version of PETSc from source using
\begin{lstlisting}[backgroundcolor=\color{background}]
  ./configure
  make all test
\end{lstlisting}

For an optimized build use
\begin{lstlisting}[backgroundcolor=\color{background}]
  ./configure --with-debugging=0
  make all test
\end{lstlisting}

You will also need to set the \Textsfc{PETSC\_ARCH} and \Textsfc{PETSC\_DIR} variables in your
\Textsfc{.bashrc} file.  For example,
\begin{lstlisting}[backgroundcolor=\color{background}]
PETSC_ARCH=arch-linux-c-debug
export PETSC_ARCH 
PETSC_DIR=/home/banerjee/Packages/petsc-3.12.5
export PETSC_DIR
\end{lstlisting}

\subsection{Hypre Installation}
Hypre can be installed by executing the instructions in the HYPRE site.  The ICE-Hypre 
implementation has not been tested extensively and should be used with care.

Download hypre-2.8.0b from
\url{https://computation.llnl.gov/casc/hypre/software.html}

\chapter{Configuring and Building Vaango}
\section{Basic instructions}
\subsection{ The Vaango repository}
After you get the code from GitHub, follow these steps:
\begin{itemize}
  \item Go to the \Textsfc{Vaango} directory:

\begin{lstlisting}[backgroundcolor=\color{background}]
     cd ParSim/Vaango
\end{lstlisting}

  \item  The source code is in the directory \Textsfc{src}.
\end{itemize}

\subsection{ Out-of-source optimized build}
\begin{itemize}
  \item For the optimized build, create a new directory called \Textsfc{opt} under \Textsfc{Vaango}:

\begin{lstlisting}[backgroundcolor=\color{background}]
    mkdir opt
\end{lstlisting}

  \item followed by:

\begin{lstlisting}[backgroundcolor=\color{background}]
    cd opt
\end{lstlisting}

  \item To create the make files do:

\begin{lstlisting}[backgroundcolor=\color{background}]
    cmake ../src
\end{lstlisting}
\end{itemize}

\subsection{ Out-of-source debug build}
\begin{itemize}
 \item For the debug build, create a directory \Textsfc{dbg} under \Textsfc{Vaango}:

\begin{lstlisting}[backgroundcolor=\color{background}]
    mkdir dbg
    cd dbg
\end{lstlisting}

 \item To create the makefiles for the debug build, use

\begin{lstlisting}[backgroundcolor=\color{background}]
    cmake -DCMAKE_BUILD_TYPE=Debug ../src
\end{lstlisting}

\end{itemize}

\subsection{ Unit tests}
If you want the units tests to be compiled, use the alternative command

\begin{lstlisting}[backgroundcolor=\color{background}]
    cmake ../src -DBUILD_UNITS_TESTS=1
\end{lstlisting}

\subsection{ Clang compiler:}
If you want to used the \Textsfc{clang} compiler instead of \Textsfc{gcc}:

\begin{lstlisting}[backgroundcolor=\color{background}]
    cmake ../src -DUSE_CLANG=1
\end{lstlisting}

\subsection{ Visit build:}
Older versions of \Textsfc{Visit} required the following extra step if you want to make sure that Visit is able to read Uintah output format files (also called UDA files) you will need to use

\begin{lstlisting}[backgroundcolor=\color{background}]
   cmake ../src -DVISIT_DIR=/path/to/Visit
\end{lstlisting}

\section{Compiling the code:}
Next you need to compile the files from \Textsfc{src}. So go to the \Textsfc{opt} directory and then type : 
\begin{lstlisting}[backgroundcolor=\color{background}]
make -j4
\end{lstlisting}

After this compilation you have all the executable files in your \Textsfc{opt} directory.
The same process can  be used for the debug build.

\chapter{Installing VisIt}
Visualization of \Vaango data is currently possible using
\Visit. The \Visit package from LLNL is general purpose
visualization software that offers all of the usual capabilities for
rendering scientific data.  It is still developed and maintained by
LLNL staff, and its interface to \Vaango data is supported by
the \Uintah team. 

\section{Directory structure}
The following directory structure is suggested for building VisIt:

\begin{lstlisting}[backgroundcolor=\color{background}]
/myPath/VisIt
/myPath/VisIt/thirdparty
/myPath/VisIt/2.10.2
\end{lstlisting}

\section{Using the VisIt install script}
Download the build script from \\
\url{http://portal.nersc.gov/project/visit/releases/2.10.2/visit-install2_10_2}.
Follow the instructions at \url{http://portal.nersc.gov/project/visit/releases/2.10.2/INSTALL_NOTES}.
\Vaango writes its output in the \Uintah file format.  If \Visit does not recognize the
\Uintah file format, you may have to follow the detailed instructions below.

\section{Using the VisIt build script}
\begin{enumerate}
  \item Get the \Textsfc{build\_visit} script from \url{http://portal.nersc.gov/project/visit/releases/2.10.2/build_visit2_10_2}.

  \item Set the MPI compiler to be used (must be the same version used to build Uintah):
\begin{lstlisting}[backgroundcolor=\color{background}]
export PAR_COMPILER=`which mpic++` 
export PAR_COMPILER_CXX=`which mpicxx` 
export PAR_INCLUDE ="-I/usr/lib/openmpi/include/"
\end{lstlisting}

  \item Use the \Textsfc{build\_visit} script to download, build, and install the thirdparty 
        libraries required by VisIt. 

  \item Go the the \Visit thirdparty directory:
\begin{lstlisting}[backgroundcolor=\color{background}]
cd /mypath/VisIt/thirdparty
\end{lstlisting}

  \item Run \Textsfc{build\_visit}:
\begin{lstlisting}[backgroundcolor=\color{background}]
/myPath/VisIt/2.10.2/build_visit --console --thirdparty-path /myPath/VisIt/thirdparty --no-visit --parallel --fortran --uintah
\end{lstlisting}
       \begin{NoteBox}
       To add additional libraries to enable other database readers using the \Textsfc{build\_visit} script 
       you will have to add the appropriate flags when you run \Textsfc{build\_visit}.
       \end{NoteBox}

  \item The \Textsfc{build\_visit} script will take a while to finish.  After the script finishes it 
        will produce a machine specific \Textsfc{cmake} file called 
        \Textsfc{\textless your\_machine\_name \textgreater.cmake}. This file will be used when 
        compiling visit. 

  \item Move the \Textsfc{\textless your\_machine\_name \textgreater.cmake} file
        to the \Visit source \Textsfc{config-site} directory:
\begin{lstlisting}[backgroundcolor=\color{background}]
mv <your_machine_name>.cmake /myPath/VisIt/2.10.2/src/config-site
\end{lstlisting}

  \item Go to the \Visit source directory:
\begin{lstlisting}[backgroundcolor=\color{background}]
cd /myPath/VisIt/2.10.2/src
\end{lstlisting}

  \item Run \Textsfc{cmake} using the \Textsfc{cmake} that was built by the \Textsfc{build\_visit} script
\begin{lstlisting}[backgroundcolor=\color{background}]
/myPath/VisIt/thirdparty/cmake/<version>/<OS-gcc_version>bin/cmake .
\end{lstlisting}

  \item Finally make \Visit:
\begin{lstlisting}[backgroundcolor=\color{background}]
make -j 8
\end{lstlisting}
        One can install VisIt or run it directly from the bin directory.
\end{enumerate}


\subsection{Remote visualization}
In order to read data on a remote machine, you will need to have a
version of \Visit and the \Uintah data format reader plugin installed on that
machine. You will also need to make sure that the \Visit executable is
in your path. You may need to edit your \Textsfc{.\textless shell\textgreater rc} 
file so that the path to \Visit is included in your shell's \Textsfc{PATH} variable.

\begin{NoteBox}
The version of VisIt installed on your local desktop and the
remote machine should be the same.
\end{NoteBox}

\subsection{Host Profile}
Next you will want to set up a Host Profile for your remote
machine. Select \Textsfc{Host Profiles} from the \Textsfc{Options} menu and set up a
new profile as shown in figure~\ref{VisItHostProfile}.
\begin{figure}
  \centering
  \includegraphics[width=.425\textwidth]{VisItHostProfile.png}
  \caption{Setting up Host Profile}
  \label{VisItHostProfile}
\end{figure}

After filling in the remote machine information, select the \Textsfc{Advanced
options} tab, then \Textsfc{Networking} and check the \Textsfc{Tunnel data connections
through SSH} option. This is illustrated in
figure~\ref{VisItHostProfileAdv}(a). Click on \Textsfc{Apply} and then do a \Textsfc{Save
Settings} in the \Textsfc{Options} menu.
\begin{figure}[h]
  \centering
  \includegraphics[width=.5\textwidth]{VisItHostProfileAdv.png} \\
  (a) Setting up options \\
  \includegraphics[width=.5\textwidth]{VisItHostProfileAdv2.png} \\
  (b) Setting up advanced options
  \caption{Visit remote visualization options}
  \label{VisItHostProfileAdv}
\end{figure}

For the remote visualization option to work, you must have ports 5600
- 5609 open. You can try this by running the following on your local
desktop (on Linux distributions),
\begin{lstlisting}
traceroute -p 560[0-9] <remote machine>
\end{lstlisting}

If the tunneling option doesn't works, try the option as shown in
figure~\ref{VisItHostProfileAdv}(b).

Now you should be able select \Textsfc{Open} from the \Visit \Textsfc{File} menu. After
selecting your host from the host entry list, you will be prompted for
a password on the remote machine (unless you have set up passwordless
ssh access).

Once the ssh login has completed, you should see the directory
listing. You can then change directories to your UDA and load the
data.

%
%----------------------------------------------------------------------------------------
%       BIBLIOGRAPHY
%----------------------------------------------------------------------------------------

\chapter*{Bibliography}
\addcontentsline{toc}{chapter}{\textcolor{ocre}{Bibliography}}

%\bibliographystyle{plain}
%\bibliography{Bibliography}
\printbibliography[heading=bibempty]




\end{document}
